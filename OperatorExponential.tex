\documentclass[12pt]{article}

\usepackage{amsthm}
\usepackage{amsmath}
\usepackage{amsfonts}
\usepackage{mathrsfs}
\usepackage{array}
\usepackage{amssymb}
\usepackage{units}
\usepackage{graphicx}
\usepackage{tikz-cd}
\usepackage{nicefrac}
\usepackage{hyperref}
\usepackage{bbm}
\usepackage{color}
\usepackage{tensor}
\usepackage{tipa}
\usepackage{bussproofs}
\usepackage{ stmaryrd }
\usepackage{ textcomp }
\usepackage{leftidx}
\usepackage{afterpage}
\usepackage{varwidth}
\usepackage{tasks}
\usepackage{ cmll }
\usepackage{makecell}
\usepackage{MnSymbol}
\usepackage{quiver}
\usepackage{adjustbox}
\usepackage{multirow}
\usepackage{booktabs}
\usepackage{xparse}
\usepackage{calc}

\newcommand\blankpage{
	\null
	\thispagestyle{empty}
	\addtocounter{page}{-1}
	\newpage
}

\newcommand{\PhantC}{\phantom{\colon}}%
\newcommand{\PhantSQ}{\phantom{\sqrt{\hspace{0.3ex}}}}

% https://tex.stackexchange.com/questions/63355/wrapping-cmidrule-in-a-macro
\ExplSyntaxOn
\makeatletter
\newcommand{\CMidRule}{\noalign\bgroup\@CMidRule{}}
\NewDocumentCommand{\@CMidRule}{
	m % Material to reinsert before cmidrule.
	O{0.0ex} % #1 = left adjust
	O{0.0ex} % #1 = right adjust
	m  %       #3 = columns to span
}{
	\peek_meaning_remove_ignore_spaces:NTF \CMidRule
	{ \@CMidRule { #1 \cmidrule[\cmidrulewidth](l{#2}r{#3}){#4} } }
	{ \egroup #1 \cmidrule[\cmidrulewidth](l{#2}r{#3}){#4} }
}
\makeatother
\ExplSyntaxOff

\graphicspath{ {images/} }

\theoremstyle{plain}
\newtheorem{thm}{Theorem}[subsection] % reset theorem numbering for each chapter
\newtheorem{proposition}[thm]{Proposition}
\newtheorem{lemma}[thm]{Lemma}
\newtheorem{fact}[thm]{Fact}
\newtheorem{cor}[thm]{Corollary}

\theoremstyle{definition}
\newtheorem{defn}[thm]{Definition} % definition numbers are dependent on theorem numbers
\newtheorem{exmp}[thm]{Example} % same for example numbers
\newtheorem{notation}[thm]{Notation}
\newtheorem{remark}[thm]{Remark}
\newtheorem{condition}[thm]{Condition}
\newtheorem{question}[thm]{Question}
\newtheorem{construction}[thm]{Construction}
\newtheorem{exercise}[thm]{Exercise}
\newtheorem{example}[thm]{Example}
\newtheorem{aside}[thm]{Aside}
\newtheorem{algorithm}[thm]{Algorithm}

\def\doubleunderline#1{\underline{\underline{#1}}}
\newcommand{\bb}[1]{\mathbb{#1}}
\newcommand{\scr}[1]{\mathscr{#1}}
\newcommand{\call}[1]{\mathcal{#1}}
\newcommand{\psheaf}{\text{\underline{Set}}^{\scr{C}^{\text{op}}}}
\newcommand{\und}[1]{\underline{\hspace{#1 cm}}}
\newcommand{\adj}[1]{\text{\textopencorner}{#1}\text{\textcorner}}
\newcommand{\comment}[1]{}
\newcommand{\lto}{\longrightarrow}
\newcommand{\rone}{(\operatorname{R}\bold{1})}
\newcommand{\lone}{(\operatorname{L}\bold{1})}
\newcommand{\rimp}{(\operatorname{R} \multimap)}
\newcommand{\limp}{(\operatorname{L} \multimap)}
\newcommand{\rtensor}{(\operatorname{R}\otimes)}
\newcommand{\ltensor}{(\operatorname{L}\otimes)}
\newcommand{\rtrue}{(\operatorname{R}\top)}
\newcommand{\rwith}{(\operatorname{R}\&)}
\newcommand{\lwithleft}{(\operatorname{L}\&)_{\operatorname{left}}}
\newcommand{\lwithright}{(\operatorname{L}\&)_{\operatorname{right}}}
\newcommand{\rplusleft}{(\operatorname{R}\oplus)_{\operatorname{left}}}
\newcommand{\rplusright}{(\operatorname{R}\oplus)_{\operatorname{right}}}
\newcommand{\lplus}{(\operatorname{L}\oplus)}
\newcommand{\prom}{(\operatorname{prom})}
\newcommand{\ctr}{(\operatorname{ctr})}
\newcommand{\der}{(\operatorname{der})}
\newcommand{\weak}{(\operatorname{weak})}
\newcommand{\exi}{(\operatorname{exists})}
\newcommand{\fa}{(\operatorname{for\text{ }all})}
\newcommand{\ex}{(\operatorname{ex})}
\newcommand{\cut}{(\operatorname{cut})}
\newcommand{\ax}{(\operatorname{ax})}
\newcommand{\negation}{\sim}
\newcommand{\true}{\top}
\newcommand{\false}{\bot}
\DeclareRobustCommand{\diamondtimes}{%
	\mathbin{\text{\rotatebox[origin=c]{45}{$\boxplus$}}}%
}
\newcommand{\tagarray}{\mbox{}\refstepcounter{equation}$(\theequation)$}
\newcommand{\startproof}[1]{
	\AxiomC{#1}
	\noLine
	\UnaryInfC{$\vdots$}
}
\newcommand\showdiv[1]{\overline{\smash{)}#1}}
\newcommand{\set}{\operatorname{\underline{Set}}}



\newenvironment{scprooftree}[1]%
{\gdef\scalefactor{#1}\begin{center}\proofSkipAmount \leavevmode}%
	{\scalebox{\scalefactor}{\DisplayProof}\proofSkipAmount \end{center} }

\usepackage[margin=1cm]{geometry}

\title{Operator exponential}
\author{William Troiani}
\date{\today}

\begin{document}
	\maketitle
	\tableofcontents
	
	\section{Introduction}
	The (completely fulfilling) motivation of this note is to solve a differential equation using the exponential of an operator.
	
	\section{Defining the exponential of an operator}
	Throughout, $\bb{F}$ denotes either $\bb{R}$ or $\bb{C}$. All vector spaces will be over $\bb{F}$.
	
	First we recall the definition of a norm.
	
	\begin{defn}
		A \textbf{norm} on $V$ is a function $\| \cdot \|: V \lto \bb{F}$ satisfying the following axioms.
		\begin{itemize}
			\item $\| x \| = 0 \Longleftrightarrow x = 0$.
			\item $\forall x \in V, \| x \| \geq 0$.
			\item $\forall x, y \in V, \| x + y \| \leq \| x \| + \| y \|$.
			\item $\forall x \in V, \forall \lambda \in \bb{F}, \| \lambda x \| = |\lambda| \| x \|$., where $|\lambda|$ denotes the magnitude (ie, the standard norm) on $\bb{F}$.
			\end{itemize}
		\end{defn}
	
	We now work towards Proposition \ref{prop:all_norms_equiv} below which states that all norms on the same finite dimensional vector space are equivalent. This assists greatly with calculations.
	
	\begin{defn}
		Two norms $\| \cdot \|_1, \| \cdot \|_2$ are \textbf{Lipschitz equivalent} if there exists $c,C \in \bb{F}$ subject to the following.
		\begin{equation}\label{eq:lipschitz}
			\forall x \in V, c\| x \|_1 \leq \| x \|_2 \leq C\| x \|_1
			\end{equation}
		\end{defn}
	
	\begin{lemma}\label{lem:homeomorphic_F}
		Let $\| \cdot \|_1$ denote the $L_1$ norm on $\bb{F}^n$, that is,
		\begin{equation}
			\forall x = (x_1,\ldots, x_n) \in \bb{F}^n, \| x \| = \sum_{i = 1}^n | x_i |
			\end{equation}
		Let $\| \cdot \|$ denote the standard norm, given by the following equation.
		\begin{equation}
			\forall x = (x_1, \ldots, x_n) \in \bb{F}^n, \| x \| = \sqrt{\sum_{i = 1}^n |x_i|^2}
			\end{equation}
		Then $\bb{F}^n$ endowed with the topology induced by $\| \cdot \|_1$ is homeomorphic to $\bb{F}^n$ endowed with the topology induced by $\| \cdot \|$.
		\end{lemma}
	\begin{proof}
		We let $(\bb{F}, \| \cdot \|_1)$ and $(\bb{F}, \| \cdot \|)$ respectively denote these two topological spaces. We claim the identity function $\operatorname{id}: (\bb{F}, \| \cdot \|_1) \lto (\bb{F}, \| \cdot \|)$ is a homeomorphism. The only check to do is that both identity functions 
		\begin{align}
			\label{eq:id_1}(\bb{F}, \| \cdot \|_1) &\lto (\bb{F}, \| \cdot \|)\\
			\label{eq:id_2}(\bb{F}, \| \cdot \|) &\lto (\bb{F}, \| \cdot \|_1)
			\end{align}
		are continuous.
		
		First, consider \eqref{eq:id_1}. We use the simple fact that $\sqrt{x_1 + \ldots + x_n} \leq \sqrt{x_1} + \ldots + \sqrt{x_n}$ for all $x_1,\ldots, x_n \in \bb{F}$ to show that for all $x = (x_1,\ldots, x_n)$ and all $x' = (x_1',\ldots, x_n')$ we have
		\begin{equation}
			\| x - x' \| = \sqrt{\sum_{i=1}^n |x_i = x_i'|^2} \leq \sum_{i = 1}^n \sqrt{|x_i - x_i'|^2} = \sum_{i = 1}^n |x_i - x_i'| = \| x - x' \|_1
			\end{equation}
		This proves continuity of \eqref{eq:id_1}.
		
		On the other hand, for all $x = (x_1,\ldots, x_n)$ and all $x' = (x_1',\ldots, x_n')$ the following holds.
		\begin{equation}
			\| x - x' \|_1 = \sum_{i = 1}^n | x_i - x_i' | \leq n \operatorname{sup}_{i = 1,..., n}\{ |x_i - x_i'| \} \leq n \sqrt{\sum_{i = 1}^n | x_i - x_i' |^2} \leq \| x - x' \| n
			\end{equation}
		It follows that \eqref{eq:id_2} is continuous.
		\end{proof}
	\begin{remark}
		We remark that the proof of Lemma \ref{lem:homeomorphic_F} shows that in fact the functions \eqref{eq:id_1}, \eqref{eq:id_2} are \emph{uniformly} continuous.
		\end{remark}
	
	\begin{proposition}\label{prop:all_norms_equiv}
		Any two norms on a finite dimensional vector space $V$ are Lipschitz equivalent.
		\end{proposition}
	\begin{proof}
		First, we claim that Lipschitz equivalence is an equivalence relation on the set of norms on $V$. Reflexivity is immediate, one takes $c = C = 1$. For symmetric, say $\| \cdot \|_1$ and $\| \cdot \|_2$ are Lipschitz equivalent. Then there exists $c,C \in \bb{F}$ satisfying \eqref{eq:lipschitz} and we notice that
		\begin{equation}
			\forall x \in V, \frac{1}{C}\| x \|_2 \leq \| x \|_1 \leq \frac{1}{c}\| x \|
			\end{equation}
		establishing symmetry. Moreover, if there was a third norm $\| \cdot \|_3$ for which there exists $k, K \in \bb{F}$ subject to
		\begin{equation}
			\forall x \in V, k\| x \|_2 \leq \| x \|_3 \leq K\| x \|_2
			\end{equation}
		then we have the following chain of inequalities.
		\begin{equation}
			\forall x \in V, ck \|x\|_1 \leq k \| x \|_2 \leq \| x \|_3 \leq K \| x \|_2 \leq CK \| x \|_1
			\end{equation}
		which proves transitivity.
		
		Thus, it suffices to show that any norm $\| \cdot \|$ is Lipschitz equivalent to the following norm. Pick a basis $\{ v_1, \ldots, v_n \}$ for $V$ and define
		\begin{align}
			\| \cdot \|_1 : V &\lto \bb{F}\\
			x = \sum_{i = 1}^n \alpha_i v_i &\longmapsto \sum_{i = 1}^n |\alpha_i|
			\end{align}
		It is easy to check that this is a norm, indeed it is the norm induced by the normed space isomorphism $V \cong \bb{F}^n$ where $\bb{F}^n$ is endowed with the $L_1$ norm.
		
		Our next claim is that this norm $\| \cdot \|$ is uniformly continuous, when $V$ is endowed with the $\| \cdot \|_1$ norm. We must show
		\begin{equation}
			\forall x, x' \in V, \forall \epsilon > 0, \exists \delta > 0, \| x - x' \|_1 < \delta \Longrightarrow | \| x \| - \| x' \| | < \epsilon
			\end{equation}
		Towards this end, let $x,x' \in V, \epsilon > 0$ be arbitrary and define
		\begin{equation}
			M := \operatorname{sup}_{i = 1,..., n}\{ \| v_i \| \}
			\end{equation}
		Then we have the following calculation, where we write $x = \sum_{i = 1}^n \alpha_i v_i, x' = \sum_{i = 1}^n \alpha_i' v_i$.
		\begin{align*}
			\| x - x' \| &= \| \sum_{i = 1}^n(\alpha_i - \alpha_i')v_i \|\\
			&\leq | \sum_{i = 1}^n(\alpha_i - \alpha_i') | \| v \|\\
			&\leq M | \sum_{i = 1}^n(\alpha_i - \alpha_i') |\\
			& = M \| x - x' \|_1
			\end{align*}
		So $\delta = \epsilon/M$ is an appropriate choice for $\delta$. This establishes the claim.
		
		Now we consider the unit ball $\{ v \in V \mid \| v \| \} \subseteq V$ which is compact by Lemma \ref{lem:homeomorphic_F}; the image of this set under the homeomorphism $V \cong \bb{F}^n$ is closed and bounded, thus compact by the Heine-Borel Theorem.
		
		Thus, by the extreme value Theorem, we have that $\| \cdot \|$ admits its maximum and minimum on this set.
		\begin{align*}
			c &:= \operatorname{inf}\{ \| v \| \mid  \| v \|_1 = 1\}\\
			C &:= \operatorname{sup}\{ \| v \| \mid \| v \|_1 = 1 \}
			\end{align*}
		Thus, if $v \in V$ is such that $\| v \|_1 = 1$, we have
		\begin{equation}\label{eq:lipschitz_standard}
			c\| v \|_1 \leq \| v \| \leq C\| v \|_1
			\end{equation}
		If $\| v \|_1 = 0$ then $v = 0$ and \eqref{eq:lipschitz_standard} clearly holds. If $\| v \|_1 \neq 1, 0$ then we divide by $\| v \|_1$ and reduce to the case of \eqref{eq:lipschitz_standard}.
		\end{proof}
	
	
	
	\end{document}


























