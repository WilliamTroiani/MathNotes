\documentclass[12pt]{article}

\usepackage{amsthm}
\usepackage{amsmath}
\usepackage{amsfonts}
\usepackage{mathrsfs}
\usepackage{array}
\usepackage{amssymb}
\usepackage{units}
\usepackage{graphicx}
\usepackage{tikz-cd}
\usepackage{nicefrac}
\usepackage{hyperref}
\usepackage{bbm}
\usepackage{color}
\usepackage{tensor}
\usepackage{tipa}
\usepackage{bussproofs}
\usepackage{ stmaryrd }
\usepackage{ textcomp }
\usepackage{leftidx}
\usepackage{afterpage}
\usepackage{varwidth}
\usepackage{tasks}
\usepackage{ cmll }
\usepackage{quiver}
\usepackage{adjustbox}
\usepackage{kbordermatrix}

\newcommand\blankpage{
	\null
	\thispagestyle{empty}
	\addtocounter{page}{-1}
	\newpage
}

\graphicspath{ {images/} }

\theoremstyle{plain}
\newtheorem{thm}{Theorem}[subsection] % reset theorem numbering for each chapter
\newtheorem{proposition}[thm]{Proposition}
\newtheorem{lemma}[thm]{Lemma}
\newtheorem{fact}[thm]{Fact}
\newtheorem{cor}[thm]{Corollary}

\theoremstyle{definition}
\newtheorem{defn}[thm]{Definition} % definition numbers are dependent on theorem numbers
\newtheorem{exmp}[thm]{Example} % same for example numbers
\newtheorem{notation}[thm]{Notation}
\newtheorem{remark}[thm]{Remark}
\newtheorem{condition}[thm]{Condition}
\newtheorem{question}[thm]{Question}
\newtheorem{construction}[thm]{Construction}
\newtheorem{exercise}[thm]{Exercise}
\newtheorem{example}[thm]{Example}
\newtheorem{aside}[thm]{Aside}

\def\doubleunderline#1{\underline{\underline{#1}}}
\newcommand{\bb}[1]{\mathbb{#1}}
\newcommand{\scr}[1]{\mathscr{#1}}
\newcommand{\call}[1]{\mathcal{#1}}
\newcommand{\psheaf}{\text{\underline{Set}}^{\scr{C}^{\text{op}}}}
\newcommand{\und}[1]{\underline{\hspace{#1 cm}}}
\newcommand{\adj}[1]{\text{\textopencorner}{#1}\text{\textcorner}}
\newcommand{\comment}[1]{}
\newcommand{\lto}{\longrightarrow}
\newcommand{\rone}{(\operatorname{R}\bold{1})}
\newcommand{\lone}{(\operatorname{L}\bold{1})}
\newcommand{\rimp}{(\operatorname{R} \multimap)}
\newcommand{\limp}{(\operatorname{L} \multimap)}
\newcommand{\rtensor}{(\operatorname{R}\otimes)}
\newcommand{\ltensor}{(\operatorname{L}\otimes)}
\newcommand{\rtrue}{(\operatorname{R}\top)}
\newcommand{\rwith}{(\operatorname{R}\&)}
\newcommand{\lwithleft}{(\operatorname{L}\&)_{\operatorname{left}}}
\newcommand{\lwithright}{(\operatorname{L}\&)_{\operatorname{right}}}
\newcommand{\rplusleft}{(\operatorname{R}\oplus)_{\operatorname{left}}}
\newcommand{\rplusright}{(\operatorname{R}\oplus)_{\operatorname{right}}}
\newcommand{\lplus}{(\operatorname{L}\oplus)}
\newcommand{\prom}{(\operatorname{prom})}
\newcommand{\ctr}{(\operatorname{ctr})}
\newcommand{\der}{(\operatorname{der})}
\newcommand{\weak}{(\operatorname{weak})}
\newcommand{\exi}{(\operatorname{exists})}
\newcommand{\fa}{(\operatorname{for\text{ }all})}
\newcommand{\ex}{(\operatorname{ex})}
\newcommand{\cut}{(\operatorname{cut})}
\newcommand{\ax}{(\operatorname{ax})}
\newcommand{\negation}{\sim}
\newcommand{\true}{\top}
\newcommand{\false}{\bot}
\newcommand{\imp}{\supset}
\DeclareRobustCommand{\diamondtimes}{%
	\mathbin{\text{\rotatebox[origin=c]{45}{$\boxplus$}}}%
}
\newcommand{\tagarray}{\mbox{}\refstepcounter{equation}$(\theequation)$}
\newcommand{\startproof}[1]{
	\AxiomC{#1}
	\noLine
	\UnaryInfC{$\vdots$}
}
\newenvironment{scprooftree}[1]%
{\gdef\scalefactor{#1}\begin{center}\proofSkipAmount \leavevmode}%
	{\scalebox{\scalefactor}{\DisplayProof}\proofSkipAmount \end{center} }

\usepackage[margin=1.5cm]{geometry}


\title{Linear Logic: Exponentials}
\author{Will Troiani}
\date{\today}

\begin{document}
	\maketitle
	
	\section{Introduction}
	
	\section{Sequent style MELL}
	\begin{defn}
		There is an infinite set of \textbf{atoms} $X, Y, Z, \ldots$ The set of \textbf{formulas} is defined as follows.
		\begin{itemize}
			\item Any atomic formula is a pre-formula.
			\item If $A,B$ are pre-formulas then so is $A \otimes B, A \parr B$.
			\item If $A$ is a pre-formula, then so is $\neg A, ! A, ?A$.
		\end{itemize}
		The set of \textbf{formulas} is the quotient of the set of pre-formulas by the equivalence relation $\sim$ generated by, for any pre-formulas $A,B$ and atomic formula $X$, the following.
		\begin{align*}
			\neg(A \otimes B) &\sim \neg A \parr \neg B & \neg(A \parr B) &\sim \neg A \otimes \neg B\\
			\neg (X,+) &\sim (X,-) & \neg (X,-) &\sim (X,+)\\
			\neg !A &\sim ? \neg A & \neg ? A &\sim ! \neg A
		\end{align*}
	\end{defn}
	
	\begin{defn}\label{def:exponential_deduction_rules}
		An \textbf{exponential deduciton rules} result from one of the schemata below by a substitutiong of the following kind: replace $A, B$ by arbitrary formulas, and $\Gamma, \Gamma', \Delta, \Delta'$ by arbitrary (possibly empty) sequences of formulas separated by commas:
		\begin{itemize}
			\item \textbf{Dereliction}:
			\begin{center}
			\AxiomC{$\vdash \Gamma, A, \Gamma'$}
			\RightLabel{$\der$}
			\UnaryInfC{$\vdash \Gamma, ?A, \Gamma'$}
			\DisplayProof
			\end{center}
			%
			\item \textbf{Promotion}:
			\begin{center}
			\AxiomC{$\vdash ?\Gamma, A, ?\Gamma'$}
			\RightLabel{$\prom$}
			\UnaryInfC{$\vdash ?\Gamma, !A, ?\Gamma'$}
			\DisplayProof
			\end{center}
			%
			\item \textbf{Weakening}:
			\begin{center}
			\AxiomC{$\vdash \Gamma, \Gamma'$}
			\RightLabel{$\weak$}
			\UnaryInfC{$\vdash \Gamma, ?A, \Gamma'$}
			\DisplayProof
			\end{center}
			%
			\item \textbf{Contraction}:
			\begin{center}
			\AxiomC{$\vdash \Gamma, ?A, ?A, \Gamma'$}
			\RightLabel{$\ctr$}
			\UnaryInfC{$\vdash \Gamma, ?A, \Gamma'$}
			\DisplayProof
			\end{center}
			\end{itemize}
		\end{defn}
	\begin{defn}
		A \textbf{proof in MELL} (multiplicative, exponential linear logic) is a finite, rooted, planar, tree where each edge is labelled by a sequent and each node except for the root is labelled by a valid deduction rule (out of those in Definition \ref{def:exponential_deduction_rules} or \cite[Definition 1.0.5]{LL}). If the edge connected to the root is labelled by the sequent $\vdash \Gamma$ then we call the proof a \textbf{proof of $\Gamma$} and in such a situation, $\Gamma$ is the conclusion of $\pi$.
		\end{defn}
	
	\begin{remark}
	There is also an ``intuitionistic" version of MELL, for which there is no negation $(\neg)$, no ``why not" $(?)$, and no par $(\parr)$. This consists of the intuitionistic, multiplicagive deduction rules \cite[Definition 1.0.10]{LL} along with the following, which are just the rules of Definition \ref{def:exponential_deduction_rules} written with only one hypothesis on the right side of the turnstile $(\vdash)$.
	\begin{itemize}
	\item \textbf{Dereliction}:
	\begin{center}
		\AxiomC{$ \Gamma, A \vdash B$}
		\RightLabel{$\der$}
		\UnaryInfC{$\Gamma, !A, \Gamma' \vdash B$}
		\DisplayProof
	\end{center}
	%
	\item \textbf{Promotion}:
	\begin{center}
		\AxiomC{$!\Gamma \vdash A$}
		\RightLabel{$\prom$}
		\UnaryInfC{$!\Gamma \vdash !A$}
		\DisplayProof
	\end{center}
	%
	\item \textbf{Weakening}:
	\begin{center}
		\AxiomC{$\Gamma, \Gamma' \vdash B$}
		\RightLabel{$\weak$}
		\UnaryInfC{$\Gamma, !A, \Gamma' \vdash B$}
		\DisplayProof
	\end{center}
	%
	\item \textbf{Contraction}:
	\begin{center}
		\AxiomC{$\Gamma, !A, !A, \Gamma' \vdash B$}
		\RightLabel{$\ctr$}
		\UnaryInfC{$ \Gamma, !A, \Gamma'\vdash B$}
		\DisplayProof
	\end{center}
\end{itemize}
	\end{remark}

There is a standard translation of intuitionistic sequent calculus into intuitionistic MELL which we now describe. We references \cite[Definition 2.2]{GMZ} for the formal definitions of intuitionistic (pre)proofs, however here we will not use names in our variables. We present two translations, one of which performs derelictions as early as possible, and the other as late as possible.

\begin{defn}
	Let $\Pi_!$ denote the set of MELL linear logic proofs and let $I_\imp$ denote the set of intuitionistic proofs. We define a translation $T: I_\imp \lto \Pi_!$.
	\begin{center}
		\begin{tabular}{ >{\centering}m{2cm} >{\centering}m{5cm} >{\centering}m{0.5cm} >{\centering}m{5cm}}
			\textbf{Axiom}: &
				\begin{prooftree}
				\AxiomC{}
				\RightLabel{$\ax$}
				\UnaryInfC{$\Gamma \vdash X$}
				\end{prooftree}
			&
			$\stackrel{T}{\lto}$
			&
				\begin{prooftree}
				\AxiomC{}
				\RightLabel{$\ax$}
				\UnaryInfC{$X \vdash X$}
				\end{prooftree}
			\end{tabular}
		\begin{tabular}{ >{\centering}m{2cm} >{\centering}m{5cm} >{\centering}m{3cm} >{\centering}m{5cm}}
		\textbf{Cut}: &
		\begin{prooftree}
			\AxiomC{$\Gamma \vdash A$}
			\AxiomC{$\Delta, X, \Theta \vdash B$}
			\RightLabel{$\cut$}
			\BinaryInfC{$\Gamma, \Delta, \Theta \vdash B$}
			\end{prooftree}
		&
		$\stackrel{T}{\lto}$
		&
		\begin{prooftree}
			\AxiomC{$\Gamma \vdash A$}
			\AxiomC{$\Delta, X, \Theta \vdash B$}
			\RightLabel{$\cut$}
			\BinaryInfC{$\Gamma, \Delta, \Theta \vdash B$}
			\end{prooftree}
		\end{tabular}
	\begin{tabular}{ >{\centering}m{4cm} >{\centering}m{5cm} >{\centering}m{0.5cm} >{\centering}m{5cm}}
		\textbf{Contraction}: &
		\begin{prooftree}
			\AxiomC{$\Gamma, X, X, \Delta \vdash A$}
			\RightLabel{$\ctr$}
			\UnaryInfC{$\Gamma, X, \Delta \vdash A$}
		\end{prooftree}
		&
		$\stackrel{T}{\lto}$
		&
		\begin{prooftree}
			\AxiomC{$\Gamma, X, X, \Delta \vdash A$}
			\RightLabel{$\der$}
			\UnaryInfC{$\Gamma, !X, X, \Delta \vdash A$}
			\RightLabel{$\der$}
			\UnaryInfC{$\Gamma, !X, !X, \Delta \vdash A$}
			\RightLabel{$\ctr$}
			\UnaryInfC{$\Gamma, !X, \Delta \vdash A$}
		\end{prooftree}
	\end{tabular}
\begin{tabular}{ >{\centering}m{2cm} >{\centering}m{5cm} >{\centering}m{0.5cm} >{\centering}m{5cm}}
	\textbf{Weakening}: &
	\begin{prooftree}
		\AxiomC{$\Gamma, \Delta \vdash A$}
		\RightLabel{$\weak$}
		\UnaryInfC{$\Gamma, X, \Delta \vdash A$}
	\end{prooftree}
	&
	$\stackrel{T}{\lto}$
	&
	\begin{prooftree}
		\AxiomC{$\Gamma, \Delta \vdash A$}
		\RightLabel{$\weak$}
		\UnaryInfC{$\Gamma, !X, \Delta \vdash A$}
	\end{prooftree}
\end{tabular}
\begin{tabular}{ >{\centering}m{2cm} >{\centering}m{5cm} >{\centering}m{0.5cm} >{\centering}m{5cm}}
\textbf{Right introduction}: &
\begin{prooftree}
	\AxiomC{$\Gamma, X, \Delta \vdash A$}
	\RightLabel{$(R \imp)$}
	\UnaryInfC{$\Gamma, \Delta \vdash X \imp A$}
\end{prooftree}
&
$\stackrel{T}{\lto}$
&
\begin{prooftree}
	\AxiomC{$\Gamma, X, \Delta \vdash A$}
	\RightLabel{$\rimp$}
	\UnaryInfC{$\Gamma, \Delta \vdash X \multimap A$}
\end{prooftree}
			\end{tabular}
		\begin{tabular}{ >{\centering}m{3cm} >{\centering}m{5cm} >{\centering}m{1cm} >{\centering}m{5cm}}
			\textbf{Left introduction}: &
			\begin{prooftree}
				\AxiomC{$\Gamma \vdash A$}
				\AxiomC{$\Delta, X, \Theta \vdash B$}
				\RightLabel{$(\operatorname{L} \imp)$}
				\BinaryInfC{$A \imp X, \Gamma, \Delta, \Theta \vdash B$}
			\end{prooftree}
			&
			$\stackrel{T}{\lto}$
			&
			\begin{prooftree}
				\AxiomC{$\Gamma \vdash A$}
				\AxiomC{$\Delta, X, \Theta \vdash B$}
				\RightLabel{$\limp$}
				\BinaryInfC{$A \multimap X, \Gamma, \Delta, \Theta \vdash B$}
			\end{prooftree}
		\end{tabular}
	\end{center}
	\end{defn}

\begin{defn}\label{def:translation}
	We present an alternate translation which performs dereliction as \emph{early} as possible.
	\begin{center}
		\begin{tabular}{ >{\centering}m{2cm} >{\centering}m{5cm} >{\centering}m{0.5cm} >{\centering}m{5cm}}
			\textbf{Axiom}: &
			\begin{prooftree}
				\AxiomC{}
				\RightLabel{$\ax$}
				\UnaryInfC{$A \vdash A$}
			\end{prooftree}
			&
			$\stackrel{T'}{\lto}$
			&
			\begin{prooftree}
				\AxiomC{}
				\RightLabel{$\ax$}
				\UnaryInfC{$A \vdash A$}
				\RightLabel{$\der$}
				\UnaryInfC{$!A \vdash A$}
			\end{prooftree}
		\end{tabular}
		\begin{tabular}{ >{\centering}m{2cm} >{\centering}m{5cm} >{\centering}m{3cm} >{\centering}m{5cm}}
			\textbf{Cut}: &
			\begin{prooftree}
				\AxiomC{$\Gamma \vdash A$}
				\AxiomC{$\Delta, X, \Theta \vdash B$}
				\RightLabel{$\cut$}
				\BinaryInfC{$\Gamma, \Delta, \Theta \vdash B$}
			\end{prooftree}
			&
			$\stackrel{T'}{\lto}$
			&
			\begin{prooftree}
				\AxiomC{$!\Gamma \vdash A$}
				\AxiomC{$!\Delta, !X, !\Theta \vdash B$}
				\RightLabel{$\cut$}
				\BinaryInfC{$!\Gamma, !\Delta, !\Theta \vdash B$}
			\end{prooftree}
		\end{tabular}
		\begin{tabular}{ >{\centering}m{4cm} >{\centering}m{5cm} >{\centering}m{0.5cm} >{\centering}m{5cm}}
			\textbf{Contraction}: &
			\begin{prooftree}
				\AxiomC{$\Gamma, X, X, \Delta \vdash A$}
				\RightLabel{$\ctr$}
				\UnaryInfC{$\Gamma, X, \Delta \vdash A$}
			\end{prooftree}
			&
			$\stackrel{T'}{\lto}$
			&
			\begin{prooftree}
				\AxiomC{$!\Gamma, !X, !X, !\Delta \vdash A$}
				\RightLabel{$\ctr$}
				\UnaryInfC{$!\Gamma, !X, !\Delta \vdash A$}
			\end{prooftree}
		\end{tabular}
		\begin{tabular}{ >{\centering}m{2cm} >{\centering}m{5cm} >{\centering}m{0.5cm} >{\centering}m{5cm}}
			\textbf{Weakening}: &
			\begin{prooftree}
				\AxiomC{$\Gamma, \Delta \vdash A$}
				\RightLabel{$\weak$}
				\UnaryInfC{$\Gamma, X, \Delta \vdash A$}
			\end{prooftree}
			&
			$\stackrel{T'}{\lto}$
			&
			\begin{prooftree}
				\AxiomC{$!\Gamma, !\Delta \vdash A$}
				\RightLabel{$\weak$}
				\UnaryInfC{$!\Gamma, !X, !\Delta \vdash A$}
			\end{prooftree}
		\end{tabular}
		\begin{tabular}{ >{\centering}m{2cm} >{\centering}m{5cm} >{\centering}m{0.5cm} >{\centering}m{5cm}}
			\textbf{Right introduction}: &
			\begin{prooftree}
				\AxiomC{$\Gamma, X, \Delta \vdash A$}
				\RightLabel{$(\operatorname{R} \imp)$}
				\UnaryInfC{$\Gamma, \Delta \vdash X \imp A$}
			\end{prooftree}
			&
			$\stackrel{T'}{\lto}$
			&
			\begin{prooftree}
				\AxiomC{$!\Gamma, !X, !\Delta \vdash A$}
				\RightLabel{$\rimp$}
				\UnaryInfC{$!\Gamma, !\Delta \vdash !X \multimap A$}
			\end{prooftree}
		\end{tabular}
		\begin{tabular}{ >{\centering}m{4cm} >{\centering}m{5cm} >{\centering}m{1cm} >{\centering}m{5cm}}
			\textbf{Light introduction}: &
			\begin{prooftree}
				\AxiomC{$\Gamma \vdash A$}
				\AxiomC{$\Delta, X, \Theta \vdash B$}
				\RightLabel{$(\operatorname{L} \imp)$}
				\BinaryInfC{$A \imp X, \Gamma, \Delta, \Theta \vdash B$}
			\end{prooftree}
			&
			$\stackrel{T'}{\lto}$
			&
			\begin{prooftree}
				\AxiomC{$!\Gamma \vdash A$}
				\RightLabel{$\prom$}
				\UnaryInfC{$!\Gamma \vdash !A$}
				\AxiomC{$!\Delta, !X, !\Theta \vdash B$}
				\RightLabel{$\limp$}
				\BinaryInfC{$!A \multimap !X, !\Gamma,!\Delta, !\Theta \vdash B$}
				\RightLabel{$\der$}
				\UnaryInfC{$!(!A \multimap !X), !\Gamma,!\Delta, !\Theta \vdash B$}
			\end{prooftree}
		\end{tabular}
	\end{center}
\end{defn}
	
	\begin{example}
		The following is denoted $\underline{2}_A$ and is the translation of the Church numeral $\underline{2}$.
		\begin{center}
			\AxiomC{}
			\RightLabel{$\ax$}
			\UnaryInfC{$A \vdash A$}
			\AxiomC{}
			\RightLabel{$\ax$}
			\UnaryInfC{$A \vdash A$}
			\AxiomC{}
			\RightLabel{$\ax$}
			\UnaryInfC{$A \vdash A$}
			\RightLabel{$\limp$}
			\BinaryInfC{$A, A \multimap A \vdash A$}
			\RightLabel{$\limp$}
			\BinaryInfC{$A, A \multimap A, A \multimap A \vdash A$}
			\RightLabel{$\der$}
			\UnaryInfC{$A, !(A \multimap A), A \multimap A \vdash A$}
			\RightLabel{$\der$}
			\UnaryInfC{$A, !(A \multimap A), !(A \multimap A) \vdash A$}
			\RightLabel{$\ctr$}
			\UnaryInfC{$A, !(A \multimap A) \vdash A$}
			\RightLabel{$\der$}
			\UnaryInfC{$!A, !(A \multimap A) \vdash A$}
			\RightLabel{$\rimp$}
			\UnaryInfC{$!(A \multimap A) \vdash !A \multimap A$}
			\DisplayProof
			\end{center}
		\end{example}
	
	
	
	\section{Persistent paths}
	
	
	
	
	
	
	
	
	
	
	
	
	
	
	
	
	
	
	
	
	
	
	
	
	
	
	
	
	\begin{thebibliography}{9}
		\bibitem{LL} W. Troiani, \emph{Linear Logic: Multiplicatives}
		
		\bibitem{GMZ} D. Murfet, W. Troiani, \emph{Gentzen-Mints-Zucker Duality}
		
		
		
		\end{thebibliography}
	
	\end{document}

































