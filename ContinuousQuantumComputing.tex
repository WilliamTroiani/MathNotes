\documentclass[12pt]{article}

\usepackage{amsthm}
\usepackage{amsmath}
\usepackage{amsfonts}
\usepackage{mathrsfs}
\usepackage{amssymb}
\usepackage{units}
\usepackage{graphicx}
\usepackage{tikz-cd}
\usepackage{nicefrac}
\usepackage{hyperref}
\usepackage{bbm}
\usepackage{color}
\usepackage{tensor}
\usepackage{tipa}
\usepackage{bussproofs}
\usepackage{ stmaryrd }
\usepackage{ textcomp }
\usepackage{leftidx}
\usepackage{afterpage}
\usepackage{varwidth}
\usepackage{physics}

\newcommand\blankpage{
	\null
	\thispagestyle{empty}
	\addtocounter{page}{-1}
	\newpage
}

\graphicspath{ {images/} }

\theoremstyle{plain}
\newtheorem{thm}{Theorem}[subsection] % reset theorem numbering for each chapter
\newtheorem{proposition}[thm]{Proposition}
\newtheorem{lemma}[thm]{Lemma}
\newtheorem{fact}[thm]{Fact}
\newtheorem{cor}[thm]{Corollary}
\newtheorem{post}[thm]{Postulate}

\theoremstyle{definition}
\newtheorem{defn}[thm]{Definition} % definition numbers are dependent on theorem numbers
\newtheorem{exmp}[thm]{Example} % same for example numbers
\newtheorem{notation}[thm]{Notation}
\newtheorem{remark}[thm]{Remark}
\newtheorem{condition}[thm]{Condition}
\newtheorem{question}[thm]{Question}
\newtheorem{construction}[thm]{Construction}
\newtheorem{exercise}[thm]{Exercise}
\newtheorem{example}[thm]{Example}
\newtheorem{observation}[thm]{Observation}
\newtheorem{algorithm}[thm]{Algorithm}

\newcommand{\bb}[1]{\mathbb{#1}}
\newcommand{\scr}[1]{\mathscr{#1}}
\newcommand{\call}[1]{\mathcal{#1}}
\newcommand{\psheaf}{\text{\underline{Set}}^{\scr{C}^{\text{op}}}}
\newcommand{\und}[1]{\underline{\hspace{#1 cm}}}
\newcommand{\adj}[1]{\text{\textopencorner}{#1}\text{\textcorner}}
\newcommand{\comment}[1]{}
\newcommand{\lto}{\longrightarrow}

\title{Continuous quantum computing}
\author{Will Troiani}
\date{August 2020}

\begin{document}
	\maketitle

\begin{defn}\label{def:cont_time_evolution}
	A \textbf{continuous time evolution} of $\bb{H}$ is a Hermitian operator $H$ on $\bb{H}$, this is the \textbf{Hamiltonian}.
\end{defn}

\begin{remark}
	One may be tempted to psychologically project mathematical depth onto: ``as \emph{discrete} is to \emph{continuous}, unitary is to Hermitian". This would be pareidolia though. What is suppressed in these notes is that the \textbf{evolution} of a continuous time evolution is a vectorial differential equation (Schr\"{o}dinger's equation)
	\begin{equation}\label{eq:schrodinger}
		i\hbar \frac{d\ket{\psi}}{dt} = H\ket{\psi}
	\end{equation}
	Single step time evolution can be modelled via continuous time evolution, this involves solving the differnetial equation \eqref{eq:schrodinger}, which is too far abroad from the targetted focus of these notes.
	
	What is important, is the mathematical \emph{definition} \ref{def:cont_time_evolution}. We will not need any continuous analogue to the second half of Definition \ref{def:time_evolution}.
\end{remark}

\section{introduction}
\textcolor{red}{Insert explanation as to why $\ket{0}, \ket{1}$ are labelled the way they are (because they are eigenvalues of particular operators). Then generalise this to continuous quantum variables.}

Let $p \in \bb{R}$ and consider the function
\begin{align*}
	\bb{C} &\lto \bb{C}\\
	x &\longmapsto e^{-i x p}
\end{align*}
We notice that
\begin{equation}
	-i\frac{d}{dx}e^{-i x p} = p e^{-i x p}
\end{equation}
That is, $p$ is an eigenvalue of the \textcolor{blue}{linear function} $-i\frac{d}{dx}$. We therefore let $\ket{p}$ denote the function $e^{-i x p}$, and this is an example of a continuous quantum variable.

\section{Triple modula redundancy for continuous variables}
Triple modula redundancy is a simple binary error-correcting routine for classical discrete systems, where each bit $x \in \{ 0, 1 \}$ is sent three times $xxx$ so that if a single error occurs $xx\overline{x}$(where $\overline{0} = 1, \overline{1} = 0$) we can achieve correction by looking at the majority bit $xx\overline{x} \longmapsto xxx$. We now adapt this methodology to continuous classical variables.

Let $x_1, x_2, x_3$ be continuous variables, that is $x_1, x_2, x_3 \in \bb{R}$ and assume that they are all initially set to some value $x \in \bb{R}$. That is, let $x = x_1 = x_2 = x_3$. Then assume errors occured $x_1 \longmapsto x_1', x_2 \longmapsto x_2', x_3 \longmapsto x_3'$, that is, let $x_1', x_2', x_3' \in \bb{R}$ and assume that the classical information $x_1 x_2 x_3$ was sent through some channel of communication and the received information was $x_1'x_2'x_3'$. Assume without loss of generality that $|x_1'| < |x_2'| < |x_3'|$, that is, $x_1'$ experienced the least dramatic change in error, and $x_3'$ the most, with $x_2'$ in the middle. Then we let $x' = (x_1' + x_2')/2$ and replace $x_1'x_2'x_3'$ by $x'x'x'$. We remark this process does not achieve perfect error correction, nor is it reversible.

We now describe a similar routine which can be used to protect against some forms of quantum error. Consider three continuous quantum variables $\ket{x_1 x_2 x_3}$ and 





\end{document}