\documentclass[12pt]{article}

\usepackage{amsthm}
\usepackage{amsmath}
\usepackage{amsfonts}
\usepackage{mathrsfs}
\usepackage{array}
\usepackage{amssymb}
\usepackage{units}
\usepackage{graphicx}
\usepackage{tikz-cd}
\usepackage{nicefrac}
\usepackage{hyperref}
\usepackage{bbm}
\usepackage{color}
\usepackage{tensor}
\usepackage{tipa}
\usepackage{bussproofs}
\usepackage{ stmaryrd }
\usepackage{ textcomp }
\usepackage{leftidx}
\usepackage{afterpage}
\usepackage{varwidth}
\usepackage{tasks}
\usepackage{ cmll }
\usepackage{quiver}
\usepackage{kbordermatrix}

\newcommand\blankpage{
	\null
	\thispagestyle{empty}
	\addtocounter{page}{-1}
	\newpage
}

\graphicspath{ {images/} }

\theoremstyle{plain}
\newtheorem{thm}{Theorem}[subsection] % reset theorem numbering for each chapter
\newtheorem{proposition}[thm]{Proposition}
\newtheorem{lemma}[thm]{Lemma}
\newtheorem{fact}[thm]{Fact}
\newtheorem{cor}[thm]{Corollary}

\theoremstyle{definition}
\newtheorem{defn}[thm]{Definition} % definition numbers are dependent on theorem numbers
\newtheorem{exmp}[thm]{Example} % same for example numbers
\newtheorem{notation}[thm]{Notation}
\newtheorem{remark}[thm]{Remark}
\newtheorem{condition}[thm]{Condition}
\newtheorem{question}[thm]{Question}
\newtheorem{construction}[thm]{Construction}
\newtheorem{exercise}[thm]{Exercise}
\newtheorem{example}[thm]{Example}
\newtheorem{aside}[thm]{Aside}

\def\doubleunderline#1{\underline{\underline{#1}}}
\newcommand{\bb}[1]{\mathbb{#1}}
\newcommand{\scr}[1]{\mathscr{#1}}
\newcommand{\call}[1]{\mathcal{#1}}
\newcommand{\psheaf}{\text{\underline{Set}}^{\scr{C}^{\text{op}}}}
\newcommand{\und}[1]{\underline{\hspace{#1 cm}}}
\newcommand{\adj}[1]{\text{\textopencorner}{#1}\text{\textcorner}}
\newcommand{\comment}[1]{}
\newcommand{\lto}{\longrightarrow}
\newcommand{\rone}{(\operatorname{R}\bold{1})}
\newcommand{\lone}{(\operatorname{L}\bold{1})}
\newcommand{\rimp}{(\operatorname{R} \multimap)}
\newcommand{\limp}{(\operatorname{L} \multimap)}
\newcommand{\rtensor}{(\operatorname{R}\otimes)}
\newcommand{\ltensor}{(\operatorname{L}\otimes)}
\newcommand{\rtrue}{(\operatorname{R}\top)}
\newcommand{\rwith}{(\operatorname{R}\&)}
\newcommand{\lwithleft}{(\operatorname{L}\&)_{\operatorname{left}}}
\newcommand{\lwithright}{(\operatorname{L}\&)_{\operatorname{right}}}
\newcommand{\rplusleft}{(\operatorname{R}\oplus)_{\operatorname{left}}}
\newcommand{\rplusright}{(\operatorname{R}\oplus)_{\operatorname{right}}}
\newcommand{\lplus}{(\operatorname{L}\oplus)}
\newcommand{\prom}{(\operatorname{prom})}
\newcommand{\ctr}{(\operatorname{ctr})}
\newcommand{\der}{(\operatorname{der})}
\newcommand{\weak}{(\operatorname{weak})}
\newcommand{\exi}{(\operatorname{exists})}
\newcommand{\fa}{(\operatorname{for\text{ }all})}
\newcommand{\ex}{(\operatorname{ex})}
\newcommand{\cut}{(\operatorname{cut})}
\newcommand{\ax}{(\operatorname{ax})}
\newcommand{\negation}{\sim}
\newcommand{\true}{\top}
\newcommand{\false}{\bot}
\DeclareRobustCommand{\diamondtimes}{%
	\mathbin{\text{\rotatebox[origin=c]{45}{$\boxplus$}}}%
}
\newcommand{\tagarray}{\mbox{}\refstepcounter{equation}$(\theequation)$}
\newcommand{\startproof}[1]{
	\AxiomC{#1}
	\noLine
	\UnaryInfC{$\vdots$}
}
\newenvironment{scprooftree}[1]%
{\gdef\scalefactor{#1}\begin{center}\proofSkipAmount \leavevmode}%
	{\scalebox{\scalefactor}{\DisplayProof}\proofSkipAmount \end{center} }

\usepackage[margin=1.5cm]{geometry}


\title{$\lambda$-terms as polynomials}
\author{Will Troiani}
\date{\today}

\begin{document}
	\maketitle
	Let $t$ be a $\lambda$-term and let $\{ x_1, \ldots, x_n \}$ be a valid context for $t$, that is
	\begin{equation}
		\operatorname{FV}(t) \subseteq \{ x_1, \ldots, x_n \}
	\end{equation}
We define an integer $m$ and an interpretation for $t$ as a polynomial map:
	\begin{align*}
		\llbracket x_1, \ldots, x_n \mid t \rrbracket: \bb{N}[x_1, \ldots, x_n]^{m} &\lto \bb{N}[x_1,\ldots, x_n]\\
		(X_1, \ldots, X_{m}) &\longmapsto q(X_1, \ldots, X_{m})
	\end{align*}
	What we mean by $\llbracket x_1, \ldots, x_n \mid t \rrbracket$ being a polynomial, is that
	\begin{equation}
		q(X_1, \ldots, X_{m}) \in (\bb{N}[x_1, \ldots, x_n])[X_1, \ldots, X_{m}]
	\end{equation}
	
	\begin{defn}
		\textbf{Say $t = x_i$ is a variable}. Then $m = 1$ and:
		\begin{align*}
			\llbracket x_1, \ldots, x_n \mid x_i \rrbracket: \bb{N}[x_1, \ldots, x_n] = \bb{N}[x_1, \ldots, x_n] &\lto \bb{N}[x_1, \ldots, x_n]\\
			X &\longmapsto x_i
			\end{align*}
		\textbf{Say $t = \lambda x_{n+1}. t$ is an abstraction}: assume we have
		\begin{align*}
			\llbracket x_1, \ldots, x_n, x_{n+1} \mid t \rrbracket: \bb{N}[x_1, \ldots, x_n, x_{n+1}]^{m} &\longmapsto \bb{N}[x_1, \ldots, x_{n}, x_{n+1}]\\
			(X_1, \ldots, X_{m}) &\longmapsto q(X_1, \ldots, X_{m})
			\end{align*}
		We notice that
		\begin{equation}
			q(X_1, \ldots, X_{m}) \in (\bb{N}[x_1, \ldots, x_{n+1}])[X_1, \ldots, X_{m}]
			\end{equation}
		and so there exists a polynomial $q' \in \bb{N}[x_1, \ldots, x_{n+1}, X_1, \ldots, X_{m}]$ such that
		\begin{equation}
			q'(x_1, \ldots, x_{n+1}, X_1, \ldots, X_{m}) = q(X_1, \ldots, X_{m})
			\end{equation}
		We introduce a new variable $X_{m+1}$ and consider
		\begin{equation}
			q'(x_1, \ldots, x_{n}, X_{m+1}, X_1, \ldots, X_{m}) \in (\bb{N}[x_1, \ldots, x_n])[X_1, \ldots, X_{m+1}]
			\end{equation}
		There exists a polynomial $q'' \in (\bb{N}[x_1, \ldots, x_n])[X_1, \ldots, X_{m+1}]$ such that
		\begin{equation}
			q''(X_1, \ldots, X_{m+1}) = q'(x_1, \ldots, x_n, X_{m+1}, X_1, \ldots, X_{m})
			\end{equation}
		We define
		\begin{align*}
			\llbracket x_1, \ldots, x_n \mid \lambda x_{n+1}. t\rrbracket: \bb{N}[x_1, \ldots, x_{n}]^{m+1} &\lto \bb{N}[x_1, \ldots, x_n]\\
			(X_1, \ldots, X_{m+1}) &\longmapsto q''(X_1, \ldots, X_{m+1})
			\end{align*}
		\textbf{Say $t = uv$ is an application}: say we have
		\begin{align*}
			\llbracket x_1,\ldots, x_n \mid u \rrbracket: \bb{N}[x_1, \ldots, x_n]^{m_1} &\lto \bb{N}[x_1, \ldots, x_n]\\
			(X_1, \ldots, X_{m_1}) &\lto q_1(X_1, \ldots, X_{m_1})
			\end{align*}
		and
		\begin{align*}
			\llbracket x_1, \ldots, x_n \mid v \rrbracket: \bb{N}[x_1, \ldots, x_n]^{m_2} &\lto \bb{N}[x_1, \ldots, x_n]\\
			(X_1, \ldots, X_{m_2}) &\longmapsto q_2(X_1, \ldots, X_{m_2})
			\end{align*}
		We define
		\begin{align*}
			\llbracket x_1, \ldots, x_n \mid uv \rrbracket: \bb{N}[x_1, \ldots, x_n]^{m_1 + m_2 -1} &\lto \bb{N}[x_1, \ldots, x_n]\\
			(X_1, \ldots, X_{m_1 + m_2 -1}) &\longmapsto q_1(X_{m_2 + 1}, \ldots, X_{m_2 + m_1 - 1}, q_2(X_{1}, \ldots, X_{m_2}))
			\end{align*}
		\end{defn}
	
\begin{proposition}
	This is a model of the untyped $\lambda$-calculus.
	\end{proposition}
\begin{proof}
	We show that
	\begin{equation}
		\llbracket x_1, \ldots, x_n \mid (\lambda x_{n+1}.t)s\rrbracket = \llbracket x_1, \ldots, x_n \mid t[x_{n+1} := s] \rrbracket
		\end{equation}
	We prove this by induction on the structure of $t$.
	
	\textbf{Say $t = x_i$ is a variable}. If $i \neq n+1$ then $t[x_{n+1} := s] = x_i$ and
	\begin{align*}
		\llbracket x_1, \ldots, x_n \mid x_i\rrbracket: \bb{N}[x_1, \ldots, x_n] &\lto \bb{N}[x_1, \ldots, x_n]\\
		X &\longmapsto x_i
		\end{align*}
	On the other hand,
	\begin{align*}
		\llbracket x_1, \ldots, x_n \mid \lambda x_{n+1}. x_i\rrbracket: \bb{N}[x_1, \ldots, x_n]^2 &\lto \bb{N}[x_1, \ldots, x_n]\\
		(X_1, X_2) &\longmapsto x_i
		\end{align*}
	and so
	\begin{align*}
		\llbracket x_1, \ldots, x_n \mid (\lambda x_{n+1}. x_i)s \rrbracket: \bb{N}[x_1, \ldots, x_n]^2 &\lto \bb{N}[x_1, \ldots, x_n]\\
		(X_1, X_2d) &\longmapsto x_i
		\end{align*}
	If $i = n+1$ then $t[x_{n+1} := s] = s$ and
	\begin{align*}
		\llbracket x_1, \ldots, x_n \mid \lambda x_{n+1}. x_{n+1} \rrbracket: \bb{N}[x_1, \ldots, x_n]^m &\lto \bb{N}[x_1, \ldots, x_n]\\
		(X_1, \ldots, X_m) &\longmapsto X_m
		\end{align*}
	Thus
	\begin{align*}
		\llbracket x_1, \ldots, x_n \mid (\lambda x_{n+1}.x_{n+1})s \rrbracket: \bb{N}[x_1, \ldots, x_n]^m &\lto \bb{N}[x_1, \ldots, x_n]\\
		(X_1, \ldots, X_m) &\longmapsto \llbracket x_1, \ldots, x_n \mid s\rrbracket(X_1, \ldots, X_m)
		\end{align*}
	\textbf{Say $t = \lambda x_{n+2}.u$ is an abstraction}. Write
	\begin{equation}
		\llbracket x_1, \ldots, x_n \mid u \rrbracket = q(X_1, \ldots, X_m),\quad \llbracket x_1, \ldots, x_n \mid s \rrbracket = p(X_1, \ldots, X_{m'})
		\end{equation}
	Then
	\begin{align*}
		\llbracket x_1, \ldots, x_n &\mid \lambda x_{n+2}. (\lambda x_{n+1}.u)s \rrbracket\\
		&= q(x_1, \ldots, x_n, p(X_1, \ldots, X_{m'}), X_{m + m' + 1}, X_{m' + 1}, \ldots, X_{m' + m})
		\end{align*}
	Also,
	\begin{equation}
		\llbracket x_1, \ldots, x_n \mid \lambda x_{n+1}x_{n+2}. u \rrbracket = q(x_1, \ldots, x_n, X_{m+2}, X_{m+1}, X_1, \ldots, X_m)
		\end{equation}
	it follows that
	\begin{align*}
		\llbracket x_1, \ldots, x_n &\mid (\lambda x_{n+1}x_{n+2}.u)s \rrbracket\\
		&= q(x_1, \ldots, x_n, p(X_1, \ldots, X_{m'}), X_{m' + m + 1}, X_{m' + 1}, \ldots, X_{m' + m})
		\end{align*}
	By the inductive hypothesis, we have
	\begin{equation}
		\llbracket x_1, \ldots, x_n, x_{n+2} \mid u[x_{1} := s] \rrbracket = 
		\llbracket x_1, \ldots, x_n, x_{n+2} \mid (\lambda x_{n+1}.u)s \rrbracket
		\end{equation}
	It follows that
	\begin{equation}
		\llbracket x_1, \ldots, x_n \mid \lambda x_{n+2}(u[x_1 := s])\rrbracket = \llbracket x_1, \ldots, x_n \mid \lambda x_{n+2}.(\lambda x_{n+1}.u)s\rrbracket
		\end{equation}
	Combining this with the above, we have
	\begin{align*}
		\llbracket x_1, \ldots, x_n \mid (\lambda x_{n+1}x_{n+2}.u)s \rrbracket &= \llbracket x_1, \ldots, x_n \mid \lambda x_{n+2}.(u[x_1 := s])\rrbracket\\
		&= \llbracket x_1, \ldots, x_n \mid (\lambda x_{n+2}. u)[x_1 := s]\rrbracket
		\end{align*}
	as required.
	\textbf{Say $t = t_1t_2$ is an application}. Write
	\begin{equation}
		\llbracket x_1, \ldots, x_n\mid t_i\rrbracket = q_1(X_1, \ldots, X_{m_i})\text{, for }i = 1,2
		\end{equation}
	and again we write
	\begin{equation}
		\llbracket x_1, \ldots, x_n \mid s \rrbracket = p(x_1, \ldots, x_n, X_1, \ldots, X_{m'})
		\end{equation}
	For $i = 1, 2 $ we have
	\begin{align*}
		&\llbracket x_1, \ldots, x_n \mid (\lambda x_{n+1}. t_i)s\rrbracket\\
		&= q_i(x_1, \ldots, x_n, p(x_1, \ldots, x_n, X_{1}, \ldots, X_{m'}), X_{m' + 1}, \ldots, X_{m' + m_i})
		\end{align*}
	Thus,
	\begin{align*}
		&\llbracket x_1, \ldots, x_n \mid \big[(\lambda x_{n+1}. t_1)s\big]\big[ (\lambda x_{n+1}.t_2)s\big]\rrbracket\\
		&= q_1(x_1, \ldots, x_n, p(x_1, \ldots, x_n, X_1, \ldots, X_{m'}),\\
		&\qquad X_{m' + m_2 + 1}, \ldots, X_{m' + m_2 + m_1 - 1},  q_2(x_1, \ldots, x_n, p(x_1, \ldots, x_n, X_1, \ldots, X_{m'})\\
		&\qquad X_{m' + 1}, \ldots, X_{m' + m_2}))
		\end{align*}
	On the other hand, we have
	\begin{align*}
		&\llbracket x_1, \ldots, x_n, x_{n+1} \mid t_1 t_2 \rrbracket\\
		&= q_1(x_1, \ldots, x_n, x_{n+1}, X_{m_2 + 1}, \ldots, X_{m_2 + m_1 - 1}, q_2(x_1, \ldots, x_n, x_{n+1}, X_{1}, \ldots, X_{m_2}))
		\end{align*}
	Thus,
	\begin{align*}
		&\llbracket x_1, \ldots, x_n \mid (\lambda x_{n+1}. t_1 t_2)s\rrbracket\\
		&= q_1(x_1, \ldots, x_n, p(x_1, \ldots, x_n, X_{1}, \ldots, X_{m'}), \\
		&\qquad X_{m' + m_2 + 1}, \ldots, X_{m' + m_2 + m_1 - 1}, q_2(x_1, \ldots, x_n, p(x_1, \ldots, x_n, X_{1}, \ldots, X_{ m'}),\\
		&\qquad X_{m' + 1}, \ldots, X_{m' + m_2}))
		\end{align*}
	By the inductive hypothesis, we have for $i = 1,2$:
	\begin{equation}
		\llbracket x_1, \ldots, x_n \mid (\lambda x_{n+1}. t_i)s\rrbracket = \llbracket x_1, \ldots, x_n \mid t_i[x_{n+1} := s]\rrbracket
		\end{equation}
	It follows that
	\begin{equation}
		\llbracket x_1, \ldots, x_n \mid \big[(\lambda x_{n+1}. t_1)s\big]\big[(\lambda x_{n+1}. t_2)s\big] = \llbracket x_1, \ldots, x_n \mid t_1[x_{n+1} := s]t_2[x_{n+1} := s]\rrbracket\\
		\end{equation}
	Combining this with above we have
	\begin{align*}
		\llbracket x_1, \ldots, x_n \mid (\lambda x_{n+1}.t_1t_2)s\rrbracket &= \llbracket x_1, \ldots, x_n \mid t_1[x_{n+1} := s]t_2[x_{n+1} := s]\rrbracket\\
		&= \llbracket x_1, \ldots, x_n \mid (t_1 t_2)[x_{n+1} := s] \rrbracket
		\end{align*}
	as required.
	\end{proof}
	
	
	
	
	
	
	
	
	
	
	
	
	
	
	
	
	
	
	
	
	
	
	
	\end{document}