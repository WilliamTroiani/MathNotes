\documentclass[12pt]{article}

\usepackage{amsthm}
\usepackage{amsmath}
\usepackage{amsfonts}
\usepackage{mathrsfs}
\usepackage{amssymb}
\usepackage{units}
\usepackage{graphicx}
\usepackage{tikz-cd}
\usepackage{nicefrac}
\usepackage{hyperref}
\usepackage{bbm}
\usepackage{color}
\usepackage{tensor}
\usepackage{tipa}
\usepackage{bussproofs}
\usepackage{ stmaryrd }
\usepackage{ textcomp }
\usepackage{leftidx}
\usepackage{afterpage}
\usepackage{varwidth}

\newcommand\blankpage{
    \null
    \thispagestyle{empty}
    \addtocounter{page}{-1}
    \newpage
    }

\graphicspath{ {images/} }

\theoremstyle{plain}
\newtheorem{thm}{Theorem}[subsection] % reset theorem numbering for each chapter
\newtheorem{proposition}[thm]{Proposition}
\newtheorem{lemma}[thm]{Lemma}
\newtheorem{fact}[thm]{Fact}
\newtheorem{cor}[thm]{Corollary}

\theoremstyle{definition}
\newtheorem{defn}[thm]{Definition} % definition numbers are dependent on theorem numbers
\newtheorem{exmp}[thm]{Example} % same for example numbers
\newtheorem{notation}[thm]{Notation}
\newtheorem{remark}[thm]{Remark}
\newtheorem{condition}[thm]{Condition}
\newtheorem{question}[thm]{Question}
\newtheorem{construction}[thm]{Construction}
\newtheorem{exercise}[thm]{Exercise}
\newtheorem{example}[thm]{Example}

\newcommand{\bb}[1]{\mathbb{#1}}
\newcommand{\scr}[1]{\mathscr{#1}}
\newcommand{\call}[1]{\mathcal{#1}}
\newcommand{\psheaf}{\text{\underline{Set}}^{\scr{C}^{\text{op}}}}
\newcommand{\und}[1]{\underline{\hspace{#1 cm}}}
\newcommand{\adj}[1]{\text{\textopencorner}{#1}\text{\textcorner}}
\newcommand{\comment}[1]{}
\newcommand{\lto}{\longrightarrow}

\title{Thesis}
\author{William Troiani}
\date{August 2020}

\begin{document}

\maketitle
\tableofcontents

\section{Equality of variables in Geometry and logic}


\section{Clifford algebras}

We work over the complex numbers $k = \bb{C}$.

\begin{defn}
	
	\end{defn}

\begin{defn}\label{def:the_idempotent}
	For $n \geq 0$ let $e_n \in C_n$ denote the element
	\begin{equation}
		e_n = \gamma_1 \ldots \gamma_n \gamma_n^\dagger \ldots \gamma_1^\dagger
		\end{equation}
\end{defn}

\section{Koszul Complex}

Recall that a $\bb{Z}_2$-graded ring $R$ comes equipt with a choice of isomorphism $R \cong R_0 \oplus R_1$ where $R_0, R_1$ are subgroups of $R$. In such a setting we call elements of $R_1$ \textbf{odd}.

\begin{defn}
	Let $E$ be a $\bb{Z}_2$-graded ring and consider a set of odd elements $\theta_1, \ldots, \theta_n, \theta^\ast_1, \ldots, \theta^\ast_n \in E$. These \textbf{satisfy the canonical anticommutation relations} if the following hold for all $i,j = 1, \ldots, n$.
	\begin{itemize}
		\item $\theta_i \theta_j + \theta_j \theta_i = 0$
		\item $\theta_i^\ast \theta_j^\ast + \theta_j^\ast \theta_i^\ast = 0$
		\item $\theta_i \theta_j^\ast + \theta_j \theta_i^\ast = \delta_{ij}$
	\end{itemize}
\end{defn}

Recall that when $A,B$ are $\bb{Z}_2$-graded modules (over a graded ring $R$, say) then $\operatorname{Hom}(A, B)$ is also a $\bb{Z}_2$-graded module over $R$.

\begin{defn}
	Let $R$ be a commutative ring and $E$ a finitely generated free $R$-module (of rank $r$ say). Assume we are given an $R$-linear map $s: E \lto R$. Then we define the \textbf{Koszul Complex} to be the following chain complex of $R$-modules.
	\begin{equation}
		0 \lto \bigwedge^rE \stackrel{d_r}{\lto} \bigwedge^{r-1}E \stackrel{d_{r-1}}{\lto} \ldots\lto \bigwedge^1E \stackrel{d_1}{\lto} R \lto 0
	\end{equation}
	where for $i = 1, \ldots, r$:
	\begin{align*}
		d_i : \bigwedge^i E &\lto \bigwedge^{i-1} E\\
		e_{j_1} \wedge \ldots \wedge e_{j_i} &\longmapsto \sum_{k = 1}^i s(e_{j_k})(-1)^{k-1}e_{j_1} \wedge \ldots \wedge \hat{e}_{j_k} \wedge \ldots \wedge e_{j_i}
	\end{align*}
	where $\hat{e_{j_k}}$ means to omit $e_{j_k}$.
\end{defn}

When $E$ is a $\bb{Z}_2$-graded ring of the form $E = \operatorname{End}(A)$ for some $\bb{Z}_2$-graded ring $A$, then $E$ admitting a set of odd elements satisfying the anticommutation relations is sufficient for $A$ to admit a Clifford algebra represenation. See Lemma \cite[Lemma 5.6.2]{Hitchcock}.

Given a free $R$-module $A = R\theta_1 \oplus \ldots \oplus R\theta_r$ of rank $r$, the standard multiplication and contraction operators
\begin{equation}
	\theta_i \wedge \und{0.2}\qquad \theta_i^\ast \lrcorner \und{0.2}
\end{equation}
satisfy the canonical anticommutation relations. Thus, $A$ admits a Clifford algebra representaiton.

\section{Matrix factorisations}\label{sec:MatrixFactorisations}
Let $k$ denote a ring.

Recall that if $A,B$ are $\bb{Z}$-graded $k$-modules then $\operatorname{Hom}(A,B)$ is also $\bb{Z}$-graded \cite{TroianiSCA}. We have a similar definition for $\bb{Z}_2$-graded modules.
\begin{defn}
	Let $A,B$ be $\bb{Z}_2$-graded $k$-modules. A homomorphism $f: A \lto B$ is \textbf{even} if $f(A_0) \subseteq B_0$ and $f(A_1) \subseteq A_1$. The homomorphism $f$ is \textbf{odd} if $f(A_0) \subseteq B_1$ and $f(A_1) \subseteq B_0$.
\end{defn}
\begin{remark}
	If $f: A \lto B$ is any module homomorphism then $f$ can be written as a matrix
	\begin{equation}
		\begin{pmatrix}
			f_{00} & f_{01}\\
			f_{10} & f_{11}
		\end{pmatrix}
	\end{equation}
	The morphism $f$ is thus even if $f_{01} = f_{10} = 0$ and is odd if $f_{00} = f_{11} = 0$. This also shows that the morphism $f$ can be written as the sum of an even and an odd component.
\end{remark}
\begin{defn}
	Let $f \in k$ be a non-zero divisor. A \textbf{linear factorisation of $f \in k$ over $k$} is a pair $(X,\partial_X)$ consisting of a $\bb{Z}_2$-graded $k$-module $X = X_0 \oplus X_1$ and an odd homomorphism $\partial_X: X \lto X$ satisfying
	\begin{equation}
		\partial_X^2 = f \cdot \operatorname{id}_{X}
	\end{equation}
	If $X$ is free then $(X, \partial_X)$ is a \textbf{matrix factorisation}.
\end{defn}
The theory of Matrix factorisations is motivated by the search for square roots to operators. As a toy example, multiplication by $x^2 - y^2$ in $\bb{C}[x,y]$ does not admit a square root, but it does if we allow matrix solutions.
\begin{equation}
	\begin{pmatrix}
		0 & x - y\\
		x+ y & 0
	\end{pmatrix}^2 = (x^2 - y^2)
	\begin{pmatrix}
		1 & 0\\
		0 & 1
	\end{pmatrix}
\end{equation}

A more serious example is given by the square root of the Laplacian operator, see the Introduction of \cite{Friedrich}.

Our interest in matrix factorsations comes from the fact that appropriate homotopy categories of matrix factorsations form the homcategories of the bicategory of Landau-Ginzburg models, which we anticipate to find within a model of multiplicative linear logic (proofs as hypersurface singularities).

\begin{defn}\label{def:morphism_mf}
	A \textbf{morphism of linear factorisations} $$\alpha: \Big(X = X_0 \oplus X_1, d_X = \begin{pmatrix}0 & p_X\\
		q_X & 0\end{pmatrix}\Big) \lto \Big(Y = Y_0 \oplus Y_1, d_Y = \begin{pmatrix}0 & p_Y\\
		q_Y & 0\end{pmatrix}\Big)$$
	of $f \in R$ is a pair of morphisms $\alpha_0: X_0 \lto Y_0, \alpha_1:X_1 \lto Y_1$ rendering the following diagram commutative.
	\begin{equation}
		\begin{tikzcd}
			X_0\arrow[r,"{p_X}"]\arrow[d,"{\alpha_0}"] & X_1\arrow[r, "{q_X}"]\arrow[d,"{\alpha_1}"] & X_0\arrow[d,"{\alpha_0}"]\\
			Y_0\arrow[r,"{p_Y}"] & Y_1\arrow[r, "{q_Y}"] & Y_0
		\end{tikzcd}
	\end{equation}
\end{defn}

Given a matrix factorisation $(X = X_0 \oplus X_1, d_X) = \begin{pmatrix} 0 & p_X\\ q_X & 0\end{pmatrix}$ there is a sequence
\begin{equation}
	\ldots \stackrel{p_X}{\lto} X_1 \stackrel{q_X}{\lto} X_0 \stackrel{p_X}{\lto} X_1 \stackrel{q_X}{\lto} \ldots
	\end{equation}
however, we note that in general $d_X^2 = f\cdot I \neq 0$ and so strictly speaking this is \emph{not} a chain complex.
\begin{defn}
	We use the notation of Definition \ref{def:morphism_mf}. Let $\beta = (\beta_0, \beta_1)$ be another morphism of linear factorisations $(X, d_X) \lto (Y, d_Y)$. The morphisms $\alpha, \beta$ are \textbf{homotopic} if there exists a pair of morphisms $h_0: X_0 \lto Y_1, h_1: X_1 \lto Y_0$ such that the following holds
	\begin{equation}
		\alpha_0 - \beta_0 = q_Yh_0 + h_1 p_X,\qquad  \alpha_1 - \beta_1 = h_0 q_X + p_Y h_1
		\end{equation}
	\end{defn}
% https://q.uiver.app/?q=WzAsMTIsWzAsMCwiXFxsZG90cyJdLFsxLDAsIlhfMCJdLFsyLDAsIlhfMSJdLFszLDAsIlhfMCJdLFs0LDAsIlhfMSJdLFs1LDAsIlxcbGRvdHMiXSxbMCwxLCJcXGxkb3RzIl0sWzEsMSwiWV8wIl0sWzIsMSwiWV8xIl0sWzMsMSwiWV8wIl0sWzQsMSwiWV8xIl0sWzUsMSwiXFxsZG90cyJdLFsxLDcsIlxcYWxwaGFfMCIsMix7Im9mZnNldCI6MX1dLFsxLDcsIlxcYmV0YV8wIiwwLHsib2Zmc2V0IjotMX1dLFsyLDgsIlxcYmV0YV8xIiwwLHsib2Zmc2V0IjotMX1dLFsyLDgsIlxcYWxwaGFfMSIsMix7Im9mZnNldCI6MX1dLFszLDksIlxcYmV0YV8wIiwwLHsib2Zmc2V0IjotMX1dLFszLDksIlxcYWxwaGFfMCIsMix7Im9mZnNldCI6MX1dLFs0LDEwLCJcXGJldGFfMSIsMCx7Im9mZnNldCI6LTF9XSxbNCwxMCwiXFxhbHBoYV8xIiwyLHsib2Zmc2V0IjoxfV0sWzEsMiwicF9YIl0sWzAsMSwicV9YIl0sWzIsMywicV9YIl0sWzMsNCwicF9YIl0sWzQsNSwicV9YIl0sWzYsNywicV9ZIiwyXSxbNyw4LCJwX1kiLDJdLFs4LDksInFfWSIsMl0sWzksMTAsInBfWSIsMl0sWzEwLDExLCJxX1kiLDJdLFsxLDYsImhfMCIsMl0sWzIsNywiaF8xIiwyXSxbMyw4LCJoXzAiLDJdLFs0LDksImhfMSIsMl0sWzUsMTAsImhfMCIsMl1d
\[\begin{tikzcd}[column sep = large, row sep = large]
	\ldots & {X_0} & {X_1} & {X_0} & {X_1} & \ldots \\
	\ldots & {Y_0} & {Y_1} & {Y_0} & {Y_1} & \ldots
	\arrow["{\alpha_0}"', shift right=1, from=1-2, to=2-2]
	\arrow["{\beta_0}", shift left=1, from=1-2, to=2-2]
	\arrow["{\beta_1}", shift left=1, from=1-3, to=2-3]
	\arrow["{\alpha_1}"', shift right=1, from=1-3, to=2-3]
	\arrow["{\beta_0}", shift left=1, from=1-4, to=2-4]
	\arrow["{\alpha_0}"', shift right=1, from=1-4, to=2-4]
	\arrow["{\beta_1}", shift left=1, from=1-5, to=2-5]
	\arrow["{\alpha_1}"', shift right=1, from=1-5, to=2-5]
	\arrow["{p_X}", from=1-2, to=1-3]
	\arrow["{q_X}", from=1-1, to=1-2]
	\arrow["{q_X}", from=1-3, to=1-4]
	\arrow["{p_X}", from=1-4, to=1-5]
	\arrow["{q_X}", from=1-5, to=1-6]
	\arrow["{q_Y}"', from=2-1, to=2-2]
	\arrow["{p_Y}"', from=2-2, to=2-3]
	\arrow["{q_Y}"', from=2-3, to=2-4]
	\arrow["{p_Y}"', from=2-4, to=2-5]
	\arrow["{q_Y}"', from=2-5, to=2-6]
	\arrow["{h_0}"', from=1-2, to=2-1]
	\arrow["{h_1}"', from=1-3, to=2-2]
	\arrow["{h_0}"', from=1-4, to=2-3]
	\arrow["{h_1}"', from=1-5, to=2-4]
	\arrow["{h_0}"', from=1-6, to=2-5]
\end{tikzcd}\]

The relation of homotopy defines an equivalence relation on the set of morphisms of linear factorisations.

\begin{defn}
	A linear transformation whose underlying $\bb{Z}_2$-graded $k$-module is free and of finite rank is a \textbf{matrix factorisation}. There is a category $\operatorname{hmf}(k[\underline{x}], f)$ where the objects are matrix factorisations of $f$ and the morphisms are homotopy equivalence classes of morphisms of matrix factorisations.
	\end{defn}

\begin{defn}
	If $(X,d_X)$ is a matrix factorisation then so is $(X[1], -d_X)$. If we denote this by $\Psi(X,d_X)$ then $\Psi: \operatorname{hmf}(k[\underline{x}], f) \lto \operatorname{hmf}(k[\underline{x}], f)$ is extends to an endofunctor which induces a supercategorical structure on $\operatorname{hmf}(k[\underline{x}], f)$ if we take $\xi: \Psi^2 \lto 1_{\operatorname{hmf}(k[\underline{x}], f)}$ to be the identity.
	\end{defn}

\begin{defn}
	Let $(X, \partial_X)$ be a linear factorisation of $f \in k$ over $k$ and $(Y, \partial_Y)$ a linear factorisation of $g \in k$ also over $k$. Then the \textbf{tensor product} of $(X, \partial_X)$ and $(Y, \partial_Y)$ consists of the following data:
	\begin{equation}
		X \otimes_k Y, \qquad \partial_{X \otimes_k Y} = d_X \otimes 1 + 1 \otimes d_Y
	\end{equation}
	where $X \otimes_k Y$ is the \emph{graded} tensor product, which satisfies the following for all $x_1, x_2 \in X, y_1, y_2 \in Y$.
	\begin{equation}
		(x_1 \otimes y_1)(x_2 \otimes y_2) = (-1)^{\operatorname{deg}(x_2)\operatorname{deg}(y_1)}(x_1 x_2 \otimes y_1 y_2)
	\end{equation}
\end{defn}

\begin{lemma}
	The tensor product $(X \otimes_k Y, \partial_{X \otimes_k Y})$ is a linear factorisation of $f + g$.
\end{lemma}
\begin{proof}
	See \cite{Hitchcock}[Page 35].
\end{proof}

In the special case where there exists $f \in k[\underline{x}], g \in k[\underline{y}], h \in k[\underline{z}]$ and $(X, \partial_X)$ is a linear factorisation of $f - g \in k[\underline{x}, \underline{y}]$ and $(Y, \partial_Y)$ is a linear factorisation of $g - h \in k[\underline{y}, \underline{z}]$ then we also have the \emph{cut} of $(X, \partial_X)$ and $(Y, \partial_Y)$.

\begin{defn}
	For each $y_1, \ldots, y_n \in \underline{y}$ let $\partial_{y_i}g$ denote the formal partial derivative of $g$ with respect to $y_i$. Denote by $J_g$ the following $k[\underline{y}]$-module.
	\begin{equation}
		J_g := k[\underline{y}]/(\partial_{y_1}g, \ldots, \partial_{y_n}g)
	\end{equation}
	
	The \textbf{cut} of $(X, \partial_X), (Y, \partial_Y)$ is the data of
	\begin{equation}
		X | Y := (X \otimes_{k[\underline{y}]} J_g \otimes_{k[\underline{y}]} Y), \qquad \partial_{X|Y} = d_X \otimes 1 \otimes 1 + 1 \otimes 1 \otimes d_Y
	\end{equation}
\end{defn}
\begin{lemma}
	The cut $X | Y$ is a matrix factorisation of $f - h$.
\end{lemma}
\begin{proof}
	\textcolor{red}{Check this.}
\end{proof}

We will use the following Lemma to indirectly talk about matrix factorisations using $\bb{Z}_2$-graded modules over Clifford algebras.

\begin{defn}\label{def:clifford_modules}
	Let $k[\underline{x}], k[\underline{y}]$ denote polynomial rings over variables $x_1, \ldots, x_n$ and $y_1, \ldots, y_n$ respectively. Let $U(\underline{x}) = \sum_{i = 1}^n x_i^2$.
	
	We let $C_{U}$ denote the $\bb{Z}_2$-graded $k$-algebra with multiplicative generators $\mu_1, \ldots, \mu_n, \nu_1, \ldots, \mu_n$ satisfying the relations
	\begin{equation}
		[\mu_i, \mu_j] = -2\delta_{ij}\qquad [\mu_i, \nu_j] = 0\qquad [\nu_i, \nu_j] = 2\delta_{ij}
	\end{equation}
	where $\delta_{ij} = 1\text{ if and only if }i = j\text{ and }\delta_{ij} = 0\text{ otherwise}$ is the Kronecker delta.
\end{defn}

\begin{remark}
	The algebra $C_{U}$ described in Definition \ref{def:clifford_modules} is the Clifford algebra corresponding to the quadratic form
	\begin{equation}
		\begin{pmatrix}
			I_n & 0\\
			0 & -I_n
		\end{pmatrix}
	\end{equation}
	where $I_n$ is the $n \times n$ identity matrix, on the space $k^{2n}$.
\end{remark}

\begin{lemma}\label{lem:Matrix_fact_from_Cliff}
	Let $\tilde{X}$ be a $\bb{Z}_2$-graded $C_{U}$-module which is free and finitely generated over $k$. Then $X := \tilde{X} \otimes_k k[\underline{x}, \underline{y}]$ coupled with the map
	\begin{equation}\label{eq:differential_Cliff_mod}
		\partial_X = \sum_{i = 1}^n \mu_i x_i + \sum_{j = 1}^n \nu_j y_j
	\end{equation}
	is a matrix factorisation of $U(\underline{y}) - U(\underline{x}) \in k[\underline{x}, \underline{y}]$.
\end{lemma}
\begin{proof}
	See \cite{Hitchcock}[Lemma 5.6.1].
\end{proof}
\begin{remark}
	The map \eqref{eq:differential_Cliff_mod} is odd because we consider $k[\underline{x}, \underline{y}]$ to admit the $\bb{Z}_2$-grading
	\begin{equation}
		k[\underline{x}, \underline{y}] \oplus 0
	\end{equation}
	That is, $k[\underline{x},\underline{y}]$ has $k[\underline{x},\underline{y}]$ entirely in degree $0$, and the zero module $0$ in degree $1$. For example, if $\underline{x},\underline{y}$ are both singleton sets $\underline{x} = \{ x \}, \underline{y} = \{ y \}$ then
	\begin{align*}
		\operatorname{deg}(\partial_X(x \otimes p)) &= \operatorname{deg}( \mu x \otimes x p+ \nu x \otimes y_j p )\\
		&= \operatorname{deg}(\mu x)\text{ }(= \operatorname{deg}(\nu x))\\
		&= \operatorname{deg}(x) + 1
	\end{align*}
\end{remark}

\begin{example}\label{ex:}
	Let $\underline{x}$ be a set of variables $\{ x_1, \ldots, x_n \}$ and $\sigma \in S_n$ a permutation on this set. Let $\tilde{X}$ denote the $\bb{Z}_2$-graded $k$-algebra
	\begin{equation}
		\tilde{X} := \bigwedge(k \theta_1 \oplus \ldots \oplus k \theta_n)
	\end{equation}
	which is a $C_{U}$-algebra (Definition \ref{def:clifford_modules}) with $C_{U}$-action induced by the following
	\begin{align*}
		\mu_i &= \theta_{\sigma^{-1}i} + \theta_{\sigma^{-1}i}^\ast\\
		\nu_i &= \theta_i - \theta_i^\ast
	\end{align*}
	Thus, by Lemma \ref{lem:Matrix_fact_from_Cliff} we obtain a matrix factorisation $(X = \tilde{X} \otimes k[\underline{x}, \underline{y}], \partial_{X})$.
	
	Now say we had another similar matrix factorisation; let $\underline{y} = \{ y_1, \ldots, y_m \}$ be another set of variables and let $\tau$ be a permutation on this set. Let $\tilde{Y}$ denote the $\bb{Z}_2$-graded $k$-algebra
	\begin{equation}
		\tilde{Y} := \bigwedge (k\psi_1 \oplus \ldots \oplus k\psi_m)
	\end{equation}
	This is a $C_{U}$-module with $C_{U}$-action induced by the following
	\begin{align*}
		\overline{\nu_i} &= \psi_{\tau^{-1}i} + \psi_{\tau^{-1}i}^\ast\\
		\omega_i &= \psi_i - \psi_i^\ast
	\end{align*}
	This induces a matrix factorisation $(Y = \tilde{Y} \otimes k[\underline{y}, \underline{z}], \partial_Y)$ of $U(\underline{y}) - U(\underline{z})$ where
	\begin{equation}
		\partial_Y := \sum_{i = 1}^n \overline{\nu}_i y_i + \sum_{i = 1}^n \omega_i z_i
	\end{equation}
	and $\underline{z} = \{ z_1, \ldots, z_n \}$ is another set of variables.
	
	We will first consider the cut $X | Y$. The sequence of partial derivatives $(\partial_{y_1}U(\underline{y}), \ldots, \partial_{y_n}U(\underline{y})) = (2y_1, \ldots, 2y_n)$ and so
	\begin{equation}
		J_{U(\underline{y})} = k[\underline{y}]/(y_1, \ldots, y_n) = k
	\end{equation}
	as a $k[\underline{y}]$-module with trivial $k[\underline{y}]$-action. We thus have
	\begin{equation}
		X | Y = X \otimes_{k[\underline{x}, \underline{y}]} k \otimes_{k[\underline{y}, \underline{z}]} Y,\qquad \partial_{X | Y} = d_X \otimes 1 \otimes 1 + 1 \otimes 1 \otimes d_Y
	\end{equation}
\end{example}

Say $R$ is a commutative ring. We will consider an element $f \in R$ of the following particular form: say we have $a_1, \ldots, a_n, b_1, \ldots, b_n \in R$ such that $f = \sum_{i = 1}^n a_i b_n$. The guiding example of such an element is when $R$ is a polynomial ring and $f$ is a quadratic form.

\begin{lemma}
	Suppose $M$ is a $\bb{Z}_2$-graded $R$-module with odd $R$-linear maps $\theta_i, \theta_i^\ast : M \lto M, i = 1, \ldots, n$ satisfying the canonical anticommutation relations. Then, setting $\delta_+ = \sum_{i = 1}^n a_i \theta_i$ and $\delta_- = \sum_{i = 1}^n b_i \theta_i^\ast$ we have that $(M, \delta_- + \delta_+)$ is a linear factorisation of $f$. 
\end{lemma}
\begin{proof}
	See \cite[Lemma 4.2.3]{Hitchcock}.
\end{proof}

Therefore, $(\bigwedge R^n, \delta_- + \delta_+)$ is a matrix factorisation of $f$. This is the \textbf{Koszul matrix factorisation of $f$}.

\section{The Bicategory of Landau-Ginzburg models (over $k$)}
Let $k$ be a commutative ring.
\begin{defn}
	A polynomial $U(\underline{x}) \in k[\underline{x}] = k[x_1, \ldots, x_n]$ is a \textbf{potential} if
	\begin{itemize}
		\item The sequence of partial derivatives $(\partial_{x_1}U(\underline{x}), \ldots, \partial_{x_{n}}U(\underline{x}))$ is Koszul-regular.
		\item The Jacobi ring $k[\underline{x}]/(\partial_{x_1}U(\underline{x}), \ldots, \partial_{x_n}U(\underline{x}))$ is a finitely generated free $k$-module.
		\end{itemize}
	\end{defn}

\begin{defn}
	The bicategory of Landau-Ginzburg models over $k$, denoted $\call{LG}_k$, consists of the following data:
	\begin{itemize}
		\item The objects of $\call{LG}_k$ are paris $(k[\underline{x}], U(\underline{x}))$ where $k[\underline{x}]$ is a polynomial ring and $U(\underline{x}) \in k[\underline{x}]$ is a potential.
		\item The category of $1$-morphisms $(k[\underline{x}], U(\underline{x})) \lto (k[\underline{y}], V(\underline{y}))$ is $\big(\operatorname{hmf}(k[\underline{x}, \underline{y}], V(\underline{y}) - U(\underline{x}))\big)^\omega$, the idempotent completion of the category of finite rank matrix factorisations (with morphisms homotopy equivalence classes of morphisms of linear factorisations).
		\item Composition of the 1-morphisms
		\begin{equation}
			\begin{tikzcd}
			(k[\underline{x}], U(\underline{x}))\arrow[r,"{(X, d_X)}"] & (k[\underline{y}], V(\underline{y}))\arrow[r,"{(Y, d_Y)}"] & (k[\underline{z}], W(\underline{z}))
			\end{tikzcd}
			\end{equation}
		is given by taking the tensor product of linear factorisations over $k[\underline{y}]$
		\begin{equation}
			(X, d_X) \otimes_{k[\underline{y}]} (Y, d_Y) = (X \otimes_{k[\underline{y}]} Y, d_X \otimes 1 + 1 \otimes d_Y)
			\end{equation}
		\item Consider an object $(k[\underline{x}], U(\underline{x}))$ where $k[\underline{x}] = k[x_1, \ldots, x_n]$ and the polynomial ring $k[\underline{x}, \underline{x}'] = k[x_1, \ldots, x_n, x_1', \ldots, x_n']$. The unit $1$-morphism $(I_{U(\underline{x})}, d_{I_{U(\underline{x})}})$ of $(k[\underline{x}], U(\underline{x}))$ is a Koszul matrix factorisation of $U(\underline{x}') - U(\underline{x}) \in k[\underline{x}, \underline{x}']$ arising from the Koszul complex of the sequence $(x_1 - x_1', \ldots, x_n - x_n')$ in $k[\underline{x}, \underline{x}']$.
		\end{itemize}
	\end{defn}

\section{Splitting idempotents}
In this thesis we defend the proposition that the splitting of idempotents has fundamental relevance to computation.

Throughout, we work with a $k$-linear category $\call{C}$, that is, the category $\call{C}$ has $k$-modules for homsets.

\begin{defn}
	Let $\call{C}$ be a category. An \textbf{idempotent} in $\call{C}$ is an endomorphism $e: C \lto C$ such that $e^2 = e$.
	
	An idempotent $e$ is \textbf{split} if there exists a pair of morphisms $s: R \lto C, r: C \lto R$ such that $sr = e, rs = \operatorname{id}_R$.
\end{defn}

\begin{lemma}\label{lem:split_idempotent_equiv}
	Let $e: C \lto C$ be an idempotent in $\call{C}$. Then the following are equivalent.
	\begin{itemize}
		\item $e = sr$ is split where $s: R \lto C, r: C \lto R$.
		\item The Equaliser $\operatorname{Eq}(e, \operatorname{id}_e)$ exists and is equal to $s: R \lto C$.
		\item The Coequaliser $\operatorname{Coeq}(e, \operatorname{id}_e)$ exists and is equal to $r: C \lto R$.
	\end{itemize}
\end{lemma}
\begin{proof}
	See \cite{Hitchcock}[Lemma B.1] or \cite{Borceux}[Proposition 6.5.4].
\end{proof}



\begin{lemma}
	Assume $\call{C}$ is the category of vector spaces over some field $k$. Let $e:C \lto C$ be an idempotent. Assume $e = sr$ is split with $s: R \lto C, r: C \lto R$ and $1 - e = s'r'$ is also split with $s': R' \lto C, r': C \lto R'$. Then there is a split short exact sequence
	\begin{equation}
		0\lto R \stackrel{s}{\lto} C \stackrel{r'}{\lto} R' \lto 0
	\end{equation}
\end{lemma}
\begin{proof}
	Consider the morphism $(r,r'): C \lto R \oplus R'$. Then $\forall x \in R$ we have
	\begin{align*}
		(r,r')s(x) &= (rs(x), r's(x))\\
		&= (x, r's(x))
	\end{align*}
	we claim $r's(x) = 0$. By Lemma \ref{lem:split_idempotent_equiv} we have that $r': C \lto R'$ is the coequaliser $\operatorname{Coeq}(1-e, \operatorname{id}_C)$. Thus $r's(x) = r'(1-e)s(x)) = r's(x) - r'es(x)$. On the other hand, $s: R \lto C$ is the equaliser $\operatorname{Eq}(e, \operatorname{id}_C)$ and so $es = s$.
	
	Thus we have a commuting diagram
	% https://q.uiver.app/?q=WzAsNixbMCwwLCIwIl0sWzEsMCwiUiJdLFsyLDAsIkMiXSxbMywwLCJSJyJdLFs0LDAsIjAiXSxbMiwxLCJSIFxcb3BsdXMgUiciXSxbMCwxXSxbMSw1XSxbNSwzXSxbMyw0XSxbMiw1LCIocixyJykiLDJdLFsxLDIsInIiXSxbMiwzLCJyJyJdXQ==
	\[\begin{tikzcd}
		0 & R & C & {R'} & 0 \\
		&& {R \oplus R'}
		\arrow[from=1-1, to=1-2]
		\arrow[from=1-2, to=2-3]
		\arrow[from=2-3, to=1-4]
		\arrow[from=1-4, to=1-5]
		\arrow["{(r,r')}"', from=1-3, to=2-3]
		\arrow["r", from=1-2, to=1-3]
		\arrow["{r'}", from=1-3, to=1-4]
	\end{tikzcd}\]
	Moreover, the homomorphism $(r,r')$ is an isomorphism. To see this, say $x,x' \in C$ are such that $(r,r')(x) = (r,r')(x')$. Then $r(x) = r(x')$ implies $s(r(x)) = s(r(x'))$ which implies $e(x) = e(x')$ and similarly $(1 - e)(x) = (1 - e)(x')$. Thus we have
	\begin{align*}
		x &= (1 - e)(x) + e(x)\\
		&= (1 - e)(x') + e(x')\\
		&= x'
	\end{align*}
	For surjectivity, notice if $(x,x') \in R \oplus R'$ are given, then $(r,r')(s,s')(x,x') = (x,x')$.
\end{proof}

The next Lemma states that splitting an idempotent is equivalent to finding its image.

\begin{lemma}\label{lem:split_idemp_and_im}
	Let $\call{C}$ be linear and admit kernels and cokernels. Then if $e:C \lto C$ is split we have
	\begin{equation}
		\operatorname{Eq}(\operatorname{id}, e) \cong \operatorname{im}(e) \cong \operatorname{Coeq}(\operatorname{id},e)
	\end{equation}
\end{lemma}
\begin{proof}[Sketch]
	We have
	\begin{equation}
		\operatorname{Eq}(\operatorname{id},e) \cong \operatorname{ker}(\operatorname{id}-e) \cong \operatorname{im}(e)
	\end{equation}
	and
	\begin{equation}
		\operatorname{Coeq}(\operatorname{id},e) \cong \operatorname{Coker}(\operatorname{id} - e) \cong \operatorname{im}(e)
	\end{equation}
\end{proof}
\begin{remark}
	Another way of understanding Lemma \ref{lem:split_idemp_and_im} is that given a vector space $C$ and $v \in C$ along with an idempotent $e: C \lto C$ we have $x = e(x) + (\operatorname{id} - e)x$. It follows that
	\begin{equation}
		C \cong \operatorname{im}e \oplus \operatorname{im}(\operatorname{id} - e)
	\end{equation}
	From this it is clear that $\operatorname{Coker}(\operatorname{id} - e) \cong \operatorname{im}(e)$.
\end{remark}

Thus, to split an idempotent is to calculate the image of the idempotent. This is where the intuition that the splitting of idempotents is a fundamental component of the abstract study of computation; idempotents dictate the projection onto states of knowledge, which reduces entropy, and the calculation of the image of these spaces is the arrival at such a state of knowledge.



\subsection{The superbicategorical structure on the homotopy category of matrix factorisations}


\section{The superbicategory $\call{LG}$}
\textcolor{red}{Note: so far we have only defined the bicategory (and we haven't even done that), do we even need the ``super"-structure?}

Throughout, $k$ is a Notherian, commutative ring (but note that has already been understood how the following theory works in a more general context, see \cite{Cut_Operation}).

\begin{defn}
	A polynomial $U \in k[x] = k[x_1, \ldots, x_n]$ is a \textbf{potential} if:
	\begin{itemize}
		\item The sequence of partial derivatives $(\partial_{x_1}U, \ldots, \partial_{x_n}U)$ is quasi-regular.
		\item The Jacobi ring $k[x]/(\partial_{x_1}U, \ldots, \partial_{x_n}U)$ is a finitely generated free $k$-module.
		\end{itemize}
	\end{defn}
\textcolor{red}{Check with Dan if this is correct.}

\begin{defn}\label{def:LG_partial_data}
	The \textbf{bicategory of Landau-Ginzburg models over $k$}, denoted $\call{LG}_k$, consists of the following data:
	\begin{itemize}
		\item The objects of $\call{LG}_k$ are pairs $(k[x], U)$ where $k[x] = k[x_1, \ldots, x_n]$ is a polynomial ring and $U \in k[x]$ is a potential.
		\item The category of $1$-morphisms $(k[x], U) \lto (k[y], V)$ is $h(U,V)^\omega$. \textcolor{red}{What is this?}
		\item Composition of $1$-morphisms
		\begin{equation}
			\begin{tikzcd}
				(k[x], U)\arrow[r,"{(X, d_X)}"] & (k[y], V)\arrow[r,"{(Y,d_Y)}"] & (k[z], W)
				\end{tikzcd}
			\end{equation}
		is given by taking the tensor product of linear factorisations over $k[y]$:
		\begin{equation}
			(X, d_X) \otimes_{k[y]} (Y, d_Y) = (X \otimes_{k[y]} Y, d_X \otimes 1 + 1 \otimes d_Y)
			\end{equation}
		\item Consider an object $(k[x], U)$ where $k[x] = k[x_1, \ldots, x_n]$ and the polynomial ring $k[x,x'] = k[x_1, \ldots, x_n, x_1', \ldots, x_n']$. The unit 1-morphism $(I_U, d_{I_U})$ of $(k[x], U)$ is a Koszul matrix factorisation of $U(x') - U(x) \in k[x,x']$ arising from the Koszul complex of the sequence $(x_1 - x_1', \ldots, x_n - x_n')$ in $k[x,x']$.
		\end{itemize}
	\end{defn}
We remark that Definition \ref{def:LG_partial_data} is incomplete.

Note that it is not clear that the tensor product is a well-defined composition functor. Consider the 1-morphisms
\begin{equation}
	\begin{tikzcd}
		(k[x], U)\arrow[r,"{(X, d_X)}"] & (k[y], V)\arrow[r,"{(Y,d_Y)}"] & (k[z], W)
	\end{tikzcd}
\end{equation}
in $\call{LG}_k$. Writing $X = k[x,y]^m$ and $Y = k[y,z]^{m'}$ for some $m,m' \in \bb{N}$ we have that
\begin{equation}
	X \otimes_{k[y]} Y = (k[x,y] \otimes_{k[y]} k[y,z])^{m m'} = k[x,y,z]^{mm'}
	\end{equation}
which is free, but not finitely generated over $k[x,z]$. Hence it is only clear that the composition of $(X,d_X)$ and $(Y,d_Y)$ belongs to $\operatorname{HMF}(k[x,z], W(z) - U(x))$, rathare than $h(U,W)^\omega$.

\section{Clifford thickening}
Let $\call{T}$ be a small idempotent complete supercategory. We construct a new supercategory $\call{T}^\bullet$ called the \textbf{Clifford thickening} in which the objects are pairs $(X,n)$ of an integer $n \geq 0$ and a left $C_n$-module $X$ in $\call{T}$. Recall that if $A$ is a $\bb{Z}_2$-graded $k$-algebra then $\call{T}_A$ denotes the supercategory of $A$-modules in $\call{T}$ with $A$-linear maps.

\begin{defn}
	Let $\call{M}$ denote the superbicategory of Morita trivial $\bb{Z}_2$-graded $k$-algebras. The 1-morphisms are $\bb{Z}_2$-graded bimodules which are finitely generated and projective over $k$ and the 2-morphisms are degree zero bimodule maps.
	\end{defn} 

\begin{proposition}
	The assignment of the supercategory $\call{T}_A$ to an algebra $A$ and of the superfunctor $\Phi_V = V \otimes_A (\und{0.2})$ to a $B-A-$bimodule $V$ determines a strong superfunctor
	\begin{equation}
		\Phi^\call{T}: \call{M} \lto \operatorname{Cat}_k^{\operatorname{sup}}
		\end{equation}
	to the superbicategory of small supercategories and superfunctors
	\end{proposition}
\begin{proof}
	See \cite[Proposition 2.22]{Cut_Operation}.
	\end{proof}
\begin{defn}
	Let $\bb{N}$ denote the category of integers $n \geq 0$ with a unique morphism $\phi_{m,n}L n \lto m$ for each pair $m, n$.
	\end{defn}
We view $\bb{N}$ as a bicategory with only identity 2-morphisms. \textcolor{red}{Rewrite this and check with Dan if this is the correct definition, do we really have morphsims for \emph{any} pair $m,n$?}

\begin{lemma}
	There is a strong functor $\bb{N} \lto \call{M}$ defined by
	\begin{align*}
		n &\longmapsto C_n = \operatorname{End}_k(S_n)\\
		\phi_{m,n} &\longmapsto S_{m,n} = S_m \otimes_k S_n^\ast
		\end{align*}
	\end{lemma}
The composite of these strong functors is a strong functor
\begin{equation}\label{eq:strong_functor_composite}
	\begin{tikzcd}
		\bb{N}\arrow[r] & \call{M}\arrow[r] & \operatorname{Cat}_k^{\operatorname{sup}}
		\end{tikzcd}
	\end{equation}
sending $n$ to the category of left $C_n$-modules in $\call{T}$ and $\phi_{m,n}$ to the functor $S_{m,n} \otimes_{C_n} \und{0.2}$.

\begin{defn}
	The \textbf{Clifford thickening} $\call{T}^\bullet$ of the supercategory $\call{T}$ is the category which results from the Grothendieck construction applied to the strong functor \eqref{eq:strong_functor_composite}.
	\end{defn}

\textcolor{red}{There is a more concrete definition circumnavigating the Grothendieck construction but I do not understand it yet.}



\begin{lemma}
	Let $X$ be a $C_n$-module in $\call{T}$. The idempotent $e_n: X \lto X$
	\end{lemma}

\section{Material I don't think I need}
\subsection{The superbicategory $\call{C}$}
In this section we define a superbicategory without units $\call{C}$. The objects of $\call{C}$ are the same as $\call{LG}$, namely potentials $(x,W)$.
\begin{defn}
	Given potentials $(x,W)$ and $(y,V)$ we define
	\begin{equation}
		\call{C}(W,V) := \Big(\operatorname{hmf}(k[x,y], V - W)^\omega\Big)
	\end{equation}
	where $(\und{0.2})^\omega$ denotes th eidempotent completion and $(\und{0.2})^\bullet$ is the Clifford thickening.
\end{defn}
Thus a $1$-morphism $W \lto V$ in $\call{C}$ is a finite rank matrix factorisation of $V(y) - W(x)$ together with an idempotent endomprhism $e$ and a family of odd operators $\gamma_i, \gamma_i^\dagger$ satisfying Clifford relations and the equations (all up to homotopy)
\begin{equation}
	\gamma_i e = \gamma_i = e\gamma_i,\qquad \gamma_i^\dagger e = \gamma_i^\dagger = e\gamma_i^\dagger
\end{equation}

For matrix factorisations $X, Y$ of $V(y) - W(x)$ with respective Clifford actions $\{ \gamma_i, \gamma_i^\dagger \}_{i = 1}^a$ and $\{ \rho_j, \rho_j^\dagger \}_{j = 1}^b$ the 2-morphisms $\phi: (X,a) \lto (Y,b)$ in $\call{C}$ are in bijection with morphisms of matrix factorisations $\phi: X \lto Y$ satisfying
\begin{equation}
	\phi \gamma_i^\dagger = 0, \qquad \rho_j \phi = 0,\qquad 1 \leq i \leq a, \quad 1 \leq j \leq b
\end{equation}
\textcolor{red}{Lemma missing here}
The first aim of this section is to define, for any object $(z,U)$ of $\call{LG}$, a functor
\begin{align*}
	\call{C}(V,U) \otimes_k \call{C}(W,V) &\lto \call{C}(W,U)\\
	(Y,X) &\longmapsto Y \mid X
\end{align*}
\textcolor{red}{What does $\otimes$ mean here?}

which we call the \textbf{cut functor}. The cut operation is defined on matrix factorisations $X$ of $V - W$ and $Y$ of $U - V$ as follows, assuming $y = (y_1, \ldots, y_m)$ and writing
\begin{equation}
	J_V = k[y]/(\partial_{y_1}V, \ldots, \partial_{y_m}V)
\end{equation}
\begin{lemma}
	The $\bb{Z}_2$-graded $k[x,z]$-module
	\begin{equation}
		Y \mid X = Y \otimes_{k[y]} J_V \otimes_{k[y]} X
	\end{equation}
	with the differential $d_Y \otimes 1 + 1 \otimes d_X$ is a finite rank matrix factorisation of $U - W$.
\end{lemma}
\begin{proof}
	Since $V$ is a potential $J_V$ is a finite rank free $k$-module, and it follows that $Y\mid X$ is a finite rank free $k[x,z]$-module.
\end{proof}

There exists a Clifford action on $Y \mid X$, however it is difficult to describe in general and so we now describe how multiplicative proof nets will be modeled and then consider this Clifford action in this particular case.

\subsection{Super(bi)categories}
Throughout $k$ is a notherian $\bb{Q}$-alebra. By default categories and functors are $k$-linear.

\subsubsection{Supercategories}
\begin{defn}
	A \textbf{supercategory} is a category $\call{C}$ together with a functor $\Psi: \call{C} \lto \call{C}$ and a natural isomoprhism $\xi: \Psi^2 \lto 1_\call{C}$ satisfying the condition
	\begin{equation}\label{eq:super_condition}
		\xi \ast 1_\Psi = 1_\Psi \ast \xi
	\end{equation}
	as natural transformations $\Psi^3 \lto \Psi$. A \textbf{superfunctor} $(F,\gamma)$ from a supercategory $(\call{C}, \Psi_\call{C})$ to a supercategory $(\call{D}, \Psi_{\call{D}})$ is a functor $F: \call{C} \lto \call{D}$ together with a natural isomoprhism $\gamma: F\Psi_\call{C} \lto \Psi_\call{D}F$ satisfying
	\begin{equation}\label{eq:super_functor}
		1_F \ast \xi = (\xi \ast 1_F)(1_\Psi \ast \gamma)(\gamma \ast 1_\Psi)
	\end{equation}
	A \textbf{supernatural transformation} $\varphi: F \lto G$ between superfunctors $(F, \gamma_F), (G, \gamma_G)$ is a natural transformation making the following diagram commute:
	\begin{equation}\label{eq:super_nat_trans}
		\begin{tikzcd}
			F\Psi\arrow[r,"{\varphi \Psi}"]\arrow[d,"{\gamma_F}"] & G\Psi\arrow[d,"{\gamma_G}"]\\
			\Psi F\arrow[r,"{\Psi \varphi}"] & \Psi G
		\end{tikzcd}
	\end{equation}
\end{defn}

\begin{remark}
	Condition \eqref{eq:super_condition} means that for all $C \in \call{C}$ we have the following equality.
	\begin{equation}
		\xi_{\Psi C} = \Psi \xi_C
	\end{equation}
	Condition \eqref{eq:super_functor} means that for all $C \in \call{C}$ we have the following equality.
	\begin{equation}
		F\xi_C = \xi_{FC}\Psi_{\call{D}}\gamma_C \gamma_{\Psi_{\call{C}C}}
	\end{equation}
	which can be written as requiring commutativity of the following diagram.
	\begin{equation}
		\begin{tikzcd}
			F\Psi_C C\arrow[r,"{\gamma_{\Psi_\call{C} C}}"]\arrow[d,swap,"{F\xi_C}"] & \Psi_\call{D} F \Psi_{\call{C}}C\arrow[d,"{\Psi_{\call{D}}\gamma_C}"]\\
			FC & \Psi_{\call{D}}^2 FC\arrow[l,"{\xi_{\call{D}FC}}"]
		\end{tikzcd}
	\end{equation}
\end{remark}

Lower case letters $a, b, \ldots$ denote objects of a bicategory, while upper case letters $X, Y, \ldots$ and greek letters $\alpha, \beta, \ldots$ respectively denote 1-morphisms and 2-morphisms. Units are denotes $\Delta$, associators are $\alpha$, and unitors are $\lambda, \rho$. The composition of $Y, X$ is denoted $YX$ or $Y \circ X$.

\begin{defn}
	A \textbf{superbicategory} is a bicategory $\call{B}$ together with the data:
	\begin{itemize}
		\item For each object $a$ a $1$-morphism $\Psi_a: a \lto a$ and a 2-isomorphism $\xi_a: \Psi_a^2 \lto \Delta_a$.
		\item For each 1-morphism $X: a \lto b$ a natural 2-isomoprhism $\gamma_X: X\Psi_a \lto \Psi_b X$.
	\end{itemize}
	This data is required to satsify the following axioms:
	\begin{itemize}
		\item For each composable pair $X, Y$ of 1-morphisms the diagram
		\begin{equation}
			\begin{tikzcd}
				(YX)\Psi\arrow[rrr,"{\gamma_{YX}}"]\arrow[d,"{\alpha}"] &&& \Psi(YX)\\
				Y(X\Psi)\arrow[r,"{1_Y \ast \gamma_X}"] & Y(\Psi X)\arrow[r,"{\alpha^{-1}}"] & (Y\Psi)X\arrow[r,"{\gamma_Y \ast 1_X}"] & (\Psi Y)X\arrow[u,"{\alpha}"]
			\end{tikzcd}
		\end{equation}
		commutes.
		\item For every object $a$, $\xi_a \ast 1_\Psi = 1_\Psi \ast \xi_a$.
		\item For every 1-morphism $X: a \lto b, 1_X \ast \xi_a = (\xi_b \ast 1)(1_\Psi \ast \gamma_X)(\gamma_X \ast 1_\Psi)$.
	\end{itemize}
\end{defn}

\begin{example}
	There is a bicategory of $\bb{Z}_2$-graded $k$-algebras where 1-morphisms are $\bb{Z}_2$-graded bimodules and 2-morphisms are degree zero bimodule maps. Given a $B-A-$module $M$ the shift $M[1]$ has the grading $M[1]_i = M_{i+1}$ and the left and right action given by 
	\begin{equation}
		b \cdot m = (-1)^{|b|}bm,\qquad m \cdot a = (-1)^{|a|} m a
	\end{equation}
	Thsi is a functor $\Psi = (\und{0.2})[1]$ on the category of $B-A-$bimodules and with $\xi = 1$ this defines the structure of a supercategory on this category of bimodules. The suual isomoprhisms of bimodules $\tau$
	\begin{align}
		N[1] \otimes M &\lto (N \otimes M)[1], & n \otimes m &\longmapsto n \otimes m\\
		N \otimes M[1] &\lto (N \otimes M)[1], & n \otimes m &\longmapsto (-1)^{|n|}n \otimes m
	\end{align}
	satisfy the conditions in Appendix \cite[B]{Cut_Operation} and therefore give the bicategory of $\bb{Z}_2$-graded algberas and bimodules the structure of a superbicategory.
\end{example}

\begin{thebibliography}{99}
	\bibitem{Hitchcock} R. Hitchcock. \emph{Differentiation, Division and the Bicategory of Landau-Ginzburg Models}. Master's thesis. \url{http://therisingsea.org/notes/MScThesisRohanHitchcock.pdf}
	
	\bibitem{Borceux} F. Borceux. \emph{Handbook of Categorical Algebra 1: Basic Category Theory}. Vol. 50. Encyclopedia of Mathematics and its Applications. Cambridge University Press, 1994.
	
	\bibitem{TroianiSCA} W. Troiani. \emph{Secondary Commutative Algebra}.
	
	\bibitem{Friedrich} T. Friedrich, \emph{Dirac Operators in Riemannian Geometry} 20MathematicsSubjectClasification.Primary58Jx; Secondary 53C27, 53C28, 57R57, 58J05, 58J20, 58J50, 81R25.
	
	\bibitem{Cut_Operation} D. Murfet. \emph{The cut operation on matrix factorisations}
	
	
	
	
	
	
	
	
	
	\end{thebibliography}

\end{document}



























\maketitle
\tableofcontents