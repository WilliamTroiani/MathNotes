\documentclass[english,letter paper,12pt,leqno]{article}

\usepackage{amsthm}
\usepackage{amsmath}
\usepackage{amsfonts}
\usepackage{mathrsfs}
\usepackage{amssymb}
\usepackage{units}
\usepackage{graphicx}
\usepackage{tikz-cd}
\usepackage{nicefrac}
\usepackage{hyperref}
\usepackage{bbm}
\usepackage{color}
\usepackage{tensor}
\usepackage{tipa}
\usepackage{bussproofs}
\usepackage{ stmaryrd }
\usepackage{ textcomp }
\usepackage{leftidx}
\usepackage{afterpage}\usepackage{array}
\usepackage{stmaryrd}
\usepackage[all]{xy}
\xyoption{rotate}
\usepackage{dsfont}
\usepackage{bussproofs}
\usepackage{tikz}
\usepackage{epigraph}
\usepackage{tikz-cd}
\usepackage{bbm}
\usepackage{csquotes}
\newcommand{\call}[1]{\mathcal{#1}}
\newcommand{\psheaf}{\text{\underline{Set}}^{\scr{C}^{\text{op}}}}
\newcommand{\und}[1]{\underline{\hspace{#1 cm}}}
\newcommand{\adj}[1]{\text{\textopencorner}{#1}\text{\textcorner}}
\newcommand{\comment}[1]{}

% Operators
\def\can{\operatorname{can}}
\def\Hom{\operatorname{Hom}}
\def\be{\begin{equation}}
	\def\ee{\end{equation}}
\def\nN{\mathds{N}}
\def\nZ{\mathds{Z}}
\def\nQ{\mathds{Q}}
\def\nR{\mathds{R}}
\def\nC{\mathds{C}}
\def\ldot{\,.\,}
\def\FV{\operatorname{FV}}
\def\jac{\operatorname{Jac}_W}
\DeclareMathOperator{\id}{id}
\def\typearrow{\rightarrow}
\def\imp{\supset}
\newcommand{\bb}[1]{\mathbb{#1}}
\newcommand{\scr}[1]{\mathscr{#1}}

\def\can{\operatorname{can}}
\def\Hom{\operatorname{Hom}}
\def\be{\begin{equation}}
	\def\ee{\end{equation}}
\def\nN{\mathds{N}}
\def\nZ{\mathds{Z}}
\def\nQ{\mathds{Q}}
\def\nR{\mathds{R}}
\def\nC{\mathds{C}}
\def\ldot{\,.\,}
\def\typearrow{\Rightarrow}
\def\FV{\operatorname{FV}}
\def\jac{\operatorname{Jac}_W}
% Commands
\newcommand{\proofvdots}[1]{\overset{\displaystyle #1}{\rvdots}}
\def\Res{\res\!}
\newcommand{\ud}{\mathrm{d}}
\newcommand{\Ress}[1]{\res_{#1}\!}
\newcommand{\cat}[1]{\mathcal{#1}}
\newcommand{\lto}{\longrightarrow}
\newcommand{\xlto}[1]{\stackrel{#1}\lto}
\newcommand{\md}[1]{\mathscr{#1}}
\def\sus{\l}
\def\l{\,|\,}
\def\sgn{\textup{sgn}}
\def\samp{\zeta}
\def\Samp{Z}
\def\traff{N}


\SelectTips{cm}{}

\setlength{\evensidemargin}{0.1in}
\setlength{\oddsidemargin}{0.1in}
\setlength{\textwidth}{6.3in}
\setlength{\topmargin}{0.0in}
\setlength{\textheight}{8.5in}
\setlength{\headheight}{0in}

\newtheorem{theorem}{Theorem}[section]
\newtheorem{thm}[theorem]{Theorem}
\newtheorem{proposition}[theorem]{Proposition}
\newtheorem{lemma}[theorem]{Lemma}
\newtheorem{corollary}[theorem]{Corollary}
\newtheorem{cor}[theorem]{Corollary}
\newtheorem{setup}[theorem]{Setup}
\newtheorem{defn}[theorem]{Definition}
\newtheorem{algorithm}[theorem]{Algorithm}
\newtheorem{question}{Question}

% Labels in tabular
\newcommand{\tagarray}{\mbox{}\refstepcounter{equation}$(\theequation)$}

\newtheoremstyle{example}{\topsep}{\topsep}
{}
{}
{\bfseries}
{.}
{2pt}
{\thmname{#1}\thmnumber{ #2}\thmnote{ #3}}

\theoremstyle{example}
\newtheorem{definition}[theorem]{Definition}
\newtheorem{example}[theorem]{Example}
\newtheorem{remark}[theorem]{Remark}
\newtheorem{desi}[theorem]{Desiderata}

\numberwithin{equation}{section}

% Will stuff}

% Operators
\def\can{\operatorname{can}}
\def\Hom{\operatorname{Hom}}
\def\be{\begin{equation}}
\def\ee{\end{equation}}
\def\nN{\mathds{N}}
\def\nZ{\mathds{Z}}
\def\nQ{\mathds{Q}}
\def\nR{\mathds{R}}
\def\nC{\mathds{C}}
\def\ldot{\,.\,}
\def\FV{\operatorname{FV}}
\def\jac{\operatorname{Jac}_W}
\def\typearrow{\rightarrow}
\def\imp{\supset}

\title{Shadows of Computation, Lecture 5}
\author{Will Troiani}
\date{September 2022}

\begin{document}

\maketitle

\section{The untyped $\lambda$-calculus}
\begin{defn}
	\label{lambdacalc}
	Let $\scr{V}$ be a (countably) infinite set of variables, and let $\scr{L}$ be the language consisting of $\scr{V}$ along with the special symbols 
	\[\lambda \qquad . \qquad ( \qquad )\]
	Let $\scr{L}^\ast$ be the set of words of $\scr{L}$, more precisely, an element $w \in \scr{L}^\ast$ is a finite sequence $(w_1,...,w_n)$ where each $w_i$ is in $\scr{L}$, for convenience, such an element will be written as $w_1...w_n$. Now let $\Lambda_p$ denote the smallest subset of $\scr{L}^\ast$ such that
	\begin{itemize}
		\item if $x \in \scr{V}$ then $x \in \Lambda_p$,
		\item if $M,N \in \Lambda_p$ then $(MN) \in \Lambda_p$,
		\item if $x \in \scr{V}$ and $M \in \Lambda_p$ then $(\lambda x. M) \in \Lambda_p$
	\end{itemize}
	$\Lambda_p$ is the set of \textbf{preterms}. A preterm $M$ such that $M \in \scr{V}$ is a \textbf{variable}, if $M = (M_1M_2)$ for some preterms $M_1,M_2$, then $M$ is an \textbf{application}, and if $M = (\lambda x, M')$ for some $x \in \scr{V}$ and $M' \in \Lambda_p$ then $M$ is an \textbf{abstraction}.
\end{defn}
In practice, it becomes unwieldy to use this notation for the preterms exactly, and so the following notation is adopted:
\begin{defn}
	\begin{itemize}
		\item For preterms $M_1,M_2,M_3$, the preterm $M_1M_2M_3$ means $((M_1M_2)M_3))$,
		\item For variables $x,y$ and a preterm $M$, the preterm $\lambda xy. M$ means $(\lambda x. (\lambda y. M))$.
	\end{itemize}
\end{defn}
The variables $x$ which appear in the subpreterm $M$ of a preterm $\lambda x. M$ are viewed as ``markers for substitution", (see Remark \ref{markersforsubstitution}). For this reason, a distinction is made between the variable $x$ and the variable $y$ in, for example, the preterm $\lambda x. xy$:
\begin{defn}
	Given a preterm $M$, let $\text{FV}(M)$ be the following set of variables, defined recursively
	\begin{itemize}
		\item if $M = x$ where $x$ is a variable then $\text{FV}(M) = \lbrace x \rbrace$,
		\item if $M = M_1M_2$ then $\text{FV}(M) = \text{FV}(M_1) \cup \text{FV}(M_2)$,
		\item if $M = \lambda x. M'$ then $\text{FV}(M) = \text{FV}(M') \setminus \lbrace x \rbrace$.
	\end{itemize}
	A variable $x \in \text{FV}(M)$ is a \textbf{free variable} of $M$, a variable $x$ which appears in $M$ but is not a free variable is a \textbf{bound variable}.
\end{defn}
As mentioned, bound variables will be viewed as ``markers for substitution", so we define the following equivalence relation on $\Lambda_p$ which relates a preterm $M$ to $M'$ if $M$ can be obtained by replacing every bound occurrence of a variable $x$ in $M'$ with another variable $y$:
\begin{defn}
	For any term $M$, let $M[x := y]$ be the preterm given by replacing every bound occurrence of $x$ in $M$ with $y$. Define the following equivalence relation on $\Lambda_p$: $M \sim_\alpha M'$ if there exists $x,y \in \scr{V}$ such that $M[x := y] = M'$, where no free variable of $M$ becomes bound in $M[x := y]$. In such a case, we say that $M$ is \textbf{$\alpha$-equivalent} to $M'$.
\end{defn}
\begin{remark}
	The reason why we need to let $x$ and $y$ be such that no free variable of $M$ becomes bound in $M[x:=y]$ is so that a preterm such as $\lambda x. y$ does not get identified with the preterm $\lambda y. y$.
\end{remark}
We are now in a position to define the underlying language of $\lambda$-calculus:
\begin{defn}
	Let $\Lambda = \nicefrac{\Lambda_p}{\sim_\alpha}$ be the set of \textbf{$\lambda$-terms}. The set of \textbf{free variables} of a $\lambda$-term $[M]$ is $\text{FV}(M)$, which can be shown to be well defined. For convenience, $M$ will be written instead of $[M]$.
\end{defn}
Now the dynamics of the computation of $\lambda$-terms will be defined.
\begin{defn}
	\textbf{Single step $\beta$-reduction} $\to_\beta$ is the smallest relation on $\Lambda$ satisfying:
	\begin{itemize}
		\item the \textbf{reduction axiom}:
		\begin{itemize}
			\item for all variables $x$ and $\lambda$-terms $M,M'$, $(\lambda x. M)M' \to_\beta M[x := M']$, where $M[x:= M']$ is the term given by replacing every free occurrence of $x$ in $M$ with $M'$,
		\end{itemize}
		\item the following \textbf{compatibility axioms}:
		\begin{itemize}
			\item if $M \to_\beta M'$ then $(MN) \to_\beta (M'N)$ and $(NM) \to_\beta (NM')$,
			\item if $M \to_\beta M'$ then for any variable $x$, $\lambda x. M \to_\beta \lambda x M'$.
		\end{itemize}
	\end{itemize}
	A subterm of the form $(\lambda x. M)M'$ is a \textbf{$\beta$-redex}, and $(\lambda x. M)M'$ \textbf{single step $\beta$-reduces} to $M[x := M']$.
\end{defn}
\begin{remark}
	Strictly, single step $\beta$-reduction should be defined on preterms and then shown that a well defined relation is induced on terms, but this level of detail has been omitted for the sake of clarity.
\end{remark}
\begin{remark}
	\label{markersforsubstitution}
	The reduction axiom shows precisely in what sense a bound variable is a ``marker for substitution". For example, $(\lambda x.x)M \to_\beta M$ and $(\lambda y.y)M \to_\beta M$, which is why $\lambda x.x$ is identified with $\lambda y.y$.
\end{remark}
It is through single step $\beta$-reduction that computation may be performed. In fact, $\lambda$-calculus is capable of performing natural number addition:
\begin{example}
	\label{plusonetwo}
	Define the following $\lambda$-terms:
	\begin{itemize}
		\item ONE := $\lambda f x . fx$,
		\item TWO := $\lambda f x . ffx$,
		\item THREE := $\lambda fx. fffx$,
		\item PLUS := $\lambda mnfx. mf(nfx)$
	\end{itemize}
	then
	\begin{align*}
		PLUS\text{ }ONE\text{ }TWO &= (\lambda mnfx. \underline{m}f(nfx))\underline{(\lambda f x . fx)}(\lambda f x . ffx)\\
		&\to_\beta (\lambda nfx. (\lambda f x . \underline{f}x)\underline{f}(nfx))(\lambda f x . ffx)\\
		&\to_\beta (\lambda nfx. (\lambda x . f\underline{x})\underline{(nfx)})(\lambda f x . ffx)\\
		&\to_\beta (\lambda nfx. f\underline{n}fx)\underline{(\lambda f x . ffx)}\\
		&\to_\beta (\lambda fx. f(\lambda f x . \underline{f}\underline{f}x)\underline{f}x)\\
		&\to_\beta (\lambda fx. f(\lambda  x . ff\underline{x})\underline{x})\\
		&\to_\beta (\lambda fx. fffx) = THREE
	\end{align*}
	where each step is obtained by substituting the right most underlined $\lambda$-term inplace of the left most underlined variable.
\end{example}
Historically, this is how Church first defined computable functions.

There is also $\eta$-expansion, which is defined similarly.

\begin{defn}\label{def:eta}
	\textbf{Single step $\eta$-expansion} $\lto_\eta$ is the smallest, compatible relation on $\Lambda$ satisfying:
	\begin{equation}
		M \lto_\eta \lambda x. Mx
	\end{equation}
	where $x$ is a variable not in the free variable set of $M$. \textbf{Multi step $\eta$-expansion} is the reflexive closure of single step $\eta$-expansion. \textbf{$\eta$-equivalence} is the reflexive, symmetric closure of multi step $\eta$-expansion.
	
	\textbf{$\beta\eta$-equivalence} is the union of $\eta$-equivalence and $\beta$-equivalence.
\end{defn}

\section{Simply typed $\lambda$-calculus}
In the simply-typed $\lambda$-calculus \cite[Chapter 3]{sorensen} there is an infinite set of \emph{atomic types} and the set $\Phi_{\typearrow}$ of \emph{simple types} is built up from the atomic types using $\typearrow$. Let $\Lambda'$ denote the set of untyped $\lambda$-calculus preterms in these variables, as defined in \cite[Chapter 1]{sorensen}. We define a subset $\Lambda'_{wt} \subseteq \Lambda'$ of \emph{well-typed} preterms, together with a function $t: \Lambda'_{wt} \lto \Phi_{\typearrow}$ by induction:
\begin{itemize}
	\item all variables $x : \sigma$ are well-typed and $t(x) = \sigma$,
	\item if $M = (P \, Q)$ and $P,Q$ are well-typed with $t(P) = \sigma \typearrow \tau$ and $t(Q) = \sigma$ for some $\sigma, \tau$ then $M$ is well-typed and $t(M) = \tau$,
	\item if $M = \lambda x\ldot N$ with $N$ well-typed, then $M$ is well-typed and $T(M) = t(x) \typearrow t(N)$.
\end{itemize}
We define $\Lambda'_\sigma = \{ M \in \Lambda'_{wt} \l t(M) = \sigma \}$ and call these \emph{preterms of type $\sigma$}. Next we observe that $\Lambda'_{wt} \subseteq \Lambda'$ is closed under the relation of $\alpha$-equivalence on $\Lambda'$, as long as we understand $\alpha$-equivalence type by type, that is, we take
\[
\lambda x \ldot M =_\alpha \lambda y\ldot M[x := y]
\]
as long as $t(x) = t(y)$. Denoting this relation by $=_\alpha$, we may therefore define the sets of \emph{well-typed $\lambda$-terms} and \emph{well-typed $\lambda$-terms of type $\sigma$}, respectively:
\begin{align}
	\Lambda_{wt} &= \Lambda'_{wt} / =_\alpha\,\\
	\Lambda_\sigma &= \Lambda'_\sigma / =_\alpha\,.
\end{align}
Note that $\Lambda_{wt}$ is the disjoint union over all $\sigma \in \Phi_{\typearrow}$ of $\Lambda_\sigma$. We write $M: \sigma$ as a synonym for $[M] \in \Lambda_\sigma$, and call these equivalence classes \emph{terms of type $\sigma$}. Since terms are, by definition, $\alpha$-equivalence classes, the expression $M = N$ henceforth means $M =_\alpha N$ unless indicated otherwise. We denote the set of free variables of a term $M$ by $\FV(M)$.

\section{The category of $\lambda$-terms}
We define a category $\call{L}$ whose objects are the types of simply-typed $\lambda$-calculus, and whose morphisms are the terms of that calculus. The natural desiderata for such a category are that the fundamental algebraic structure of $\lambda$-calculus, function application and lambda abstraction, should be realised by categorical algebra.

Following Church's original presentation our $\lambda$-calculus only contains function types and $\Phi_{\typearrow}$ denotes the set of simple types. We write $\Lambda_\sigma$ for the set of $\alpha$-equivalence classes of $\lambda$-terms of type $\sigma$, and we write $=_{\beta\eta}$ for the equivalence relation generated by $\beta\eta$ equivalence.

\begin{definition}[(Category of $\lambda$-terms)]\label{definition:lambda_cat} The category $\call{L}$ has objects
	\[
	\operatorname{ob}(\call{L}) = \Phi_{\typearrow} \cup \{ \bold{1} \}
	\]
	and morphisms given for types $\sigma, \tau \in \Phi_{\typearrow}$ by
	\begin{align*}
		\call{L}(\sigma, \tau) &= \Lambda_{\sigma \typearrow \tau}/\!=_{\beta\eta}\,\\
		\call{L}(\bold{1}, \sigma) &= \Lambda_{\sigma}/\!=_{\beta\eta}\,\\
		\call{L}(\sigma, \bold{1}) &= \{ \star \}\,\\
		\call{L}(\bold{1},\bold{1}) &= \{ \star \}\,,
	\end{align*}
	where $\star$ is a new symbol. For $\sigma, \tau, \rho \in \Phi_{\typearrow}$ the composition rule is the function
	\begin{align}
		\label{eq:tttt}\call{L}(\tau, \rho) \times \call{L}(\sigma, \tau) &\longrightarrow \call{L}(\sigma, \rho)\\
		(N,M) &\longmapsto \lambda x^\sigma \ldot (N (M x))
	\end{align}
	where $x \not\in \FV(N) \cup \FV(M)$. We write the composite as $N \circ M$. In the remaining special cases the composite is given by the rules
	\begin{align}
		\label{eq:ttct}\call{L}(\tau, \rho) \times \call{L}(\bold{1}, \tau) \longrightarrow \call{L}(\bold{1}, \rho)\,, \qquad & N \circ M = (N \, M)\,,\\
		\label{eq:ctcc}\call{L}(\bold{1}, \rho) \times \call{L}(\bold{1}, \bold{1}) \longrightarrow \call{L}(\bold{1}, \rho)\,, \qquad & N \circ \star = N\,,\\
		\label{eq:cttc}\call{L}(\bold{1}, \rho) \times \call{L}(\sigma, \bold{1}) \longrightarrow \call{L}(\sigma, \rho)\,, \qquad & N \circ \star = \lambda t^\sigma \ldot N\,,
	\end{align}
	where in the final rule $t \notin \FV(N)$. Notice that these functions, although their rules depend on representatives of equivalence classes, are none-the-less well defined.
\end{definition}

For terms $M,N$ the expression $M = N$ always means equality of terms (that is, up to $\alpha$-equivalence) and we write $M =_{\beta\eta} N$ if we want to indicate equality up to $\beta\eta$-equivalence (for example as morphisms in the category $\call{L}$). Let $\twoheadrightarrow_\beta$ denote multi-step $\beta$-reduction \cite[Definition 1.3.3]{sorensen}.

\begin{lemma}\label{lemma:beta_reduce_FV} If $M \twoheadrightarrow_\beta N$ then $\FV(N) \subseteq \FV(M)$.
\end{lemma}
%\begin{proof}
%In a one-step $\beta$-reduction $((\lambda x \ldot P) Q) \rightarrow_\beta P[x:=Q]$ the free variables on the reduced term were either already free in $P$ or $Q$.
%\end{proof}

\begin{definition} Given a term $M$ we define
	\[
	\FV_\beta(M) = \bigcap_{N =_{\beta} M} \FV(N)
	\]
	where the intersection is over all terms $N$ which are $\beta$-equivalent to $M$. 
\end{definition}

We prove that we have a category.

The following calculation shows that $\id_\sigma \in \cat{L}(\sigma, \sigma)$ is an identity at $\sigma$.
Observe that for a term $M: \sigma \typearrow \tau$, we have
\begin{align*}
	\lambda t^\sigma \ldot (M (\id_\sigma t)) &= \lambda t^\sigma \ldot (M ((\lambda x^\sigma \ldot x) t))\\
	&=_\beta \lambda t \ldot (M t)\\
	&=_\eta M\,,
\end{align*}
and similarly $\lambda s^\tau \ldot (\id_\tau (M s)) =_{\beta\eta} M$. Moreover, $\star$ is clearly an identity at $\bold{1}$. For associativity there are a few cases to check:
\begin{itemize}
	\item Consider a diagram of objects and morphisms in $\cat{L}$ of the form:
	\be
	\xymatrix@C+2pc{
		\delta & \rho \ar[l]_-{P} & \tau \ar[l]_-{N} & \sigma \ar[l]_-{M}\,.
	}
	\ee
	\begin{align*}
		P \circ (N \circ M) &= \lambda y^\sigma \ldot (P (N \circ M \,y))\\
		&= \lambda y^\sigma \ldot (P ((\lambda x^\sigma \ldot (N (M x))) y))\\
		&=_\beta \lambda y^\sigma \ldot (P (N (M y)))\\
		&=_\beta (P \circ N) \circ M\,.
	\end{align*}
	\item Consider a diagram of objects and morphisms in $\cat{L}$ of the form
	\be
	\xymatrix@C+2pc{
		\delta & \rho \ar[l]_-{P} & \tau \ar[l]_-{N} & \bold{1} \ar[l]_-{M}\,.
	}
	\ee
	\begin{align*}
		P \circ (N \circ M) &= P \circ (N M)\\
		&= (P (N M))\\
		&= (\lambda y^\tau \ldot (P (N y)) M)\\
		&= (P \circ N) \circ M\,.
	\end{align*}
	\item Consider a diagram of objects and morphisms in $\cat{L}$ of the form
	\be
	\xymatrix@C+2pc{
		\delta & \rho \ar[l]_-{P} & \bold{1} \ar[l]_-{N} & \sigma \ar[l]_-{\star}\,.
	}
	\ee
	\begin{align*}
		(P \circ N) \circ \star &= (P N) \circ \star\\
		&= \lambda t^\sigma \ldot (P N)\\
		&= \lambda t^\sigma (P ((\lambda z^\sigma \ldot N) t))\\
		&= P \circ (N \circ \star)\,.
	\end{align*}
	\item Consider a diagram of objects and morphisms in $\cat{L}$ of the form
	\be
	\xymatrix@C+2pc{
		\delta & \bold{1} \ar[l]_-{P} & \tau \ar[l]_-{\star} & \sigma \ar[l]_-{M}\,.
	}
	\ee
	\begin{align*}
		(P \circ \star) \circ M &= (\lambda t^\tau \ldot P) \circ M\\
		&= \lambda q^\sigma \ldot ( (\lambda t^\tau \ldot P) (M q))\\
		&= \lambda q^\sigma \ldot P\\
		&= P \circ ( \star \circ M)\,.
	\end{align*}
\end{itemize}
The other cases are trivial.

\bibliographystyle{amsalpha}
\providecommand{\bysame}{\leavevmode\hbox to3em{\hrulefill}\thinspace}
\providecommand{\href}[2]{#2}
\begin{thebibliography}{BHLS03}
	
	\bibitem{atiyah}
	M.~Atiyah, \textsl{Duality in mathematics and physics}, in: lecutre notes from Institut de Matematica de la Universitat de Barcelona (IMUB), (2007) available at: \url{https://fme.upc.edu/ca/arxius/butlleti-digital/riemann/071218_conferencia_atiyah-d_article.pdf}.
	
	\bibitem{cutctr}
	M.~Borisavljevic, K.~Dosen, Z.~Petric, \textsl{On Permuting Cut with Contraction}, preprint arXiv:math/9911065v1.
	
	\bibitem{church}
	A.~Church, \textsl{A set of postulates for the foundation of logic}, Annals of Mathematics \textbf{33}, no.2 pp.346--366 (1932).
	
	\bibitem{curry}
	H.~B.~Curry and R.~Feys, \textsl{Combinatory logic, volume I}, Studies in Logic and the Foundations of Mathematics, North-Holland, Amsterdam, (1958).
	
	\bibitem{dosen_1996}
	K.~Do\v{s}en, \textsl{Deductive completeness}, Bulletin of symbolic logic 2.3, pp.243--283 (1996).
	
	\bibitem{dosen_2001}
	K.~Do\v{s}en, \textsl{Abstraction and application in adjunction}, arXiv preprint math/0111061 (2001).
	
	\bibitem{dyckhoffpinto}
	R.~Dyckhoff, L.~Pinto, \textsl{Permutability of proofs in intuitionistic sequent calculi}, Theoretical Computer Science \textbf{212}, pp.141-155, (1999).
	
	\bibitem{gallier}
	J.~H.~Gallier, \textsl{Constructive logics Part I: A tutorial on proof systems and typed lambda-calculi}, Theoretical Computer Science, 110(2) pp.249--339, (1993).
	
	\bibitem{gentzen}
	G.~Gentzen, \textsl{Untersuchungen \"uber das logische Schliessen}, Mathematische Zeitschrift \textbf{39} (1935) 176--210, 405--431, translation in \textsl{The collected papers of Gerhard Gentzen}, edited by M.~E.~Szabo, 1969.
	
	\bibitem{girard_blind}
	J.-Y.~Girard, \textsl{The Blind Spot: lectures on logic}, European Mathematical Society, (2011).
	
	\bibitem{girard}
	J.-Y.~Girard, Y.~Lafont, and P.~Taylor, \textsl{Proofs and Types}, Cambridge Tracts in Theoretical Computer Science 7 ,Cambridge University Press, 1989.
	
	\bibitem{herbelin}
	H.~Herbelin, \textsl{A Lambda-Calculus structure isomorphic to a Gentzen-style sequent calculus structure}, in L.~Pacholski and J.~Tiuryn, editors, \textsl{Computer Science Logic}, 8th workshop, CSL'94, volume 933 of Lecture Notes in Computer Science, pp.61--75, Springer-Verlag (1995).
	
	\bibitem{hermida}
	C.~Hermida and B.~Jacobs, \textsl{Fibrations with indeterminates: Contextual and functional completeness for polymorphic lambda calculi}, Math. Structures Comput. Sci. 5 (1995), 501--531.
	
	\bibitem{heyting}
	A.~Heyting, \textsl{Intuitionism, an introduction}, Studies in Logic and the Foundations of Mathematics, North-Holland, (1956). Third edition (1971).
	
	\bibitem{howard}
	W.~A.~Howard, \textsl{The formulae-as-types notion of construction}, in Seldin and Hindley \textsl{To H.B.Curry: essays on Combinatory logic, Lambda calculus and Formalism}, Academic press (1980).
	
	\bibitem{virtues_eta}
	C.~B.~Jay, N.~Ghani, \textsl{The virtues of eta-expansion}, J. Functional Programming \textbf{1} (1): 1--000, Cambridge University Press, (1993).
	
	\bibitem{kleene}
	S.~C.~Kleene, \textsl{Two papers on the predicate calculus}, Memoirs of the American Mathematical Society, \textbf{10}, (1952).
	
	\bibitem{lambek_func} J.~Lambek, \textsl{Functional completeness of cartesian categories}, Annals of Mathematical Logic \emph{6.3-4} pp.259--292, (1974).
	
	\bibitem{lambek_scott}
	J.~Lambek and P.J.~Scott, \textsl{Introduction to higher-order categorical logic}, Camrbidge Studies in Advanced Mathematics, Cambridge University Press, (1986).
	
	\bibitem{lawvere}
	F.W.~Lawvere, \textsl{Adjointness in foundations}, Dialectica \textbf{23} No. 3/4, pp.281--296, (1969).
	
	\bibitem{maclane}
	S.~Mac Lane, \textsl{The Lambda Calculus and Adjoint Functors}, Logic, Meaning and Computation, Springer Netherlands, pp.181--184 (2001).
	
	\bibitem{mints}
	G.~Mints, \textsl{Normal forms for sequent derivations}, in: P.~Odifreddi (Ed.), Kreiseliana, A.~K.~Peters, Wellesley, Massachusetts, 1996, pp. 469--492; also part of Stanford Univ. Report CSLI-94-193, November 1994.
	
	\bibitem{negri}
	S.~Negri, \textsl{Varieties of linear calculi}, Journal of Philosophical Logic \textbf{31}, pp.569--590, (2002).
	
	\bibitem{negriplato}
	S.~Negri and J.~von Plato, \textsl{Structural proof theory}, Cambridge University Press, (2008).
	
	\bibitem{pottinger}
	G.~Pottinger, \textsl{Normalization as a homomorphic image of cut-elimination}, Annals of Mathematical Logic \textbf{12} pp.323--357, (1977).
	
	\bibitem{prawitz}
	D.~Prawitz, \textsl{Natural deduction: a proof-theoretical study}, Almqvist \& Wicksell, Stockholm (1965).
	
	\bibitem{prawitz_phil}
	D.~Prawitz, \textsl{Philosophical aspects of proof theory}, Contemporary philosophy: a new survey, \textbf{1}, pp.235--277, Martinus Nijhoff Publishers, The Hague/Boston/London (1981).
	
	\bibitem{selinger}
	P.~Selinger, \textsl{Lecture notes on the lambda calculus}, preprint \href{https://arxiv.org/abs/0804.3434}[arXiv:0804.3434], (2008).
	
	\bibitem{sorensen}
	M.~S\o rensen and P.~Urzyczyn, \textsl{Lectures on the Curry-Howard isomorphism}, Studies in Logic and the Foundations of Mathematics Vol. 149, Elsevier New York, (2006).
	
	\bibitem{troelstra_margin}
	A.~S.~Troelstra, \textsl{Marginalia on Sequent Calculi}, Studia Logica \textbf{62}, pp.291--303, (1999).
	
	\bibitem{troelstra}
	A~.S.~Troelstra and D.~van Dalen, \textsl{Constructivism in Mathematics}, Vol. 1, Studies in Logic and the Foundations of Mathematics, 121, Amsterdam: North-Holland, (1988).
	
	\bibitem{troelstra_sch}
	A.~S.~Troelstra, H.~Schwichtenberg, \textsl{Basic proof theory}, Cambridge University Press, Cambridge, (1996).
	
	\bibitem{wadler}
	P.~L.~Wadler, \textsl{Proofs are programs: 19th century logic and 21st century computing}, Manuscript (2000).
	
	\bibitem{ungar}
	A.~M.~Ungar, \textsl{Normalization, cut-elimination and the theory of proofs}, Center for the Study of Language and Information Lecture Notes No. 28, (1992).
	
	\bibitem{zucker}
	J.~Zucker, \textsl{The correspondence between cut-elimination and normalization}, Annals of Mathematical Logic \textbf{7} 1--112 (1974).
	
	
\end{thebibliography}



\end{document}