\documentclass[12pt]{article}

\usepackage{amsthm}
\usepackage{amsmath}
\usepackage{amsfonts}
\usepackage{mathrsfs}
\usepackage{array}
\usepackage{amssymb}
\usepackage{units}
\usepackage{graphicx}
\usepackage{tikz-cd}
\usepackage{nicefrac}
\usepackage{hyperref}
\usepackage{bbm}
\usepackage{color}
\usepackage{tensor}
\usepackage{tipa}
\usepackage{bussproofs}
\usepackage{ stmaryrd }
\usepackage{ textcomp }
\usepackage{leftidx}
\usepackage{afterpage}
\usepackage{varwidth}
\usepackage{tasks}
\usepackage{ cmll }
\usepackage{quiver}
\usepackage{adjustbox}
\usepackage{kbordermatrix}

\newcommand\blankpage{
	\null
	\thispagestyle{empty}
	\addtocounter{page}{-1}
	\newpage
}

\graphicspath{ {images/} }

\theoremstyle{plain}
\newtheorem{thm}{Theorem}[subsection] % reset theorem numbering for each chapter
\newtheorem{proposition}[thm]{Proposition}
\newtheorem{lemma}[thm]{Lemma}
\newtheorem{fact}[thm]{Fact}
\newtheorem{cor}[thm]{Corollary}

\theoremstyle{definition}
\newtheorem{defn}[thm]{Definition} % definition numbers are dependent on theorem numbers
\newtheorem{exmp}[thm]{Example} % same for example numbers
\newtheorem{notation}[thm]{Notation}
\newtheorem{remark}[thm]{Remark}
\newtheorem{condition}[thm]{Condition}
\newtheorem{question}[thm]{Question}
\newtheorem{construction}[thm]{Construction}
\newtheorem{exercise}[thm]{Exercise}
\newtheorem{example}[thm]{Example}
\newtheorem{aside}[thm]{Aside}

\def\doubleunderline#1{\underline{\underline{#1}}}
\newcommand{\bb}[1]{\mathbb{#1}}
\newcommand{\scr}[1]{\mathscr{#1}}
\newcommand{\call}[1]{\mathcal{#1}}
\newcommand{\psheaf}{\text{\underline{Set}}^{\scr{C}^{\text{op}}}}
\newcommand{\und}[1]{\underline{\hspace{#1 cm}}}
\newcommand{\adj}[1]{\text{\textopencorner}{#1}\text{\textcorner}}
\newcommand{\comment}[1]{}
\newcommand{\lto}{\longrightarrow}
\newcommand{\rone}{(\operatorname{R}\bold{1})}
\newcommand{\lone}{(\operatorname{L}\bold{1})}
\newcommand{\rimp}{(\operatorname{R} \multimap)}
\newcommand{\limp}{(\operatorname{L} \multimap)}
\newcommand{\rtensor}{(\operatorname{R}\otimes)}
\newcommand{\ltensor}{(\operatorname{L}\otimes)}
\newcommand{\rtrue}{(\operatorname{R}\top)}
\newcommand{\rwith}{(\operatorname{R}\&)}
\newcommand{\lwithleft}{(\operatorname{L}\&)_{\operatorname{left}}}
\newcommand{\lwithright}{(\operatorname{L}\&)_{\operatorname{right}}}
\newcommand{\rplusleft}{(\operatorname{R}\oplus)_{\operatorname{left}}}
\newcommand{\rplusright}{(\operatorname{R}\oplus)_{\operatorname{right}}}
\newcommand{\lplus}{(\operatorname{L}\oplus)}
\newcommand{\prom}{(\operatorname{prom})}
\newcommand{\ctr}{(\operatorname{ctr})}
\newcommand{\der}{(\operatorname{der})}
\newcommand{\weak}{(\operatorname{weak})}
\newcommand{\exi}{(\operatorname{exists})}
\newcommand{\fa}{(\operatorname{for\text{ }all})}
\newcommand{\ex}{(\operatorname{ex})}
\newcommand{\cut}{(\operatorname{cut})}
\newcommand{\ax}{(\operatorname{ax})}
\newcommand{\negation}{\sim}
\newcommand{\true}{\top}
\newcommand{\false}{\bot}
\DeclareRobustCommand{\diamondtimes}{%
	\mathbin{\text{\rotatebox[origin=c]{45}{$\boxplus$}}}%
}
\newcommand{\tagarray}{\mbox{}\refstepcounter{equation}$(\theequation)$}
\newcommand{\startproof}[1]{
	\AxiomC{#1}
	\noLine
	\UnaryInfC{$\vdots$}
}
\newenvironment{scprooftree}[1]%
{\gdef\scalefactor{#1}\begin{center}\proofSkipAmount \leavevmode}%
	{\scalebox{\scalefactor}{\DisplayProof}\proofSkipAmount \end{center} }

\usepackage[margin=1.5cm]{geometry}


\title{Execution formula for \emph{all} of MELL}
\author{Will Troiani}
\date{\today}

\begin{document}

\maketitle

\begin{defn}
\label{def:atomic_atoms}
There is an infinite set of \textbf{unoriented atoms} $X,Y,Z,\ldots$ and an \textbf{oriented atom} (or \textbf{atomic proposition}) is a pair $(X,+)$ or $(X,-)$ where $X$ is an unoriented atom. Let $\call{A}$ denote the set of oriented atoms.
\end{defn}

For $x \in \{ +, - \}$ we write $\overline{x}$ for the negation, so $\overline{+} = -, \overline{-} = +$.

\begin{defn}
The set of \textbf{pre-formulas} is defined as follows:
\begin{itemize}
	\item Any atomic proposition is a preformula.
	\item If $A,B$ are pre-formulas then so are $A \otimes B, A \parr B$.
	\item If $A$ is a pre-formula then so are $\neg A, ! A, ? A$.
\end{itemize}
The set of \textbf{multiplicative exponential linear logic formulas (MELL formulas)} is the quotient of the set of pre-formulas by the equivalence relation generated, for arbitrary formulas $A, B$ and unoriented atom $X$, by
\begin{align*}
\neg(A \otimes B) &= \neg B \parr \neg A,\quad \neg(A \parr B) = \neg B \otimes A,\quad \neg(X, x) = (X, \overline{x})\\
\neg !A &= ? \neg A,\quad \neg ?A = !\neg A
\end{align*}
\end{defn}

Recall that in the multiplicative case, the set of words $\call{A}^\ast$ over $\call{A}$ forms a monoid under the operation of concatination. This monoidal structure extends to maps reflecting the connectives $\otimes, \parr, \neg$, for instance if $c: \call{A}^\ast \times \call{A}^\ast \lto \call{A}^\ast$ denotes concatination, and $\otimes: \call{F} \times \call{F} \lto \call{F}$ is the map sending a pair of formulas $A,B$ to the formula $A \otimes B$ then the following diagram commutes for some unique map $a: \call{F} \lto \call{A}^\ast$.
\begin{equation}
\begin{tikzcd}
\call{F} \times \call{F}\arrow[r,"{a \times a}"]\arrow[d,swap,"{\otimes}"] & \call{A}^\ast \times \call{A}^\ast\arrow[d,"{c}"]\\
\call{F}\arrow[r,"{a}"] & \call{A}^\ast
\end{tikzcd}
\end{equation}
See \cite[Definition 3.5, 3.6]{MT} for the full list of commutative diagrams. This map $a$ induces the \textbf{sequence of oriented atoms} of a formula $A$.
\begin{equation}
a(A) = (X_1, x_1), \ldots, (X_n, x_n)
\end{equation}
The \textbf{set of unoriented atoms} of $A$ is then the disjoint union
\begin{equation}
U_A = \{ X_1 \} \coprod \ldots \coprod \{ X_n \}
\end{equation}
We use this notation in the following Definition.
\begin{defn}\label{def:unoriented_atoms}
Let $A$ be a multiplicative exponential linear logic formula. The \textbf{set of unoriented atoms} $U_A$ of $A$ is defined by induction on the structure of $A$ as follows.
\begin{itemize}
	\item If $A = A_1 \otimes A_2$ or $A_1 \parr A_2$ then $U_A := U_{A_1} \coprod U_{A_2}$.
	\item If $A = \neg A'$ then $U_{A} := U_{A'}$.
	\item If $A = ?A'$ or $A = !A'$ then $U_A := \coprod_{j = 0}^\infty U_{A'}$.
\end{itemize}
\end{defn}

The set of unoriented atoms only depends on the formula, and not its placement inside some proof. Next we take into account the \emph{depth} of a formula inside a proof net.

\begin{defn}
Let $A$ be an occurrence of a formula inside a proof net and say $A$ has depth $d$. Then we define the set
\begin{equation}
\bb{N}^d \times U_A
\end{equation}
\end{defn}

\begin{defn}
Let $\pi$ be a proof net and let $E$ be its set of edges. Let $k$ denote a ring. The \textbf{polynomial ring of $\pi$} $P_{\pi}$ is
\begin{equation}
P_{\pi} = \bigotimes_{e \in E}k[\operatorname{Dep}U_{A_e}]
\end{equation}
where $A_e$ is the formula labelling edge $e$.
\end{defn}

Recall that in \cite{AlgPntExponentials} we defined MELL proof nets and we had the following clause for promotion links: Each promotion link must come equipt with a subset $(V, E)$ of the links and edges of the proof structure such that the following conditions hold:
	\begin{itemize}
		\item The following process must result in a proof structure: for every edge $e \in E$ such that the target $t(e)$ is \emph{not} an element of $V$, we introduce a conclusion vertex and set $t(e)$ to be this conclusion vertex.
		\item All edges $e \in E$ such that $t(e) \not\in V$, the label of $e$ is $?A$ for some $A$.
		\item The premise to the promotion link is an element of $E$.
			\item Each vertex labelled $c$ has exactly one premise and no conclusion. Such a premise of a vertex labelled $c$ is called a \textbf{conclusion} of the proof structure.
	\end{itemize}


\begin{defn}
If $e \in E$ is an edge in a subset $(V,E)$ of $\pi$ corresponding to some promotion link, and the target of $e$ does not lie in $V$, then the source of $e$ is on the \textbf{boarder of a box}.
\end{defn}

\begin{remark}
We notice that all vertices labelled $!$ are on the boarder of a box.
\end{remark}






















\bibliographystyle{amsalpha}
	\providecommand{\bysame}{\leavevmode\hbox to3em{\hrulefill}\thinspace}
	\providecommand{\href}[2]{#2}
	\begin{thebibliography}{99}
		\bibitem{AlgPntExponentials} \emph{AlgPntExponentials}

		\bibitem{linearlogic} \emph{Linear Logic}, J.Y. Girard. Theoretical Computer Science, Volume 50, Issue 1, Jan. 30, 1987.
		
		\bibitem{multiplicatives} \emph{Multiplicatives}, J.Y. Girard. Logic and Computer Science: New Trends and Applications. Rosenberg \& Sellier. pp. 11--34 (1987).
		
		\bibitem{girard} \emph{Geometry of Interaction: Interpretation of System F}, J.Y. Girard. Categories in Computer Science and Logic, pages 69 – 108, Providence, 1989.
		
		\bibitem{GoI2} \emph{Geometry of Interaction II, Deadlock Free Agorithms} Part of the Lecture Notes in Computer Science book series (LNCS,volume 417). 2005.
		
		\bibitem{GoI3} \emph{Geometry of Interaction III, Accomodation the Additives}, J.Y. Girard. Proceedings of the workshop on Advances in linear logic. June 1995
		
		\bibitem{GoI4} \emph{Geometry of Interaction IV, the Feedback Equation}, J.Y. Girard. Logic Colloquium 2003, December 9.
		
		\bibitem{GoI5} \emph{Geometry of Interaction V}, J.Y. Girard. Theoretical Computer ScienceVolume 412Issue 20April, 2011
		
		\bibitem{Regnier} \emph{Linear Logic and the Hilbert Space} Advances in Linear Logic , pp. 307 - 328, Cambridge University Press, 1995.
		
		\bibitem{Seiller1} \emph{Interaction Graphs: Multiplicatives} Annals of Pure and Applied Logic 163 (2012), pp. 1808-1837.
		
		\bibitem{Seiller2} \emph{Interaction Graphs: Additives} Annals of Pure and Applied Logic 167 (2016), pp. 95-154.
		
		\bibitem{Seiller3} \emph{Interaction Graphs: Nondeterministic Automata}, ACM Transactions in Computational Logic 19(3), 2018.
		
		\bibitem{Seiller4} \emph{Interaction Graphs: Exponentials} Logical Methods in Computer Science 15, 2019.
		
		\bibitem{Laurent} \emph{Olivier Laurent. A Token Machine for Full Geometry of Interaction}. 2001, pp.283-297. ⟨hal-00009137⟩
		
		\bibitem{HaghverdiScott} \emph{Towards a Typed Geometry of Interaction} CSL 2005: Computer Science Logic pp 216–231.
		
		\bibitem{Hines} \emph{From a conjecture of Collatz to Thompson's group F, via a conjunction of Girard}, \url{https://arxiv.org/abs/2202.04443}
		
		\bibitem{BS} \emph{The Blind Spot}. J.Y. Girard.
		
		\bibitem{AnalyticFunctors} \emph{Normal functors, power series and lambda-calculus} Annals of Pure and Applied Logic
		Volume 37, Issue 2, February 1988, Pages 129-177.
		
		\bibitem{GMZ} \emph{Gentzen-Mints-Zucker Duality} D. Murfet, W. Troiani. \url{https://arxiv.org/abs/2008.10131}
		
		\bibitem{laurent} \emph{An introduction to proof nets}. O. Laurent. \url{http://perso.ens-lyon.fr/olivier.laurent/pn.pdf}
		
		\bibitem{AlgPnt} \emph{Elimination and cut-elimination in multiplicative linear logic}, W. Troiani, D. Murfet.
		
		\bibitem{Frege} \emph{Sense and Reference} G. Frege. Philosophical Review 57 (3):209-230 (1948)
		
		\bibitem{Sorensen} \emph{Lectures on the Curry-Howard Isomorphism} Published: July 4, 2006 Imprint: Elsevier Science

		\bibitem{MT} \emph{Elimination and cut-elimination in multiplicative linear logic}
		
	\end{thebibliography}

\end{document}