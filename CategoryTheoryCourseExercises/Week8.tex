\documentclass[12pt]{article}

\usepackage{amsthm}
\usepackage{amsmath}
\usepackage{amsfonts}
\usepackage{mathrsfs}
\usepackage{array}
\usepackage{amssymb}
\usepackage{units}
\usepackage{graphicx}
\usepackage{tikz-cd}
\usepackage{nicefrac}
\usepackage{hyperref}
\usepackage{bbm}
\usepackage{color}
\usepackage{tensor}
\usepackage{tipa}
\usepackage{bussproofs}
\usepackage{ stmaryrd }
\usepackage{ textcomp }
\usepackage{leftidx}
\usepackage{afterpage}
\usepackage{varwidth}
\usepackage{tasks}
\usepackage{ cmll }
\usepackage{adjustbox}

\newcommand\blankpage{
	\null
	\thispagestyle{empty}
	\addtocounter{page}{-1}
	\newpage
}

\graphicspath{ {images/} }

\theoremstyle{plain}
\newtheorem{thm}{Theorem}[subsection] % reset theorem numbering for each chapter
\newtheorem{proposition}[thm]{Proposition}
\newtheorem{lemma}[thm]{Lemma}
\newtheorem{fact}[thm]{Fact}
\newtheorem{cor}[thm]{Corollary}

\theoremstyle{definition}
\newtheorem{defn}[thm]{Definition} % definition numbers are dependent on theorem numbers
\newtheorem{exmp}[thm]{Example} % same for example numbers
\newtheorem{notation}[thm]{Notation}
\newtheorem{remark}[thm]{Remark}
\newtheorem{condition}[thm]{Condition}
\newtheorem{question}[thm]{Question}
\newtheorem{construction}[thm]{Construction}
\newtheorem{exercise}[thm]{Exercise}
\newtheorem{example}[thm]{Example}
\newtheorem{aside}[thm]{Aside}

\def\doubleunderline#1{\underline{\underline{#1}}}
\newcommand{\bb}[1]{\mathbb{#1}}
\newcommand{\scr}[1]{\mathscr{#1}}
\newcommand{\call}[1]{\mathcal{#1}}
\newcommand{\psheaf}{\text{\underline{Set}}^{\scr{C}^{\text{op}}}}
\newcommand{\und}[1]{\underline{\hspace{#1 cm}}}
\newcommand{\adj}[1]{\text{\textopencorner}{#1}\text{\textcorner}}
\newcommand{\comment}[1]{}
\newcommand{\lto}{\longrightarrow}
\newcommand{\rone}{(\operatorname{R}\bold{1})}
\newcommand{\lone}{(\operatorname{L}\bold{1})}
\newcommand{\rimp}{(\operatorname{R} \multimap)}
\newcommand{\limp}{(\operatorname{L} \multimap)}
\newcommand{\rtensor}{(\operatorname{R}\otimes)}
\newcommand{\ltensor}{(\operatorname{L}\otimes)}
\newcommand{\rtrue}{(\operatorname{R}\top)}
\newcommand{\rwith}{(\operatorname{R}\&)}
\newcommand{\lwithleft}{(\operatorname{L}\&)_{\operatorname{left}}}
\newcommand{\lwithright}{(\operatorname{L}\&)_{\operatorname{right}}}
\newcommand{\rplusleft}{(\operatorname{R}\oplus)_{\operatorname{left}}}
\newcommand{\rplusright}{(\operatorname{R}\oplus)_{\operatorname{right}}}
\newcommand{\lplus}{(\operatorname{L}\oplus)}
\newcommand{\prom}{(\operatorname{prom})}
\newcommand{\ctr}{(\operatorname{ctr})}
\newcommand{\der}{(\operatorname{der})}
\newcommand{\weak}{(\operatorname{weak})}
\newcommand{\exi}{(\operatorname{exists})}
\newcommand{\fa}{(\operatorname{for\text{ }all})}
\newcommand{\ex}{(\operatorname{ex})}
\newcommand{\cut}{(\operatorname{cut})}
\newcommand{\ax}{(\operatorname{ax})}
\newcommand{\negation}{\sim}
\newcommand{\true}{\top}
\newcommand{\false}{\bot}
\DeclareRobustCommand{\diamondtimes}{%
	\mathbin{\text{\rotatebox[origin=c]{45}{$\boxplus$}}}%
}
\newcommand{\tagarray}{\mbox{}\refstepcounter{equation}$(\theequation)$}
\newcommand{\startproof}[1]{
	\AxiomC{#1}
	\noLine
	\UnaryInfC{$\vdots$}
}
\newenvironment{scprooftree}[1]%
{\gdef\scalefactor{#1}\begin{center}\proofSkipAmount \leavevmode}%
	{\scalebox{\scalefactor}{\DisplayProof}\proofSkipAmount \end{center} }


\title{Category Theory exercise sheet 8}

\begin{document}
	
	\maketitle
	
	\section{Category theory}
	\begin{enumerate}
		\item Let $F: \scr{C} \lto \scr{D}$ be a functor which is left adjoint to $G: \scr{D} \lto \scr{C}$. Consider the endofunctor $GF: \scr{C} \lto \scr{C}$ along with the natural transformations $\eta: \operatorname{id} \Rightarrow GF$ and $\mu = G(\epsilon_F): GFGF \Rightarrow GF$ where $\epsilon: FG \Rightarrow \operatorname{id}$ is the counit of the adjunction. Prove that
		\begin{equation}
			(GF, \eta, \mu)
		\end{equation}
		is a monad.
		\item Let $\scr{C}$ be a category and consider a monad $(T, \eta, \mu)$ over $\scr{C}$. Define:
		\begin{align*}
			\operatorname{Obj}(\scr{C}_T) &= \operatorname{Obj}(\scr{C})\\
			\operatorname{Hom}_{\scr{C_T}}(X,Y) &= \operatorname{Hom}_{\scr{C}}(X, TY)
			\end{align*}
			We define composition to be:
			\begin{equation}\label{eq:kleisli_comp}
				g \circ_{\scr{C}_T} f = \mu_C \circ Tg \circ f: A \lto TB \lto T^2C \lto TC
			\end{equation}
			The identity morphism for $X \in \scr{C}_T$ is $\eta_X$.
			Prove that $\scr{C}_T$ is a category. This category is called the \textbf{Kleisli category} of $T$.
		\end{enumerate}		
		Recall from the lectures that to every monad $(T, \mu, \eta)$ there exists a corresponding Kleisli triple $(T, \eta, \und{0.2}^\ast)$, where if $g: B \lto TC$ is a morphisim in $\scr{C}$ then $g^\ast: TB \lto TC$ is $\mu_C \circ Tf$. Thus, given $f: A\lto TB$ the Kleisli triple allows us to ``compose" $g$ and $f$ using $g^\ast$:
		\begin{equation}
			g^\ast \circ f = \mu_B \circ Tg \circ f
			\end{equation}
		which is exactly \eqref{eq:kleisli_comp}. So now we see exactly why it is that monads are so helpful for modelling programs. We start by identifying programs with input/output pairs (by identifying programs with morphisms in $\underline{\operatorname{Set}})$ and then we choose a notion of computation of particular interest, modelled by a monad $T$. Programs can then be compared with consideration of this notion of computation by comparing morphisms in the Kleisli category of $T$.
	\section{Mathematics}
	Let $M$ be a monoid, define the functor
	\begin{align*}
		T: \underline{\operatorname{Set}} &\lto \underline{\operatorname{Set}}\\
		X &\longmapsto X \times M\\
		(f: X \to Y) &\lto (Tf(x,m) = (f(x),m))
		\end{align*}
	Then define $\eta: \operatorname{id} \Rightarrow T$ by
	\begin{align*}
		\eta_X: X &\lto TX\\
		x &\longmapsto (x, e)
		\end{align*}
	where $e$ is the identity element of $M$. Define $\mu: T^2 \Rightarrow T$ by
	\begin{align*}
		\mu_X: T^2(X) &\lto T(X)\\
		((x,m_1), m_2) &\lto (x, m_1 \cdot m_2)
		\end{align*}
	Prove that
	\begin{equation}
		(T, \eta, \mu)
		\end{equation}
	is a monad.
	
	\section{Computer science}
	\begin{enumerate}
	\item Consider the category $\underline{\operatorname{Set}}$ of sets. Define the following notion of computation corresponding to exceptions. Let $E$ be a set of exceptions.
	\begin{align*}
		TA &= A \coprod E\\
		\eta_A(x) &= x\text{ (the canonical inclusion map)}\\
		(f: A \lto TB) &\lto f^\ast(a) = 
		\begin{cases}
			a,& a \in A\\
			e, & e \in E
			\end{cases}
		\end{align*}
	Prove that this is a Kleisli triple.
	\item Define
	\begin{align*}
		TA &= \call{P}_{\text{fin}}(A)\\
		\eta_A(a) &= \{ a \}\\
		(f: A \lto TB) &\longrightarrow \big(f^\ast(a) = \bigcup_{x \in a}f(x)\big)
	\end{align*}
	Prove that this is a Kleisli triple. What notion of computation does this model?
	\end{enumerate}
	
	
	
	
	
	
	\end{document}