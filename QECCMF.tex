\documentclass[12pt]{article}

\usepackage{amsthm}
\usepackage{amsmath}
\usepackage{amsfonts}
\usepackage{mathrsfs}
\usepackage{array}
\usepackage{amssymb}
\usepackage{units}
\usepackage{graphicx}
\usepackage{tikz-cd}
\usepackage{nicefrac}
\usepackage{hyperref}
\usepackage{bbm}
\usepackage{color}
\usepackage{tensor}
\usepackage{tipa}
\usepackage{bussproofs}
\usepackage{ stmaryrd }
\usepackage{ textcomp }
\usepackage{leftidx}
\usepackage{afterpage}
\usepackage{varwidth}
\usepackage{tasks}
\usepackage{ cmll }
\usepackage{quiver}
\usepackage{adjustbox}
\usepackage{kbordermatrix}
\usepackage{braket}

\newcommand\blankpage{
	\null
	\thispagestyle{empty}
	\addtocounter{page}{-1}
	\newpage
}

\graphicspath{ {images/} }

\theoremstyle{plain}
\newtheorem{thm}{Theorem}[subsection] % reset theorem numbering for each chapter
\newtheorem{proposition}[thm]{Proposition}
\newtheorem{lemma}[thm]{Lemma}
\newtheorem{fact}[thm]{Fact}
\newtheorem{cor}[thm]{Corollary}

\theoremstyle{definition}
\newtheorem{defn}[thm]{Definition} % definition numbers are dependent on theorem numbers
\newtheorem{exmp}[thm]{Example} % same for example numbers
\newtheorem{notation}[thm]{Notation}
\newtheorem{remark}[thm]{Remark}
\newtheorem{condition}[thm]{Condition}
\newtheorem{question}[thm]{Question}
\newtheorem{construction}[thm]{Construction}
\newtheorem{exercise}[thm]{Exercise}
\newtheorem{example}[thm]{Example}
\newtheorem{aside}[thm]{Aside}

\def\doubleunderline#1{\underline{\underline{#1}}}
\newcommand{\bb}[1]{\mathbb{#1}}
\newcommand{\scr}[1]{\mathscr{#1}}
\newcommand{\call}[1]{\mathcal{#1}}
\newcommand{\psheaf}{\text{\underline{Set}}^{\scr{C}^{\text{op}}}}
\newcommand{\und}[1]{\underline{\hspace{#1 cm}}}
\newcommand{\adj}[1]{\text{\textopencorner}{#1}\text{\textcorner}}
\newcommand{\comment}[1]{}
\newcommand{\lto}{\longrightarrow}
\newcommand{\rone}{(\operatorname{R}\bold{1})}
\newcommand{\lone}{(\operatorname{L}\bold{1})}
\newcommand{\rimp}{(\operatorname{R} \multimap)}
\newcommand{\limp}{(\operatorname{L} \multimap)}
\newcommand{\rtensor}{(\operatorname{R}\otimes)}
\newcommand{\ltensor}{(\operatorname{L}\otimes)}
\newcommand{\rtrue}{(\operatorname{R}\top)}
\newcommand{\rwith}{(\operatorname{R}\&)}
\newcommand{\lwithleft}{(\operatorname{L}\&)_{\operatorname{left}}}
\newcommand{\lwithright}{(\operatorname{L}\&)_{\operatorname{right}}}
\newcommand{\rplusleft}{(\operatorname{R}\oplus)_{\operatorname{left}}}
\newcommand{\rplusright}{(\operatorname{R}\oplus)_{\operatorname{right}}}
\newcommand{\lplus}{(\operatorname{L}\oplus)}
\newcommand{\prom}{(\operatorname{prom})}
\newcommand{\ctr}{(\operatorname{ctr})}
\newcommand{\der}{(\operatorname{der})}
\newcommand{\weak}{(\operatorname{weak})}
\newcommand{\exi}{(\operatorname{exists})}
\newcommand{\fa}{(\operatorname{for\text{ }all})}
\newcommand{\ex}{(\operatorname{ex})}
\newcommand{\cut}{(\operatorname{cut})}
\newcommand{\ax}{(\operatorname{ax})}
\newcommand{\negation}{\sim}
\newcommand{\true}{\top}
\newcommand{\false}{\bot}
\DeclareRobustCommand{\diamondtimes}{%
	\mathbin{\text{\rotatebox[origin=c]{45}{$\boxplus$}}}%
}
\newcommand{\tagarray}{\mbox{}\refstepcounter{equation}$(\theequation)$}
\newcommand{\startproof}[1]{
	\AxiomC{#1}
	\noLine
	\UnaryInfC{$\vdots$}
}
\newenvironment{scprooftree}[1]%
{\gdef\scalefactor{#1}\begin{center}\proofSkipAmount \leavevmode}%
	{\scalebox{\scalefactor}{\DisplayProof}\proofSkipAmount \end{center} }


\title{Quantum Error Correcting Codes and Matrix Factorisations}
\author{Will Troiani}
\date{\today}

\begin{document}

\maketitle

\section{Quantum Error Correcting Codes}

\begin{defn}\label{def:qubit}
	A \textbf{qubit} is a copy of the $\bb{C}$-Hilbert space $\bb{C}^2$.
		
	The \textbf{state} of a qubit $\bb{C}^2$ is a vector $\ket{\psi} \in \bb{C}^2$ of norm 1.
		
	A pair $(\bb{C}^2,\ket{\psi})$ consisting of a qubit $\bb{C}^2$ and a state $\ket{\psi} \in \bb{C}^2$ is a \textbf{prepared qubit} and we say $\bb{C}^2$ has been \textbf{prepared} to $\ket{\psi}$.
\end{defn}

\begin{defn}\label{def:composite_system}
	Let $\call{H}_1,\call{H}_2$ be two state spaces. The \textbf{composite state space} is $\call{H}_1 \otimes \call{H}_2$. A \textbf{state} of a composite system is a vector $\ket{\psi} \in \call{H}_1 \otimes \call{H}_2$ which can be written as a linear combination of pure tensors
	\begin{equation*}\label{eq:sum_pure_tensors}
			\alpha_{1}\ket{\psi_1} + \ldots + \alpha_n\ket{\psi_n} \in \call{H}_1 \otimes \call{H}_2
	\end{equation*}
	where the coefficients satisfy $|\alpha_1|^2 + \ldots + |\alpha_n|^2 = 1$.
\end{defn}

We define a measurement as a family of possible outcomes with assocaited probabilities. We allow for the possibility that measurements effect the state, and so measurements are operators upon the state space.

\begin{defn}\label{def:measurement}
	A \textbf{measurement} on a state space $\call{H}$ is a finite family of linear operators $\lbrace M_m: \call{H}\lto \call{H}\rbrace_{m \in \call{M}}$ satisfying the \textbf{completeness condition}.
	\begin{equation}\label{eq:completeness}
		\sum_{m \in \call{M}}M_m^{\dagger}M_m = I
	\end{equation}
	An element $m \in \call{M}$ is an \textbf{outcome}.
	
	The \textbf{resulting state} after measurement $\{ M_m \}_{m \in \call{M}}$ and outcome $m$ is:
	\begin{equation}
		\frac{M_m\ket{\psi}}{\sqrt{p(m)}}
	\end{equation}
\end{defn}

\begin{remark}
	Associated to every measurement and state vector $\ket{\psi}$ there is a value
	\begin{equation}
		p(m) := \bra{\psi}M^{\dagger}_mM_m\ket{\psi} = \lVert M_m \ket{\psi} \rVert^2
	\end{equation}
	It follows from \eqref{eq:completeness} that $p(m) \leq 1$ for all $m, \ket{\psi}$. We understand $p(m)$ as the probability of outcome $m$ on the measurement $\{ M_m \}_{m \in \call{M}}$.
\end{remark}

\begin{defn}
	A linear transformation $P$ is a \textbf{projector} if $P^2 = P$.
\end{defn}

\begin{defn}\label{def:QECC}
	A \textbf{quantum error correcting code (QECC)} is a pair $\call{Q} = (\call{H}, S)$ consisting of a state space $\call{H}$ along with a set of operators $S$ on $\call{H}$. The elements of $S$ are the \textbf{stabilisers}. The \textbf{codespace} $\call{H}^S$ of $\call{Q}$ is the maximal subspace of $\call{H}$ invariant under all the operators in $S$.
	\end{defn}

See Appendix \ref{sec:QECC_ex} for an example of a simple QECC.

\section{Matrix factorisations}

\subsection{Koszul Complex}

Recall that a $\bb{Z}_2$-graded ring $R$ comes equipt with a choice of isomorphism $R \cong R_0 \oplus R_1$ where $R_0, R_1$ are subgroups of $R$. The elements of $R_1$ are \textbf{odd}.

\begin{defn}
	Let $E$ be a $\bb{Z}_2$-graded ring and consider a set of odd elements $\theta_1, \ldots, \theta_n, \theta^\ast_1, \ldots, \theta^\ast_n \in E$. These \textbf{satisfy the canonical anticommutation relations} if the following hold for all $i,j = 1, \ldots, n$.
	\begin{itemize}
		\item $\theta_i \theta_j + \theta_j \theta_i = 0$
		\item $\theta_i^\ast \theta_j^\ast + \theta_j^\ast \theta_i^\ast = 0$
		\item $\theta_i \theta_j^\ast + \theta_j \theta_i^\ast = \delta_{ij}$
	\end{itemize}
\end{defn}

When $A \cong A_0 \oplus A_1,B \cong B_1 \oplus B_2$ are $\bb{Z}_2$-graded modules (over a graded ring $R$, say), a homomorphism $\varphi: A \lto B$ can be written as a matrix
\begin{equation}
\begin{pmatrix}
\varphi_{00} & \varphi_{01}\\
\varphi_{10} & \varphi_{11}
\end{pmatrix}
\end{equation}
where $\varphi_{00}(A_0) \subseteq B_0, \varphi_{01}(A_0) \subseteq B_1, \varphi_{10}(A_1) \subseteq B_0, \varphi_{11}(A_1) \subseteq B_1$. Writing this matrix as a sum yields respectively an even and odd component of $\varphi$
\begin{equation}
\begin{pmatrix}
\varphi_{00} & \varphi_{01}\\
\varphi_{10} & \varphi_{11}
\end{pmatrix}
=
\begin{pmatrix}
\varphi_{00} & 0\\
0 & \varphi_{11}
\end{pmatrix}
+
\begin{pmatrix}
0 & \varphi_{01}\\
\varphi_{10} & 0
\end{pmatrix}
\end{equation}
In this way, $\operatorname{Hom}(A, B)$ is also a $\bb{Z}_2$-graded module over $R$.

When $E$ is a $\bb{Z}_2$-graded ring of the form $E = \operatorname{End}(A)$ for some $\bb{Z}_2$-graded ring $A$, then $E$ admitting a set of odd elements satisfying the anticommutation relations is sufficient for $A$ to admit a Clifford algebra represenation \cite[Lemma 5.6.2]{Hitchcock}.

\begin{defn}
The Clifford Algebra $C_n$ is generated by elements $\mu_1, \ldots, \mu_n, \nu_1, \ldots, \nu_n$ subject to:
	\begin{equation*}
		[\mu_i, \mu_j] = -2\delta_{ij}\quad [\nu_i, \nu_j] = 2\delta_{ij}\quad [\mu_i, \nu_j] = 0
		\end{equation*}
	where $[\xi, \zeta] = \xi\zeta + \zeta\xi$ for $\xi, \zeta \in \{ \mu_1, \ldots, \mu_n, \nu_1, \ldots, \nu_n \}$.
\end{defn}

There is a $C_m$ action on $S_m$ and hence on $S_m \otimes_\bb{C} (Y \otimes X)$. This is induced by two canonical endomorphisms which exist on $S_m$. The \textbf{wedge} and \textbf{contraction} maps.
\begin{align*}
	\theta_i: \bigwedge^{d-1}(\bb{C}\theta_1 \oplus \ldots \oplus \bb{C}\theta_n) &\lto \bigwedge^d(\bb{C}\theta_1 \oplus \ldots \oplus \bb{C}\theta_n)\\
	\theta_{j_1} \wedge \ldots \wedge \theta_{j_{d-1}} &\longmapsto \theta_i \wedge \theta_{j_1} \wedge \ldots \wedge \theta_{j_{d-1}}
\end{align*}
and
\begin{align*}
	\theta_i^\ast: \bigwedge^d(\bb{C}\theta_1 \oplus \ldots \oplus \bb{C}\theta_n) &\lto \bigwedge^{d-1}(\bb{C}\theta_1 \oplus \ldots \oplus \bb{C}\theta_n)\\
	\theta_{j_1} \wedge \ldots \wedge \theta_{j_d} &\longmapsto \sum_{k = 1}^d(-1)^{k+1} \delta_{j_k = i}\theta_{j_1} \wedge \ldots \wedge \hat{\theta}_{j_k} \wedge \ldots \wedge \theta_{j_d}
\end{align*}

Given a free $R$-module $A = R\theta_1 \oplus \ldots \oplus R\theta_r$ of rank $r$, the multiplication and contraction operators satisfy the canonical anticommutation relations. Thus, $A$ admits a Clifford algebra representaiton.

\subsection{Matrix factorisations}\label{sec:MatrixFactorisations}
Let $k$ denote a ring.

Recall that if $A,B$ are $\bb{Z}$-graded $k$-modules then $\operatorname{Hom}(A,B)$ is also $\bb{Z}$-graded \cite{TroianiSCA}. We have a similar definition for $\bb{Z}_2$-graded modules.
\begin{defn}
	Let $A,B$ be $\bb{Z}_2$-graded $k$-modules. A homomorphism $f: A \lto B$ is \textbf{even} if $f(A_0) \subseteq B_0$ and $f(A_1) \subseteq A_1$. The homomorphism $f$ is \textbf{odd} if $f(A_0) \subseteq B_1$ and $f(A_1) \subseteq B_0$.
\end{defn}
\begin{remark}
	If $f: A \lto B$ is any module homomorphism then $f$ can be written as a matrix
	\begin{equation}
		\begin{pmatrix}
			f_{00} & f_{01}\\
			f_{10} & f_{11}
		\end{pmatrix}
	\end{equation}
	The morphism $f$ is thus even if $f_{01} = f_{10} = 0$ and is odd if $f_{00} = f_{11} = 0$. This also shows that the morphism $f$ can be written as the sum of an even and an odd component.
\end{remark}
\begin{defn}
	Let $f \in k$ be a non-zero divisor. A \textbf{linear factorisation of $f \in k$ over $k$} is a pair $(X,\partial_X)$ consisting of a $\bb{Z}_2$-graded $k$-module $X = X_0 \oplus X_1$ and an odd homomorphism $\partial_X: X \lto X$ satisfying
	\begin{equation}
		\partial_X^2 = f \cdot \operatorname{id}_{X}
	\end{equation}
	If $X$ is free then $(X, \partial_X)$ is a \textbf{matrix factorisation}.
\end{defn}
The theory of Matrix factorisations is motivated by the search for square roots to operators. As a toy example, multiplication by $x^2 - y^2$ in $\bb{C}[x,y]$ does not admit a square root, but it does if we allow matrix solutions.
\begin{equation}
	\begin{pmatrix}
		0 & x - y\\
		x+ y & 0
	\end{pmatrix}^2 = (x^2 - y^2)
	\begin{pmatrix}
		1 & 0\\
		0 & 1
	\end{pmatrix}
\end{equation}

A more serious example is given by the square root of the Laplacian operator, see the Introduction of \cite{Friedrich}.

Our interest in matrix factorsations comes from the fact that appropriate homotopy categories of matrix factorsations form the homcategories of the bicategory of Landau-Ginzburg models, which we anticipate to find within a model of multiplicative linear logic (proofs as hypersurface singularities).

\begin{defn}\label{def:morphism_mf}
	A \textbf{morphism of linear factorisations} $$\alpha: \Big(X = X_0 \oplus X_1, d_X = \begin{pmatrix}0 & p_X\\
		q_X & 0\end{pmatrix}\Big) \lto \Big(Y = Y_0 \oplus Y_1, d_Y = \begin{pmatrix}0 & p_Y\\
		q_Y & 0\end{pmatrix}\Big)$$
	of $f \in R$ is a pair of morphisms $\alpha_0: X_0 \lto Y_0, \alpha_1:X_1 \lto Y_1$ rendering the following diagram commutative.
	\begin{equation}
		\begin{tikzcd}
			X_0\arrow[r,"{p_X}"]\arrow[d,"{\alpha_0}"] & X_1\arrow[r, "{q_X}"]\arrow[d,"{\alpha_1}"] & X_0\arrow[d,"{\alpha_0}"]\\
			Y_0\arrow[r,"{p_Y}"] & Y_1\arrow[r, "{q_Y}"] & Y_0
		\end{tikzcd}
	\end{equation}
\end{defn}

Given a matrix factorisation $(X = X_0 \oplus X_1, d_X) = \begin{pmatrix} 0 & p_X\\ q_X & 0\end{pmatrix}$ there is a sequence
\begin{equation}
	\ldots \stackrel{p_X}{\lto} X_1 \stackrel{q_X}{\lto} X_0 \stackrel{p_X}{\lto} X_1 \stackrel{q_X}{\lto} \ldots
	\end{equation}
however, we note that in general $d_X^2 = f\cdot I \neq 0$ and so strictly speaking this is \emph{not} a chain complex.
\begin{defn}
	We use the notation of Definition \ref{def:morphism_mf}. Let $\beta = (\beta_0, \beta_1)$ be another morphism of linear factorisations $(X, d_X) \lto (Y, d_Y)$. The morphisms $\alpha, \beta$ are \textbf{homotopic} if there exists a pair of morphisms $h_0: X_0 \lto Y_1, h_1: X_1 \lto Y_0$ such that the following holds
	\begin{equation}
		\alpha_0 - \beta_0 = q_Yh_0 + h_1 p_X,\qquad  \alpha_1 - \beta_1 = h_0 q_X + p_Y h_1
		\end{equation}
	\end{defn}
% https://q.uiver.app/?q=WzAsMTIsWzAsMCwiXFxsZG90cyJdLFsxLDAsIlhfMCJdLFsyLDAsIlhfMSJdLFszLDAsIlhfMCJdLFs0LDAsIlhfMSJdLFs1LDAsIlxcbGRvdHMiXSxbMCwxLCJcXGxkb3RzIl0sWzEsMSwiWV8wIl0sWzIsMSwiWV8xIl0sWzMsMSwiWV8wIl0sWzQsMSwiWV8xIl0sWzUsMSwiXFxsZG90cyJdLFsxLDcsIlxcYWxwaGFfMCIsMix7Im9mZnNldCI6MX1dLFsxLDcsIlxcYmV0YV8wIiwwLHsib2Zmc2V0IjotMX1dLFsyLDgsIlxcYmV0YV8xIiwwLHsib2Zmc2V0IjotMX1dLFsyLDgsIlxcYWxwaGFfMSIsMix7Im9mZnNldCI6MX1dLFszLDksIlxcYmV0YV8wIiwwLHsib2Zmc2V0IjotMX1dLFszLDksIlxcYWxwaGFfMCIsMix7Im9mZnNldCI6MX1dLFs0LDEwLCJcXGJldGFfMSIsMCx7Im9mZnNldCI6LTF9XSxbNCwxMCwiXFxhbHBoYV8xIiwyLHsib2Zmc2V0IjoxfV0sWzEsMiwicF9YIl0sWzAsMSwicV9YIl0sWzIsMywicV9YIl0sWzMsNCwicF9YIl0sWzQsNSwicV9YIl0sWzYsNywicV9ZIiwyXSxbNyw4LCJwX1kiLDJdLFs4LDksInFfWSIsMl0sWzksMTAsInBfWSIsMl0sWzEwLDExLCJxX1kiLDJdLFsxLDYsImhfMCIsMl0sWzIsNywiaF8xIiwyXSxbMyw4LCJoXzAiLDJdLFs0LDksImhfMSIsMl0sWzUsMTAsImhfMCIsMl1d
\[\begin{tikzcd}[column sep = large, row sep = large]
	\ldots & {X_0} & {X_1} & {X_0} & {X_1} & \ldots \\
	\ldots & {Y_0} & {Y_1} & {Y_0} & {Y_1} & \ldots
	\arrow["{\alpha_0}"', shift right=1, from=1-2, to=2-2]
	\arrow["{\beta_0}", shift left=1, from=1-2, to=2-2]
	\arrow["{\beta_1}", shift left=1, from=1-3, to=2-3]
	\arrow["{\alpha_1}"', shift right=1, from=1-3, to=2-3]
	\arrow["{\beta_0}", shift left=1, from=1-4, to=2-4]
	\arrow["{\alpha_0}"', shift right=1, from=1-4, to=2-4]
	\arrow["{\beta_1}", shift left=1, from=1-5, to=2-5]
	\arrow["{\alpha_1}"', shift right=1, from=1-5, to=2-5]
	\arrow["{p_X}", from=1-2, to=1-3]
	\arrow["{q_X}", from=1-1, to=1-2]
	\arrow["{q_X}", from=1-3, to=1-4]
	\arrow["{p_X}", from=1-4, to=1-5]
	\arrow["{q_X}", from=1-5, to=1-6]
	\arrow["{q_Y}"', from=2-1, to=2-2]
	\arrow["{p_Y}"', from=2-2, to=2-3]
	\arrow["{q_Y}"', from=2-3, to=2-4]
	\arrow["{p_Y}"', from=2-4, to=2-5]
	\arrow["{q_Y}"', from=2-5, to=2-6]
	\arrow["{h_0}"', from=1-2, to=2-1]
	\arrow["{h_1}"', from=1-3, to=2-2]
	\arrow["{h_0}"', from=1-4, to=2-3]
	\arrow["{h_1}"', from=1-5, to=2-4]
	\arrow["{h_0}"', from=1-6, to=2-5]
\end{tikzcd}\]

The relation of homotopy defines an equivalence relation on the set of morphisms of linear factorisations.

\begin{defn}
	A linear transformation whose underlying $\bb{Z}_2$-graded $k$-module is free and of finite rank is a \textbf{matrix factorisation}. There is a category $\operatorname{hmf}(k[\underline{x}], f)$ where the objects are matrix factorisations of $f$ and the morphisms are homotopy equivalence classes of morphisms of matrix factorisations.
	\end{defn}

\begin{defn}
	If $(X,d_X)$ is a matrix factorisation then so is $(X[1], -d_X)$. If we denote this by $\Psi(X,d_X)$ then $\Psi: \operatorname{hmf}(k[\underline{x}], f) \lto \operatorname{hmf}(k[\underline{x}], f)$ is extends to an endofunctor which induces a supercategorical structure on $\operatorname{hmf}(k[\underline{x}], f)$ if we take $\xi: \Psi^2 \lto 1_{\operatorname{hmf}(k[\underline{x}], f)}$ to be the identity.
	\end{defn}

\begin{defn}
	Let $(X, \partial_X)$ be a linear factorisation of $f \in k$ over $k$ and $(Y, \partial_Y)$ a linear factorisation of $g \in k$ also over $k$. Then the \textbf{tensor product} of $(X, \partial_X)$ and $(Y, \partial_Y)$ consists of the following data:
	\begin{equation}
		X \otimes_k Y, \qquad \partial_{X \otimes_k Y} = d_X \otimes 1 + 1 \otimes d_Y
	\end{equation}
	where $X \otimes_k Y$ is the \emph{graded} tensor product, which satisfies the following for all $x_1, x_2 \in X, y_1, y_2 \in Y$.
	\begin{equation}
		(x_1 \otimes y_1)(x_2 \otimes y_2) = (-1)^{\operatorname{deg}(x_2)\operatorname{deg}(y_1)}(x_1 x_2 \otimes y_1 y_2)
	\end{equation}
\end{defn}

\begin{lemma}
	The tensor product $(X \otimes_k Y, \partial_{X \otimes_k Y})$ is a linear factorisation of $f + g$.
\end{lemma}
\begin{proof}
	See \cite{Hitchcock}[Page 35].
\end{proof}

In the special case where there exists $f \in k[\underline{x}], g \in k[\underline{y}], h \in k[\underline{z}]$ and $(X, \partial_X)$ is a linear factorisation of $f - g \in k[\underline{x}, \underline{y}]$ and $(Y, \partial_Y)$ is a linear factorisation of $g - h \in k[\underline{y}, \underline{z}]$ then we also have the \emph{cut} of $(X, \partial_X)$ and $(Y, \partial_Y)$.

\begin{defn}
	For each $y_1, \ldots, y_n \in \underline{y}$ let $\partial_{y_i}g$ denote the formal partial derivative of $g$ with respect to $y_i$. Denote by $J_g$ the following $k[\underline{y}]$-module.
	\begin{equation}
		J_g := k[\underline{y}]/(\partial_{y_1}g, \ldots, \partial_{y_n}g)
	\end{equation}
	
	The \textbf{cut} of $(X, \partial_X), (Y, \partial_Y)$ is the data of
	\begin{equation}
		X | Y := (X \otimes_{k[\underline{y}]} J_g \otimes_{k[\underline{y}]} Y), \qquad \partial_{X|Y} = d_X \otimes 1 \otimes 1 + 1 \otimes 1 \otimes d_Y
	\end{equation}
\end{defn}
\begin{lemma}
	The cut $X | Y$ is a matrix factorisation of $f - h$.
\end{lemma}
\begin{proof}
	\textcolor{red}{Check this.}
\end{proof}

We will use the following Lemma to indirectly talk about matrix factorisations using $\bb{Z}_2$-graded modules over Clifford algebras.

\begin{defn}\label{def:clifford_modules}
	Let $k[\underline{x}], k[\underline{y}]$ denote polynomial rings over variables $x_1, \ldots, x_n$ and $y_1, \ldots, y_n$ respectively. Let $U(\underline{x}) = \sum_{i = 1}^n x_i^2$.
	
	We let $C_{U}$ denote the $\bb{Z}_2$-graded $k$-algebra with multiplicative generators $\mu_1, \ldots, \mu_n, \nu_1, \ldots, \mu_n$ satisfying the relations
	\begin{equation}
		[\mu_i, \mu_j] = -2\delta_{ij}\qquad [\mu_i, \nu_j] = 0\qquad [\nu_i, \nu_j] = 2\delta_{ij}
	\end{equation}
	where $\delta_{ij} = 1\text{ if and only if }i = j\text{ and }\delta_{ij} = 0\text{ otherwise}$ is the Kronecker delta.
\end{defn}

\begin{lemma}\label{lem:Matrix_fact_from_Cliff}
	Let $\tilde{X}$ be a $\bb{Z}_2$-graded $C_{U}$-module which is free and finitely generated over $k$. Then $X := \tilde{X} \otimes_k k[\underline{x}, \underline{y}]$ coupled with the map
	\begin{equation}\label{eq:differential_Cliff_mod}
		\partial_X = \sum_{i = 1}^n \mu_i x_i + \sum_{j = 1}^n \nu_j y_j
	\end{equation}
	is a matrix factorisation of $U(\underline{y}) - U(\underline{x}) \in k[\underline{x}, \underline{y}]$.
\end{lemma}
\begin{proof}
	See \cite{Hitchcock}[Lemma 5.6.1].
\end{proof}
\begin{remark}
	The map \eqref{eq:differential_Cliff_mod} is odd because we consider $k[\underline{x}, \underline{y}]$ to admit the $\bb{Z}_2$-grading
	\begin{equation}
		k[\underline{x}, \underline{y}] \oplus 0
	\end{equation}
	That is, $k[\underline{x},\underline{y}]$ has $k[\underline{x},\underline{y}]$ entirely in degree $0$, and the zero module $0$ in degree $1$. For example, if $\underline{x},\underline{y}$ are both singleton sets $\underline{x} = \{ x \}, \underline{y} = \{ y \}$ then
	\begin{align*}
		\operatorname{deg}(\partial_X(x \otimes p)) &= \operatorname{deg}( \mu x \otimes x p+ \nu x \otimes y_j p )\\
		&= \operatorname{deg}(\mu x)\text{ }(= \operatorname{deg}(\nu x))\\
		&= \operatorname{deg}(x) + 1
	\end{align*}
\end{remark}

\begin{example}\label{ex:}
	Let $\underline{x}$ be a set of variables $\{ x_1, \ldots, x_n \}$ and $\sigma \in S_n$ a permutation on this set. Let $\tilde{X}$ denote the $\bb{Z}_2$-graded $k$-algebra
	\begin{equation}
		\tilde{X} := \bigwedge(k \theta_1 \oplus \ldots \oplus k \theta_n)
	\end{equation}
	which is a $C_{U}$-algebra (Definition \ref{def:clifford_modules}) with $C_{U}$-action induced by the following
	\begin{align*}
		\mu_i &= \theta_{\sigma^{-1}i} + \theta_{\sigma^{-1}i}^\ast\\
		\nu_i &= \theta_i - \theta_i^\ast
	\end{align*}
	Thus, by Lemma \ref{lem:Matrix_fact_from_Cliff} we obtain a matrix factorisation $(X = \tilde{X} \otimes k[\underline{x}, \underline{y}], \partial_{X})$.
	
	Now say we had another similar matrix factorisation; let $\underline{y} = \{ y_1, \ldots, y_m \}$ be another set of variables and let $\tau$ be a permutation on this set. Let $\tilde{Y}$ denote the $\bb{Z}_2$-graded $k$-algebra
	\begin{equation}
		\tilde{Y} := \bigwedge (k\psi_1 \oplus \ldots \oplus k\psi_m)
	\end{equation}
	This is a $C_{U}$-module with $C_{U}$-action induced by the following
	\begin{align*}
		\overline{\nu_i} &= \psi_{\tau^{-1}i} + \psi_{\tau^{-1}i}^\ast\\
		\omega_i &= \psi_i - \psi_i^\ast
	\end{align*}
	This induces a matrix factorisation $(Y = \tilde{Y} \otimes k[\underline{y}, \underline{z}], \partial_Y)$ of $U(\underline{y}) - U(\underline{z})$ where
	\begin{equation}
		\partial_Y := \sum_{i = 1}^n \overline{\nu}_i y_i + \sum_{i = 1}^n \omega_i z_i
	\end{equation}
	and $\underline{z} = \{ z_1, \ldots, z_n \}$ is another set of variables.
	
	We will first consider the cut $X | Y$. The sequence of partial derivatives $(\partial_{y_1}U(\underline{y}), \ldots, \partial_{y_n}U(\underline{y})) = (2y_1, \ldots, 2y_n)$ and so
	\begin{equation}
		J_{U(\underline{y})} = k[\underline{y}]/(y_1, \ldots, y_n) = k
	\end{equation}
	as a $k[\underline{y}]$-module with trivial $k[\underline{y}]$-action. We thus have
	\begin{equation}
		X | Y = X \otimes_{k[\underline{x}, \underline{y}]} k \otimes_{k[\underline{y}, \underline{z}]} Y,\qquad \partial_{X | Y} = d_X \otimes 1 \otimes 1 + 1 \otimes 1 \otimes d_Y
	\end{equation}
\end{example}

Say $R$ is a commutative ring. We will consider an element $f \in R$ of the following particular form: say we have $a_1, \ldots, a_n, b_1, \ldots, b_n \in R$ such that $f = \sum_{i = 1}^n a_i b_n$. The guiding example of such an element is when $R$ is a polynomial ring and $f$ is a quadratic form.

\begin{lemma}
	Suppose $M$ is a $\bb{Z}_2$-graded $R$-module with odd $R$-linear maps $\theta_i, \theta_i^\ast : M \lto M, i = 1, \ldots, n$ satisfying the canonical anticommutation relations. Then, setting $\delta_+ = \sum_{i = 1}^n a_i \theta_i$ and $\delta_- = \sum_{i = 1}^n b_i \theta_i^\ast$ we have that $(M, \delta_- + \delta_+)$ is a linear factorisation of $f$. 
\end{lemma}
\begin{proof}
	See \cite[Lemma 4.2.3]{Hitchcock}.
\end{proof}

Therefore, $(\bigwedge R^n, \delta_- + \delta_+)$ is a matrix factorisation of $f$. This is the \textbf{Koszul matrix factorisation of $f$}.

\appendix
\section{Quantum error correcting codes examples}
\label{sec:QECC_ex}
\begin{defn}[Bit flip correction]\label{alg:bit_flip_correction}
	Input: a received message $\ket{\psi}$,
	\begin{enumerate}
		\item perform the following projective measurements:
		\begin{equation}
			\bra{\psi} Z_1 Z_2 \ket{\psi}\text{ with resulting state }\ket{\psi'},
		\end{equation}
		followed by
		\begin{equation}
			\bra{\psi'} Z_2 Z_3 \ket{\psi'}
		\end{equation}
		let $(r_1,r_2)$ be the pair of results from these measurements.
		\item It will be shown that $r_1,r_2 \in \lbrace 1,-1\rbrace$, and the resulting state of the second measurement is $\ket{\psi}$.
		\item Now retrieve $\ket{\varphi}$ based on the values of $r_1,r_2$:
		\begin{itemize}
			\item if $(r_1, r_2) = (1,1)$, return $\ket{\psi}$,
			\item if $(r_1,r_2) = (-1,1)$, return $X_1 \ket{\psi}$,
			\item if $(r_1,r_2) = (1,-1)$, return $X_3 \ket{\psi}$,
			\item if $(r_1,r_2) = (-1,-1)$, return $X_2 \ket{\psi}$
		\end{itemize}
	\end{enumerate}
\end{defn}
We now prove correctness of Algorithm \ref{alg:bit_flip_correction}:
\begin{proof}
	It will be helpful to first notice:
	\begin{align*}
		Z_1Z_2\ket{000} &= \ket{000} & Z_1Z_2\ket{001} &= \ket{001}\\
		Z_1Z_2\ket{010} &= -\ket{010} & Z_1Z_2\ket{011} &= -\ket{011}\\
		Z_1Z_2\ket{100} &= -\ket{100} & Z_1Z_2\ket{101} &= -\ket{101}\\
		Z_1Z_2\ket{110} &= \ket{110} &Z_1Z_2 \ket{111} &= \ket{111}
	\end{align*}
	Let $\ket{\psi}:= a\ket{010} + b\ket{101}$ be a state, ie, an element of $\bb{H}^{\otimes 3}$. We perform the measurement $Z_1Z_2$ followed by $Z_2Z_3$:
	\begin{align*}
		\bra{\psi}Z_1Z_2\ket{\psi} &= (a\bra{010} + b\bra{101})Z_1Z_2(a\ket{010} + b\ket{101})\\
		&= (a\bra{010} + b\bra{101})(-a\ket{010} - b\ket{101})\\
		&= -a^2 - b^2 = -1
	\end{align*}
	and
	\begin{align*}
		\bra{\psi}Z_2Z_3\ket{\psi} &= (a\bra{010} + b\bra{101})Z_1Z_2(a\ket{010} + b\ket{101})\\
		&= (a\bra{010} + b\bra{101})(-a\ket{010} - b\ket{101})\\
		&= -a^2 - b^2 = -1
	\end{align*}
	We can infer from the fact that both of these came out as $-1$ that it was the second bit which was flipped, and so we can correct this. However, what is the impact of this measurement on the state? Again we calculate:
	\begin{align*}
		Z_1Z_2(a\ket{010} + b\ket{101}) &= Z_1(-a\ket{010} + b\ket{101})\\
		&= -a\ket{010} - b\ket{101}
	\end{align*}
	and
	\begin{align*}
		Z_2Z_3(-a\ket{010} - b\ket{101}) &= Z_2(-a\ket{010} + b\ket{101})\\
		&= a\ket{010} + b\ket{101}
	\end{align*}
	and so the measurements (in the end) did not impact our state.
\end{proof}

\begin{thebibliography}{99}
	\bibitem{Hitchcock} R. Hitchcock. \emph{Differentiation, Division and the Bicategory of Landau-Ginzburg Models}. Master's thesis. \url{http://therisingsea.org/notes/MScThesisRohanHitchcock.pdf}
	
	\bibitem{Borceux} F. Borceux. \emph{Handbook of Categorical Algebra 1: Basic Category Theory}. Vol. 50. Encyclopedia of Mathematics and its Applications. Cambridge University Press, 1994.
	
	\bibitem{TroianiSCA} W. Troiani. \emph{Secondary Commutative Algebra}.
	
	\bibitem{Friedrich} T. Friedrich, \emph{Dirac Operators in Riemannian Geometry} 20MathematicsSubjectClasification.Primary58Jx; Secondary 53C27, 53C28, 57R57, 58J05, 58J20, 58J50, 81R25.
	
	\bibitem{Cut_Operation} D. Murfet. \emph{The cut operation on matrix factorisations}
	
	\bibitem{quantum_computing} M. Nielsen, I. Chuang \emph{Quantum Computation and Quantum Information tenth anniversary} Cambridge University Press, 09 December 2010
		
	\bibitem{CommutativeAlgebra} W. Troiani \emph{Elementary Commutative Algebra} \url{https://williamtroiani.github.io/pdfs/CommutativeAlgebraWithLin.pdf}
		
	\bibitem{Baez} J. Baez. \url{https://math.ucr.edu/home/baez/quantum/node4.html}
	
	
	
	
	
	
	
	\end{thebibliography}





















\end{document}





















