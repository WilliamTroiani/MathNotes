\documentclass[12pt]{article}

\usepackage{amsthm}
\usepackage{amsmath}
\usepackage{amsfonts}
\usepackage{mathrsfs}
\usepackage{array}
\usepackage{amssymb}
\usepackage{units}
\usepackage{graphicx}
\usepackage{tikz-cd}
\usepackage{nicefrac}
\usepackage{hyperref}
\usepackage{bbm}
\usepackage{color}
\usepackage{tensor}
\usepackage{tipa}
\usepackage{bussproofs}
\usepackage{ stmaryrd }
\usepackage{ textcomp }
\usepackage{leftidx}
\usepackage{afterpage}
\usepackage{varwidth}
\usepackage{tasks}
\usepackage{ cmll }
\usepackage{quiver}
\usepackage{adjustbox}

\newcommand\blankpage{
	\null
	\thispagestyle{empty}
	\addtocounter{page}{-1}
	\newpage
}

\graphicspath{ {images/} }

\theoremstyle{plain}
\newtheorem{thm}{Theorem}[subsection] % reset theorem numbering for each chapter
\newtheorem{proposition}[thm]{Proposition}
\newtheorem{lemma}[thm]{Lemma}
\newtheorem{fact}[thm]{Fact}
\newtheorem{cor}[thm]{Corollary}

\theoremstyle{definition}
\newtheorem{defn}[thm]{Definition} % definition numbers are dependent on theorem numbers
\newtheorem{exmp}[thm]{Example} % same for example numbers
\newtheorem{notation}[thm]{Notation}
\newtheorem{remark}[thm]{Remark}
\newtheorem{condition}[thm]{Condition}
\newtheorem{question}[thm]{Question}
\newtheorem{construction}[thm]{Construction}
\newtheorem{exercise}[thm]{Exercise}
\newtheorem{example}[thm]{Example}
\newtheorem{aside}[thm]{Aside}

\def\doubleunderline#1{\underline{\underline{#1}}}
\newcommand{\bb}[1]{\mathbb{#1}}
\newcommand{\scr}[1]{\mathscr{#1}}
\newcommand{\call}[1]{\mathcal{#1}}
\newcommand{\psheaf}{\text{\underline{Set}}^{\scr{C}^{\text{op}}}}
\newcommand{\und}[1]{\underline{\hspace{#1 cm}}}
\newcommand{\adj}[1]{\text{\textopencorner}{#1}\text{\textcorner}}
\newcommand{\comment}[1]{}
\newcommand{\lto}{\longrightarrow}
\newcommand{\rone}{(\operatorname{R}\bold{1})}
\newcommand{\lone}{(\operatorname{L}\bold{1})}
\newcommand{\rimp}{(\operatorname{R} \multimap)}
\newcommand{\limp}{(\operatorname{L} \multimap)}
\newcommand{\rtensor}{(\operatorname{R}\otimes)}
\newcommand{\ltensor}{(\operatorname{L}\otimes)}
\newcommand{\rtrue}{(\operatorname{R}\top)}
\newcommand{\rwith}{(\operatorname{R}\&)}
\newcommand{\lwithleft}{(\operatorname{L}\&)_{\operatorname{left}}}
\newcommand{\lwithright}{(\operatorname{L}\&)_{\operatorname{right}}}
\newcommand{\rplusleft}{(\operatorname{R}\oplus)_{\operatorname{left}}}
\newcommand{\rplusright}{(\operatorname{R}\oplus)_{\operatorname{right}}}
\newcommand{\lplus}{(\operatorname{L}\oplus)}
\newcommand{\prom}{(\operatorname{prom})}
\newcommand{\ctr}{(\operatorname{ctr})}
\newcommand{\der}{(\operatorname{der})}
\newcommand{\weak}{(\operatorname{weak})}
\newcommand{\exi}{(\operatorname{exists})}
\newcommand{\fa}{(\operatorname{for\text{ }all})}
\newcommand{\ex}{(\operatorname{ex})}
\newcommand{\cut}{(\operatorname{cut})}
\newcommand{\ax}{(\operatorname{ax})}
\newcommand{\negation}{\sim}
\newcommand{\true}{\top}
\newcommand{\false}{\bot}
\DeclareRobustCommand{\diamondtimes}{%
	\mathbin{\text{\rotatebox[origin=c]{45}{$\boxplus$}}}%
}
\newcommand{\tagarray}{\mbox{}\refstepcounter{equation}$(\theequation)$}
\newcommand{\startproof}[1]{
	\AxiomC{#1}
	\noLine
	\UnaryInfC{$\vdots$}
}
\newenvironment{scprooftree}[1]%
{\gdef\scalefactor{#1}\begin{center}\proofSkipAmount \leavevmode}%
	{\scalebox{\scalefactor}{\DisplayProof}\proofSkipAmount \end{center} }

\usepackage[margin=1cm]{geometry}


\title{Linear Logic}
\author{Will Troiani}
\date{February 2021}

\begin{document}
	\maketitle
	\tableofcontents
	\section{Multiplicatives}
	\begin{defn}\label{def:formulas}
		There is an infinite set of \textbf{unoriented atoms} $X,Y,Z,...$ and an \textbf{oriented atom} (or \textbf{atomic proposition}) is a pair $(X,+)$ or $(X,-)$ where $X$ is an unoriented atom. The set of \textbf{pre-formulas} is defined as follows.
		\begin{itemize}
			\item Any atomic proposition is a preformula.
			\item If $A,B$ are pre-formulas then so are $A \otimes B$, $A \parr B$.
			\item If $A$ is a pre-formula then so is $\neg A$.
		\end{itemize}
		The set of \textbf{formulas} is the quotient of the set of pre-formulas by the equivalence relation $\sim$ generated by, for arbitrary formulas $A,B$ and unoriented atom $X$, the following.
		\begin{equation}\label{eq:negation}
			\neg (A \otimes B) \sim \neg B \parr \neg A,\qquad \neg (A \parr B) \sim \neg B \otimes \neg A,\qquad \neg (X, +) \sim (X, -),\qquad \neg (X,-) \sim (X,+)
		\end{equation}
	\end{defn}
	\begin{lemma}
		For all formulas $A$ we have $\neg \neg A = A$.
	\end{lemma}
	\begin{proof}
		We proceed by induction on the number $n$ given by the sum of the occurrences of $\otimes$ and $\parr$ in $A$.
		
		If $n = 0$ then $A = (X,+)$ or $A = (X,-)$. In either situation the fact that $\neg \neg A = A$ follows from \eqref{eq:negation}.
		
		Now say $n > 0$ and the result holds for all $k < n$. We either have $A = A_1 \otimes A_2$ or $A = A_1 \parr A_2$, each case is proved in a similar way, we show the deatils for when $A = A_1 \otimes A_2$.
		\begin{equation}
			\neg \neg A = \neg \neg (A_1 \otimes A_2) = \neg (\neg A_2 \parr \neg A_1) = \neg \neg A_1 \otimes \neg \neg A_2
		\end{equation}
		By the inductive hypothesis we ahve that $\neg \neg A_1 = A_1$ and $\neg \neg A_2 = A_2$. It thus follows that $\neg \neg A = A$.
	\end{proof}
	\begin{defn}
		A finite sequence of formulas is a \textbf{sequent} and we write $\vdash  A_1,..., A_n$ for the sequent $(A_1,...,A_n)$.
	\end{defn}
	\begin{defn}\label{def:mult_lin_log_ded_rule}
		A \textbf{multiplicative, linear logic deduction rule} (or simply \textbf{deduction rule}) results from one of the schemata below by a substitution of the following kind: replace $A,B$ by arbitrary formulas, and $\Gamma, \Gamma', \Delta, \Delta'$ by arbitrary (possibly empty) sequences of formulas separated by commas:
		\begin{itemize}
			\item the \textbf{identity group}:
			\begin{itemize}
				\item \textbf{Axiom}
				\begin{prooftree}
					\AxiomC{}
					\RightLabel{$\ax$}
					\UnaryInfC{$\vdash A, \sim A$}
				\end{prooftree}
				\item \textbf{Cut}:
				\begin{prooftree}
					\AxiomC{$\vdash \Gamma, A, \Gamma'$}
					\AxiomC{$\vdash \Delta, \negation A, \Delta'$}
					\RightLabel{$\cut$}
					\BinaryInfC{$\vdash \Gamma, \Gamma', \Delta, \Delta'$}
				\end{prooftree}
			\end{itemize}
			\item the \textbf{multiplicative rules}
			\begin{itemize}
				\item \textbf{Times}:
				\begin{prooftree}
					\AxiomC{$\vdash \Gamma, A, \Gamma'$}
					\AxiomC{$\vdash \Delta, B, \Delta'$}
					\RightLabel{$\otimes$}
					\BinaryInfC{$\vdash \Gamma, \Gamma', A \otimes B, \Delta, \Delta'$}
				\end{prooftree}
				\item \textbf{Par}
				\begin{prooftree}
					\AxiomC{$\vdash \Gamma, A, B, \Gamma'$}
					\RightLabel{$\parr$}
					\UnaryInfC{$\vdash \Gamma, A \parr B, \Gamma'$}
				\end{prooftree}
			\end{itemize}
			\item the \textbf{structural rule}:
			\begin{itemize}
				\item \textbf{Exchange}
				\begin{prooftree}
					\AxiomC{$\vdash \Gamma, A, B, \Gamma'$}
					\RightLabel{$\ex$}
					\UnaryInfC{$\vdash \Gamma, B, A, \Gamma'$}
				\end{prooftree}
			\end{itemize}
		\end{itemize}
	\end{defn}
	\begin{defn}\label{def:proof}
		A \textbf{proof in MLL} is a finite, rooted, planar, tree where each edge is labelled by a sequent and each node except for the root is labelled by a valid deduction rule. If the edge connected to the root is labelled by the sequent $\vdash \Gamma$ then we call the proof a \textbf{proof of $\vdash \Gamma$} and in such a situation, $\Gamma$ is the \textbf{conclusion} of $\pi$.
	\end{defn}
	A proof in MLL is a concept inspired by the proofs that mathematicians write in, for example, algebraic geometry. One of the reasons why it is interesting to consider a formal notion of a proof so that we can analyse the true \emph{essence} of a proof. Many arbitrary choices are made when one writes a proof on paper, for instance, what order two arguments are written down when one does not depend on the other. Hence, a natural idea is to define an equivalence relation on the set of a proofs in MLL which identify proofs which differ only in these insignificant ways. However, there is an even better idea. We begin with the question: ``what is good notation"?
	
	Let us begin with an example of \emph{bad} notation. Say we considered proofs in MLL up to equivalence, as suggested by the previous paragraph. For clarity, just in this paragraph and the next we will refer to proofs in MLL as \emph{preproofs}, and equivalence classes of such things as \emph{proofs}. The only reasonable way to denote a proof would be to denote it \emph{indirectly} via a preproof. Hence, this notation \emph{encompasses more than the pure essence of a proof}, as the arbitrary choices belonging to that particular preproof are also included in the notation. The result: a notation which obfuscates the mathematical structure of proofs.
	
	Hence, the desiderata for what constitutes ``good notation" at least includes the following condition.
	\begin{equation}
		\text{There should be exactly one way of writing the object in question.}
	\end{equation}
	This is what proof structures (Definition \ref{def:proof_structures}) do, and indeed, we will see that this notation leads to many insights (the Sequentialisation Theorem \ref{thm:sequentialisation}, Geometry of Interaction \ref{thm:geometry_of_interaction_zero}, etc) which capture some structure of proofs, without reference to the structure of a choice of representing preproof.
	
	First, we recall the definition of a directed multigraph.
	\begin{defn}
		A \textbf{directed multigraph} is a triple $(V,E,\varphi)$ where:
		\begin{itemize}
			\item $V$ is a set of \textbf{vertices}, or \textbf{nodes}.
			\item $E$ is a set of \textbf{edges}, or \textbf{lines}.
			\item $r: E \lto \lbrace (x,y) \mid x, y \in V\rbrace$ is a function from the set of edges to the set of ordered pairs of vertices.
			\end{itemize}
		For all edges $e \in E$, the first element of $r(e)$ is the \textbf{source} and the second element is the \textbf{target}.
		\end{defn}
	
	\begin{defn}\label{def:proof_structures}
		A \textbf{proof structure} is a directed multigraph with edges labelled with formulas (Definition \ref{def:formulas}) and with nodes labelled with an element of the following set: $\lbrace \ax, \cut, \otimes, \parr, \operatorname{c} \rbrace$. A proof structure may not admit any loops (however it may admit cycles). The incoming edges of a node are called its \textbf{premisses}, the outgoing edges are its \textbf{conclusions}. Proof structures are required to adhere to the following conditions.
		\begin{itemize}
			\item Each node labelled $\ax$ has exactly two conclusions and no premisse, the conclusions are labelled $A$ and $\neg A$ for some $A$.
			\item Each node labelled $\cut$ has exactly two premisses and no conclusion, where the premisses are labelled $A$ and $\neg A$ for some $A$. 
			\item Each node labelled $\otimes$ has exactly two premisses and one conclusion. These two premisses are ordered. The smallest one is called the \emph{left} premise of the node, the biggest one is called the \emph{right} premise. The left premise is labelled $A$, the right premise is labelled $B$ and the conclusion is labelled $A \otimes B$, for some $A,B$.
			\item Each node labelled $\parr$ has exactly two ordered premisses and one conclusion. The left premise is labelled $A$, the right premise is labelled $B$ and the conclusion sis labelled $A \parr B$, for some $A,B$.
			\item Each node labelled $\operatorname{c}$ has exactly one premise and no conclusion. Such a premise of a node labelled $\operatorname{c}$ is called the \textbf{conclusion} of the proof structure.
		\end{itemize}
		Let $\pi$ be a proof structure. An \textbf{axiom link} of $\pi$ is a subgraph consisting of a node labelled $\ax$ along with its conclusions labelled $A, \neg A$ for some formula $A$. A \textbf{tensor link} of $\pi$ consists of a node labelled $\otimes$ along with its premises and conclusion labelled respectively $A,B, A \otimes B$ for some pair of formulas $(A,B)$. A \textbf{par link} consists of a node labelled $\parr$ along with its premises and conclusion labelled respectively $A,B, A \parr B$ for some pair of formulas $(A,B)$. A $\cut$ link consists of a node labelled $\cut$ along with its premises labelled $A, \neg A$ for some formula $A$.
	\end{defn}
	\begin{remark}
		Our reference for the definition of a proof structure is \cite{laurent}, see there for more details.
	\end{remark}
	\begin{defn}\label{def:occurrences_labels}
		An \textbf{occurrence of a formula} $A$ in a proof structure $\pi$ is an edge $e$ labelled by $A$.
	\end{defn}
	Loosely speaking, logic is about determining correct arguments. That is, from the space of arguments (either correct or incorrect), logic determines whether an argument $A$ lies in the subspace of \emph{correct arguments} or not. In the current context, we take the set of proof structures to be the space of proofs, both correct and incorrect, and we take the subset of so called \emph{proof nets} to be the subspace of correct arguments. Proof nets are the proof structures which lie in the image of a translation map (Definition \ref{def:translation_map} below) between sequent style proofs and proof structures, we now explain this sequent style logical system.
	
	\begin{defn}\label{def:int_mult_ded_rule} 
		The set of \textbf{intuitionistic formulas} is defined as follows.
		\begin{itemize}
			\item Any atomic proposition (Definition \ref{def:formulas}) is an intuitionistic formula.
			\item If $A,B$ are formulas then so are $A \otimes B, A \multimap B$.
		\end{itemize}
		Let $\call{P}^n$ be the set of all length $n$ sequences of labelled intuitionistic variables with $\call{P}^0 := \lbrace \varnothing \rbrace$, and $\call{P} := \cup_{n = 0}^\infty \call{P}^n$. A \textbf{sequent} is a pair $(\Gamma,A)$ where $\Gamma \in \call{P}$ and $A \in I\Psi$, written $\Gamma \vdash A$. We call $\Gamma$ the \textbf{antecedent} and $A$ the \textbf{succedent} of the sequent. Given $\Gamma$ and a labelled intuitionistic formula $A$ we write $\Gamma, A$ for the element of $\call{P}$ given by appending $A$ to the end of $\Gamma$. We write $\vdash A$ for $\varnothing \vdash A$.
		
		An \textbf{intuitionistic, multiplicative deduction rule} (or simply \textbf{deduction rule}) results from one of the schemata below by a substitution of the following kind: replace $A,B,C$ by arbitrary labelled intuitionistic formulas, and $\Gamma,\Delta, \Theta$ by arbitrary (possibly empty) sequences of labelled intuitionistic formulas separated by commas:
		\begin{itemize}
			\item The \textbf{identity group}:
			\begin{center}
				\begin{tabular}{ >{\centering}m{2cm} >{\centering}m{7cm} >{\centering}m{0.5cm} }
					\textbf{Axiom}
					&
					\begin{prooftree}
						\AxiomC{}
						\RightLabel{$({\operatorname{ax}})$}
						\UnaryInfC{$A \vdash A$}
					\end{prooftree}
					&
					\tagarray{\label{LL:ax}}
				\end{tabular}
			\end{center}
			
			\begin{center}
				\begin{tabular}{ >{\centering}m{2cm} >{\centering}m{7cm} >{\centering}m{0.5cm} }
					\textbf{Cut}
					&
					\begin{prooftree}
						\AxiomC{$\Gamma \vdash A$}
						\AxiomC{$\Delta, A,\Theta \vdash B$}
						\RightLabel{$({\operatorname{cut}})$}
						\BinaryInfC{$\Gamma, \Delta, \Theta \vdash B$}
					\end{prooftree}
					&
					\tagarray{\label{LL:cut}}
				\end{tabular}
			\end{center}
			\item The \textbf{logical rules}:
			\begin{center}
				\begin{tabular}{ >{\centering}m{3cm} >{\centering}m{5cm} >{\centering}m{5cm} >{\centering}m{0.5cm} }
					\textbf{Left/right times}
					&
					\AxiomC{$\Gamma, A, B, \Gamma' \vdash C$}
					\RightLabel{$\ltensor$}
					\UnaryInfC{$\Gamma, A \otimes B, \Gamma' \vdash C$}
					\DisplayProof
					&
					\AxiomC{$\Gamma \vdash A$}
					\AxiomC{$\Delta\vdash B$}
					\RightLabel{$\rtensor$}
					\BinaryInfC{$\Gamma, \Delta \vdash A \otimes B$}
					\DisplayProof
					&
					\tagarray{\label{LL:times}}
				\end{tabular}
			\end{center}
			
			\begin{center}
				\begin{tabular}{ >{\centering}m{3cm} >{\centering}m{5cm} >{\centering}m{7cm} >{\centering}m{0.5cm} }
					\textbf{Right/left implication}
					&
					\AxiomC{$\Gamma, A, \Gamma' \vdash B$}
					\RightLabel{$\rimp$}
					\UnaryInfC{$\Gamma, \Gamma' \vdash A \multimap B$}
					\DisplayProof
					&
					\AxiomC{$\Gamma \vdash A$}
					\AxiomC{$\Delta, B, \Delta' \vdash C$}
					\RightLabel{$\limp$}
					\BinaryInfC{$A \multimap B, \Gamma, \Delta \vdash C$}
					\DisplayProof
					&
					\tagarray{\label{LL:times}}
				\end{tabular}
			\end{center}
			
			\item The \textbf{structural rule}
			
			\begin{center}
				\begin{tabular}{ >{\centering}m{2cm} >{\centering}m{7cm} >{\centering}m{0.5cm} }
					\textbf{Exchange}
					&
					\begin{prooftree}
						\AxiomC{$\Gamma, A, B, \Gamma' \vdash C$}
						\RightLabel{$({\operatorname{ex}})$}
						\UnaryInfC{$\Gamma, B, A, \Gamma' \vdash C$}
					\end{prooftree}
					&
					\tagarray{\label{LL:exchange}}
				\end{tabular}
			\end{center}
		\end{itemize}
		A \textbf{proof in IMLL} is defined the same way as in Definition \ref{def:proof}.
	\end{defn}
	\begin{defn}\label{def:translation_map}
		Let $\Sigma$ denote the set of multiplicative, linear logic proofs and $\operatorname{MPS}$ the set of multiplicative proof structures. We let $T: \Sigma \lto \operatorname{MPS}$ denote the function defined inductively by associating to each deduction rule of Definition \ref{def:mult_lin_log_ded_rule} a multiplicative proof structure. More precisely, we simultaneously inductively prove that if $\pi$ has height $n$ and is constructed from either one proof $\pi'$ with height less than $n$ or from two proofs $\pi_1,\pi_2$ each with height less than $n$, then $T(\pi'), T(\pi_1), T(\pi_2)$ have conclusions corresponding to the conclusions of $\pi',\pi_1,\pi_2$, and we use this fact to inductively define $T(\pi)$ which in turn has conclusions corresponding to those of $\pi$.
		
		Given a proof $\pi$, the following notation:
		\begin{equation}
			\begin{tikzcd}
				T(\pi)\arrow[d,dash]\\
				A
			\end{tikzcd}
		\end{equation}
		means the translation $T(\pi)$, which admits a conclusion $A$, with the conclusion node $\operatorname{c}$ removed.
		\begin{center}
			\begin{tabular}{ >{\centering}m{2cm} >{\centering}m{5cm} >{\centering}m{0.5cm} >{\centering}m{5cm}}
				\text{Axiom} & \begin{prooftree}
					\AxiomC{}
					\RightLabel{$\ax$}
					\UnaryInfC{$\vdash A, \sim A$}
				\end{prooftree} & $\stackrel{T}{\lto}$ &
				% https://q.uiver.app/?q=WzAsNSxbMSwwLCJcXGF4Il0sWzAsMSwiQSJdLFsyLDEsIlxcbmVnIEEiXSxbMCwyLCJcXG9wZXJhdG9ybmFtZXtjfSJdLFsyLDIsIlxcb3BlcmF0b3JuYW1le2N9Il0sWzAsMSwiIiwwLHsiY3VydmUiOjIsInN0eWxlIjp7ImhlYWQiOnsibmFtZSI6Im5vbmUifX19XSxbMCwyLCIiLDIseyJjdXJ2ZSI6LTIsInN0eWxlIjp7ImhlYWQiOnsibmFtZSI6Im5vbmUifX19XSxbMSwzXSxbMiw0XV0=
				$\begin{tikzcd}[column sep = small, row sep = small]
					& \ax \\
					A && {\neg A} \\
					{\operatorname{c}} && {\operatorname{c}}
					\arrow[curve={height=12pt}, no head, from=1-2, to=2-1]
					\arrow[curve={height=-12pt}, no head, from=1-2, to=2-3]
					\arrow[from=2-1, to=3-1]
					\arrow[from=2-3, to=3-3]
				\end{tikzcd}$
			\end{tabular}
		\end{center}
		
		\begin{center}
			\begin{tabular}{ >{\centering}m{2cm} >{\centering}m{7cm} >{\centering}m{0.5cm} >{\centering}m{7cm}}
				\text{Cut} &
				\begin{prooftree}
					\startproof{$\pi_1$}
					\UnaryInfC{$\vdash \Gamma, A, \Gamma'$}
					\startproof{$\pi_2$}
					\UnaryInfC{$\vdash \Delta, \negation A, \Delta'$}
					\RightLabel{$\cut$}
					\BinaryInfC{$\vdash \Gamma, \Gamma', \Delta, \Delta'$}
				\end{prooftree} & $\stackrel{T}{\lto}$ &
				% https://q.uiver.app/?q=WzAsNSxbMCwwLCJUKFxccGlfMSkiXSxbMiwwLCJUKFxccGlfMikiXSxbMCwxLCJBIl0sWzIsMSwiXFxuZWcgQSJdLFsxLDIsIlxcY3V0Il0sWzIsNCwiIiwwLHsiY3VydmUiOjJ9XSxbMyw0LCIiLDIseyJjdXJ2ZSI6LTJ9XSxbMCwyLCIiLDAseyJzdHlsZSI6eyJoZWFkIjp7Im5hbWUiOiJub25lIn19fV0sWzEsMywiIiwyLHsic3R5bGUiOnsiaGVhZCI6eyJuYW1lIjoibm9uZSJ9fX1dXQ==
				$\begin{tikzcd}[column sep = small, row sep = small]
					{T(\pi_1)} && {T(\pi_2)} \\
					A && {\neg A} \\
					& \cut
					\arrow[curve={height=12pt}, from=2-1, to=3-2]
					\arrow[curve={height=-12pt}, from=2-3, to=3-2]
					\arrow[no head, from=1-1, to=2-1]
					\arrow[no head, from=1-3, to=2-3]
				\end{tikzcd}$
			\end{tabular}
		\end{center}
		
		\begin{center}
			\begin{tabular}{ >{\centering}m{2cm} >{\centering}m{7cm} >{\centering}m{0.5cm} >{\centering}m{7cm}}
				\text{Times} &
				\begin{prooftree}
					\startproof{$\pi_1$}
					\UnaryInfC{$\vdash \Gamma, A, \Gamma'$}
					\startproof{$\pi_2$}
					\UnaryInfC{$\vdash \Delta, B, \Delta'$}
					\RightLabel{$\otimes$}
					\BinaryInfC{$\vdash \Gamma, \Gamma', A \otimes B, \Delta, \Delta'$}
				\end{prooftree} & $\stackrel{T}{\lto}$ &
				% https://q.uiver.app/?q=WzAsNyxbMiwxLCJcXG5lZyBBIl0sWzAsMSwiQSJdLFsxLDIsIlxcb3RpbWVzIl0sWzEsNCwiXFxvcGVyYXRvcm5hbWV7Y30iXSxbMSwzLCJBIFxcb3RpbWVzIFxcbmVnIEEiXSxbMiwwLCJUKFxccGlfMikiXSxbMCwwLCJUKFxccGlfMSkiXSxbMCwyLCIiLDAseyJjdXJ2ZSI6LTJ9XSxbMSwyLCIiLDEseyJjdXJ2ZSI6Mn1dLFsyLDQsIiIsMSx7InN0eWxlIjp7ImhlYWQiOnsibmFtZSI6Im5vbmUifX19XSxbNCwzXSxbNiwxLCIiLDEseyJzdHlsZSI6eyJoZWFkIjp7Im5hbWUiOiJub25lIn19fV0sWzUsMCwiIiwwLHsic3R5bGUiOnsiaGVhZCI6eyJuYW1lIjoibm9uZSJ9fX1dXQ==
				$\begin{tikzcd}[column sep = small, row sep = small]
					{T(\pi_1)} && {T(\pi_2)} \\
					A && {B} \\
					& \otimes \\
					& {A \otimes B} \\
					& {\operatorname{c}}
					\arrow[curve={height=-12pt}, from=2-3, to=3-2]
					\arrow[curve={height=12pt}, from=2-1, to=3-2]
					\arrow[no head, from=3-2, to=4-2]
					\arrow[from=4-2, to=5-2]
					\arrow[no head, from=1-1, to=2-1]
					\arrow[no head, from=1-3, to=2-3]
				\end{tikzcd}$
			\end{tabular}
		\end{center}
		
		\begin{center}
			\begin{tabular}{ >{\centering}m{2cm} >{\centering}m{7cm} >{\centering}m{0.5cm} >{\centering}m{7cm}}
				\text{Par} &
				\begin{prooftree}
					\startproof{$\pi$}
					\UnaryInfC{$\vdash \Gamma, A, B, \Gamma'$}
					\RightLabel{$\parr$}
					\UnaryInfC{$\vdash \Gamma, A \parr B, \Gamma'$}
				\end{prooftree} & $\stackrel{T}{\lto}$ &
				% https://q.uiver.app/?q=WzAsNixbMiwxLCJcXG5lZyBBIl0sWzAsMSwiQSJdLFsxLDIsIlxccGFyciJdLFsxLDQsIlxcb3BlcmF0b3JuYW1le2N9Il0sWzEsMywiQSBcXHBhcnIgXFxuZWcgQSJdLFsxLDAsIlQoXFxwaSkiXSxbMCwyLCIiLDAseyJjdXJ2ZSI6LTJ9XSxbMSwyLCIiLDEseyJjdXJ2ZSI6Mn1dLFsyLDQsIiIsMSx7InN0eWxlIjp7ImhlYWQiOnsibmFtZSI6Im5vbmUifX19XSxbNCwzXSxbNSwxLCIiLDEseyJjdXJ2ZSI6Miwic3R5bGUiOnsiaGVhZCI6eyJuYW1lIjoibm9uZSJ9fX1dLFs1LDAsIiIsMSx7ImN1cnZlIjotMiwic3R5bGUiOnsiaGVhZCI6eyJuYW1lIjoibm9uZSJ9fX1dXQ==
				$\begin{tikzcd}[column sep = small, row sep = small]
					& {T(\pi)} \\
					A && {B} \\
					& \parr \\
					& {A \parr B} \\
					& {\operatorname{c}}
					\arrow[curve={height=-12pt}, from=2-3, to=3-2]
					\arrow[curve={height=12pt}, from=2-1, to=3-2]
					\arrow[no head, from=3-2, to=4-2]
					\arrow[from=4-2, to=5-2]
					\arrow[curve={height=12pt}, no head, from=1-2, to=2-1]
					\arrow[curve={height=-12pt}, no head, from=1-2, to=2-3]
				\end{tikzcd}$
			\end{tabular}
		\end{center}
		
		\begin{center}
			\begin{tabular}{ >{\centering}m{2cm} >{\centering}m{7cm} >{\centering}m{0.5cm} >{\centering}m{7cm}}
				\text{Exchange} &
				\begin{prooftree}
					\startproof{$\pi$}
					\UnaryInfC{$\vdash \Gamma, A, B, \Gamma'$}
					\RightLabel{$\ex$}
					\UnaryInfC{$\vdash \Gamma, B, A, \Gamma'$}
				\end{prooftree} & $\stackrel{T}{\lto}$ &
				\begin{tikzcd}[column sep = small]
					T(\pi)
				\end{tikzcd}
			\end{tabular}
		\end{center}
		A \textbf{multiplicative proof net} (or simply \textbf{proof net}) is a multiplicative proof structure which lies in the image of $T$.
	\end{defn}
	\begin{defn}
		Let $\Pi$ denote the set of intuitionistic, multiplicative, linear logic proofs. Then again, there is a translation
		\begin{equation}
			S: \Pi \lto \operatorname{MPS}
		\end{equation}
		defined inductively:
		\begin{center}
			\begin{tabular}{ >{\centering}m{2cm} >{\centering}m{5cm} >{\centering}m{0.5cm} >{\centering}m{5cm}}
				\text{Axiom} & \begin{prooftree}
					\AxiomC{}
					\RightLabel{$({\operatorname{ax}})$}
					\UnaryInfC{$A \vdash A$}
				\end{prooftree} & $\stackrel{S}{\lto}$ &
				% https://q.uiver.app/?q=WzAsNSxbMSwwLCJcXGF4Il0sWzAsMSwiQSJdLFsyLDEsIlxcbmVnIEEiXSxbMCwyLCJcXG9wZXJhdG9ybmFtZXtjfSJdLFsyLDIsIlxcb3BlcmF0b3JuYW1le2N9Il0sWzAsMSwiIiwwLHsiY3VydmUiOjIsInN0eWxlIjp7ImhlYWQiOnsibmFtZSI6Im5vbmUifX19XSxbMCwyLCIiLDIseyJjdXJ2ZSI6LTIsInN0eWxlIjp7ImhlYWQiOnsibmFtZSI6Im5vbmUifX19XSxbMSwzXSxbMiw0XV0=
				$\begin{tikzcd}[column sep = small, row sep = small]
					& \ax \\
					A && {\neg A} \\
					{\operatorname{c}} && {\operatorname{c}}
					\arrow[curve={height=12pt}, no head, from=1-2, to=2-1]
					\arrow[curve={height=-12pt}, no head, from=1-2, to=2-3]
					\arrow[from=2-1, to=3-1]
					\arrow[from=2-3, to=3-3]
				\end{tikzcd}$
			\end{tabular}
		\end{center}
		
		\begin{center}
			\begin{tabular}{ >{\centering}m{2cm} >{\centering}m{7cm} >{\centering}m{0.5cm} >{\centering}m{7cm}}
				\text{Cut} &
				\begin{prooftree}
					\startproof{$\pi_1$}
					\UnaryInfC{$\Gamma \vdash A$}
					\startproof{$\pi_2$}
					\UnaryInfC{$\Delta, A,\Theta \vdash B$}
					\RightLabel{$({\operatorname{cut}})$}
					\BinaryInfC{$\Gamma, \Delta, \Theta \vdash B$}
				\end{prooftree} & $\stackrel{S}{\lto}$ &
				% https://q.uiver.app/?q=WzAsNSxbMCwwLCJUKFxccGlfMSkiXSxbMiwwLCJUKFxccGlfMikiXSxbMCwxLCJBIl0sWzIsMSwiXFxuZWcgQSJdLFsxLDIsIlxcY3V0Il0sWzIsNCwiIiwwLHsiY3VydmUiOjJ9XSxbMyw0LCIiLDIseyJjdXJ2ZSI6LTJ9XSxbMCwyLCIiLDAseyJzdHlsZSI6eyJoZWFkIjp7Im5hbWUiOiJub25lIn19fV0sWzEsMywiIiwyLHsic3R5bGUiOnsiaGVhZCI6eyJuYW1lIjoibm9uZSJ9fX1dXQ==
				$\begin{tikzcd}[column sep = small, row sep = small]
					{S(\pi_1)} && {S(\pi_2)} \\
					A && {\neg A} \\
					& \cut
					\arrow[curve={height=12pt}, from=2-1, to=3-2]
					\arrow[curve={height=-12pt}, from=2-3, to=3-2]
					\arrow[no head, from=1-1, to=2-1]
					\arrow[no head, from=1-3, to=2-3]
				\end{tikzcd}$
			\end{tabular}
		\end{center}
		
		\begin{center}
			\begin{tabular}{ >{\centering}m{2cm} >{\centering}m{7cm} >{\centering}m{0.5cm} >{\centering}m{7cm}}
				\text{Left times} &
				\startproof{$\pi$}
				\UnaryInfC{$\Gamma, A, B, \Gamma' \vdash C$}
				\RightLabel{$\ltensor$}
				\UnaryInfC{$\Gamma, A \otimes B, \Gamma' \vdash C$}
				\DisplayProof & $\stackrel{S}{\lto}$ &
				$\begin{tikzcd}[column sep = small, row sep = small]
					& {S(\pi)} \\
					\neg A && {\neg B} \\
					& \parr \\
					& {\neg A \parr \neg B} \\
					& {\operatorname{c}}
					\arrow[curve={height=-12pt}, from=2-3, to=3-2]
					\arrow[curve={height=12pt}, from=2-1, to=3-2]
					\arrow[no head, from=3-2, to=4-2]
					\arrow[from=4-2, to=5-2]
					\arrow[curve={height=12pt}, no head, from=1-2, to=2-1]
					\arrow[curve={height=-12pt}, no head, from=1-2, to=2-3]
				\end{tikzcd}$
			\end{tabular}
		\end{center}
		
		\begin{center}
			\begin{tabular}{ >{\centering}m{2cm} >{\centering}m{7cm} >{\centering}m{0.5cm} >{\centering}m{7cm}}
				\text{Right times} &
				\startproof{$\pi_1$}
				\UnaryInfC{$\Gamma \vdash A$}
				\startproof{$\pi_2$}
				\UnaryInfC{$\Delta\vdash B$}
				\RightLabel{$\rtensor$}
				\BinaryInfC{$\Gamma, \Delta \vdash A \otimes B$}
				\DisplayProof & $\stackrel{S}{\lto}$ &
				% https://q.uiver.app/?q=WzAsNyxbMiwxLCJcXG5lZyBBIl0sWzAsMSwiQSJdLFsxLDIsIlxcb3RpbWVzIl0sWzEsNCwiXFxvcGVyYXRvcm5hbWV7Y30iXSxbMSwzLCJBIFxcb3RpbWVzIFxcbmVnIEEiXSxbMiwwLCJUKFxccGlfMikiXSxbMCwwLCJUKFxccGlfMSkiXSxbMCwyLCIiLDAseyJjdXJ2ZSI6LTJ9XSxbMSwyLCIiLDEseyJjdXJ2ZSI6Mn1dLFsyLDQsIiIsMSx7InN0eWxlIjp7ImhlYWQiOnsibmFtZSI6Im5vbmUifX19XSxbNCwzXSxbNiwxLCIiLDEseyJzdHlsZSI6eyJoZWFkIjp7Im5hbWUiOiJub25lIn19fV0sWzUsMCwiIiwwLHsic3R5bGUiOnsiaGVhZCI6eyJuYW1lIjoibm9uZSJ9fX1dXQ==
				$\begin{tikzcd}[column sep = small, row sep = small]
					{S(\pi_1)} && {S(\pi_2)} \\
					A && {B} \\
					& \otimes \\
					& {A \otimes B} \\
					& {\operatorname{c}}
					\arrow[curve={height=-12pt}, from=2-3, to=3-2]
					\arrow[curve={height=12pt}, from=2-1, to=3-2]
					\arrow[no head, from=3-2, to=4-2]
					\arrow[from=4-2, to=5-2]
					\arrow[no head, from=1-1, to=2-1]
					\arrow[no head, from=1-3, to=2-3]
				\end{tikzcd}$
			\end{tabular}
		\end{center}
		
		\begin{center}
			\begin{tabular}{ >{\centering}m{2cm} >{\centering}m{7cm} >{\centering}m{0.5cm} >{\centering}m{7cm}}
				\text{Right implication} &
				\startproof{$\pi$}
				\UnaryInfC{$\Gamma, A, \Gamma' \vdash B$}
				\RightLabel{$\rimp$}
				\UnaryInfC{$\Gamma, \Gamma' \vdash A \multimap B$}
				\DisplayProof & $\stackrel{S}{\lto}$ &
				$\begin{tikzcd}[column sep = small, row sep = small]
					& {S(\pi)} \\
					\neg A && {B} \\
					& \parr \\
					& {\neg A \parr B} \\
					& {\operatorname{c}}
					\arrow[curve={height=-12pt}, from=2-3, to=3-2]
					\arrow[curve={height=12pt}, from=2-1, to=3-2]
					\arrow[no head, from=3-2, to=4-2]
					\arrow[from=4-2, to=5-2]
					\arrow[curve={height=12pt}, no head, from=1-2, to=2-1]
					\arrow[curve={height=-12pt}, no head, from=1-2, to=2-3]
				\end{tikzcd}$
			\end{tabular}
		\end{center}
		
		\begin{center}
			\begin{tabular}{ >{\centering}m{2cm} >{\centering}m{7cm} >{\centering}m{0.5cm} >{\centering}m{7cm}}
				\text{Left implication} &
				\startproof{$\pi_1$}
				\UnaryInfC{$\Gamma \vdash A$}
				\startproof{$\pi_2$}
				\UnaryInfC{$\Delta, B, \Delta' \vdash C$}
				\RightLabel{$\limp$}
				\BinaryInfC{$A \multimap B, \Gamma, \Delta \vdash C$}
				\DisplayProof & $\stackrel{S}{\lto}$ &
				% https://q.uiver.app/?q=WzAsNyxbMiwxLCJcXG5lZyBBIl0sWzAsMSwiQSJdLFsxLDIsIlxcb3RpbWVzIl0sWzEsNCwiXFxvcGVyYXRvcm5hbWV7Y30iXSxbMSwzLCJBIFxcb3RpbWVzIFxcbmVnIEEiXSxbMiwwLCJUKFxccGlfMikiXSxbMCwwLCJUKFxccGlfMSkiXSxbMCwyLCIiLDAseyJjdXJ2ZSI6LTJ9XSxbMSwyLCIiLDEseyJjdXJ2ZSI6Mn1dLFsyLDQsIiIsMSx7InN0eWxlIjp7ImhlYWQiOnsibmFtZSI6Im5vbmUifX19XSxbNCwzXSxbNiwxLCIiLDEseyJzdHlsZSI6eyJoZWFkIjp7Im5hbWUiOiJub25lIn19fV0sWzUsMCwiIiwwLHsic3R5bGUiOnsiaGVhZCI6eyJuYW1lIjoibm9uZSJ9fX1dXQ==
				$\begin{tikzcd}[column sep = small, row sep = small]
					{S(\pi_1)} && {S(\pi_2)} \\
					A && {\neg B} \\
					& \otimes \\
					& {A \otimes \neg B} \\
					& {\operatorname{c}}
					\arrow[curve={height=-12pt}, from=2-3, to=3-2]
					\arrow[curve={height=12pt}, from=2-1, to=3-2]
					\arrow[no head, from=3-2, to=4-2]
					\arrow[from=4-2, to=5-2]
					\arrow[no head, from=1-1, to=2-1]
					\arrow[no head, from=1-3, to=2-3]
				\end{tikzcd}$
			\end{tabular}
		\end{center}
		
		\begin{center}
			\begin{tabular}{ >{\centering}m{2cm} >{\centering}m{7cm} >{\centering}m{0.5cm} >{\centering}m{7cm}}
				\text{Exchange} &
				\begin{prooftree}
					\startproof{$\pi$}
					\UnaryInfC{$ \Gamma, A, B, \Gamma' \vdash C$}
					\RightLabel{$\ex$}
					\UnaryInfC{$\Gamma, B, A, \Gamma' \vdash C$}
				\end{prooftree} & $\stackrel{S}{\lto}$ &
				\begin{tikzcd}[column sep = small]
					S(\pi)
				\end{tikzcd}
			\end{tabular}
		\end{center}
		A \textbf{intuitionistic, multiplicative proof net} (or simply \textbf{intuitionistic proof net}) is an proof structure which lies in the image of $S$.
	\end{defn}
	\begin{defn}
		There is also a translation $R: \Pi \lto \Sigma$ which we now present. In what follows, if $\Gamma = A_1,...,A_n$ is a sequence of formulas, then $\neg \Gamma$ denotes the sequence given by $\neg A_1, ..., \neg A_n$.
		\begin{center}
			\begin{tabular}{ >{\centering}m{2cm} >{\centering}m{5cm} >{\centering}m{0.5cm} >{\centering}m{5cm}}
				\text{Axiom} & \begin{prooftree}
					\AxiomC{}
					\RightLabel{$({\operatorname{ax}})$}
					\UnaryInfC{$A \vdash A$}
				\end{prooftree} & $\stackrel{R}{\lto}$ &
				% https://q.uiver.app/?q=WzAsNSxbMSwwLCJcXGF4Il0sWzAsMSwiQSJdLFsyLDEsIlxcbmVnIEEiXSxbMCwyLCJcXG9wZXJhdG9ybmFtZXtjfSJdLFsyLDIsIlxcb3BlcmF0b3JuYW1le2N9Il0sWzAsMSwiIiwwLHsiY3VydmUiOjIsInN0eWxlIjp7ImhlYWQiOnsibmFtZSI6Im5vbmUifX19XSxbMCwyLCIiLDIseyJjdXJ2ZSI6LTIsInN0eWxlIjp7ImhlYWQiOnsibmFtZSI6Im5vbmUifX19XSxbMSwzXSxbMiw0XV0=
				\begin{prooftree}
					\AxiomC{}
					\RightLabel{$({\operatorname{ax}})$}
					\UnaryInfC{$\vdash \neg A,  A$}
				\end{prooftree}
			\end{tabular}
		\end{center}
		
		\begin{center}
			\begin{tabular}{ >{\centering}m{2cm} >{\centering}m{7cm} >{\centering}m{0.5cm} >{\centering}m{7cm}}
				\text{Cut} &
				\begin{prooftree}
					\startproof{$\pi_1$}
					\UnaryInfC{$\Gamma \vdash A$}
					\startproof{$\pi_2$}
					\UnaryInfC{$\Delta, A,\Theta \vdash B$}
					\RightLabel{$({\operatorname{cut}})$}
					\BinaryInfC{$\Gamma, \Delta, \Theta \vdash B$}
				\end{prooftree} & $\stackrel{R}{\lto}$ &
				% 
				\begin{prooftree}
					\startproof{$R(\pi_1)$}
					\UnaryInfC{$ \vdash \neg \Gamma, A$}
					\startproof{$R(\pi_2)$}
					\UnaryInfC{$ \vdash \neg \Delta, \neg A,\neg\Theta, B$}
					\RightLabel{$({\operatorname{cut}})$}
					\BinaryInfC{$ \vdash \neg \Gamma, \neg\Delta, \neg \Theta, B$}
				\end{prooftree}
			\end{tabular}
		\end{center}
		
		\begin{center}
			\begin{tabular}{ >{\centering}m{2cm} >{\centering}m{7cm} >{\centering}m{0.5cm} >{\centering}m{7cm}}
				\text{Left times} &
				\startproof{$\pi$}
				\UnaryInfC{$\Gamma, A, B, \Gamma' \vdash C$}
				\RightLabel{$\ltensor$}
				\UnaryInfC{$\Gamma, A \otimes B, \Gamma' \vdash C$}
				\DisplayProof & $\stackrel{R}{\lto}$ &
				\startproof{$R(\pi)$}
				\UnaryInfC{$ \vdash\neg \Gamma, \neg A, \neg B,\neg \Gamma', C$}
				\RightLabel{$\otimes$}
				\UnaryInfC{$ \vdash\neg\Gamma, \neg A \otimes \neg B, \neg \Gamma', C$}
				\DisplayProof 
			\end{tabular}
		\end{center}
		
		\begin{center}
			\begin{tabular}{ >{\centering}m{2cm} >{\centering}m{7cm} >{\centering}m{0.5cm} >{\centering}m{7cm}}
				\text{Right times} &
				\startproof{$\pi_1$}
				\UnaryInfC{$\Gamma \vdash A$}
				\startproof{$\pi_2$}
				\UnaryInfC{$\Delta\vdash B$}
				\RightLabel{$\rtensor$}
				\BinaryInfC{$\Gamma, \Delta \vdash A \otimes B$}
				\DisplayProof & $\stackrel{R}{\lto}$ &
				% 
				\startproof{$R(\pi_1)$}
				\UnaryInfC{$ \vdash \neg\Gamma, A$}
				\startproof{$R(\pi_2)$}
				\UnaryInfC{$\vdash\neg \Delta, B$}
				\RightLabel{$\otimes$}
				\BinaryInfC{$ \vdash\neg \Gamma, \neg \Delta, A \otimes B$}
				\DisplayProof
			\end{tabular}
		\end{center}
		
		\begin{center}
			\begin{tabular}{ >{\centering}m{2cm} >{\centering}m{7cm} >{\centering}m{0.5cm} >{\centering}m{7cm}}
				\text{Right implication} &
				\startproof{$\pi$}
				\UnaryInfC{$\Gamma, A, \Gamma' \vdash B$}
				\RightLabel{$\rimp$}
				\UnaryInfC{$\Gamma, \Gamma' \vdash A \multimap B$}
				\DisplayProof & $\stackrel{R}{\lto}$ &
				\startproof{$R(\pi)$}
				\UnaryInfC{$ \vdash\neg\Gamma, \neg A, \neg\Gamma', B$}
				\RightLabel{$\ex$}
				\doubleLine
				\UnaryInfC{$\vdash \neg \Gamma, \neg \Gamma', \neg A, B$}
				\RightLabel{$\parr$}
				\UnaryInfC{$ \vdash\neg \Gamma,\neg \Gamma', \neg A \parr B$}
				\DisplayProof
			\end{tabular}
		\end{center}
		
		\begin{center}
			\begin{tabular}{ >{\centering}m{2cm} >{\centering}m{7cm} >{\centering}m{0.5cm} >{\centering}m{7cm}}
				\text{Left implication} &
				\startproof{$\pi_1$}
				\UnaryInfC{$\Gamma \vdash A$}
				\startproof{$\pi_2$}
				\UnaryInfC{$\Delta, B, \Delta' \vdash C$}
				\RightLabel{$\limp$}
				\BinaryInfC{$A \multimap B, \Gamma, \Delta \vdash C$}
				\DisplayProof & $\stackrel{R}{\lto}$ &
				% \startproof{$\pi_1$}
				\startproof{$R(\pi_1)$}
				\UnaryInfC{$ \vdash \neg\Gamma, A$}
				\startproof{$R(\pi_2)$}
				\UnaryInfC{$ \vdash\neg \Delta, \neg B, \neg \Delta',C$}
				\RightLabel{$\otimes$}
				\BinaryInfC{$ \vdash \neg\Gamma, A \otimes \neg B, \neg\Delta, C$}
				\DisplayProof
			\end{tabular}
		\end{center}
		
		\begin{center}
			\begin{tabular}{ >{\centering}m{2cm} >{\centering}m{7cm} >{\centering}m{0.5cm} >{\centering}m{7cm}}
				\text{Exchange} &
				\begin{prooftree}
					\startproof{$\pi$}
					\UnaryInfC{$ \Gamma, A, B, \Gamma' \vdash C$}
					\RightLabel{$\ex$}
					\UnaryInfC{$\Gamma, B, A, \Gamma' \vdash C$}
				\end{prooftree} & $\stackrel{R}{\lto}$ &
				\begin{prooftree}
					\startproof{$\pi$}
					\UnaryInfC{$  \vdash \neg\Gamma, \neg A, \neg B, \neg \Gamma',C$}
					\RightLabel{$\ex$}
					\UnaryInfC{$ \vdash \neg\Gamma, \neg B, \neg A, \neg \Gamma', C$}
				\end{prooftree} 
			\end{tabular}
		\end{center}
	\end{defn}
	It is easy to see that the following diagram commutes.
	\begin{equation}
		\begin{tikzcd}
			\Pi\arrow[r,"R"]\arrow[dr,swap,"S"] & \Sigma\arrow[d,"T"]\\
			& \operatorname{MPN}
		\end{tikzcd}
	\end{equation}
	\begin{lemma}
		The map $R$ is injective.
	\end{lemma}
	\begin{proof}
		By inspection of the defining rules of $R$.
	\end{proof}
	\begin{lemma}\label{lem:negation_surj}
		The map $R$ is not surjective.
	\end{lemma}
	\begin{proof}
		There is not proof $\pi$ in $\Pi$ such that $R(\pi)$ is the following.
		\begin{prooftree}
			\AxiomC{}
			\RightLabel{$\ax$}
			\UnaryInfC{$\vdash \neg A, A$}
		\end{prooftree}
	\end{proof}
	\begin{remark}
		Neither of the maps $T,S$ are injective. By definition of proof nets, the map $T$ is surjective. The image of $S$ is the set of \textbf{multiplicative, intuitionistic proof nets}. These will not be considered again in these notes, but it would be interesting to find a correctness criterion for intuitionistic proof nets similar to the \emph{long trip condition} (Section \ref{sec:sequentialisation}).
	\end{remark}
	In Section \ref{sec:sequentialisation} we will see what the image of the map $T$ is.
	
	\section{The Sequentialisation Theorem}\label{sec:sequentialisation}
	\begin{defn}
		Let $\pi$ be a proof structure and denote the set of tensor and par links of $\pi$ by $\operatorname{Link}_{\otimes,\parr}\pi$ (or simply $\operatorname{Link}\pi$). A \textbf{switching} of $\pi$ is a function
		\begin{equation}
			S: \operatorname{Link}\pi \lto \lbrace L,R\rbrace
		\end{equation}
		A \textbf{switching} of a particular link $l$ is a choice of $L,R$ associated to $l$.
	\end{defn}
	\begin{defn}\label{def:trip}
		Let $\pi$ be a proof structure. Let $\call{O}(\pi)$ denote the set of occurrences of formulas in $\pi$ (Definition \ref{def:occurrences_labels}).  We consider two disjoint copies of this set
		\begin{equation}
			\call{U}(\pi) := \call{O}(\pi) \coprod \call{O}(\pi)
		\end{equation}
		where elements from the first copy are the \textbf{up elements}, and elements from the second copy are the \textbf{down elements}. We write $\uparrow A$ for the up element corresponding to an occurrence of a formula $A$ in $\pi$, and similarly for $A\downarrow$. Given a switching $S$ of $\pi$, a \textbf{pretrip of $\pi$ with respect to $S$} is a finite sequence $(x_1,...,x_n)$ of elements of $\call{U}(\pi)$ satisfying the following.
		\begin{enumerate}
			\item The sequence is a loop, that is, $x_1 = x_n$, and all elements (except the first and the last) are distinct.
			\item\label{def:trip_axiom} If $x_j=\uparrow A$ and $A$ is part of an axiom link then $x_{j+1} = \negation A\downarrow$.
			\item If $x_j = A\downarrow$ and $A$ is part of a cut link then $x_{j+1} = \uparrow \negation A$.
			\item For any tensor link $l$ with premises $A,B$ such that $l$ has switching $L$, we have:
			\begin{itemize}
				\item if $x_j = A \downarrow$ then $x_{j+1} = (A \otimes B)\downarrow$,
				\item if $x_j = \uparrow (A \otimes B)$ then $x_{j+1} = \uparrow B_j$,
				\item if $x_j = B \downarrow$ then $x_{j+1} = \uparrow A$.
			\end{itemize}
			and if $l$ has switching $R$, we have:
			\begin{itemize}
				\item if $x_j = A \downarrow$ then $x_{j+1} = \uparrow B$,
				\item if $x_j = \uparrow (A \otimes B)$ then $x_{j+1} = \uparrow A$,
				\item if $x_j = B \downarrow$ then $x_{j+1} = (A \otimes B) \downarrow$.
			\end{itemize}
			(see Figure \ref{fig:tensorswitching})
			\item\label{def:trip_right_par} for any par link $l$ with premises $A,B$ such that $l$ has switching $L$, we have:
			\begin{itemize}
				\item if $x_j = \uparrow (A \parr B)$ then $x_{j+1} = \uparrow A$,
				\item if $x_j = A\downarrow$ then $x_{j+1} = (A \parr B)\downarrow$,
				\item if $x_j = B\downarrow$ then $x_{j+1} = \uparrow B$.
			\end{itemize}
			and if $l$ evaluates under $S$ to $R$, we have:
			\begin{itemize}
				\item if $x_j = A\downarrow$ then $x_{j+1} = \uparrow A$,
				\item if $x_j = \uparrow (A\parr B)$ then $x_{j+1} = \uparrow B$,
				\item if $x_j = B\downarrow$ then $x_{j+1} = (A\parr B)\downarrow$.
			\end{itemize}
			(see Figure \ref{fig:parrswitching})
		\end{enumerate}
		\begin{figure}[h]
			\centering
			\includegraphics[width = 0.8\textwidth]{TensorSwitch.jpg}
			\caption{Tensor link, $L$ switching, $R$ switching}
			\label{fig:tensorswitching}
		\end{figure}
		\begin{figure}[h]
			\centering
			\includegraphics[width = 0.8\textwidth]{ParSwitch.jpg}
			\caption{Par link, $L$ switching, $R$ switching.}
			\label{fig:parrswitching}
		\end{figure}
		\begin{defn}
			Let $\operatorname{Pre}\call{T}(\pi,S)$ denote the set of all pretrips of $\pi$ with respect to $S$. We define an equivalence relation on this set $\sim$ where two pretrips $(x_1,...,x_n)$ and $(y_1,...,y_m)$ are equivalent if $n = m$, and there exists an integer $k$ such that $x_{i + k} = y_i$ (where $i + k$ means $\operatorname{mod} n$) for all $i = 1,...,n$.
			
			A \textbf{trip} of $\pi$ with respect to $S$ is an equivalence class of pretrips. We denote the set of all trips by $\call{T}(\pi,S)$.  If the set $\call{T}(\pi,S)$ admits more than one element, these elements are called \textbf{short trips}, and if it admits only one element, this element is the \textbf{long trip}. We refer to the statement ``for all switchings $S$, the set $\call{T}(\pi,S)$ contains exactly one element" as the \textbf{long trip condition}.
			
			A \textbf{short pretrip} is a choice of representative for a pretrip, and a \textbf{long pretrip} is a choice of representatitive of a long trip.
		\end{defn}
		
	\end{defn}
	Given a proof structure $\pi$ satisfying the long trip condition and a tensor link $l$ with premises $A,B$ say, let $S$ be a switching of $\pi$ and $t := (x_1,...,x_n)$ be the long pretrip of $\pi$ satisfying $x_1 =A\downarrow$. Since $\pi$ satisfies the long trip condition, it must be the case that $\uparrow (A \otimes B)$ and $B\downarrow$ occur somewhere in $t$, can we determine which occurs earlier? Let $m,l > 0$ be such that $x_m = \uparrow (A \otimes B), x_l = B\downarrow$ and assume $l < m$. Say $S(\tau) = L$, then $t$ has the shape
	\begin{equation}\label{eq:sequence_left_switch_tensor}
		(A\downarrow, (A \otimes B)\downarrow, ..., B \downarrow, \uparrow A, ..., \uparrow (A \otimes B), \uparrow B, ..., A\downarrow)
	\end{equation}
	Now consider the switching given by
	\[\hat{S}(\sigma) = \begin{cases}
		S(\sigma),& \sigma \neq \tau\\
		R, & \sigma = \tau
	\end{cases}
	\]
	Then \eqref{eq:sequence_left_switch_tensor} becomes:
	\begin{equation}
		(A\downarrow, \uparrow B, ..., A\downarrow)
	\end{equation}
	which is a short pretrip, contradicting the assumption that $\pi$ satisfies the long trip condition. Thus $m < l$. We have proven (the first half) of the following.
	\begin{lemma}\label{lem:stays_contained_tensor}
		Let $\pi$ be a proof structure satisfying the long trip condition, $l$ be a tensor link with premises $A,B$ say, $S$ be a switching of $\pi$ and $(x_1,...,x_n)$ the long pretrip satisfying $x_1 = A \downarrow$. If $m,l > 0$ are such that $x_m = \uparrow (A \otimes B), x_l = B \downarrow$, then:
		\begin{itemize}
			\item if $S(\tau) = L$ then $m < l$,
			\item if $S(\tau) = R$ then $l < m$.
		\end{itemize}
	\end{lemma}
	The proof of the other half is similar to what has already been written, however since Lemma \ref{lem:stays_contained_tensor} contradicts \cite[Lemma 2.9.1]{linearlogic} we write out the details here:
	\begin{proof}
		Say $m < l$, then $t$ has the shape
		\begin{equation}\label{eq:sequence_right_switch_tensor}
			(A\downarrow, \uparrow B, ..., \uparrow (A \otimes B), \uparrow A, ..., B \downarrow, (A \otimes B)\downarrow, ..., A\downarrow)
		\end{equation}
		Now consider the switching given by
		\[
		S'(\sigma) = 
		\begin{cases}
			S(\sigma),& \sigma \neq \tau\\
			L, & \sigma = \tau
		\end{cases}
		\]
		Then \eqref{eq:sequence_right_switch_tensor} becomes:
		\begin{equation}
			(A\downarrow, (A \otimes B)\downarrow,..., A\downarrow)
		\end{equation}
		which is a short pretrip.
	\end{proof}
	\begin{lemma}\label{lem:stays_contained_par}
		Let $\pi$ be a proof structure satisfying the long trip condition, $l$ be a par link with premises $A,B$ say, $S$ be a switching of $\pi$ and $(x_1,...,x_n)$ be the long pretrip satisfying $x_1 = A\downarrow$. If $m,l > 0$ are such that $x_m = \uparrow (A \parr B), x_l = B\downarrow$, then
		\begin{itemize}
			\item if $S(\tau) = L$ then $m < l$,
			\item if $S(\tau) = R$ then $l < m$
		\end{itemize}
	\end{lemma}
	\begin{remark}
		Lemma \ref{lem:stays_contained_tensor} gives a nice interpretation of Lemma \ref{lem:stays_contained_tensor} that long trips \emph{return to where they left} at each tensor link.
		
		The situation is a bit different for par links; the relevant slogan is long tripes \emph{visit the premises before returning to the conclusion}.
	\end{remark}
	Say $\pi$ satisfies the long trip condition and moreover $\pi$ admits a tensor link $l$ (with premises $A,B$ say) such that if $l$ is removed, the resulting proof structure consists of two disjoint proof structures $\pi_1,\pi_2$ each satisfying the long trip condition. It is necessarily the case that any pretrip $\rho$ of $\pi$ starting at $\uparrow A$ visits the entirety of $\call{U}(\pi_1)$ before returning to the tensor link $l$, lest $\pi_1$ admit a short trip. Moreover, it must be the case that $\sigma$ admits no occurrence of formulas in $\pi_2$ lest the result of removing the tensor link $l$ not result in disjoint proof structures. Thus, if such a link $l$ exists, it is \emph{maximal} in the sense that there is no other tensor link $l'$ where a pretrip starting at a premise of $l'$ contains the entirety of any pretrip starting at $A$. The remainder of this Section will amount to proving the converse, that any such maximal tensor link ``splits" $\pi$.
	\begin{defn}\label{def:pretrip_from_A}
		Let $\pi$ be a proof structure satisfying the long trip condition, $S$ a switching of $\pi$, and $A$ an occurrence of a formula in $\pi$. Consider the long pretrip $(x_1,...,x_n)$ satisfying $x_1 = \uparrow A$. We denote by
		\begin{equation}
			\operatorname{PTrip}(\pi,S,A,\uparrow)
		\end{equation}
		the subsequence $(x_1,...,x_m)$ of $(x_1,...,x_n)$ satisfying $x_m = A\downarrow$. We define
		\begin{equation}
			\operatorname{PTrip}(\pi, S, A, \downarrow)
		\end{equation}
		similarly.
		
		Also, for $a \in \lbrace \uparrow,\downarrow\rbrace $ we define the following set
		\begin{equation}
			\operatorname{Visit}_S(A,a) := \lbrace C \in \call{O}(\pi) \mid \uparrow C, C\downarrow \text{ occur in } \operatorname{PTrip}(\pi,S,A,a)\rbrace
		\end{equation}
		The \textbf{up empire of $A$} is the following set:
		\begin{equation}
			\operatorname{Emp}_{\uparrow}A := \lbrace C \in \call{O}(\pi) \mid \text{For all switchings }S\text{ we have } \uparrow C, C\downarrow \text{ occur in } \operatorname{PTrip}(\pi,S, A,\uparrow)\rbrace
		\end{equation}
		The \textbf{down empire of $A$} is defined symmetrically.
	\end{defn}
	With this new terminology we now have some corollaries of Lemmas \ref{lem:stays_contained_tensor} and \ref{lem:stays_contained_par}:
	\begin{cor}\label{cor:pretrip_innards}
		Let $\pi$ be a proof structure satisfying the long trip condition, and let $S$ be a switching of $\pi$, for a formula $A$ and $a \in \lbrace \uparrow, \downarrow\rbrace$, denote $\operatorname{PTrip}(\pi,S, A, a)$ by $\operatorname{PTrip}(A,a)$:
		\begin{enumerate}
			\item if $A$ is part of an axiom link then
			\begin{equation}
				\operatorname{PTrip}(A,\uparrow) = \uparrow A, \operatorname{PTrip}(\negation A, \downarrow), A \downarrow
			\end{equation}
			\item if $l$ is a tensor link with conclusion $A \otimes B$:
			\begin{enumerate}
				\item if $S(l) = L$:
				\begin{equation}
					\operatorname{Ptrip}(A, \downarrow) = A\downarrow, \operatorname{PTrip}(A \otimes B, \downarrow), \operatorname{PTrip}(B, \uparrow), \uparrow A
				\end{equation}
				\begin{equation}
					\operatorname{PTrip}(B, \downarrow) = B\downarrow, \operatorname{PTrip}(A,\uparrow), \operatorname{PTrip}(A \otimes B, \downarrow), \uparrow B
				\end{equation}
				\begin{equation}
					\operatorname{PTrip}(A \otimes B, \uparrow) = \uparrow A \otimes B, \operatorname{PTrip}(B, \uparrow), \operatorname{PTrip}(A,\uparrow), A \otimes B \downarrow
				\end{equation}
				\item if $S(l) = R$:
				\begin{equation}
					\operatorname{PTrip}(A, \downarrow) = A\downarrow, \operatorname{PTrip}(B, \uparrow),\operatorname{PTrip}(A \otimes B, \downarrow), \uparrow A
				\end{equation}
				\begin{equation}
					\operatorname{PTrip}(B, \downarrow) = B\downarrow,  \operatorname{PTrip}(A \otimes B, \downarrow), \operatorname{PTrip}(A,\uparrow),\uparrow B
				\end{equation}
				\begin{equation}
					\operatorname{PTrip}(A \otimes B, \uparrow) = \uparrow A \otimes B,  \operatorname{PTrip}(A,\uparrow), \operatorname{PTrip}(B, \uparrow),A \otimes B \downarrow
				\end{equation}
			\end{enumerate}
			\item if $A$ is a premise of a par link $l$ with conclusion $A \parr B$:
			\begin{enumerate}
				\item if $S(l) = L$:
				\begin{equation}
					\operatorname{PTrip}(A,\downarrow) = A\downarrow, \operatorname{PTrip}(A \parr B, \downarrow), \uparrow A
				\end{equation}
				\begin{equation}
					\operatorname{PTrip}(B, \downarrow) = B\downarrow, \uparrow B
				\end{equation}
				\begin{equation}
					\operatorname{PTrip}(A \parr B, \uparrow) = \uparrow A \parr B, \operatorname{PTrip}(A, \uparrow), A \parr B \downarrow
				\end{equation}
				\item if $S(l) = R$:
				\begin{equation}
					\operatorname{PTrip}(A,\downarrow) = A\downarrow, \uparrow A
				\end{equation}
				\begin{equation}
					\operatorname{PTrip}(B, \downarrow) = B\downarrow, \operatorname{PTrip}(A \parr B, \downarrow), \uparrow B
				\end{equation}
				\begin{equation}
					\operatorname{PTrip}(A \parr B, \uparrow) = \uparrow A \parr B, \operatorname{PTrip}(B, \uparrow), A \parr B \downarrow
				\end{equation}
			\end{enumerate}
		\end{enumerate}
	\end{cor}
	In particular:
	\begin{cor}\label{cor:stays_contained_corollary}
		For any formula $A$ which is a premise to either a tensor or par link, and any $a \in \lbrace \uparrow, \downarrow \rbrace$, we have: $$\uparrow C \text{ occurs in } \operatorname{PTrip}(\pi,S,A,\uparrow)\qquad\text{ if and only if }\qquad C\downarrow \text{ occurs in } \operatorname{PTrip}(\pi,S,A,\downarrow)$$ and similarly for $\operatorname{PTrip}(\pi,S,A,\downarrow)$.
	\end{cor}
	\begin{proof}
		By induction on the length of the sequence $\operatorname{PTrip}(\pi,S,A,a)$ and appealing to Corollary \ref{cor:pretrip_innards}.
	\end{proof}
	\begin{cor}\label{cor:empire_features}
		Let $\pi$ be a proof structure satisfying the long trip condition,
		we have the following.
		\begin{enumerate}
			\item\label{cor:empire_features_up_ax} For any axiom link with conclusions $A, \neg A$:
			\begin{equation}
				\operatorname{Emp}_{\uparrow}A = \operatorname{Emp}_{\downarrow}(\neg A) \cup \lbrace A \rbrace
			\end{equation}
			\item\label{cor:empire_features_down_cut} For any cut link with premises $A, \neg A$:
			\begin{equation}
				\operatorname{Emp}_{\downarrow}A = \operatorname{Emp}_{\uparrow}(\neg A) \cup \lbrace A \rbrace
			\end{equation}
			\item\label{cor:empire_features_empty_int} For any tensor link with premises $A,B$:
			\begin{equation}
				\operatorname{Emp}_{\uparrow}A \cap \operatorname{Emp}_{\uparrow}B = \varnothing
			\end{equation}
			\item\label{cor:empire_features_up_tens_par} For any tensor or par link with premises $A,B$ and conclusion $C$:
			\begin{equation}
				\operatorname{Emp}_{\uparrow}C = \operatorname{Emp}_{\uparrow}A \cup \operatorname{Emp}_{\uparrow}B \cup \lbrace C \rbrace
			\end{equation}
			\item\label{cor:empire_features_down_tens} For any tensor link with premises $A,B$:
			\begin{equation}
				\operatorname{Emp}_{\downarrow}B = \operatorname{Emp}_{\uparrow}A \cup \operatorname{Emp}_{\downarrow}(A \otimes B) \cup \lbrace B \rbrace
			\end{equation}
			and similarly:
			\begin{equation}
				\operatorname{Emp}_{\downarrow}A = \operatorname{Emp}_{\uparrow}B \cup \operatorname{Emp}_{\downarrow}(A \otimes B) \cup \lbrace A\rbrace
			\end{equation}
		\end{enumerate}
	\end{cor}
	\begin{defn}
		Given any link $l$ we write $B \in l$ if $B$ occurs as either a premise or a conclusion of $l$.
		
		Let $\pi$ be a proof structure satisfying the long trip condition, and $a \in \lbrace \uparrow, \downarrow\rbrace$. The set of \textbf{links of $A$ with respect to $S$} is the set
		\begin{equation}
			\operatorname{Link}_aA := \lbrace l \in \operatorname{Link}\pi \mid \forall B \in l, B \in \operatorname{Emp}_aA \rbrace
		\end{equation}
	\end{defn}
	\begin{defn}
		Let $\pi$ be a proof structure satisfying the long trip condition and let $a \in \lbrace \uparrow, \downarrow \rbrace$. Define the set
		\begin{equation}
			\operatorname{Link}^0_{\parr,a}A := \lbrace l \in \operatorname{Link}\pi \mid \text{Exactly one premise of }l\text{ is in }\operatorname{Emp}_{a}A\rbrace
		\end{equation}
	\end{defn}
	\begin{lemma}[Realisation Lemma]\label{lem:realisation_switching}
		Let $\pi$ be a cut-free proof structure satisfying the long trip condition, let $a \in \lbrace \uparrow, \downarrow \rbrace$ and $A$ an occurrence of a formula in $\pi$. Define the following function:
		\begin{align*}
			S: \operatorname{Link}_{\parr,a}^0A &\lto \lbrace L,R\rbrace\\
			l &\longmapsto
			\begin{cases}
				L, & \text{if the right premise of }l\text{ is in }\operatorname{Emp}_{a}A\\
				R, & \text{if the left premise of }l\text{ is in }\operatorname{Emp}_{a}A
			\end{cases}
		\end{align*}
		and extend this to a switching $\hat{S}: \operatorname{Link}\pi \lto \lbrace L,R \rbrace$ arbitrarily. Then
		\begin{equation}
			\operatorname{Emp}_aA = \operatorname{Visit}_{\hat{S}}(A,a)
		\end{equation}
	\end{lemma}
	\begin{proof}
		We proceed by induction on the size $|\operatorname{Link}_a(A)|$ of the set $\operatorname{Link}_a(A)$. For the base case, assume $|\operatorname{Link}_a(A)| = 0$. The formula $A$ is part of an axiom link and so $\operatorname{Emp}_{\uparrow}A = A, \negation A$ and $\operatorname{Emp}_{\downarrow}A = A$, the result follows easily.
		
		Now assume that $|\operatorname{Link}_a A| = n > 0$ and the result holds for any formula $B$ such that $|\operatorname{Link}_aB| < n$. First say $a = \uparrow$, and $A$ is a conclusion of either a tensor or a par link
		% https://q.uiver.app/?q=WzAsNSxbMCwwLCJBXzEiXSxbMiwwLCJBXzIiXSxbMSwxLCJcXGJveHRpbWVzIl0sWzEsMiwiQSJdLFsxLDMsIlxcdmRvdHMiXSxbMyw0XSxbMSwyLCIiLDAseyJjdXJ2ZSI6LTIsInN0eWxlIjp7ImhlYWQiOnsibmFtZSI6Im5vbmUifX19XSxbMCwyLCIiLDIseyJjdXJ2ZSI6Miwic3R5bGUiOnsiaGVhZCI6eyJuYW1lIjoibm9uZSJ9fX1dLFsyLDNdXQ==
		\[\begin{tikzcd}
			{A_1} && {A_2} \\
			& \boxtimes \\
			& A \\
			& \vdots
			\arrow[from=3-2, to=4-2]
			\arrow[curve={height=-12pt}, no head, from=1-3, to=2-2]
			\arrow[curve={height=12pt}, no head, from=1-1, to=2-2]
			\arrow[from=2-2, to=3-2]
		\end{tikzcd}\]
		where $\boxtimes \in \lbrace \otimes, \parr \rbrace$ and $A = A_1 \otimes A_2$ or $A = A_1 \parr A_2$. By \eqref{cor:empire_features_up_tens_par} we have
		\begin{align*}
			\operatorname{Emp}_{\uparrow}A &= \operatorname{Emp}_{\uparrow}A_1 \cup \operatorname{Emp}_{\uparrow}A_2 \cup \lbrace A\rbrace\\
			&= \operatorname{Visit}_{\hat{S}}(A_1,\uparrow) \cup \operatorname{Visit}_S(A_2,\uparrow) \cup \lbrace A \rbrace\\
			&= \operatorname{Visit}_{\hat{S}}(A, \uparrow)
		\end{align*}
		where the second equality follows from the inductive hypothesis.
		
		Assume $A$ is part of an axiom link. By \eqref{cor:empire_features_up_ax}
		\begin{equation}
			\operatorname{Emp}_{\uparrow}A = \operatorname{Emp}_{\downarrow}(\neg A) \cup \lbrace A \rbrace
		\end{equation}
		with
		\begin{equation}
			|\operatorname{Link}_{\uparrow}A| = |\operatorname{Link}_{\downarrow}(\neg A)|
		\end{equation}
		Since $|\operatorname{Link}_{\downarrow}(\negation A)| > 0$ we necessarily have that $\negation A$ is not a conclusion. Thus, since $\pi$ is cut-free, $A$ is connected to an occurrence $\negation A$ which is a premise to either a tensor link or a par link. In the case of the former, we have:
		% https://q.uiver.app/?q=WzAsNSxbMCwwLCJDIl0sWzIsMCwiXFxuZWcgQSJdLFsxLDEsIlxcb3RpbWVzIl0sWzEsMiwiQyBcXG90aW1lcyBcXG5lZyBBIl0sWzEsMywiXFx2ZG90cyJdLFszLDRdLFsxLDIsIiIsMCx7ImN1cnZlIjotMiwic3R5bGUiOnsiaGVhZCI6eyJuYW1lIjoibm9uZSJ9fX1dLFswLDIsIiIsMix7ImN1cnZlIjoyLCJzdHlsZSI6eyJoZWFkIjp7Im5hbWUiOiJub25lIn19fV0sWzIsM11d
		\[\begin{tikzcd}[column sep = small, row sep = small]
			C && {\neg A} \\
			& \otimes \\
			& {C \otimes \neg A} \\
			& \vdots
			\arrow[from=3-2, to=4-2]
			\arrow[curve={height=-12pt}, no head, from=1-3, to=2-2]
			\arrow[curve={height=12pt}, no head, from=1-1, to=2-2]
			\arrow[from=2-2, to=3-2, no head]
		\end{tikzcd}\]
		then by \eqref{cor:empire_features_down_tens}:
		\begin{align*}
			\operatorname{Emp}_{\downarrow}(\neg A) &= \operatorname{Emp}_{\uparrow}C \cup \operatorname{Emp}_{\downarrow}(C \otimes \neg A) \cup \lbrace \neg A\rbrace\\
			&= \operatorname{Visit}_{\hat{S}}(C,\uparrow) \cup \operatorname{Visit}_{\hat{S}}(C \otimes \neg A,\downarrow) \cup \lbrace \neg A\rbrace\\
			&= \operatorname{Visit}_{\hat{S}}(\neg A, \downarrow)
		\end{align*}
		where the second equality follows from the inductive hypothesis.
		
		If $\negation A$ is a premise of a par link
		\[\begin{tikzcd}[column sep = small, row sep = small]
			C && {\neg A} \\
			& \parr \\
			& {C \parr \neg A} \\
			& \vdots
			\arrow[from=3-2, to=4-2]
			\arrow[curve={height=-12pt}, no head, from=1-3, to=2-2]
			\arrow[curve={height=12pt}, no head, from=1-1, to=2-2]
			\arrow[from=2-2, to=3-2, no head]
		\end{tikzcd}\]
		then by construction of $\hat{S}$, where we use the specific definition of $S$ for the first time,
		\begin{align*}
			\operatorname{Emp}_{\downarrow}(\neg A) &= \lbrace \neg A\rbrace\\
			&= \operatorname{Visit}_{\hat{S}}(\neg A,\downarrow)
		\end{align*}
		The case when $a = \downarrow$ is exactly similar and so we omit the proof.
	\end{proof}
	\begin{defn}
		A tensor or par link is \textbf{terminal} if it is a conclusion.
	\end{defn}
	\begin{cor}\label{lem:par_link_existence}
		Let $\pi$ be a cut-free proof structure satisfying the long trip condition. Let
		\[l = \begin{tikzcd}[column sep = small, row sep = small]
			A && {B} \\
			& \otimes \\
			& {A \otimes B} \\
			& \operatorname{c}
			\arrow[from=3-2, to=4-2]
			\arrow[curve={height=-12pt}, no head, from=1-3, to=2-2]
			\arrow[curve={height=12pt}, no head, from=1-1, to=2-2]
			\arrow[from=2-2, to=3-2, no head]
		\end{tikzcd}\]
		be a terminal tensor link of $\pi$. Then $\pi$ admits a par link
		\[l' := \begin{tikzcd}[column sep = small, row sep = small]
			C && {D} \\
			& \otimes \\
			& {C \parr D} \\
			& \vdots
			\arrow[from=3-2, to=4-2]
			\arrow[curve={height=-12pt}, no head, from=1-3, to=2-2]
			\arrow[curve={height=12pt}, no head, from=1-1, to=2-2]
			\arrow[from=2-2, to=3-2, no head]
		\end{tikzcd}\]
		such that either $C \in \operatorname{Emp}_{\uparrow}A$ and $D \in \operatorname{Emp}_{\uparrow}B$ or $C \in \operatorname{Emp}_{\uparrow}B$ and $D \in \operatorname{Emp}_{\uparrow}A$ if and only if for any switching $S$ of $\pi$ we have that either
		\[\operatorname{Emp}_{\uparrow}A \subsetneq \operatorname{Visit}_S(A,\uparrow)\qquad\text{or}\qquad \operatorname{Emp}_{\uparrow}B \subsetneq \operatorname{Visit}_S(B,\uparrow)\]
	\end{cor}
	\begin{proof}
		Say $\pi$ admitted $l'$ and $C \in \operatorname{Emp}_{\uparrow}A$ and $D \in \operatorname{Emp}_{\uparrow}B$. If the switching $S$ is such that $S(l) = L$ then $C \parr D \in \operatorname{Visit}_S(B)\setminus \operatorname{Emp}_{\uparrow}B$ and if $S(\tau) = R$ then $C \parr D \in \operatorname{Visit}_S(A) \setminus \operatorname{Emp}_{\uparrow}A$. The other case is similar.
		
		Conversely, say $\pi$ admits no such par link $l'$, that is, assume
		\begin{equation}
			\operatorname{Link}^0_{\parr,\uparrow}(A) \cap \operatorname{Link}^0_{\parr,\uparrow}(B) = \varnothing
		\end{equation}
		Then there is by Lemma \ref{lem:realisation_switching} a well defined function $$S: \operatorname{Link}_{\parr,\uparrow}^0(A) \cup \operatorname{Link}^0_{\parr,\uparrow}(B) \lto \lbrace L, R \rbrace$$ which extends to a switching $\hat{S}$ such that
		\begin{equation}
			\operatorname{Emp}_{\uparrow}A = \operatorname{Visit}_{\hat{S}}(A,\uparrow)\qquad\text{and}\qquad \operatorname{Emp}_{\uparrow}B = \operatorname{Visit}_{\hat{S}}(B,\uparrow)
		\end{equation}
	\end{proof}
	
	\begin{lemma}[Separation Lemma]
		A cut-free proof structure $\pi$ satisfying the long trip condition, with only tensor links amongst its conclusions admits a tensor link
		\[l := \begin{tikzcd}[column sep = small, row sep = small]
			A && {B} \\
			& \otimes \\
			& {A \otimes B} \\
			& \operatorname{c}
			\arrow[from=3-2, to=4-2]
			\arrow[curve={height=-12pt}, no head, from=1-3, to=2-2]
			\arrow[curve={height=12pt}, no head, from=1-1, to=2-2]
			\arrow[from=2-2, to=3-2, no head]
		\end{tikzcd}\]
		satisfying
		\begin{equation}
			\call{O}(\pi) = \operatorname{Emp}_{\uparrow}A \cup \operatorname{Emp}_{\uparrow}B \cup \lbrace A \otimes B\rbrace
		\end{equation}
		Moreover, removing $A \otimes B$ results in a disconnected graph with each component a proof structure satisfying the long trip condition.
	\end{lemma}
	\begin{proof}
		Consider the set of tensor links $\operatorname{Link}_{\otimes}(\pi)$ of $\pi$. We endow this with the following partial order $\leq$: a pair of links:
		\[
		l' := \begin{tikzcd}[column sep = small, row sep = small]
			A && {B} \\
			& \otimes \\
			& {A \otimes B} \\
			& \vdots
			\arrow[from=3-2, to=4-2]
			\arrow[curve={height=-12pt}, no head, from=1-3, to=2-2]
			\arrow[curve={height=12pt}, no head, from=1-1, to=2-2]
			\arrow[from=2-2, to=3-2, no head]
		\end{tikzcd}
		%
		\qquad
		%
		l'' :=
		\begin{tikzcd}[column sep = small, row sep = small]
			C && {D} \\
			& \otimes \\
			& {C \otimes D} \\
			& \vdots
			\arrow[from=3-2, to=4-2]
			\arrow[curve={height=-12pt}, no head, from=1-3, to=2-2]
			\arrow[curve={height=12pt}, no head, from=1-1, to=2-2]
			\arrow[from=2-2, to=3-2, no head]
		\end{tikzcd}
		\]
		are such that $l' \leq l''$ if $\operatorname{Emp}_{\uparrow}A \cup \operatorname{Emp}_{\uparrow}B \subseteq \operatorname{Emp}_{\uparrow}C \cup \operatorname{Emp}_{\uparrow}D$. Let $l$ (with conclusion $A \otimes B$ say) be a tensor link maximal with respect to $\leq$. We show that $l$ satisfies the required property.
		
		Say $\call{O}(\pi) \neq \operatorname{Emp}_{\uparrow}A \cup \operatorname{Emp}_{\uparrow}B \cup \lbrace A \otimes B\rbrace$. Then by Lemma \ref{lem:par_link_existence} there exists a par link
		\[
		l' = \begin{tikzcd}[column sep = small, row sep = small]
			C && {D} \\
			& \parr \\
			& {C \parr D} \\
			& \vdots
			\arrow[from=3-2, to=4-2]
			\arrow[curve={height=-12pt}, no head, from=1-3, to=2-2]
			\arrow[curve={height=12pt}, no head, from=1-1, to=2-2]
			\arrow[from=2-2, to=3-2, no head]
		\end{tikzcd}
		\]
		such that either $C \in \operatorname{Emp}_{\uparrow}A$ and $D \in \operatorname{Emp}_{\uparrow}B$ or $C \in \operatorname{Emp}_{\uparrow}B$ and $D \in \operatorname{Emp}_{\uparrow}A$. We show the proof in the case of the former. Since $\pi$ admits no terminal par links, the unique maximal length directed path of $\pi$ beginning at the node $\parr$ of $l'$ terminates at an edge labelled $E \otimes F$, for some $E,F$.
		\[l'':= \begin{tikzcd}[column sep = small, row sep = small]
			E && {F} \\
			& \otimes \\
			& {E \otimes F} \\
			& \vdots
			\arrow[from=3-2, to=4-2]
			\arrow[curve={height=-12pt}, no head, from=1-3, to=2-2]
			\arrow[curve={height=12pt}, no head, from=1-1, to=2-2]
			\arrow[from=2-2, to=3-2, no head]
		\end{tikzcd}\]
		Notice that if $l'' = l$, then either $C \parr D \in \operatorname{Emp}_{\uparrow}A$ or $C \parr D \in \operatorname{Emp}_{\uparrow}B$ which in either case implies $\operatorname{Emp}_{\uparrow}A \cap \operatorname{Emp}_{\uparrow}B \neq \varnothing$, contradicting Corollary \ref{cor:empire_features}, \ref{cor:empire_features_empty_int}, and so $l'' \neq l$. Without any loss of generality, assume that $l'$ sits above $F$. The situation looks as follows.
		% https://q.uiver.app/?q=WzAsMjEsWzMsMywiQSBcXG90aW1lcyBCIl0sWzMsNSwiXFxvcGVyYXRvcm5hbWV7Y30iXSxbMiwyLCJBIl0sWzIsMSwiXFx2ZG90cyJdLFsxLDAsIlxcYXgiXSxbMCw1LCJDIl0sWzMsNiwiXFxwYXJyIl0sWzYsNSwiRCJdLFszLDcsIkMgXFxwYXJyIEQiXSxbMCwxLCJcXHZkb3RzIl0sWzQsMiwiQiJdLFs0LDEsIlxcdmRvdHMiXSxbNSwwLCJcXGF4Il0sWzYsMSwiXFx2ZG90cyJdLFszLDgsIlxcdmRvdHMiXSxbMyw5LCJGIl0sWzIsMTAsIlxcb3RpbWVzIl0sWzIsMTIsIlxcb3BlcmF0b3JuYW1le2N9Il0sWzIsMTEsIkUgXFxvdGltZXMgRiJdLFsxLDksIkUiXSxbMSw4LCJcXHZkb3RzIl0sWzAsMV0sWzIsMCwiIiwwLHsiY3VydmUiOjJ9XSxbMywyLCIiLDAseyJzdHlsZSI6eyJoZWFkIjp7Im5hbWUiOiJub25lIn19fV0sWzQsMywiIiwwLHsiY3VydmUiOi0yfV0sWzUsNiwiIiwwLHsiY3VydmUiOjJ9XSxbNyw2LCIiLDIseyJjdXJ2ZSI6LTJ9XSxbNiw4LCIiLDIseyJzdHlsZSI6eyJoZWFkIjp7Im5hbWUiOiJub25lIn19fV0sWzQsOSwiIiwyLHsiY3VydmUiOjJ9XSxbOSw1LCIiLDEseyJzdHlsZSI6eyJoZWFkIjp7Im5hbWUiOiJub25lIn19fV0sWzEyLDExLCIiLDEseyJjdXJ2ZSI6Mn1dLFsxMiwxMywiIiwxLHsiY3VydmUiOi0yfV0sWzEzLDcsIiIsMSx7InN0eWxlIjp7ImhlYWQiOnsibmFtZSI6Im5vbmUifX19XSxbOCwxNF0sWzIwLDE5LCIiLDIseyJzdHlsZSI6eyJoZWFkIjp7Im5hbWUiOiJub25lIn19fV0sWzE5LDE2LCIiLDIseyJjdXJ2ZSI6Mn1dLFsxNSwxNiwiIiwwLHsiY3VydmUiOi0yfV0sWzE2LDE4LCIiLDAseyJzdHlsZSI6eyJoZWFkIjp7Im5hbWUiOiJub25lIn19fV0sWzE4LDE3XSxbMTQsMTUsIiIsMSx7InN0eWxlIjp7ImhlYWQiOnsibmFtZSI6Im5vbmUifX19XSxbMTEsMTAsIiIsMSx7InN0eWxlIjp7ImhlYWQiOnsibmFtZSI6Im5vbmUifX19XSxbMTAsMCwiIiwxLHsiY3VydmUiOi0yfV1d
		\[\begin{tikzcd}[column sep = small, row sep = small]
			& \ax &&&& \ax \\
			\vdots && \vdots && \vdots && \vdots \\
			&& A && B \\
			&&& {A \otimes B} \\
			\\
			C &&& {\operatorname{c}} &&& D \\
			&&& \parr \\
			&&& {C \parr D} \\
			& \vdots && \vdots \\
			& E && F \\
			&& \otimes \\
			&& {E \otimes F} \\
			&& {\operatorname{c}}
			\arrow[from=4-4, to=6-4]
			\arrow[curve={height=12pt}, from=3-3, to=4-4]
			\arrow[no head, from=2-3, to=3-3]
			\arrow[curve={height=-12pt}, from=1-2, to=2-3]
			\arrow[curve={height=12pt}, from=6-1, to=7-4]
			\arrow[curve={height=-12pt}, from=6-7, to=7-4]
			\arrow[no head, from=7-4, to=8-4]
			\arrow[curve={height=12pt}, from=1-2, to=2-1]
			\arrow[no head, from=2-1, to=6-1]
			\arrow[curve={height=12pt}, from=1-6, to=2-5]
			\arrow[curve={height=-12pt}, from=1-6, to=2-7]
			\arrow[no head, from=2-7, to=6-7]
			\arrow[from=8-4, to=9-4]
			\arrow[no head, from=9-2, to=10-2]
			\arrow[curve={height=12pt}, from=10-2, to=11-3]
			\arrow[curve={height=-12pt}, from=10-4, to=11-3]
			\arrow[no head, from=11-3, to=12-3]
			\arrow[from=12-3, to=13-3]
			\arrow[no head, from=9-4, to=10-4]
			\arrow[no head, from=2-5, to=3-5]
			\arrow[curve={height=-12pt}, from=3-5, to=4-4]
		\end{tikzcd}\]
		
		
		
		Let $S$ be a switching of $\pi$ so that $\operatorname{Emp}_{\uparrow}F = \operatorname{Visit}_S(F,\uparrow)$ and so that $S(l') = L$, which exists by Lemma \ref{lem:par_link_existence}. Let $t = (x_1,...,x_n)$ be the long pretrip of $\pi$ with respect to $S$ satisfying $x_1 = F\uparrow$. We have by Lemma \ref{lem:stays_contained_par} that $t$ takes the following shape:
		\begin{equation}\label{eq:condemning_shape}
			\uparrow F, ..., \uparrow(C \parr D), \uparrow C, ..., D\downarrow, \uparrow D, ..., C\downarrow, (C \parr D)\downarrow, ..., F\downarrow,...
		\end{equation}
		We have that $D \in \operatorname{Emp}_{\uparrow}B$ so for simplicity, rewrite \eqref{eq:condemning_shape} as $t' = (x_{1 + k},...,x_{n+k})$ for some $k > 0$ (where $i + k$ means $i + k\operatorname{mod} n$) so that $\uparrow B$ occurs to the left of $D \downarrow$ and $B \downarrow$ occurring to the right of $\uparrow D$.
		We have that $C \not\in \operatorname{Emp}_{\uparrow}B$ and so by Corollary \ref{cor:stays_contained_corollary}:
		\begin{equation}
			\uparrow B \text{ occurrs in }\uparrow C,..., D\downarrow\text{ and } B\downarrow\text{ occurrs in }\uparrow D, ..., C\downarrow
		\end{equation}
		However, this implies that $B \in \operatorname{Visit}_S(F,\uparrow)$ which by Lemma \ref{lem:realisation_switching} implies $B \in \operatorname{Emp}_{\uparrow}F$.
		
		By reversing the switching of $l'$ we can similarly show that $A \in \operatorname{Emp}_{\uparrow}F$, contradicting the maximality of $l$. This proves the first claim.
		
		For the second claim, since $\call{O}(\pi) = \operatorname{Emp}_{\uparrow}A \cup \operatorname{Emp}_{\uparrow}B \cup \lbrace A \otimes B\rbrace$ we have by Lemma \ref{lem:par_link_existence} that
		\begin{equation}
			\operatorname{Link}_{\parr,\uparrow}^0(A\otimes B) = \varnothing
		\end{equation}
		and we saw in the proof of Lemma \ref{lem:realisation_switching} that a switching $S$ which realises $\operatorname{Emp}_{\uparrow}A$ is given by setting all switchings arbitrarily except for those in $\operatorname{Link}_{\parr,\uparrow}^0(A\otimes B)$. This means that for any switching $S$ of $\pi$:
		\begin{equation}
			\operatorname{Visit}_{S}(A,\uparrow) = \operatorname{Emp}_{\uparrow}A \qquad\text{and}\qquad \operatorname{Visit}_{S}(B,\uparrow) = \operatorname{Emp}_{\uparrow}B
		\end{equation}
		which is to say the two subproof structures given by removing $A \otimes B$ never admit a short trip, that is, they each satisfy the long trip condition.
	\end{proof}
	\begin{thm}[The Sequentialisation Theorem]\label{thm:sequentialisation}
		A proof structure $\pi$ (possibly with cuts) satisfies the long trip condition if and only $\pi$ is a proof net.
	\end{thm}
	\begin{proof}
		First assume that $\pi$ is cut-free.
		
		We proceed by induction on the size $|\operatorname{Link}\pi|$ of the set $\operatorname{Link}\pi$. If there this is zero then $\pi$ consists of a single axiom link and so the result is clear.
		
		For the inductive step, we consider two cases, first say $\pi$ admits a par link for a conclusion. Then removing this par link clearly results in two cut-free subproof structures satsifying the long trip condition and so the result follows from the inductive hypothesis. If no such terminal par link exists, then by the Separation Lemma there exists some tensor link in the conclusion for which we can remove and apply the inductive hypothesis.
		
		Now say that $\pi$ contained cuts. We replace each cut with a tensor link to create a new proof $\zeta$. That there exists a proof $\Xi$ which maps to $\zeta$ follows from the part of the result proved already as $\zeta$ is cut-free. We adapt $\Xi$ appropriately by replacing $\otimes$-rules by $\operatorname{cut}$-rules and we are done.
	\end{proof}
\section{The dynamics of MLL}\label{sec:dynamics}
\begin{defn}
	A subgraph of a proof structure $\pi$ of one of the following forms, where $v_1 \neq v_2$ are distinct vertices in $\pi$, is an \emph{$a$-redex}.
	\begin{center}
		\begin{tabular}{ c c c }
			$
			\begin{tikzcd}[column sep = small, row sep = small]
				& \ax\arrow[dr,bend left, dash]\arrow[dl,bend right, dash] &&& v_2\arrow[d,dash]\\
				\neg A\arrow[d] && A\arrow[dr,bend right] && \neg A\arrow[dl,bend left]\\
				v_1&&&\cut
			\end{tikzcd}
			$
			&
			$
			\begin{tikzcd}[column sep = small, row sep = small]
				v_1\arrow[d,dash] &&& \ax\arrow[dl, bend right, dash]\arrow[dr,dash, bend left]\\
				A\arrow[dr, bend right] && \neg A\arrow[dl,bend left] && A\arrow[d]\\
				& \cut &&& v_2
			\end{tikzcd}
			$
			&
			\tagarray{\label{eq:a_redex}}
		\end{tabular}
	\end{center}
	A subgraph of a proof structure of the following form where $v_1,v_2,v_3,v_4$ are vertices in $\pi$ is an \emph{$m$-redex}.
	\begin{center}
		\begin{tabular}{ c c }
			$
			% https://q.uiver.app/?q=WzAsMTQsWzAsMF0sWzIsMSwiQSJdLFszLDIsIlxcb3RpbWVzIl0sWzQsMSwiQiJdLFs2LDAsInZfMyJdLFs2LDEsIlxcbmVnIEEiXSxbNywyLCJcXHBhcnIiXSxbOCwxLCJcXG5lZyBCIl0sWzgsMCwidl80Il0sWzMsMywiQSBcXG90aW1lcyBCIl0sWzcsMywiXFxuZWcgQSBcXHBhcnIgXFxuZWcgQiJdLFs1LDQsIlxcY3V0Il0sWzIsMCwidl8xIl0sWzQsMCwidl8yIl0sWzEsMiwiIiwwLHsiY3VydmUiOjJ9XSxbMywyLCIiLDAseyJjdXJ2ZSI6LTJ9XSxbNSw2LCIiLDAseyJjdXJ2ZSI6Mn1dLFs3LDYsIiIsMCx7ImN1cnZlIjotMn1dLFs0LDUsIiIsMix7InN0eWxlIjp7ImhlYWQiOnsibmFtZSI6Im5vbmUifX19XSxbOCw3LCIiLDAseyJzdHlsZSI6eyJoZWFkIjp7Im5hbWUiOiJub25lIn19fV0sWzYsMTAsIiIsMCx7InN0eWxlIjp7ImhlYWQiOnsibmFtZSI6Im5vbmUifX19XSxbOSwxMSwiIiwwLHsiY3VydmUiOjJ9XSxbMTAsMTEsIiIsMSx7ImN1cnZlIjotMn1dLFsyLDksIiIsMSx7InN0eWxlIjp7ImhlYWQiOnsibmFtZSI6Im5vbmUifX19XSxbMTIsMSwiIiwwLHsic3R5bGUiOnsiaGVhZCI6eyJuYW1lIjoibm9uZSJ9fX1dLFsxMywzLCIiLDAseyJzdHlsZSI6eyJoZWFkIjp7Im5hbWUiOiJub25lIn19fV1d
			\begin{tikzcd}[column sep = small, row sep = small]
				{} && {v_1} && {v_2} && {v_3} && {v_4} \\
				&& A && B && {\neg A} && {\neg B} \\
				&&& \otimes &&&& \parr \\
				&&& {A \otimes B} &&&& {\neg A \parr \neg B} \\
				&&&&& \cut
				\arrow[curve={height=12pt}, from=2-3, to=3-4]
				\arrow[curve={height=-12pt}, from=2-5, to=3-4]
				\arrow[curve={height=12pt}, from=2-7, to=3-8]
				\arrow[curve={height=-12pt}, from=2-9, to=3-8]
				\arrow[no head, from=1-7, to=2-7]
				\arrow[no head, from=1-9, to=2-9]
				\arrow[no head, from=3-8, to=4-8]
				\arrow[curve={height=12pt}, from=4-4, to=5-6]
				\arrow[curve={height=-12pt}, from=4-8, to=5-6]
				\arrow[no head, from=3-4, to=4-4]
				\arrow[no head, from=1-3, to=2-3]
				\arrow[no head, from=1-5, to=2-5]
			\end{tikzcd}
			$
			&
			\tagarray{\label{eq:m_redex}}
		\end{tabular}
	\end{center}
\end{defn}


\begin{defn}\label{def:reduction}
	Multiplicative linear logic proof structures come equipped with three reduction rules, these reduction rules apply to proofs which admit either an $a$-redex or an $m$-redex. More precisely, given a multiplicative linear logic proof structure $\pi$ admitting an $a$-redex $\zeta$ of the form given on the left in \eqref{rule:ax_reduct_negation}, the reduction of $\pi$ is the proof $\pi'$ given by replacing the subgraph $\zeta$ in $\pi$ by what is displayed on the right in \eqref{rule:ax_reduct_negation}. Similarly for if $\pi$ admits an $a$-redex of the form given on the left of \eqref{rule:ax_reduct} or if $\pi$ admits an $m$-redex.
	\begin{center}
		\begin{tabular}{ c c c }
			$
			\begin{tikzcd}[column sep = small, row sep = small]
				& \ax\arrow[dr,bend left, dash]\arrow[dl,bend right, dash] &&& \vdots\arrow[d,dash]\\
				\neg A\arrow[d] && A\arrow[dr,bend right] && \neg A\arrow[dl,bend left]\\
				\vdots&&&\cut
			\end{tikzcd}
			$
			&
			$
			\begin{tikzcd}[column sep = small, row sep = small]
				\vdots\arrow[d,dash]\\
				\neg A\arrow[d]\\
				\vdots
			\end{tikzcd}
			$
			&
			\tagarray{\label{rule:ax_reduct_negation}}\\
			$
			\begin{tikzcd}[column sep = small, row sep = small]
				\vdots\arrow[d,dash] &&& \ax\arrow[dl, bend right, dash]\arrow[dr,dash, bend left]\\
				A\arrow[dr, bend right] && \neg A\arrow[dl,bend left] && A\arrow[d]\\
				& \cut &&& \vdots
			\end{tikzcd}
			$
			&
			$
			\begin{tikzcd}[column sep = small, row sep = small]
				\vdots\arrow[d,dash]\\
				A\arrow[d]\\
				\vdots
			\end{tikzcd}
			$
			&
			\tagarray{\label{rule:ax_reduct}}
		\end{tabular}
	\end{center}
	
	\begin{center}
		\begin{tabular}{ c c }
			$\begin{tikzcd}[column sep = small, row sep = small]
				\vdots && \vdots && \vdots && \vdots \\
				A && B && {\neg A} && {\neg B} \\
				& \otimes &&&& \parr \\
				& {A \otimes B} &&&& {\neg A \parr \neg B} \\
				&&& \cut
				%			\arrow[curve={height=12pt}, from=4-2, to=5-4]
				%			\arrow[curve={height=-12pt}, from=4-6, to=5-4]
				\arrow[from=2-5, to=3-6]
				\arrow[from=2-7, to=3-6]
				\arrow[from=3-6, to=4-6, dash]
				\arrow[from=3-2, to=4-2, dash]
				\arrow[from=2-1, to=3-2]
				\arrow[from=2-3, to=3-2]
				\arrow[from=1-1, to=2-1, dash]
				\arrow[from=1-3, to=2-3, dash]
				\arrow[from=1-5, to=2-5, dash]
				\arrow[from=1-7, to=2-7, dash]
				\arrow[from=4-2, to=5-4, bend right]
				\arrow[from=4-6, to=5-4, bend left]
			\end{tikzcd}$\\
			& \tagarray{\label{rule:tensor_reduct}}\\
			$\begin{tikzcd}[column sep = small, row sep = small]
				\vdots && \vdots && \vdots && \vdots \\
				A && B && {\neg A} && {\neg B} \\
				& \cut &&&& \cut
				\arrow[from=1-1, to=2-1, dash]
				\arrow[from=1-3, to=2-3, dash]
				\arrow[from=1-5, to=2-5, dash]
				\arrow[from=1-7, to=2-7, dash]
				\arrow[from=2-1, to=3-2, bend right]
				\arrow[from=2-5, to=3-2, bend left]
				\arrow[from=2-3, to=3-6, bend right]
				\arrow[from=2-7, to=3-6, bend left]
			\end{tikzcd}$
		\end{tabular}
	\end{center}
	A \textbf{reduction} is a pair of proof structures $(\pi,\pi')$ where $\pi'$ is the result of applying one of the reduction rules just described to $\pi$. We write $\pi \lto_{\cut} \pi'$ when $(\pi,\pi')$ is a reduction.
\end{defn}
\begin{proposition}[Church-Rosser]\label{prop:church_rosser}
	If $\pi_1$ is a proof structure and $\pi_1 \lto_{\operatorname{cut}} \pi_2, \pi_1 \lto_{\operatorname{cut}} \pi_3$ then there exists a proof structure $\pi_4$ such that $\pi_2 \lto_{\operatorname{cut}} \pi_4, \pi_3 \lto_{\operatorname{cut}} \pi_4$.
\end{proposition}
\begin{proof}[Proof sketch]
	The key observation is that reducing any redex in a proof does not eliminate any other redex.
\end{proof}
\begin{defn}
	Let $\pi$ be a proof net possibly containing cut links. A \textbf{reduction sequence} is a sequence
	\begin{equation}
		\pi = \pi_0 \lto_{\operatorname{cut}} \pi_1 \lto_{\operatorname{cut}} \hdots \lto_{\operatorname{cut}} \pi_n
	\end{equation}
	with $\pi_n$ cut-free.
\end{defn}
\begin{lemma}\label{lem:red_sequence_existence}
	Every proof net $\pi$ admits a reduction sequence.
\end{lemma}
\begin{proof}
	Given a cut link $l$ with premises $A, \neg A$ say, the \textbf{complexity of $l$}, $c(l)$ is the sum of the number of occurrences of $\otimes$ and the number of occurrences of $\parr$ in $A$. We proceed by induction on the maximum of the complexities of all cut links in $\pi$.
	
	Say this maximum is $0$. Then all cut-links have the shape of either \eqref{rule:ax_reduct_negation} or \eqref{rule:ax_reduct} (using the fact that $\pi$ is a \emph{proof net}, the hypthesis that $v_1 \neg v_2$ in Definition \ref{def:reduction} is satisfied). We can reduce these redexes (in any order) to deduce the result.
	
	Now say the maximum is $n > 0$. We then apply \eqref{rule:tensor_reduct} to all cut links of complexity $n$ (in any order) to obtain a new proof structure $\zeta$. We wish to use the inductive hypothesis on $\zeta$ but we must make sure that $\zeta$ satisfies the longtrip condition. This follows easily by considering the contrapositive. Any pretrip (long or short) of $\zeta$ appears as a subsequent of some pretrip of $\pi$, so if $\zeta$ admits a short trip so does $\pi$. We can now apply the inductive hypothesis and we are done.
\end{proof}
\begin{defn}
	Let $\operatorname{Red}\pi$ denote the set of all reduction sequences of $\pi$. The \textbf{length} $l(\underline{x})$ of a reduction sequence $\underline{x} \in \operatorname{Red}\pi$ is the length of the sequence $\underline{x}$.
\end{defn}
\begin{cor}\label{cor:stable_length}
	The length of a reduction path is independent of the choice of reduction path.
\end{cor}
\begin{proof}
	The proof is purely geometric. Let
	\begin{equation}
		\underline{x} := ( \pi = \pi_1 \lto_{\operatorname{cut}} \hdots \lto_{\operatorname{cut}} \pi_n)
	\end{equation}
	be the reduction path described by Lemma \ref{lem:red_sequence_existence} and let
	\begin{equation}
		\underline{y} := (\pi = \zeta_0 \lto_{\operatorname{cut}} \hdots \lto_{\operatorname{cut}} \zeta_n)
	\end{equation}
	be any other reduction sequence. By Lemma \ref{prop:church_rosser} we have $\pi_n = \zeta_n$. Also using \ref{prop:church_rosser}, the pair of reduction paths can be completed to some grid defined by a subset of $\bb{N} \times \bb{N}$. All paths $p$ consisting of only upwards steps or right steps such that $p$ is bound to this grid have the same length and so $l(\underline{x}) = l(\underline{y})$.
\end{proof}
\begin{defn}\label{def:normal_form}
	The proof of Corollary \ref{cor:stable_length} shows that every reduction path of a proof net $\pi$ leads to the same cut-free proof $\zeta$. We call $\zeta$ the \textbf{normal form} of $\pi$.
\end{defn}
\begin{defn}\label{def:eta_expansion}
	A pair of proof nets $(\pi,\pi')$ where $\pi'$ is obtained from $\pi$ via replacing some subgraph of $\pi$ of the form on the left of \eqref{eq:eta_expansion} is an \textbf{$\eta$-expansion}. We write $\pi \lto_{\eta} \pi'$. An \textbf{$\eta$-increx} is a subgraph of a proof structure of the form given on the left of \eqref{eq:eta_expansion}.
\end{defn}
% https://q.uiver.app/?q=WzAsNSxbMSwwLCJcXGF4Il0sWzIsMSwiXFxuZWcgQSBcXHBhcnIgXFxuZWcgQiJdLFswLDEsIkEgXFxvdGltZXMgQiJdLFswLDIsIlxcdmRvdHMiXSxbMiwyLCJcXHZkb3RzIl0sWzAsMSwiIiwyLHsiY3VydmUiOi0yLCJzdHlsZSI6eyJoZWFkIjp7Im5hbWUiOiJub25lIn19fV0sWzAsMiwiIiwwLHsiY3VydmUiOjIsInN0eWxlIjp7ImhlYWQiOnsibmFtZSI6Im5vbmUifX19XSxbMiwzXSxbMSw0XV0=
\begin{equation}\label{eq:eta_expansion}
	\begin{tikzcd}[column sep = small, row sep = small]
		& \ax \\
		{A \otimes B} && {\neg A \parr \neg B} \\
		\vdots && \vdots
		\arrow[curve={height=-12pt}, no head, from=1-2, to=2-3]
		\arrow[curve={height=12pt}, no head, from=1-2, to=2-1]
		\arrow[from=2-1, to=3-1]
		\arrow[from=2-3, to=3-3]
	\end{tikzcd}
	% https://q.uiver.app/?q=WzAsMTIsWzEsMCwiXFxheCJdLFswLDEsIkEiXSxbMiwxLCJcXG5lZyBBIl0sWzQsMCwiXFxheCJdLFszLDEsIkIiXSxbNSwxLCJcXG5lZyBCICJdLFsxLDIsIlxcb3RpbWVzIl0sWzMsMiwiXFxwYXJyIl0sWzMsMywiXFxuZWcgQSBcXHBhcnIgXFxuZWcgQiJdLFszLDQsIlxcdmRvdHMiXSxbMSwzLCJBIFxcb3RpbWVzIEIiXSxbMSw0LCJcXHZkb3RzIl0sWzMsNCwiIiwwLHsiY3VydmUiOjIsInN0eWxlIjp7ImhlYWQiOnsibmFtZSI6Im5vbmUifX19XSxbMyw1LCIiLDIseyJjdXJ2ZSI6LTIsInN0eWxlIjp7ImhlYWQiOnsibmFtZSI6Im5vbmUifX19XSxbMCwxLCIiLDIseyJjdXJ2ZSI6Miwic3R5bGUiOnsiaGVhZCI6eyJuYW1lIjoibm9uZSJ9fX1dLFswLDIsIiIsMCx7ImN1cnZlIjotMiwic3R5bGUiOnsiaGVhZCI6eyJuYW1lIjoibm9uZSJ9fX1dLFsxLDYsIiIsMix7ImN1cnZlIjoyfV0sWzQsNiwiIiwwLHsiY3VydmUiOi0yfV0sWzIsNywiIiwwLHsiY3VydmUiOjJ9XSxbNSw3LCIiLDIseyJjdXJ2ZSI6LTJ9XSxbNyw4LCIiLDIseyJzdHlsZSI6eyJoZWFkIjp7Im5hbWUiOiJub25lIn19fV0sWzgsOV0sWzYsMTAsIiIsMix7InN0eWxlIjp7ImhlYWQiOnsibmFtZSI6Im5vbmUifX19XSxbMTAsMTFdXQ==
	\begin{tikzcd}[column sep = small, row sep = small]
		& \ax &&& \ax \\
		A && {\neg A} & B && {\neg B } \\
		& \otimes && \parr \\
		& {A \otimes B} && {\neg A \parr \neg B} \\
		& \vdots && \vdots
		\arrow[curve={height=12pt}, no head, from=1-5, to=2-4]
		\arrow[curve={height=-12pt}, no head, from=1-5, to=2-6]
		\arrow[curve={height=12pt}, no head, from=1-2, to=2-1]
		\arrow[curve={height=-12pt}, no head, from=1-2, to=2-3]
		\arrow[curve={height=12pt}, from=2-1, to=3-2]
		\arrow[curve={height=-12pt}, from=2-4, to=3-2]
		\arrow[curve={height=12pt}, from=2-3, to=3-4]
		\arrow[curve={height=-12pt}, from=2-6, to=3-4]
		\arrow[no head, from=3-4, to=4-4]
		\arrow[from=4-4, to=5-4]
		\arrow[no head, from=3-2, to=4-2]
		\arrow[from=4-2, to=5-2]
	\end{tikzcd}
\end{equation}

\begin{lemma}\label{lem:eta_confluence}
	If $\pi \lto_{\eta}\pi'$ and $\pi \lto_{\eta} \pi''$ then there exists $\pi'''$ such that $\pi' \lto_\eta \pi'''$ and $\pi'' \lto_{\eta} \pi'''$.
\end{lemma}
\begin{proof}
	Any $\eta$-expansion does not alter any other $\eta$-increx.
\end{proof}
Due to the finiteness of proof nets and their formulas, the process of choosing any $\eta$-increx and reducing it eventually termintates and results in a proof net with no $\eta$-increxes.
\begin{cor}\label{cor:terminating_eta_cut}
	The process of reducing all redexes in a proof net $\pi$ and then expanding all $\eta$-increx terminates and results in a cut-free proof net where all conclusions of all axiom links are atomic.
\end{cor}
\begin{defn}
	The result of applying process \ref{cor:terminating_eta_cut} to a proof net $\pi$ is the \textbf{super normal form} of $\pi$.
\end{defn}

	\section{Proofs as permutations}
	\begin{defn}
		Let $\pi$ be a proof net. Let $\call{P}(\pi)$ denote the disjoint union of all the unoriented axioms of all formulas which are conclusions to axiom links in $\pi$.
	\end{defn}
	\begin{example}\label{ex:permutation_example}
		Let $\pi$ denote the following proof net, where $X,Y$ are atomic. For simplicity, we usually attach artificial labels to the conclusions of axiom links. So, in the following example, for any integer $i$ and any $Z = X,Y$, the notation $Z_i$ simply means $Z$, also $Z$ means $(Z,+)$ and $\neg Z$ means $(Z,-)$.
		% https://q.uiver.app/?q=WzAsMTEsWzAsMSwiKFggLCspIFxcb3RpbWVzIChZLC0pIl0sWzIsMSwiKFgsIC0pIFxccGFyciAoWSwrKSJdLFsxLDAsIlxcYXgiXSxbNSwwLCJcXGF4Il0sWzQsMSwiKFgsKykiXSxbNiwxLCIoWSwtKSJdLFszLDIsIlxcb3RpbWVzIl0sWzMsMywiKChYLC0pXFxwYXJyKFksKykpXFxvdGltZXMgKFgsKykiXSxbMCwyLCJcXG9wZXJhdG9ybmFtZXtjfSJdLFs2LDIsIlxcb3BlcmF0b3JuYW1le2N9Il0sWzMsNCwiXFxvcGVyYXRvcm5hbWV7Y30iXSxbMiwxLCIiLDAseyJjdXJ2ZSI6LTIsInN0eWxlIjp7ImhlYWQiOnsibmFtZSI6Im5vbmUifX19XSxbMiwwLCIiLDIseyJjdXJ2ZSI6Miwic3R5bGUiOnsiaGVhZCI6eyJuYW1lIjoibm9uZSJ9fX1dLFszLDQsIiIsMCx7ImN1cnZlIjoyLCJzdHlsZSI6eyJoZWFkIjp7Im5hbWUiOiJub25lIn19fV0sWzMsNSwiIiwyLHsiY3VydmUiOi0yLCJzdHlsZSI6eyJoZWFkIjp7Im5hbWUiOiJub25lIn19fV0sWzYsNywiIiwyLHsic3R5bGUiOnsiaGVhZCI6eyJuYW1lIjoibm9uZSJ9fX1dLFs0LDYsIiIsMSx7ImN1cnZlIjotMn1dLFsxLDYsIiIsMSx7ImN1cnZlIjoyfV0sWzcsMTBdLFs1LDldLFswLDhdXQ==
		\[\begin{tikzcd}[column sep = tiny, row sep = small]
			& \ax &&&& \ax \\
			{X_1 \otimes \neg Y_2} && {\neg X_3 \parr Y_4} && {X_5} && {\neg Y_6} \\
			{\operatorname{c}} &&& \otimes &&& {\operatorname{c}} \\
			&&& {(\neg X_7\parr Y_8)\otimes X_9} \\
			&&& {\operatorname{c}}
			\arrow[curve={height=-12pt}, no head, from=1-2, to=2-3]
			\arrow[curve={height=12pt}, no head, from=1-2, to=2-1]
			\arrow[curve={height=12pt}, no head, from=1-6, to=2-5]
			\arrow[curve={height=-12pt}, no head, from=1-6, to=2-7]
			\arrow[no head, from=3-4, to=4-4]
			\arrow[curve={height=-12pt}, from=2-5, to=3-4]
			\arrow[curve={height=12pt}, from=2-3, to=3-4]
			\arrow[from=4-4, to=5-4]
			\arrow[from=2-7, to=3-7]
			\arrow[from=2-1, to=3-1]
		\end{tikzcd}\]
		Then:
		\begin{align}
			\call{P}(\pi) &= \lbrace X \rbrace \coprod \lbrace Y \rbrace \coprod \lbrace X \rbrace \coprod \lbrace Y \rbrace \coprod \lbrace X \rbrace \coprod \lbrace Y \rbrace \coprod \lbrace X_7 \rbrace \coprod \lbrace Y_8 \rbrace \coprod \lbrace X_9 \rbrace\\
			&= \lbrace X_1,Y_2,X_3,Y_4,X_5,Y_6, X_7, Y_8, X_9\rbrace
		\end{align}
		
	\end{example}
	\begin{defn}\label{def:seq_set}
		Let $A$ be a formula with sequence of oriented atoms $\big((X_1,x_1),...,(X_n,x_n)\big)$. The \textbf{sequence of unoriented atoms} of $A$ is $(X_1,...,X_n)$ and the \textbf{set of unoriented atoms} of $A$ is the disjoint union $\lbrace X_1 \rbrace \coprod \hdots \coprod \lbrace X_n \rbrace$.
	\end{defn}
	\begin{defn}\label{def:permutations}
		Let $\pi$ be a proof net with axiom links $l_1,...,l_n$ say. For each $i = 1,...,n$ the link $l_i$ defines a permutation $\tau_{l_i}$ on the set $\call{P}(\pi)$ in the following way: if $l_i$ has conclusions $A, \neg A$ then the $j^\text{th}$ element of the sequence of unoriented atoms of $A$ is mapped via $\tau_{l_i}$ to the $j^\text{th}$ element of the sequence of unoriented atoms of $\neg A$. We define $\alpha_\pi$ to be the product of all these permutations.
		\begin{equation}
			\alpha_\pi := \tau_{l_1}\hdots \tau_{l_n}
		\end{equation}
		We call this permutation the \textbf{axiom link permutation associated to $\pi$}.
		
		There is another permutation on $\call{P}(\pi)$ defined symmetrically but where the cut links are considered rather than the axiom links. We denote this permutation $\gamma_\pi$.
		
		There is yet another permutation on $\call{P}(\pi)$. Let $S$ be a switching of $\pi$. For each unoriented axiom $X \in \call{P}(\pi)$, corresponding to the formula $A$ say, let $\beta_{\pi}^S(X)$ denote the unoriented axiom corresponding to the first occurrence in $\operatorname{PTrip}(\pi,S,A,\downarrow)$ (Definition \ref{def:pretrip_from_A}) of the form $\uparrow B$ where $B$ is a formula labelling a conclusion of an axiom link in $\pi$.
		
		The set of all premutations of the second form is denoted:
		\begin{equation}
			\Sigma(\pi) := \lbrace \beta_{\pi}^S \mid S\text{ is a switching of }\pi\rbrace
		\end{equation}
		We will often denote elements of $\beta_{\pi}^S \in \Sigma(\pi)$ simply by $\beta$.
	\end{defn}
	\begin{example}
		Continuing with Example \ref{ex:permutation_example}, and using the artifical labels we attributed to the atomic formulas of conclusions, we have $\alpha_\pi = (13)(24)(56)$ and since $\pi$ admits no cut links we have $\gamma_\pi = \operatorname{id}_{\call{P}(\pi)}$.
		
		The set $\Sigma(\pi)$ is uninteresting so we consider a more complicated example. Before doing this though, we introduce a simple Lemma for the sake of easing notation.
	\end{example}
	\begin{lemma}\label{lem:pars_or_not}
		Let $\pi$ be a proof net with conclusions $A_1,...,A_n$ and let $\zeta$ be a proof net obtained by beginning with $\pi$ and in any order forming par links which connect all the conclusions $A_1,...,A_n$ so that $\zeta$ has conclusions $B_1,...,B_m$ where $m \leq n$ and each $B_i$ is constructed only by $\parr$ and a subset of the formulas $A_1,...,A_n$. Then $\Sigma(\pi) = \Sigma(\zeta)$.
	\end{lemma}
	\begin{proof}
		Easy proof by induction on the integer given by the number of par links in $\zeta$ minus the number of par links in $\pi$.
	\end{proof}
	\begin{example}\label{ex:counter}
		\begin{figure}[h]
			\centering
			\includegraphics[width = 17cm]{Permutations.png}
			\caption{The set $\Sigma(\pi_1)$}
			\label{fig:counter_one}
		\end{figure}
		Let $\zeta$ be defined as follows, again, we explicitly place integer labells on the formulas, but $A_i$ denotes the formula $A$ for all $i$.
		% https://q.uiver.app/?q=WzAsMjMsWzAsMSwiQSJdLFsyLDEsIlxcbmVnIEEiXSxbMywxLCJBIl0sWzUsMSwiXFxuZWcgQSJdLFsyLDMsIlxcb3RpbWVzIl0sWzIsNCwiQSBcXG90aW1lcyBBIl0sWzEsMCwiXFxheCJdLFs0LDAsIlxcYXgiXSxbNiwxLCJBIl0sWzgsMSwiXFxuZWcgQSJdLFs5LDEsIkEiXSxbMTEsMSwiXFxuZWcgQSJdLFsxMCwwLCJcXGF4Il0sWzcsMCwiXFxheCJdLFs4LDMsIlxcb3RpbWVzIl0sWzgsNCwiQSBcXG90aW1lcyBBIl0sWzUsNSwiXFxvdGltZXMiXSxbNSw2LCIoQSBcXG90aW1lcyBBKSBcXG90aW1lcyAoQSBcXG90aW1lcyBBKSJdLFs1LDcsIlxcb3BlcmF0b3JuYW1le2N9Il0sWzIsMiwiXFxvcGVyYXRvcm5hbWV7Y30iXSxbNSwyLCJcXG9wZXJhdG9ybmFtZXtjfSJdLFs4LDIsIlxcb3BlcmF0b3JuYW1le2N9Il0sWzExLDIsIlxcb3BlcmF0b3JuYW1le2N9Il0sWzAsNCwiIiwwLHsiY3VydmUiOjJ9XSxbMiw0LCIiLDIseyJjdXJ2ZSI6LTJ9XSxbNCw1LCIiLDAseyJzdHlsZSI6eyJoZWFkIjp7Im5hbWUiOiJub25lIn19fV0sWzYsMCwiIiwwLHsiY3VydmUiOjIsInN0eWxlIjp7ImhlYWQiOnsibmFtZSI6Im5vbmUifX19XSxbNiwxLCIiLDIseyJjdXJ2ZSI6LTIsInN0eWxlIjp7ImhlYWQiOnsibmFtZSI6Im5vbmUifX19XSxbNywyLCIiLDIseyJjdXJ2ZSI6Miwic3R5bGUiOnsiaGVhZCI6eyJuYW1lIjoibm9uZSJ9fX1dLFs3LDMsIiIsMCx7ImN1cnZlIjotMiwic3R5bGUiOnsiaGVhZCI6eyJuYW1lIjoibm9uZSJ9fX1dLFsxMyw4LCIiLDAseyJjdXJ2ZSI6Miwic3R5bGUiOnsiaGVhZCI6eyJuYW1lIjoibm9uZSJ9fX1dLFsxMyw5LCIiLDIseyJjdXJ2ZSI6LTIsInN0eWxlIjp7ImhlYWQiOnsibmFtZSI6Im5vbmUifX19XSxbMTIsMTAsIiIsMix7ImN1cnZlIjoyLCJzdHlsZSI6eyJoZWFkIjp7Im5hbWUiOiJub25lIn19fV0sWzEyLDExLCIiLDAseyJjdXJ2ZSI6LTIsInN0eWxlIjp7ImhlYWQiOnsibmFtZSI6Im5vbmUifX19XSxbOCwxNCwiIiwwLHsiY3VydmUiOjJ9XSxbMTAsMTQsIiIsMix7ImN1cnZlIjotMn1dLFsxNCwxNSwiIiwyLHsic3R5bGUiOnsiaGVhZCI6eyJuYW1lIjoibm9uZSJ9fX1dLFs1LDE2LCIiLDAseyJjdXJ2ZSI6Mn1dLFsxNSwxNiwiIiwyLHsiY3VydmUiOi0yfV0sWzE2LDE3LCIiLDIseyJzdHlsZSI6eyJoZWFkIjp7Im5hbWUiOiJub25lIn19fV0sWzE3LDE4XSxbMSwxOV0sWzMsMjBdLFs5LDIxXSxbMTEsMjJdXQ==
		\[\begin{tikzcd}[column sep  = tiny, row sep = tiny]
			& \ax &&& \ax &&& \ax &&& \ax \\
			A_1 && {\neg A_2} & A_3 && {\neg A_4} & A_5 && {\neg A_6} & A_7 && {\neg A_8} \\
			&& {\operatorname{c}} &&& {\operatorname{c}} &&& {\operatorname{c}} &&& {\operatorname{c}} \\
			&& \otimes &&&&&& \otimes \\
			&& {A \otimes A} &&&&&& A \otimes A \\
			&&&&& \otimes \\
			&&&&& {(A \otimes A) \otimes (A \otimes A)} \\
			&&&&& {\operatorname{c}}
			\arrow[curve={height=12pt}, from=2-1, to=4-3]
			\arrow[curve={height=-12pt}, from=2-4, to=4-3]
			\arrow[no head, from=4-3, to=5-3]
			\arrow[curve={height=12pt}, no head, from=1-2, to=2-1]
			\arrow[curve={height=-12pt}, no head, from=1-2, to=2-3]
			\arrow[curve={height=12pt}, no head, from=1-5, to=2-4]
			\arrow[curve={height=-12pt}, no head, from=1-5, to=2-6]
			\arrow[curve={height=12pt}, no head, from=1-8, to=2-7]
			\arrow[curve={height=-12pt}, no head, from=1-8, to=2-9]
			\arrow[curve={height=12pt}, no head, from=1-11, to=2-10]
			\arrow[curve={height=-12pt}, no head, from=1-11, to=2-12]
			\arrow[curve={height=12pt}, from=2-7, to=4-9]
			\arrow[curve={height=-12pt}, from=2-10, to=4-9]
			\arrow[no head, from=4-9, to=5-9]
			\arrow[curve={height=12pt}, from=5-3, to=6-6]
			\arrow[curve={height=-12pt}, from=5-9, to=6-6]
			\arrow[no head, from=6-6, to=7-6]
			\arrow[from=7-6, to=8-6]
			\arrow[from=2-3, to=3-3]
			\arrow[from=2-6, to=3-6]
			\arrow[from=2-9, to=3-9]
			\arrow[from=2-12, to=3-12]
		\end{tikzcd}\]
		To ease notation yet further, In Figure \ref{fig:counter_one} we suppress the nodes, the edges leading to $\operatorname{c}$ nodes, and the direction of the edges. We see from Figure \ref{fig:counter_one} that we can write down $\Sigma(\pi_1)$:
		\begin{equation}
			\Sigma(\pi_1) = \lbrace (1375), (1357), (1753), (1573)\rbrace
		\end{equation}
		\begin{remark}\label{rmk:trunk}
			Clearly, the set $\Sigma(\pi_1)$ only depends on the typing tree (the cut-free sub-proof structure corresponding to a provable formula $A$ as described in Remark \ref{rmk:trunk}), which in turn only depends on the formula $A$.
		\end{remark}
	\end{example}
	We can now rephrase the longtrip condition of Section \ref{sec:sequentialisation} in terms of permutations.
	\begin{proposition}
		Let $\pi$ be a proof structure, then $\pi$ is a proof net if and only if for all $\beta \in \Sigma(\pi)$ the permutation $\alpha_{\pi}\beta$ is cyclic.
	\end{proposition}
	If $\pi$ is a proof net admitting a single conclusion, then it is \emph{not} the case that $\alpha_\pi$ uniquely determines this proof net, for instance, the following two pairs of proof nets admit the same axiom link permutations.
	% https://q.uiver.app/?q=WzAsMTYsWzEsMCwiXFxheCJdLFswLDEsIkEiXSxbMiwxLCJcXG5lZyBBIl0sWzQsMSwiQSJdLFs1LDAsIlxcYXgiXSxbNiwxLCJcXG5lZyBBIl0sWzcsMiwiXFxjdXQiXSxbOCwxLCJBIl0sWzksMCwiXFxheCJdLFsxMCwxLCJcXG5lZyBBIl0sWzEsMiwiXFxwYXJyIl0sWzEsMywiWCBcXHBhcnIgXFxuZWcgWCJdLFsxLDQsIlxcb3BlcmF0b3JuYW1le2N9Il0sWzcsMywiXFxwYXJyIl0sWzcsNCwiWCBcXHBhcnIgXFxuZWcgWCJdLFs3LDUsIlxcb3BlcmF0b3JuYW1le2N9Il0sWzAsMSwiIiwwLHsiY3VydmUiOjIsInN0eWxlIjp7ImhlYWQiOnsibmFtZSI6Im5vbmUifX19XSxbMCwyLCIiLDIseyJjdXJ2ZSI6LTIsInN0eWxlIjp7ImhlYWQiOnsibmFtZSI6Im5vbmUifX19XSxbNSw2LCIiLDIseyJjdXJ2ZSI6Mn1dLFs3LDYsIiIsMCx7ImN1cnZlIjotMn1dLFs4LDcsIiIsMCx7ImN1cnZlIjoyLCJzdHlsZSI6eyJoZWFkIjp7Im5hbWUiOiJub25lIn19fV0sWzgsOSwiIiwyLHsiY3VydmUiOi0yLCJzdHlsZSI6eyJoZWFkIjp7Im5hbWUiOiJub25lIn19fV0sWzQsNSwiIiwyLHsiY3VydmUiOi0yLCJzdHlsZSI6eyJoZWFkIjp7Im5hbWUiOiJub25lIn19fV0sWzQsMywiIiwxLHsiY3VydmUiOjIsInN0eWxlIjp7ImhlYWQiOnsibmFtZSI6Im5vbmUifX19XSxbMSwxMCwiIiwwLHsiY3VydmUiOjJ9XSxbMiwxMCwiIiwyLHsiY3VydmUiOi0yfV0sWzEwLDExLCIiLDIseyJzdHlsZSI6eyJoZWFkIjp7Im5hbWUiOiJub25lIn19fV0sWzExLDEyXSxbOSwxMywiIiwyLHsiY3VydmUiOi0yfV0sWzMsMTMsIiIsMix7ImN1cnZlIjoyfV0sWzEzLDE0LCIiLDIseyJzdHlsZSI6eyJoZWFkIjp7Im5hbWUiOiJub25lIn19fV0sWzE0LDE1XV0=
	\[\begin{tikzcd}[column sep = small, row sep = small]
		& \ax &&&& \ax &&&& \ax \\
		A && {\neg A} && A && {\neg A} && A && {\neg A} \\
		& \parr &&&&&& \cut \\
		& {X \parr \neg X} &&&&&& \parr \\
		& {\operatorname{c}} &&&&&& {X \parr \neg X} \\
		&&&&&&& {\operatorname{c}}
		\arrow[curve={height=12pt}, no head, from=1-2, to=2-1]
		\arrow[curve={height=-12pt}, no head, from=1-2, to=2-3]
		\arrow[curve={height=12pt}, from=2-7, to=3-8]
		\arrow[curve={height=-12pt}, from=2-9, to=3-8]
		\arrow[curve={height=12pt}, no head, from=1-10, to=2-9]
		\arrow[curve={height=-12pt}, no head, from=1-10, to=2-11]
		\arrow[curve={height=-12pt}, no head, from=1-6, to=2-7]
		\arrow[curve={height=12pt}, no head, from=1-6, to=2-5]
		\arrow[curve={height=12pt}, from=2-1, to=3-2]
		\arrow[curve={height=-12pt}, from=2-3, to=3-2]
		\arrow[no head, from=3-2, to=4-2]
		\arrow[from=4-2, to=5-2]
		\arrow[curve={height=-12pt}, from=2-11, to=4-8]
		\arrow[curve={height=12pt}, from=2-5, to=4-8]
		\arrow[no head, from=4-8, to=5-8]
		\arrow[from=5-8, to=6-8]
	\end{tikzcd}\]
	% https://q.uiver.app/?q=WzAsMTksWzQsMCwiXFxheCJdLFszLDEsIkEiXSxbNSwxLCJcXG5lZyBBIl0sWzcsMCwiXFxheCJdLFs2LDEsIkIiXSxbOCwxLCJcXG5lZyBCICJdLFs0LDIsIlxcb3RpbWVzIl0sWzYsMiwiXFxwYXJyIl0sWzYsMywiXFxuZWcgQSBcXHBhcnIgXFxuZWcgQiJdLFs0LDMsIkEgXFxvdGltZXMgQiJdLFs1LDQsIlxccGFyciJdLFs1LDUsIihBIFxcb3RpbWVzIEIpIFxccGFyciAoXFxuZWcgQSBcXHBhcnIgXFxuZWcgQikiXSxbNSw2LCJcXG9wZXJhdG9ybmFtZXtjfSJdLFsxLDAsIlxcYXgiXSxbMCwxLCJBIFxcb3RpbWVzIEIiXSxbMiwxLCJcXG5lZyBBIFxccGFyciBcXG5lZyBCIl0sWzEsMiwiXFxwYXJyIl0sWzEsMywiKEEgXFxvdGltZXMgQikgXFxwYXJyIChcXG5lZyBBIFxccGFyciBcXG5lZyBCKSJdLFsxLDQsIlxcb3BlcmF0b3JuYW1le2N9Il0sWzMsNCwiIiwwLHsiY3VydmUiOjIsInN0eWxlIjp7ImhlYWQiOnsibmFtZSI6Im5vbmUifX19XSxbMyw1LCIiLDIseyJjdXJ2ZSI6LTIsInN0eWxlIjp7ImhlYWQiOnsibmFtZSI6Im5vbmUifX19XSxbMCwxLCIiLDIseyJjdXJ2ZSI6Miwic3R5bGUiOnsiaGVhZCI6eyJuYW1lIjoibm9uZSJ9fX1dLFswLDIsIiIsMCx7ImN1cnZlIjotMiwic3R5bGUiOnsiaGVhZCI6eyJuYW1lIjoibm9uZSJ9fX1dLFsxLDYsIiIsMix7ImN1cnZlIjoyfV0sWzQsNiwiIiwwLHsiY3VydmUiOi0yfV0sWzIsNywiIiwwLHsiY3VydmUiOjJ9XSxbNSw3LCIiLDIseyJjdXJ2ZSI6LTJ9XSxbNyw4LCIiLDIseyJzdHlsZSI6eyJoZWFkIjp7Im5hbWUiOiJub25lIn19fV0sWzYsOSwiIiwyLHsic3R5bGUiOnsiaGVhZCI6eyJuYW1lIjoibm9uZSJ9fX1dLFs5LDEwLCIiLDIseyJjdXJ2ZSI6Mn1dLFs4LDEwLCIiLDEseyJjdXJ2ZSI6LTJ9XSxbMTAsMTEsIiIsMSx7InN0eWxlIjp7ImhlYWQiOnsibmFtZSI6Im5vbmUifX19XSxbMTEsMTJdLFsxMywxNCwiIiwwLHsiY3VydmUiOjIsInN0eWxlIjp7ImhlYWQiOnsibmFtZSI6Im5vbmUifX19XSxbMTMsMTUsIiIsMix7ImN1cnZlIjotMiwic3R5bGUiOnsiaGVhZCI6eyJuYW1lIjoibm9uZSJ9fX1dLFsxNSwxNiwiIiwyLHsiY3VydmUiOi0yfV0sWzE0LDE2LCIiLDAseyJjdXJ2ZSI6Mn1dLFsxNiwxNywiIiwwLHsic3R5bGUiOnsiaGVhZCI6eyJuYW1lIjoibm9uZSJ9fX1dLFsxNywxOF1d
	In what follows, $C$ denotes $(A \otimes B) \parr (\neg A \parr \neg B)$.
	\[\begin{tikzcd}[column sep = tiny, row sep = small]
		& \ax &&& \ax &&& \ax \\
		{A \otimes B} && {\neg A \parr \neg B} & A && {\neg A} & B && {\neg B } \\
		& \parr &&& \otimes && \parr \\
		& {C} &&& {A \otimes B} && {\neg A \parr \neg B} \\
		& {\operatorname{c}} &&&& \parr \\
		&&&&& {C} \\
		&&&&& {\operatorname{c}}
		\arrow[curve={height=12pt}, no head, from=1-8, to=2-7]
		\arrow[curve={height=-12pt}, no head, from=1-8, to=2-9]
		\arrow[curve={height=12pt}, no head, from=1-5, to=2-4]
		\arrow[curve={height=-12pt}, no head, from=1-5, to=2-6]
		\arrow[curve={height=12pt}, from=2-4, to=3-5]
		\arrow[curve={height=-12pt}, from=2-7, to=3-5]
		\arrow[curve={height=12pt}, from=2-6, to=3-7]
		\arrow[curve={height=-12pt}, from=2-9, to=3-7]
		\arrow[no head, from=3-7, to=4-7]
		\arrow[no head, from=3-5, to=4-5]
		\arrow[curve={height=12pt}, from=4-5, to=5-6]
		\arrow[curve={height=-12pt}, from=4-7, to=5-6]
		\arrow[no head, from=5-6, to=6-6]
		\arrow[from=6-6, to=7-6]
		\arrow[curve={height=12pt}, no head, from=1-2, to=2-1]
		\arrow[curve={height=-12pt}, no head, from=1-2, to=2-3]
		\arrow[curve={height=-12pt}, from=2-3, to=3-2]
		\arrow[curve={height=12pt}, from=2-1, to=3-2]
		\arrow[no head, from=3-2, to=4-2]
		\arrow[from=4-2, to=5-2]
	\end{tikzcd}\]
	The following Proposition says that these are the only cases when this happens.
	\begin{proposition}\label{prop:alpha_uniqueness}
		Let $\pi$ be a cut-free proof net with only one conclusion, assume also that all conclusions of all axiom links are atomic. Then $\pi$ is determined uniquely by $\alpha_\pi$ and the conclusion $A$.
	\end{proposition}
	\begin{proof}
		Let $A$ be the soul conclusion of $\pi$. Consider the typing tree of $A$, this is a binary tree with all leaves labelled by atomic propositions. The proof net $\pi$ is constructed from this by choosing axiom links, this choice of axiom links is determined by $\alpha_\pi$.
	\end{proof}
\begin{cor}
	Let $\pi$ be a cut-free proof net with conclusions $A_1,...,A_n$, assume also that all conclusions of all axiom links are atomic. Then $\pi$ is determined uniquely by $\alpha_\pi$ and the conclusions $A_1,...,A_n$.
\end{cor}\label{cor:super_normal_form}
\begin{proof}
	Let $\pi_1,\pi_2$ be two such proof nets. Construct a new proof net $\pi_1'$ from $\pi_1$ by creating par links (in any order) so that $\pi_1'$ has a unique conclusion $A_1 \parr \hdots \parr A_n$. Construct a new proof net $\pi_2'$ in a similar way and make the same choices of par links made when $\pi_1'$ was constructed. No axiom links were altered in this process, and so $\alpha_{\pi_1} = \alpha_{\pi_2} = \alpha_{\pi_1'} = \alpha_{\pi_2'}$. It follows from Proposition \ref{prop:alpha_uniqueness} that $\pi_1' = \pi_2'$. For $i = 1,2$, remove the par links which appear in $\pi_i'$ but not in $\pi_i$ and conclude that $\pi_1 = \pi_2$.
\end{proof}
\begin{example}
	Consider the following formula, where $X$ is atomic.
	\begin{equation}
		((X,+) \otimes (X,+)) \parr ((X,-) \parr (X,-))
	\end{equation}
	corresponds to the sub-proof structure given by ignoring the dashed lines and the axiom links of \eqref{eq:counter_one}, for clarity, we label the conclusions of axiom links explicitly with an integer $i \in \lbrace 1,2,3,4 \rbrace$, but each $X_i$ just means the formula $X$, for simplicity, we write $X$ for $(X,+)$ and $\neg X$ for $(X,-)$, the important point is that $X$ is atomic.
	% https://q.uiver.app/?q=WzAsMTUsWzMsNiwiXFxwYXJyIl0sWzEsNSwiQSBcXG90aW1lcyBBIl0sWzUsNSwiXFxuZWcgQSBcXHBhcnIgXFxuZWcgQSJdLFs0LDMsIlxcbmVnIEEiXSxbNiwzLCJcXG5lZyBBIl0sWzIsMywiQSJdLFswLDMsIkEiXSxbNSw0LCJcXHBhcnIiXSxbMSw0LCJcXG90aW1lcyJdLFszLDEsIlxcYXgiXSxbMywwLCJcXGF4Il0sWzQsMiwiXFxheCJdLFsyLDIsIlxcYXgiXSxbMyw3LCIoQSBcXG90aW1lcyBBKSBcXHBhcnIgKFxcbmVnIEEgXFxwYXJyIFxcbmVnIEEpIl0sWzMsOCwiXFxvcGVyYXRvcm5hbWV7Y30iXSxbMSwwLCIiLDAseyJjdXJ2ZSI6Mn1dLFsyLDAsIiIsMix7ImN1cnZlIjotMn1dLFs0LDddLFszLDddLFs3LDIsIiIsMCx7InN0eWxlIjp7ImhlYWQiOnsibmFtZSI6Im5vbmUifX19XSxbNSw4XSxbNiw4XSxbOCwxLCIiLDIseyJzdHlsZSI6eyJoZWFkIjp7Im5hbWUiOiJub25lIn19fV0sWzksMywiIiwwLHsiY3VydmUiOi0xLCJzdHlsZSI6eyJib2R5Ijp7Im5hbWUiOiJkYXNoZWQifSwiaGVhZCI6eyJuYW1lIjoibm9uZSJ9fX1dLFs5LDUsIiIsMSx7ImN1cnZlIjoxLCJzdHlsZSI6eyJib2R5Ijp7Im5hbWUiOiJkYXNoZWQifSwiaGVhZCI6eyJuYW1lIjoibm9uZSJ9fX1dLFsxMCw0LCIiLDIseyJjdXJ2ZSI6LTMsInN0eWxlIjp7ImJvZHkiOnsibmFtZSI6ImRhc2hlZCJ9LCJoZWFkIjp7Im5hbWUiOiJub25lIn19fV0sWzEwLDYsIiIsMSx7ImN1cnZlIjozLCJzdHlsZSI6eyJib2R5Ijp7Im5hbWUiOiJkYXNoZWQifSwiaGVhZCI6eyJuYW1lIjoibm9uZSJ9fX1dLFsxMSw0LCIiLDEseyJjdXJ2ZSI6LTEsInN0eWxlIjp7ImhlYWQiOnsibmFtZSI6Im5vbmUifX19XSxbMTEsNSwiIiwxLHsiY3VydmUiOjMsInN0eWxlIjp7ImhlYWQiOnsibmFtZSI6Im5vbmUifX19XSxbMTIsNiwiIiwxLHsiY3VydmUiOjEsInN0eWxlIjp7ImhlYWQiOnsibmFtZSI6Im5vbmUifX19XSxbMTIsMywiIiwxLHsiY3VydmUiOi0zLCJzdHlsZSI6eyJoZWFkIjp7Im5hbWUiOiJub25lIn19fV0sWzAsMTMsIiIsMCx7InN0eWxlIjp7ImhlYWQiOnsibmFtZSI6Im5vbmUifX19XSxbMTMsMTRdXQ==
	\begin{equation}\label{eq:counter_one}
		\begin{tikzcd}[column sep = tiny, row sep = small]
			&&& \ax \\
			&&& \ax \\
			&& \ax && \ax \\
			X_1 && X_2 && {\neg X_3} && {\neg X_4} \\
			& \otimes &&&& \parr \\
			& {X \otimes X} &&&& {\neg X\parr \neg X} \\
			&&& \parr \\
			&&& {(X\otimes X) \parr (\neg X \parr \neg X)} \\
			&&& {\operatorname{c}}
			\arrow[curve={height=12pt}, from=6-2, to=7-4]
			\arrow[curve={height=-12pt}, from=6-6, to=7-4]
			\arrow[from=4-7, to=5-6]
			\arrow[from=4-5, to=5-6]
			\arrow[no head, from=5-6, to=6-6]
			\arrow[from=4-3, to=5-2]
			\arrow[from=4-1, to=5-2]
			\arrow[no head, from=5-2, to=6-2]
			\arrow[curve={height=-6pt}, dashed, no head, from=2-4, to=4-5]
			\arrow[curve={height=6pt}, dashed, no head, from=2-4, to=4-3]
			\arrow[curve={height=-18pt}, dashed, no head, from=1-4, to=4-7]
			\arrow[curve={height=18pt}, dashed, no head, from=1-4, to=4-1]
			\arrow[curve={height=-6pt}, no head, from=3-5, to=4-7]
			\arrow[curve={height=18pt}, no head, from=3-5, to=4-3]
			\arrow[curve={height=6pt}, no head, from=3-3, to=4-1]
			\arrow[curve={height=-18pt}, no head, from=3-3, to=4-5]
			\arrow[no head, from=7-4, to=8-4]
			\arrow[from=8-4, to=9-4]
		\end{tikzcd}
	\end{equation}
	The proof net given by ignoring the dashed lines in \eqref{eq:counter_one} corresponds to the permutation $(12)(34)$ (ie, the permutation $X_1 \leftrightarrow X_2, X_3 \leftrightarrow X_4$), and that given by ignoring the axiom links and including the dashed lines is $(14)(23)$.
\end{example}
\begin{defn}\label{def:super_normal}
	A \textbf{super normal form} is a proof net $\pi$ which is cut-free and is such that all conclusions of all axiom links are atomic.
\end{defn}
In light of Corollary \ref{cor:super_normal_form} we wish to canonically associate an arbitrary proof net $\pi$ to a super normal form.
	
	\section{Geometry of Interaction Zero, modelling the dynamics of MLL}
	Geometry of interaction zero requires that all formulas occurring in axiom links are atomic, every proof structure not of this form can be related to one which is by $\eta$-equivalence:
	We begin with the following crucial observation:
	\begin{lemma}\label{lem:GoI_lemma}
		Let $\pi$ be a proof structure and assume there is a cut in $\pi$ of $A$ against $\neg A$. Write
		\begin{equation}
			A :=  A_1 \boxtimes_1 \hdots \boxtimes_{n-1} A_n
		\end{equation}
		where for each $i$ we have $\boxtimes_i \in \lbrace \otimes, \parr\rbrace$. Let $\zeta$ be a proof structure equivalent to $\pi$ under cut-reduction which is obtained by reducing all $m$-redexes (Definition \ref{def:reduction}). Then for each $i$ there exists a unique cut link $l_i$ with premises $A_i, \neg A_i$.
	\end{lemma}
	\begin{proof}
		By induction on $n$, where the base case follows trivially and the inductive step by inspection of \eqref{rule:tensor_reduct}.
	\end{proof}
	\begin{defn}\label{perm_cut}
		We define a permutation $\gamma_\pi$ on $\call{P}(\pi)$. Let $l$ be a cut link in $\pi$ with premises $A, \neg A$, say. For each atomic formula $A_i$ of $A$ there exists by Lemma \ref{lem:GoI_lemma} a cut link $l_i$ in the normal form $\zeta$ of $\pi$ so that the premises of $l_i$ are $A_i, \neg A_i$. Let $A_i,\neg A_i$ have corresponding unoriented atoms $X_i, Y_i$. Let $\gamma$ be the permutation which swaps $X_i$ and $Y_i$. For all other elements of $X \in \call{P}(\pi)$ we define $\gamma_\pi(X) = X$.
	\end{defn}
	\begin{example}
		We denote by $\pi$ the following proof net.
		% https://q.uiver.app/?q=WzAsMTYsWzAsMSwiQSJdLFsxLDAsIlxcYXgiXSxbMiwxLCJcXG5lZyBBIl0sWzMsMiwiXFxvdGltZXMiXSxbNCwxLCJBIl0sWzYsMSwiXFxuZWcgQSJdLFs1LDAsIlxcYXgiXSxbMywzLCJcXG5lZyBBIFxcb3RpbWVzIEEiXSxbNywxLCJBIl0sWzgsMCwiXFxheCJdLFs5LDEsIlxcbmVnIEEiXSxbOCwyLCJcXHBhcnIiXSxbOCwzLCJBIFxccGFyciBcXG5lZyBBIl0sWzYsNCwiXFxjdXQiXSxbNiwyLCJcXG9wZXJhdG9ybmFtZXtjfSJdLFswLDIsIlxcb3BlcmF0b3JuYW1le2N9Il0sWzEsMCwiIiwwLHsiY3VydmUiOjIsInN0eWxlIjp7ImhlYWQiOnsibmFtZSI6Im5vbmUifX19XSxbMSwyLCIiLDIseyJjdXJ2ZSI6LTIsInN0eWxlIjp7ImhlYWQiOnsibmFtZSI6Im5vbmUifX19XSxbNiw0LCIiLDIseyJjdXJ2ZSI6Miwic3R5bGUiOnsiaGVhZCI6eyJuYW1lIjoibm9uZSJ9fX1dLFs2LDUsIiIsMCx7ImN1cnZlIjotMiwic3R5bGUiOnsiaGVhZCI6eyJuYW1lIjoibm9uZSJ9fX1dLFs0LDMsIiIsMix7ImN1cnZlIjotMn1dLFsyLDMsIiIsMSx7ImN1cnZlIjoyfV0sWzMsNywiIiwxLHsic3R5bGUiOnsiaGVhZCI6eyJuYW1lIjoibm9uZSJ9fX1dLFs5LDgsIiIsMSx7ImN1cnZlIjoyLCJzdHlsZSI6eyJoZWFkIjp7Im5hbWUiOiJub25lIn19fV0sWzksMTAsIiIsMSx7ImN1cnZlIjotMiwic3R5bGUiOnsiaGVhZCI6eyJuYW1lIjoibm9uZSJ9fX1dLFs4LDExLCIiLDEseyJjdXJ2ZSI6Mn1dLFsxMCwxMSwiIiwxLHsiY3VydmUiOi0yfV0sWzExLDEyLCIiLDEseyJzdHlsZSI6eyJoZWFkIjp7Im5hbWUiOiJub25lIn19fV0sWzcsMTMsIiIsMSx7ImN1cnZlIjoyfV0sWzEyLDEzLCIiLDEseyJjdXJ2ZSI6LTJ9XSxbMCwxNV0sWzUsMTRdXQ==
		\begin{equation}\label{eq:standard_example}
			\begin{tikzcd}[column sep = small, row sep = small]
				& \ax &&&& \ax &&& \ax \\
				A && {\neg A} && A && {\neg A} & A && {\neg A} \\
				{\operatorname{c}} &&& \otimes &&& {\operatorname{c}} && \parr \\
				&&& {\neg A \otimes A} &&&&& {A \parr \neg A} \\
				&&&&&& \cut
				\arrow[curve={height=12pt}, no head, from=1-2, to=2-1]
				\arrow[curve={height=-12pt}, no head, from=1-2, to=2-3]
				\arrow[curve={height=12pt}, no head, from=1-6, to=2-5]
				\arrow[curve={height=-12pt}, no head, from=1-6, to=2-7]
				\arrow[curve={height=-12pt}, from=2-5, to=3-4]
				\arrow[curve={height=12pt}, from=2-3, to=3-4]
				\arrow[no head, from=3-4, to=4-4]
				\arrow[curve={height=12pt}, no head, from=1-9, to=2-8]
				\arrow[curve={height=-12pt}, no head, from=1-9, to=2-10]
				\arrow[curve={height=12pt}, from=2-8, to=3-9]
				\arrow[curve={height=-12pt}, from=2-10, to=3-9]
				\arrow[no head, from=3-9, to=4-9]
				\arrow[curve={height=12pt}, from=4-4, to=5-7]
				\arrow[curve={height=-12pt}, from=4-9, to=5-7]
				\arrow[from=2-1, to=3-1]
				\arrow[from=2-7, to=3-7]
			\end{tikzcd}
		\end{equation}
		Reducing the cut-redex we obtain:
		% https://q.uiver.app/?q=WzAsMTMsWzEsMCwiXFxheCJdLFswLDEsIkEiXSxbMiwxLCJcXG5lZyBBIl0sWzMsMSwiQSJdLFs0LDAsIlxcYXgiXSxbNSwxLCJcXG5lZyBBIl0sWzcsMCwiXFxheCJdLFs2LDEsIkEiXSxbOCwxLCJcXG5lZyBBIl0sWzQsMywiXFxjdXQiXSxbNiwzLCJcXGN1dCJdLFs1LDIsIlxcb3BlcmF0b3JuYW1le2N9Il0sWzAsMiwiXFxvcGVyYXRvcm5hbWV7Y30iXSxbMCwxLCIiLDAseyJjdXJ2ZSI6Miwic3R5bGUiOnsiaGVhZCI6eyJuYW1lIjoibm9uZSJ9fX1dLFswLDIsIiIsMix7ImN1cnZlIjotMiwic3R5bGUiOnsiaGVhZCI6eyJuYW1lIjoibm9uZSJ9fX1dLFs0LDMsIiIsMix7ImN1cnZlIjoyLCJzdHlsZSI6eyJoZWFkIjp7Im5hbWUiOiJub25lIn19fV0sWzQsNSwiIiwwLHsiY3VydmUiOi0yLCJzdHlsZSI6eyJoZWFkIjp7Im5hbWUiOiJub25lIn19fV0sWzYsNywiIiwwLHsiY3VydmUiOjIsInN0eWxlIjp7ImhlYWQiOnsibmFtZSI6Im5vbmUifX19XSxbNiw4LCIiLDIseyJjdXJ2ZSI6LTIsInN0eWxlIjp7ImhlYWQiOnsibmFtZSI6Im5vbmUifX19XSxbMiw5LCIiLDIseyJjdXJ2ZSI6Mn1dLFs3LDksIiIsMCx7ImN1cnZlIjotMn1dLFs4LDEwLCIiLDIseyJjdXJ2ZSI6LTJ9XSxbMywxMCwiIiwwLHsiY3VydmUiOjJ9XSxbNSwxMV0sWzEsMTJdXQ==
		\[\begin{tikzcd}[column sep = small, row sep = small]
			& \ax &&& \ax &&& \ax \\
			A && {\neg A} & A && {\neg A} & A && {\neg A} \\
			{\operatorname{c}} &&&&& {\operatorname{c}} \\
			&&&& \cut && \cut
			\arrow[curve={height=12pt}, no head, from=1-2, to=2-1]
			\arrow[curve={height=-12pt}, no head, from=1-2, to=2-3]
			\arrow[curve={height=12pt}, no head, from=1-5, to=2-4]
			\arrow[curve={height=-12pt}, no head, from=1-5, to=2-6]
			\arrow[curve={height=12pt}, no head, from=1-8, to=2-7]
			\arrow[curve={height=-12pt}, no head, from=1-8, to=2-9]
			\arrow[curve={height=12pt}, from=2-3, to=4-5]
			\arrow[curve={height=-12pt}, from=2-7, to=4-5]
			\arrow[curve={height=-12pt}, from=2-9, to=4-7]
			\arrow[curve={height=12pt}, from=2-4, to=4-7]
			\arrow[from=2-6, to=3-6]
			\arrow[from=2-1, to=3-1]
		\end{tikzcd}\]
		The cut link in $\pi$ consists of $\neg A \otimes A$ and $A \parr \neg A$, the order of these formulas was respected by the cut-reduction step, in the sense made precise by Lemma \ref{lem:GoI_lemma}, as seen in this example as the resulting cuts are between ``the two first elements" and ``the two last elements", ie, between $\neg A$ and $A$, and between $A$ and $\neg A$.
	\end{example}
	Lemma \ref{lem:GoI_lemma} will be used to prove both Geometry of Interaction zero and Geometry of Interaction One.
	
	\begin{lemma}\label{lem:particular_redex_form}
		The set $\call{P}(\pi)$ is invariant under reduction of $m$-redexes and $\eta$-expansion (defined in Remark \ref{rmk:trunk}). More precisely, we have the following two statements.
		\begin{itemize}
			\item Say $\pi'$ produced by reducing an $m$-redex \eqref{rule:tensor_reduct} in $\pi$, then $\call{P}(\pi) = \call{P}(\pi')$.
			\item Say $\pi \lto_\eta \pi'$ (see Definition \ref{def:eta_expansion}), then $\call{P}(\pi) = \call{P}(\pi')$.
		\end{itemize}
	\end{lemma}
	\begin{proof}
		For the first claim we simply notice that rule \eqref{rule:tensor_reduct} has no effect on the axiom links of $\pi$. For the second we see that the order of the sequence of unoriented atoms of $A,B$ is explicated by the axiom links produced by an $\eta$-expansion.
	\end{proof}
	Hence, when considering $\call{P}(\pi)$, we can always assume without loss of generality that $\pi$ contains no $m$-redexes and that all conclusions of all axiom links of $\pi$ are atomic.
	\begin{lemma}\label{lem:redexes_particular_form}
		Let $\pi$ be a proof net admitting no $m$-redexes and assume that all conclusions of all axiom links of $\pi$ are atomic. All redexes $\pi \lto_{\cut} \pi'$ are necessarily of the following form.
		% https://q.uiver.app/?q=WzAsMTQsWzEsMCwiXFxheCJdLFswLDEsIlgiXSxbMiwxLCJcXG5lZyBYIl0sWzMsMiwiXFxjdXQiXSxbNCwxLCJYIl0sWzUsMCwiXFxheCJdLFs2LDEsIlxcbmVnIFgiXSxbOSwwLCJcXGF4Il0sWzgsMSwiWCJdLFsxMCwxLCJcXG5lZyBYIl0sWzAsMiwiXFx2ZG90cyJdLFs2LDIsIlxcdmRvdHMiXSxbOCwyLCJcXHZkb3RzIl0sWzEwLDIsIlxcdmRvdHMiXSxbMCwxLCIiLDAseyJjdXJ2ZSI6Miwic3R5bGUiOnsiaGVhZCI6eyJuYW1lIjoibm9uZSJ9fX1dLFswLDIsIiIsMix7ImN1cnZlIjotMiwic3R5bGUiOnsiaGVhZCI6eyJuYW1lIjoibm9uZSJ9fX1dLFsxLDEwXSxbMiwzLCIiLDIseyJjdXJ2ZSI6Mn1dLFs0LDMsIiIsMCx7ImN1cnZlIjotMn1dLFs1LDQsIiIsMCx7ImN1cnZlIjoyLCJzdHlsZSI6eyJoZWFkIjp7Im5hbWUiOiJub25lIn19fV0sWzUsNiwiIiwyLHsiY3VydmUiOi0yLCJzdHlsZSI6eyJoZWFkIjp7Im5hbWUiOiJub25lIn19fV0sWzYsMTFdLFs3LDgsIiIsMix7ImN1cnZlIjoyLCJzdHlsZSI6eyJoZWFkIjp7Im5hbWUiOiJub25lIn19fV0sWzcsOSwiIiwwLHsiY3VydmUiOi0yLCJzdHlsZSI6eyJoZWFkIjp7Im5hbWUiOiJub25lIn19fV0sWzksMTNdLFs4LDEyXV0=
		\begin{equation}\label{eq:redex_form}
			\begin{tikzcd}[column sep = small, row sep = small]
				& \ax &&&& \ax &&&& \ax \\
				X && {\neg X} && X && {\neg X} && X && {\neg X} \\
				\vdots &&& \cut &&& \vdots && \vdots && \vdots
				\arrow[curve={height=12pt}, no head, from=1-2, to=2-1]
				\arrow[curve={height=-12pt}, no head, from=1-2, to=2-3]
				\arrow[from=2-1, to=3-1]
				\arrow[curve={height=12pt}, from=2-3, to=3-4]
				\arrow[curve={height=-12pt}, from=2-5, to=3-4]
				\arrow[curve={height=12pt}, no head, from=1-6, to=2-5]
				\arrow[curve={height=-12pt}, no head, from=1-6, to=2-7]
				\arrow[from=2-7, to=3-7]
				\arrow[curve={height=12pt}, no head, from=1-10, to=2-9]
				\arrow[curve={height=-12pt}, no head, from=1-10, to=2-11]
				\arrow[from=2-11, to=3-11]
				\arrow[from=2-9, to=3-9]
			\end{tikzcd}
		\end{equation}
	\end{lemma}
	\begin{proof}
		All redexes of $\pi$ are $a$-redexes, so all redexes of $\pi$ are of the form \eqref{rule:ax_reduct_negation} or \eqref{rule:ax_reduct}, but since all the axiom links have atoms as conclusions, it must be the case that the cut link in \eqref{rule:ax_reduct_negation}, \eqref{rule:ax_reduct} have premises which are also atoms. These atoms can only possibly exist if they are conclusions to an axiom link, and so we obtain the form given in the statement.
	\end{proof}
	\begin{defn}\label{def:injective_function}
		Say $\pi$ is a proof net with no $m$-redexes and all conclusions of all axiom links are atomic. Moreover, say there is a reduction $\pi \lto \pi'$ which by Lemma \ref{lem:particular_redex_form} is of the form \eqref{eq:redex_form}. We define a function $\iota: \call{P}(\pi') \rightarrowtail \call{P}(\pi)$ given by the following schema:
		% https://q.uiver.app/?q=WzAsMTQsWzEsMywiXFxheCJdLFswLDQsIkEiXSxbMiw0LCJcXG5lZyBBIl0sWzMsNSwiXFxjdXQiXSxbNCw0LCJBIl0sWzAsNSwiXFx2ZG90cyJdLFs1LDMsIlxcYXgiXSxbNiw0LCJcXG5lZyBBIl0sWzYsNSwiXFx2ZG90cyJdLFsyLDEsIkEiXSxbNCwxLCJcXG5lZyBBIl0sWzMsMCwiXFxheCJdLFsyLDIsIlxcdmRvdHMiXSxbNCwyLCJcXHZkb3RzIl0sWzAsMSwiIiwwLHsiY3VydmUiOjIsInN0eWxlIjp7ImhlYWQiOnsibmFtZSI6Im5vbmUifX19XSxbMCwyLCIiLDIseyJjdXJ2ZSI6LTIsInN0eWxlIjp7ImhlYWQiOnsibmFtZSI6Im5vbmUifX19XSxbMiwzLCIiLDIseyJjdXJ2ZSI6Mn1dLFs0LDMsIiIsMCx7ImN1cnZlIjotMn1dLFsxLDVdLFs2LDQsIiIsMCx7ImN1cnZlIjoyLCJzdHlsZSI6eyJoZWFkIjp7Im5hbWUiOiJub25lIn19fV0sWzYsNywiIiwyLHsiY3VydmUiOi0yLCJzdHlsZSI6eyJoZWFkIjp7Im5hbWUiOiJub25lIn19fV0sWzcsOF0sWzExLDksIiIsMix7ImN1cnZlIjoyLCJzdHlsZSI6eyJoZWFkIjp7Im5hbWUiOiJub25lIn19fV0sWzExLDEwLCIiLDAseyJjdXJ2ZSI6LTIsInN0eWxlIjp7ImhlYWQiOnsibmFtZSI6Im5vbmUifX19XSxbMTAsMTNdLFs5LDEyXSxbOSwxLCIiLDEseyJjdXJ2ZSI6NSwiY29sb3VyIjpbMzAwLDYwLDYwXX1dLFsxMCw3LCIiLDEseyJjdXJ2ZSI6LTUsImNvbG91ciI6WzMwMCw2MCw2MF19XV0=
		\begin{equation}\label{eq:inclusion_definition}
		\begin{tikzcd}[column sep = small, row sep = small]
			&&& \ax \\
			&& X && {\neg X} \\
			&& \vdots && \vdots \\
			& \ax &&&& \ax \\
			X && {\neg X} && X && {\neg X} \\
			\vdots &&& \cut &&& \vdots
			\arrow[curve={height=12pt}, no head, from=4-2, to=5-1]
			\arrow[curve={height=-12pt}, no head, from=4-2, to=5-3]
			\arrow[curve={height=12pt}, from=5-3, to=6-4]
			\arrow[curve={height=-12pt}, from=5-5, to=6-4]
			\arrow[from=5-1, to=6-1]
			\arrow[curve={height=12pt}, no head, from=4-6, to=5-5]
			\arrow[curve={height=-12pt}, no head, from=4-6, to=5-7]
			\arrow[from=5-7, to=6-7]
			\arrow[curve={height=12pt}, no head, from=1-4, to=2-3]
			\arrow[curve={height=-12pt}, no head, from=1-4, to=2-5]
			\arrow[from=2-5, to=3-5]
			\arrow[from=2-3, to=3-3]
			\arrow[color={rgb,255:red,214;green,92;blue,214}, curve={height=30pt}, from=2-3, to=5-1]
			\arrow[color={rgb,255:red,214;green,92;blue,214}, curve={height=-30pt}, from=2-5, to=5-7]
		\end{tikzcd}
\end{equation}
	\end{defn}
The following proposition states that super normal forms are completely determined by their axiom links, and conclusion.
	\begin{defn}\label{def:snf_function}
		Let $\pi$ be a proof net and consider $\call{P}(\pi)$, in light of Lemma \ref{lem:particular_redex_form} we can assume without loss of generality that $\pi$ admits no $m$-redexes and that all conclusions of all axiom links in $\pi$ are atomic. Let $\zeta$ be the corresponding super normal form established by Corollary \ref{cor:super_normal_form}. Let $(\pi = \pi_1,\hdots, \pi_n = \zeta)$ be a sequence of cut reductions. These induce a family of functions:
		\begin{equation}
			\call{P}(\zeta) = \call{P}(\pi_n) \rightarrowtail \call{P}(\pi_{n-1}) \lto \hdots \lto \call{P}(\pi_2) \lto \call{P}(\pi_1) = \call{P}(\pi)
		\end{equation}
		Composing these determines a function $\iota_{\pi}: \call{P}(\zeta) \rightarrowtail \call{P}(\pi)$.
	\end{defn}
	\begin{remark}
		It follows from Proposition \ref{prop:church_rosser} that the function $\iota_\pi$ is independent of the choice of reduction path used to define it.
		\end{remark}
	We give an alternate characterisation of the image of $\iota_\pi$.
	\begin{lemma}\label{lem:essence_classification}
		Let $\pi$ be a proof net admitting no $m$-redexes and assume all conclusions of all axiom links are atomic. A formula $A$ in $\pi$ is in $\operatorname{im}\iota_\pi$ if and only if it is \emph{not} premise to a cut link and is the conclusion to an axiom link.
	\end{lemma}
\begin{proof}
	Say $A$ is premise to a cut link. Since $A \in \call{P}(\pi)$ it is also the case that $A$ is conclusion to a cut link. Hence there exists a cut reduction which removes $A$, and so $A$ is not in the image of $\iota_\pi$.
	
	Now say $A$ is \emph{not} premise to a cut link and so $A$ is necessarily \emph{not} part of an $a$-redex. There are no $m$-redexes in $\pi$ and so all cut reductions reduce $a$-redexes. Hence $A$ survives the cut reduction process. In other words, $A \in \operatorname{im}\iota_\pi$.
\end{proof}
	
	\begin{defn}\label{def:GoI_permutation}
		Let $\pi$ be a proof net. We describe a final permutation $\delta_\pi$ on $\call{P}(\pi)$. Recall the injective function $\iota_\pi$ of Definition \ref{def:snf_function}. For each $X \in \call{P}(\pi)$ let $d_i$ denote the least integer such that 
		\begin{equation}
			(\alpha_{\pi} \circ \gamma_{\pi})^{d_i}(X) \in \operatorname{im}\iota_\pi
		\end{equation}
		Notice that such an integer $d_i$ always exists as $\pi$ is a proof net (as $\pi$ satisfies the longtrip condition, see Section \ref{sec:sequentialisation}).
		
		We then define the following permutation on $\call{P}(\pi)$, the permutations $\alpha_\pi, \gamma_\pi$ are defined in Definition \ref{def:permutations}:
		\begin{equation}
			\delta_{\pi}(X) = (\alpha_\pi \circ \gamma_\pi)^{d_i}(X)
		\end{equation}
	\end{defn}
	\begin{thm}\label{thm:geometry_of_interaction_zero}[Geometry of Interaction zero]
		Let $\pi$ be a proof net possibly with cuts and let $\zeta$ be the normal form of $\pi$ (Definition \ref{def:normal_form}). Then
		\begin{equation}
			\delta_\pi = \iota_{\pi}\alpha_{\zeta}
		\end{equation}
	\end{thm}
	\begin{proof}
		By inspection of \eqref{rule:tensor_reduct} we have that $\gamma_{\pi}$ is invariant under reduction of $m$-redexes. Also, $\alpha_\pi$ is clearly invariant under reduction of $m$-redexes, thus we can assume that $\pi$ admits no $m$-redex. Furthermore, by inspection of \eqref{eq:eta_expansion} we see that $\alpha_\pi$ is invariant under $\eta$-expansion, it is also clear that $\gamma_\pi$ is invariant under $\eta$-expansion. Thus we can also assume that all conclusions of all axiom links of $\pi$ are atomic.
		
		All cut links appear inside ``chains" of axiom and cut links, such as in the following Diagram.
		% https://q.uiver.app/?q=WzAsMTQsWzEsMCwiXFxheCJdLFswLDEsIlgiXSxbMiwxLCJcXG5lZyBYIl0sWzMsMiwiXFxjdXQiXSxbNCwxLCJYIl0sWzUsMCwiXFxoZG90cyJdLFs2LDBdLFs2LDEsIlgiXSxbNywyLCJcXGN1dCJdLFs4LDEsIlxcbmVnIFgiXSxbOSwwLCJcXGF4Il0sWzEwLDEsIlgiXSxbMCwyLCJcXHZkb3RzIl0sWzEwLDIsIlxcdmRvdHMiXSxbMTAsMTEsIiIsMCx7ImN1cnZlIjotMiwic3R5bGUiOnsiaGVhZCI6eyJuYW1lIjoibm9uZSJ9fX1dLFsxMCw5LCIiLDIseyJjdXJ2ZSI6Miwic3R5bGUiOnsiaGVhZCI6eyJuYW1lIjoibm9uZSJ9fX1dLFs5LDgsIiIsMix7ImN1cnZlIjotMn1dLFs3LDgsIiIsMCx7ImN1cnZlIjoyfV0sWzUsNywiIiwwLHsiY3VydmUiOi0yLCJzdHlsZSI6eyJoZWFkIjp7Im5hbWUiOiJub25lIn19fV0sWzUsNCwiIiwyLHsiY3VydmUiOjIsInN0eWxlIjp7ImhlYWQiOnsibmFtZSI6Im5vbmUifX19XSxbNCwzLCIiLDIseyJjdXJ2ZSI6LTJ9XSxbMiwzLCIiLDAseyJjdXJ2ZSI6Mn1dLFswLDIsIiIsMCx7ImN1cnZlIjotMiwic3R5bGUiOnsiaGVhZCI6eyJuYW1lIjoibm9uZSJ9fX1dLFswLDEsIiIsMix7ImN1cnZlIjoyfV0sWzEsMTJdLFsxMSwxM11d
		\[\begin{tikzcd}[column sep = tiny, row sep = small]
			& \ax &&&& \hdots & {} &&& \ax \\
			X && {\neg X} && X && X && {\neg X} && X \\
			\vdots &&& \cut &&&& \cut &&& \vdots
			\arrow[curve={height=-12pt}, no head, from=1-10, to=2-11]
			\arrow[curve={height=12pt}, no head, from=1-10, to=2-9]
			\arrow[curve={height=-12pt}, from=2-9, to=3-8]
			\arrow[curve={height=12pt}, from=2-7, to=3-8]
			\arrow[curve={height=-12pt}, no head, from=1-6, to=2-7]
			\arrow[curve={height=12pt}, no head, from=1-6, to=2-5]
			\arrow[curve={height=-12pt}, from=2-5, to=3-4]
			\arrow[curve={height=12pt}, from=2-3, to=3-4]
			\arrow[curve={height=-12pt}, no head, from=1-2, to=2-3]
			\arrow[curve={height=12pt}, from=1-2, to=2-1]
			\arrow[from=2-1, to=3-1]
			\arrow[from=2-11, to=3-11]
		\end{tikzcd}\]
	By Lemma \ref{lem:essence_classification}, all formulas in the ``interior" of these chains are not in $\operatorname{im}\iota_\pi$. Hence, $\delta_\pi$ is a product of transposes where the formulas on the two extreme ends of these chains are swapped. By considering the cut elimination rules \eqref{rule:ax_reduct_negation}, \eqref{rule:ax_reduct} we see that this is exactly the behaviour of $\alpha_{\zeta}$, and the that these two formulas are the images of the corresponding formulas in $\zeta$.
	\end{proof}

In \cite[Proposition A.1]{AlgPnt} the relationship between the permutation $\delta$ of Definition \ref{def:GoI_permutation} and another permutation coming from the model of Multiplicative Linear Logic as coordinate rings presented there. Here, we show a detailed example of this connection.
\begin{example}
	Let $\pi$ denote the following proof net, we let $(X,+)$ be atomic and denote it by $X$, the formula $C$ denotes $(\neg X \parr  X) \otimes (X \otimes \neg X)$. For convenience, we have artificially labelled the atomic propositions, but throughout, for any $i$ the notation $X_i$ means $X$.
	% https://q.uiver.app/?q=WzAsMTcsWzQsMiwiXFxuZWcgWF81IFxccGFyciBYXzYiXSxbNSwzLCJcXG90aW1lcyJdLFs1LDQsIkMiXSxbNiwyLCJYXzcgXFxvdGltZXMgXFxuZWcgWF84Il0sWzcsMiwiWF85IFxcb3RpbWVzIFxcbmVnIFhfezEwfSJdLFs5LDIsIlxcbmVnIFhfezExfSBcXHBhcnIgWF97MTJ9Il0sWzgsMSwiXFxheCJdLFs4LDMsIlxccGFyciJdLFs4LDQsIlxcbmVnIEMiXSxbNyw1LCJcXGN1dCJdLFszLDEsIlxcYXgiXSxbMiwyLCJYXzMgXFxvdGltZXMgXFxuZWcgWF80Il0sWzMsMCwiXFxheCJdLFswLDIsIlxcbmVnIFhfMSBcXHBhcnIgWF8yIl0sWzEsMywiXFxwYXJyIl0sWzEsNCwiXFxuZWcgQyJdLFsxLDUsIlxcb3BlcmF0b3JuYW1le2N9Il0sWzMsMSwiIiwwLHsiY3VydmUiOi0yfV0sWzAsMSwiIiwxLHsiY3VydmUiOjJ9XSxbMSwyLCIiLDEseyJzdHlsZSI6eyJoZWFkIjp7Im5hbWUiOiJub25lIn19fV0sWzYsNCwiIiwxLHsiY3VydmUiOjIsInN0eWxlIjp7ImhlYWQiOnsibmFtZSI6Im5vbmUifX19XSxbNiw1LCIiLDEseyJjdXJ2ZSI6LTIsInN0eWxlIjp7ImhlYWQiOnsibmFtZSI6Im5vbmUifX19XSxbNCw3LCIiLDEseyJjdXJ2ZSI6Mn1dLFs1LDcsIiIsMSx7ImN1cnZlIjotMn1dLFs3LDgsIiIsMSx7InN0eWxlIjp7ImhlYWQiOnsibmFtZSI6Im5vbmUifX19XSxbMiw5LCIiLDEseyJjdXJ2ZSI6Mn1dLFs4LDksIiIsMSx7ImN1cnZlIjotMn1dLFsxMCwwLCIiLDEseyJjdXJ2ZSI6LTIsInN0eWxlIjp7ImhlYWQiOnsibmFtZSI6Im5vbmUifX19XSxbMTAsMTEsIiIsMSx7ImN1cnZlIjoyLCJzdHlsZSI6eyJoZWFkIjp7Im5hbWUiOiJub25lIn19fV0sWzEyLDMsIiIsMix7ImN1cnZlIjotMywic3R5bGUiOnsiaGVhZCI6eyJuYW1lIjoibm9uZSJ9fX1dLFsxMiwxMywiIiwwLHsiY3VydmUiOjMsInN0eWxlIjp7ImhlYWQiOnsibmFtZSI6Im5vbmUifX19XSxbMTEsMTQsIiIsMCx7ImN1cnZlIjotMn1dLFsxMywxNCwiIiwxLHsiY3VydmUiOjF9XSxbMTQsMTUsIiIsMSx7InN0eWxlIjp7ImhlYWQiOnsibmFtZSI6Im5vbmUifX19XSxbMTUsMTZdXQ==
	\[\begin{tikzcd}[column sep = tiny, row sep = small]
		&&& \ax \\
		&&& \ax &&&&& \ax \\
		{\neg X_1 \parr X_2} && {X_3 \otimes \neg X_4} && {\neg X_5 \parr X_6} && {X_7 \otimes \neg X_8} & {X_9 \otimes \neg X_{10}} && {\neg X_{11} \parr X_{12}} \\
		& \parr &&&& \otimes &&& \parr \\
		& {\neg C} &&&& C &&& {\neg C} \\
		& {\operatorname{c}} &&&&&& \cut
		\arrow[curve={height=-12pt}, from=3-7, to=4-6]
		\arrow[curve={height=12pt}, from=3-5, to=4-6]
		\arrow[no head, from=4-6, to=5-6]
		\arrow[curve={height=12pt}, no head, from=2-9, to=3-8]
		\arrow[curve={height=-12pt}, no head, from=2-9, to=3-10]
		\arrow[curve={height=12pt}, from=3-8, to=4-9]
		\arrow[curve={height=-12pt}, from=3-10, to=4-9]
		\arrow[no head, from=4-9, to=5-9]
		\arrow[curve={height=12pt}, from=5-6, to=6-8]
		\arrow[curve={height=-12pt}, from=5-9, to=6-8]
		\arrow[curve={height=-12pt}, no head, from=2-4, to=3-5]
		\arrow[curve={height=12pt}, no head, from=2-4, to=3-3]
		\arrow[curve={height=-18pt}, no head, from=1-4, to=3-7]
		\arrow[curve={height=18pt}, no head, from=1-4, to=3-1]
		\arrow[curve={height=-12pt}, from=3-3, to=4-2]
		\arrow[curve={height=6pt}, from=3-1, to=4-2]
		\arrow[no head, from=4-2, to=5-2]
		\arrow[from=5-2, to=6-2]
	\end{tikzcd}\]
	First we describe $\delta_\pi$. The first observation is that $\pi$ is mapped under $\eta$-expansion to the following proof net, which we denote by $\pi'$.
	% https://q.uiver.app/?q=WzAsMzgsWzUsNSwiXFxuZWcgWF81IFxccGFyciBYXzYiXSxbNiw2LCJcXG90aW1lcyJdLFs2LDcsIkMiXSxbNyw1LCJYXzcgXFxvdGltZXMgXFxuZWcgWF84Il0sWzgsNSwiWF85IFxcb3RpbWVzIFxcbmVnIFhfezEwfSJdLFsxMCw1LCJcXG5lZyBYX3sxMX0gXFxwYXJyIFhfezEyfSJdLFs5LDYsIlxccGFyciJdLFs5LDcsIlxcbmVnIEMiXSxbOCw4LCJcXGN1dCJdLFszLDUsIlhfMyBcXG90aW1lcyBcXG5lZyBYXzQiXSxbMSw1LCJcXG5lZyBYXzEgXFxwYXJyIFhfMiJdLFsyLDYsIlxccGFyciJdLFsyLDcsIlxcbmVnIEMiXSxbMiw4LCJcXG9wZXJhdG9ybmFtZXtjfSJdLFswLDMsIlxcbmVnIFhfMSJdLFsxLDMsIlhfMiJdLFsyLDMsIlhfMyJdLFszLDMsIlxcbmVnIFhfNCJdLFs0LDMsIlxcbmVnIFhfNSJdLFs1LDMsIlhfNiJdLFs2LDMsIlhfNyJdLFs3LDMsIlxcbmVnIFhfOCJdLFs4LDMsIlhfOSJdLFs5LDMsIlhfezEwfSJdLFsxMCwzLCJYX3sxMX0iXSxbMTEsMywiWF97MTJ9Il0sWzEsNCwiXFxwYXJyIl0sWzMsNCwiXFxvdGltZXMiXSxbNSw0LCJcXHBhcnIiXSxbNyw0LCJcXG90aW1lcyJdLFs4LDQsIlxcb3RpbWVzIl0sWzEwLDQsIlxccGFyciJdLFsyLDAsIlxcYXgiXSxbNSwwLCJcXGF4Il0sWzMsMiwiXFxheCJdLFs0LDIsIlxcYXgiXSxbOSwyLCJcXGF4Il0sWzEwLDIsIlxcYXgiXSxbMywxLCIiLDAseyJjdXJ2ZSI6LTJ9XSxbMCwxLCIiLDEseyJjdXJ2ZSI6Mn1dLFsxLDIsIiIsMSx7InN0eWxlIjp7ImhlYWQiOnsibmFtZSI6Im5vbmUifX19XSxbNCw2LCIiLDEseyJjdXJ2ZSI6Mn1dLFs1LDYsIiIsMSx7ImN1cnZlIjotMn1dLFs2LDcsIiIsMSx7InN0eWxlIjp7ImhlYWQiOnsibmFtZSI6Im5vbmUifX19XSxbMiw4LCIiLDEseyJjdXJ2ZSI6Mn1dLFs3LDgsIiIsMSx7ImN1cnZlIjotMn1dLFs5LDExLCIiLDAseyJjdXJ2ZSI6LTJ9XSxbMTAsMTEsIiIsMSx7ImN1cnZlIjoxfV0sWzExLDEyLCIiLDEseyJzdHlsZSI6eyJoZWFkIjp7Im5hbWUiOiJub25lIn19fV0sWzEyLDEzXSxbMjQsMzFdLFsyNSwzMSwiIiwyLHsiY3VydmUiOi0yfV0sWzMxLDUsIiIsMCx7InN0eWxlIjp7ImhlYWQiOnsibmFtZSI6Im5vbmUifX19XSxbMjMsMzAsIiIsMCx7ImN1cnZlIjotMn1dLFsyMiwzMF0sWzMwLDQsIiIsMix7InN0eWxlIjp7ImhlYWQiOnsibmFtZSI6Im5vbmUifX19XSxbMjEsMjldLFsyMCwyOSwiIiwwLHsiY3VydmUiOjJ9XSxbMjksMywiIiwwLHsic3R5bGUiOnsiaGVhZCI6eyJuYW1lIjoibm9uZSJ9fX1dLFsxOSwyOF0sWzE4LDI4LCIiLDIseyJjdXJ2ZSI6Mn1dLFsyOCwwLCIiLDIseyJzdHlsZSI6eyJoZWFkIjp7Im5hbWUiOiJub25lIn19fV0sWzE3LDI3XSxbMTYsMjcsIiIsMCx7ImN1cnZlIjoyfV0sWzI3LDksIiIsMCx7InN0eWxlIjp7ImhlYWQiOnsibmFtZSI6Im5vbmUifX19XSxbMTQsMjYsIiIsMix7ImN1cnZlIjoyfV0sWzI2LDEwLCIiLDIseyJzdHlsZSI6eyJoZWFkIjp7Im5hbWUiOiJub25lIn19fV0sWzE1LDI2XSxbMzIsMTQsIiIsMix7ImN1cnZlIjo0LCJzdHlsZSI6eyJoZWFkIjp7Im5hbWUiOiJub25lIn19fV0sWzMyLDIwLCIiLDAseyJjdXJ2ZSI6LTQsInN0eWxlIjp7ImhlYWQiOnsibmFtZSI6Im5vbmUifX19XSxbMzMsMTUsIiIsMCx7ImN1cnZlIjo0LCJzdHlsZSI6eyJoZWFkIjp7Im5hbWUiOiJub25lIn19fV0sWzMzLDIxLCIiLDIseyJjdXJ2ZSI6LTQsInN0eWxlIjp7ImhlYWQiOnsibmFtZSI6Im5vbmUifX19XSxbMzQsMTYsIiIsMCx7ImN1cnZlIjoyLCJzdHlsZSI6eyJoZWFkIjp7Im5hbWUiOiJub25lIn19fV0sWzM0LDE4LCIiLDIseyJjdXJ2ZSI6LTIsInN0eWxlIjp7ImhlYWQiOnsibmFtZSI6Im5vbmUifX19XSxbMzUsMTcsIiIsMix7ImN1cnZlIjoyLCJzdHlsZSI6eyJoZWFkIjp7Im5hbWUiOiJub25lIn19fV0sWzM1LDE5LCIiLDAseyJjdXJ2ZSI6LTIsInN0eWxlIjp7ImhlYWQiOnsibmFtZSI6Im5vbmUifX19XSxbMzYsMjIsIiIsMix7ImN1cnZlIjoyLCJzdHlsZSI6eyJoZWFkIjp7Im5hbWUiOiJub25lIn19fV0sWzM2LDI0LCIiLDAseyJjdXJ2ZSI6LTIsInN0eWxlIjp7ImhlYWQiOnsibmFtZSI6Im5vbmUifX19XSxbMzcsMjMsIiIsMCx7ImN1cnZlIjoyLCJzdHlsZSI6eyJoZWFkIjp7Im5hbWUiOiJub25lIn19fV0sWzM3LDI1LCIiLDIseyJjdXJ2ZSI6LTIsInN0eWxlIjp7ImhlYWQiOnsibmFtZSI6Im5vbmUifX19XV0=
	\[\adjustbox{scale=0.7}{\begin{tikzcd}[column sep = small]
			&& \ax &&& \ax \\
			\\
			&&& \ax & \ax &&&&& \ax & \ax \\
			{\neg X_1} & {X_2} & {X_3} & {\neg X_4} & {\neg X_5} & {X_6} & {X_7} & {\neg X_8} & {X_9} & {X_{10}} & {X_{11}} & {X_{12}} \\
			& \parr && \otimes && \parr && \otimes & \otimes && \parr \\
			& {\neg X_1 \parr X_2} && {X_3 \otimes \neg X_4} && {\neg X_5 \parr X_6} && {X_7 \otimes \neg X_8} & {X_9 \otimes \neg X_{10}} && {\neg X_{11} \parr X_{12}} \\
			&& \parr &&&& \otimes &&& \parr \\
			&& {\neg C} &&&& C &&& {\neg C} \\
			&& {\operatorname{c}} &&&&&& \cut
			\arrow[curve={height=-12pt}, from=6-8, to=7-7]
			\arrow[curve={height=12pt}, from=6-6, to=7-7]
			\arrow[no head, from=7-7, to=8-7]
			\arrow[curve={height=12pt}, from=6-9, to=7-10]
			\arrow[curve={height=-12pt}, from=6-11, to=7-10]
			\arrow[no head, from=7-10, to=8-10]
			\arrow[curve={height=12pt}, from=8-7, to=9-9]
			\arrow[curve={height=-12pt}, from=8-10, to=9-9]
			\arrow[curve={height=-12pt}, from=6-4, to=7-3]
			\arrow[curve={height=6pt}, from=6-2, to=7-3]
			\arrow[no head, from=7-3, to=8-3]
			\arrow[from=8-3, to=9-3]
			\arrow[from=4-11, to=5-11]
			\arrow[curve={height=-12pt}, from=4-12, to=5-11]
			\arrow[no head, from=5-11, to=6-11]
			\arrow[curve={height=-12pt}, from=4-10, to=5-9]
			\arrow[from=4-9, to=5-9]
			\arrow[no head, from=5-9, to=6-9]
			\arrow[from=4-8, to=5-8]
			\arrow[curve={height=12pt}, from=4-7, to=5-8]
			\arrow[no head, from=5-8, to=6-8]
			\arrow[from=4-6, to=5-6]
			\arrow[curve={height=12pt}, from=4-5, to=5-6]
			\arrow[no head, from=5-6, to=6-6]
			\arrow[from=4-4, to=5-4]
			\arrow[curve={height=12pt}, from=4-3, to=5-4]
			\arrow[no head, from=5-4, to=6-4]
			\arrow[curve={height=12pt}, from=4-1, to=5-2]
			\arrow[no head, from=5-2, to=6-2]
			\arrow[from=4-2, to=5-2]
			\arrow[curve={height=24pt}, no head, from=1-3, to=4-1]
			\arrow[curve={height=-24pt}, no head, from=1-3, to=4-7]
			\arrow[curve={height=24pt}, no head, from=1-6, to=4-2]
			\arrow[curve={height=-24pt}, no head, from=1-6, to=4-8]
			\arrow[curve={height=12pt}, no head, from=3-4, to=4-3]
			\arrow[curve={height=-12pt}, no head, from=3-4, to=4-5]
			\arrow[curve={height=12pt}, no head, from=3-5, to=4-4]
			\arrow[curve={height=-12pt}, no head, from=3-5, to=4-6]
			\arrow[curve={height=12pt}, no head, from=3-10, to=4-9]
			\arrow[curve={height=-12pt}, no head, from=3-10, to=4-11]
			\arrow[curve={height=12pt}, no head, from=3-11, to=4-10]
			\arrow[curve={height=-12pt}, no head, from=3-11, to=4-12]
	\end{tikzcd}}\]
	The proof net $\pi'$ admits $m$-redexes, we reduce these to obtain the following proof net which we denote by $\pi''$.
	% https://q.uiver.app/?q=WzAsMjksWzksNSwiWF8zIFxcb3RpbWVzIFxcbmVnIFhfNCJdLFsxLDUsIlxcbmVnIFhfMSBcXHBhcnIgWF8yIl0sWzMsNiwiXFxwYXJyIl0sWzMsNywiXFxuZWcgQyJdLFszLDgsIlxcb3BlcmF0b3JuYW1le2N9Il0sWzAsMywiXFxuZWcgWF8xIl0sWzIsMywiWF8yIl0sWzgsMywiWF8zIl0sWzEwLDMsIlxcbmVnIFhfNCJdLFs3LDMsIlxcbmVnIFhfNSJdLFsxMSwzLCJYXzYiXSxbNCwzLCJYXzciXSxbMTQsMywiXFxuZWcgWF84Il0sWzYsMywiWF85Il0sWzEyLDMsIlhfezEwfSJdLFs1LDMsIlhfezExfSJdLFsxMywzLCJYX3sxMn0iXSxbMSw0LCJcXHBhcnIiXSxbOSw0LCJcXG90aW1lcyJdLFsxLDIsIlxcYXgiXSxbOSwwLCJcXGF4Il0sWzcsMiwiXFxheCJdLFsxMCwyLCJcXGF4Il0sWzUsMiwiXFxheCJdLFsxMiwyLCJcXGF4Il0sWzYsNCwiXFxjdXQiXSxbMTEsNCwiXFxjdXQiXSxbNCw0LCJcXGN1dCJdLFsxMyw0LCJcXGN1dCJdLFswLDIsIiIsMCx7ImN1cnZlIjotMn1dLFsxLDIsIiIsMSx7ImN1cnZlIjoxfV0sWzIsMywiIiwxLHsic3R5bGUiOnsiaGVhZCI6eyJuYW1lIjoibm9uZSJ9fX1dLFszLDRdLFs4LDE4LCIiLDAseyJjdXJ2ZSI6LTJ9XSxbNywxOCwiIiwwLHsiY3VydmUiOjJ9XSxbMTgsMCwiIiwwLHsic3R5bGUiOnsiaGVhZCI6eyJuYW1lIjoibm9uZSJ9fX1dLFs1LDE3LCIiLDIseyJjdXJ2ZSI6Mn1dLFsxNywxLCIiLDIseyJzdHlsZSI6eyJoZWFkIjp7Im5hbWUiOiJub25lIn19fV0sWzYsMTcsIiIsMCx7ImN1cnZlIjotMn1dLFsxOSw1LCIiLDIseyJjdXJ2ZSI6Miwic3R5bGUiOnsiaGVhZCI6eyJuYW1lIjoibm9uZSJ9fX1dLFsxOSwxMSwiIiwwLHsiY3VydmUiOi0yLCJzdHlsZSI6eyJoZWFkIjp7Im5hbWUiOiJub25lIn19fV0sWzIwLDYsIiIsMCx7ImN1cnZlIjozLCJzdHlsZSI6eyJoZWFkIjp7Im5hbWUiOiJub25lIn19fV0sWzIwLDEyLCIiLDIseyJjdXJ2ZSI6LTQsInN0eWxlIjp7ImhlYWQiOnsibmFtZSI6Im5vbmUifX19XSxbMjEsNywiIiwwLHsiY3VydmUiOi0xLCJzdHlsZSI6eyJoZWFkIjp7Im5hbWUiOiJub25lIn19fV0sWzIxLDksIiIsMix7InN0eWxlIjp7ImhlYWQiOnsibmFtZSI6Im5vbmUifX19XSxbMjIsOCwiIiwyLHsic3R5bGUiOnsiaGVhZCI6eyJuYW1lIjoibm9uZSJ9fX1dLFsyMiwxMCwiIiwwLHsiY3VydmUiOi0yLCJzdHlsZSI6eyJoZWFkIjp7Im5hbWUiOiJub25lIn19fV0sWzIzLDEzLCIiLDIseyJjdXJ2ZSI6LTIsInN0eWxlIjp7ImhlYWQiOnsibmFtZSI6Im5vbmUifX19XSxbMjMsMTUsIiIsMCx7InN0eWxlIjp7ImhlYWQiOnsibmFtZSI6Im5vbmUifX19XSxbMjQsMTQsIiIsMCx7InN0eWxlIjp7ImhlYWQiOnsibmFtZSI6Im5vbmUifX19XSxbMjQsMTYsIiIsMix7ImN1cnZlIjotMiwic3R5bGUiOnsiaGVhZCI6eyJuYW1lIjoibm9uZSJ9fX1dLFs5LDI1LCIiLDIseyJjdXJ2ZSI6LTJ9XSxbMTMsMjVdLFsxMCwyNl0sWzE0LDI2LCIiLDEseyJjdXJ2ZSI6LTJ9XSxbMTEsMjddLFsxNSwyNywiIiwxLHsiY3VydmUiOi0yfV0sWzEyLDI4LCIiLDIseyJjdXJ2ZSI6LTJ9XSxbMTYsMjhdXQ==
	\begin{equation}\label{eq:GoI_example}\adjustbox{scale=0.7}{\begin{tikzcd}[column sep = small]
				&&&&&&&&& \ax \\
				\\
				& \ax &&&& \ax && \ax &&& \ax && \ax \\
				{\neg X_1} && {X_2} && {X_7} & {X_{11}} & {X_9} & {\neg X_5} & {X_3} && {\neg X_4} & {X_6} & {X_{10}} & {X_{12}} & {\neg X_8} \\
				& \parr &&& \cut && \cut &&& \otimes && \cut && \cut \\
				& {\neg X_1 \parr X_2} &&&&&&&& {X_3 \otimes \neg X_4} \\
				&&& \parr \\
				&&& {\neg C} \\
				&&& {\operatorname{c}}
				\arrow[curve={height=-12pt}, from=6-10, to=7-4]
				\arrow[curve={height=6pt}, from=6-2, to=7-4]
				\arrow[no head, from=7-4, to=8-4]
				\arrow[from=8-4, to=9-4]
				\arrow[curve={height=-12pt}, from=4-11, to=5-10]
				\arrow[curve={height=12pt}, from=4-9, to=5-10]
				\arrow[no head, from=5-10, to=6-10]
				\arrow[curve={height=12pt}, from=4-1, to=5-2]
				\arrow[no head, from=5-2, to=6-2]
				\arrow[curve={height=-12pt}, from=4-3, to=5-2]
				\arrow[curve={height=12pt}, no head, from=3-2, to=4-1]
				\arrow[curve={height=-12pt}, no head, from=3-2, to=4-5]
				\arrow[curve={height=18pt}, no head, from=1-10, to=4-3]
				\arrow[curve={height=-24pt}, no head, from=1-10, to=4-15]
				\arrow[curve={height=-6pt}, no head, from=3-8, to=4-9]
				\arrow[no head, from=3-8, to=4-8]
				\arrow[no head, from=3-11, to=4-11]
				\arrow[curve={height=-12pt}, no head, from=3-11, to=4-12]
				\arrow[curve={height=-12pt}, no head, from=3-6, to=4-7]
				\arrow[no head, from=3-6, to=4-6]
				\arrow[no head, from=3-13, to=4-13]
				\arrow[curve={height=-12pt}, no head, from=3-13, to=4-14]
				\arrow[curve={height=-12pt}, from=4-8, to=5-7]
				\arrow[from=4-7, to=5-7]
				\arrow[from=4-12, to=5-12]
				\arrow[curve={height=-12pt}, from=4-13, to=5-12]
				\arrow[from=4-5, to=5-5]
				\arrow[curve={height=-12pt}, from=4-6, to=5-5]
				\arrow[curve={height=-12pt}, from=4-15, to=5-14]
				\arrow[from=4-14, to=5-14]
		\end{tikzcd}}
	\end{equation}
	As was explained in the proof of Theorem \cite[Proposition A.1]{AlgPnt} we have that $\delta_{\pi} = \delta_{\pi''}$. It was also explained in the same proof that $\delta_{\pi''}$ is a product of transpositions which swaps two formulas which are at the extreme ends of a common ``chain". Hence, we read off \eqref{eq:GoI_example} that $\delta_\pi = (13)(24)$.
	
	On the other hand, we have that $\neg C = (\neg X_1 \parr X_2) \parr (X_3 \otimes \neg X_4)$, and so in the notation of Proposition \cite[Proposition 3.9]{AlgPnt} we have a sequence $(i_1,i_2) = (2,3)$ with complement $(j_1,j_2) = (1,4)$, and so the polynomial denoted $k[X_1,X_2]$ in the statement of Proposition \cite[Proposition 3.9]{AlgPnt} here is the polynomial $k[X_2,X_3]$ and the polynomial denoted $k[Y_1,Y_2]$ in Proposition \cite[Proposition 3.9]{AlgPnt} here is the polynomial $k[X_1,X_4]$. The following are elements of the defining ideal $I_\pi$ of $\pi$.
	\begin{equation}
		X_2 - X_8, \quad X_8'' - X_{12}'', \quad X_{12} - X_{10}, \quad X_{10}'' - X_{6}'', \quad X_{6} - X_4
	\end{equation}
	and so are $X_i - X_i', X_i' - X_i''$ for $i = 2,4,6,10,12$. Hence we see that $\beta_-^{-1}\beta_+(X_2) = X_4$, so $\sigma(2) = 4$.
	
	Similarly, the following are also elements of the defining ideal $I_\pi$ of $\pi$.
	\begin{equation}
		X_1 - X_7,\quad X_7'' - X_{11}'',\quad X_{11} - X_9,\quad X_9'' - X_{5}'',\quad X_5 - X_3
	\end{equation}
	and so are $X_i - X_i', X_i' - X_i''$ for $i = 1,3,5,7$. Hence, $\sigma(1) = 3$. Thus, the following holds for all $i = 1,2,3,4$, as anticipated by Proposition \cite[Proposition A.1]{AlgPnt}.
	\begin{equation}
		\delta_{\pi}(X_i) = X_{\sigma(i)}
	\end{equation}
\end{example}
	
\section{Geometry of Interaction One}
We consider certain conclusion-conclusion paths in a proof structure and associate bounded linear operators to such paths. Moreover, given a reduction $\pi \lto \pi'$ the paths constructed from a proof structure $\pi$ are related to those constructed from $\pi'$ via an explicit formula (Definition \ref{def:execution_formula}). Section \ref{sec:interal_sum_tens} establishes the relevant operator theory, and Section \ref{sec:proofs_operators}  relates this to proof structures.
\subsection{Internalisation of direction sum and tensor product}\label{sec:interal_sum_tens}
We focus on the specific Hilbert space $\bb{H} = \ell^2$ of sequences $\underline{z} = (z_0,z_1,...)$ of complex numbers which are square summable, ie, $\sum_{n = 0}^\infty |z_n|^2$ converges. This has an inner product given as follows.
\begin{equation}
	\big\langle \underline{z}, \underline{w}\big\rangle = \sum_{n = 0}^\infty z_n\overline{w}_n
\end{equation}
In fact, the sum $\bb{H}^m$ of $m$ copies of $\bb{H}$ also has an inner product structure, defined by
\begin{equation}\label{eq:bijection}
	\Big\langle \big(\underline{z}^1,...,\underline{z}^m\big),\big(\underline{w}^1,...,\underline{w}^m\big)\Big\rangle_{\bb{H}^m} = \sum_{j = 1}^m\langle \big(\underline{z}^j,\underline{w}^j\big)\rangle_{\bb{H}}
\end{equation}
We fix the standard basis for $\ell^2$ consisting of sequences $\underline{e}^i$ such that all entries are equal to $0$ except for the $i^{\operatorname{th}}$ which is equal to $1$. We note that this basis is countably infinite. A basis for $\ell^2 \oplus \ell^2$ is given by all $(\underline{e}^i, 0)$ and $(0,\underline{e}^i)$ which is also countable, thus, bijections $\alpha: \bb{N} \coprod \bb{N} \lto \bb{N}$ induce isomorphisms $\ell^2 \lto \ell^2 \oplus \ell^2$. More explicitly, if $\alpha: \bb{N} \coprod \bb{N} \lto \bb{N}$ is such a bijection then there exists injective functions $\alpha_1,\alpha_2: \bb{N} \lto \bb{N}$ which make the following diagram commute.
\begin{equation}
	\begin{tikzcd}
		\bb{N}\arrow[dr,"{\alpha_1}"]\arrow[d,rightarrowtail]\\
		\bb{N}\arrow[r,"{\alpha}"] \coprod \bb{N}\arrow[r,"{\alpha}"] & \bb{N}\\
		\bb{N}\arrow[ur,swap,"{\alpha_2}"]\arrow[u,rightarrowtail]
	\end{tikzcd}
\end{equation}
The induced isomorphism $\hat{\alpha}: \ell^2 \lto \ell^2 \oplus \ell^2$ is then given by the following explicit formula, where $z = \sum_{i = 0}^\infty z_i\underline{e}^i$:
\begin{equation}\label{eq:alpha_hat}
	\hat{\alpha}(z) = \sum_{i = 0}^\infty\Big( z_{\alpha_1(i)}\underline{e}^i,z_{\alpha_2(i)}\underline{e}^i\Big)
\end{equation}
The following calculation shows that $\hat{\alpha}$ is an isometry:
\begin{align*}
	\big\langle \hat{\alpha}(\underline{z}), \hat{\alpha}(\underline{w})\big\rangle &= \Big\langle \sum_{i = 0}^\infty \big(z_{\alpha_1(i)}\underline{e}^i, z_{\alpha_2(i)}\underline{e}^i\big), \sum_{i = 0}^\infty \big( w_{\alpha_1(i)}\underline{e}^i, w_{\alpha_2(i)}\underline{e}^i\big) \Big\rangle\\
	&= \Big\langle \sum_{i = 0}^\infty z_{\alpha_1(i)}\underline{e}^i, \sum_{i = 0}^\infty w_{\alpha_1(i)}\underline{e}^i\Big\rangle + \Big\langle \sum_{i = 0}^\infty z_{\alpha_2(i)}\underline{e}^i, \sum_{i = 0}^\infty w_{\alpha_2(i)}\underline{e}^i \Big\rangle\\
	&= \sum_{i = 0}^\infty z_{\alpha_1(i)}\overline{w}_{\alpha_i(i)} + \sum_{i = 0}^\infty z_{\alpha_2(i)}\overline{w}_{\alpha_2(i)}\\
	&= \sum_{i = 0}^\infty z_i\overline{w}_i\\
	&=  \langle \underline{z},\underline{w}\rangle
\end{align*}
We claim that \eqref{eq:alpha_hat} can also be written as $\hat{\alpha}(z) = \big(p^\ast(z), q^\ast(z)\big)$ for operators $p,q: \ell^2 \lto \ell^2$ determined by continuity and the following conditions.
\begin{equation}
	p(\underline{e}^i) = \underline{e}^{\alpha_1(i)},\qquad q(\underline{e}^i) = \underline{e}^{\alpha_2(i)}
\end{equation}
These maps are norm preserving and so are clearly bounded, thus we have well defined bounded, linear operators. It can be established by a direct calculation that these have adjoints respectively determined by continuity and the following conditions.
\begin{align}
	p^\ast(\underline{e}^i) &= \underline{e}^{\alpha_1^{-1}(i)}\text{ if }\alpha_1^{-1}(i)\text{ exists, otherwise }p^\ast(\underline{e}^i) = 0\\
	q^\ast(\underline{e}^i) &= \underline{e}^{\alpha_2^{-1}(i)}\text{ if }\alpha_2^{-1}(i)\text{ exists, otherwise }p^\ast(\underline{e}^i) = 0
\end{align}
For example: let $w = \sum_{i = 0}^\infty w_i\underline{e}^i$, then
\begin{align*}
	\big\langle p(z),w \big\rangle = \sum_{i = 0}^\infty z_i\overline{w}_{\alpha_1(i)} = \big\langle z, p^\ast(w)\big\rangle
\end{align*}
we thus have the following formula.
\begin{equation}
	\hat{\alpha} = p^\ast \oplus q^\ast
\end{equation}
In a similar way, given any $n > 0$ along with a bijection $\alpha: \bb{N} \lto \coprod_{i = 1}^n\bb{N}$, there is a corresponding induced isometric isomorphism $\hat{\alpha}: \bb{H} \lto \bb{H}^n$ which has an explicit formula, where $z = \sum_{i = 0}^\infty z_i\underline{e}^i$:
\begin{equation}
	\hat{\alpha}(z) = \sum_{i = 0}^\infty \Big(z_{\alpha_1(i)}\underline{e}^i,...,z_{\alpha_n(i)}\underline{e}^i\Big)
\end{equation}
\begin{example}\label{ex:p_q}
	A simple example is given by the following:
	\begin{align*}
		\alpha_1: \bb{N} &\lto \bb{N} & \alpha_2: \bb{N} &\lto \bb{N}\\
		n &\longmapsto 2n & n &\longmapsto 2n + 1
	\end{align*}
	which induces $\alpha: \bb{N} \coprod \bb{N} \lto \bb{N}$, defined by $\alpha(n,1) = 2n$ and $\alpha(n,2) = 2n+1$. The functions $\alpha_1,\alpha_2,\alpha$ make the following a coproduct diagram:
	\begin{equation}
		\begin{tikzcd}
			\bb{N}\arrow[dr,"{\alpha_1}"]\arrow[d,rightarrowtail]\\
			\bb{N}\arrow[r,"{\alpha}"] \coprod \bb{N}\arrow[r,"{\alpha}"] & \bb{N}\\
			\bb{N}\arrow[ur,swap,"{\alpha_2}"]\arrow[u,rightarrowtail]
		\end{tikzcd}
	\end{equation}
	and indeed $\alpha$ is a bijection. We thus have two functions:
	\begin{align*}
		p: \ell^2 & \lto \ell^2 & q: \ell^2 &\lto \ell^2\\
		(z_0,z_1,...) &\longmapsto (z_0,0, z_1, 0, z_2, ...) & (z_1,z_2,...) &\longmapsto (0,z_0, 0, z_1, 0, ...)
	\end{align*}
	which have the following adjoints:
	\begin{align*}
		p^\ast: \ell^2 &\longmapsto \ell^2 & q^\ast: \ell^2 &\lto \ell^2\\
		(z_0,z_1,...) &\longmapsto (z_0,z_2,...) & (z_0,z_1,...) &\longmapsto (z_1,z_3,...)
	\end{align*}
	\begin{aside}
		The following calculation shows that $p^\ast$ is adjoint to $p$, the corresponding calculation for $q$ is similar:
		\begin{align*}
			\big\langle p(z_0,z_1,...),(w_0,w_1,...)\big\rangle &= \big\langle (z_0, 0, z_1, 0,...),(w_1,w_2,...)\big\rangle\\
			&= \big\langle (z_0,z_1,...),(w_0,w_2,...)\big\rangle\\
			&= \big\langle (z_0,z_1,...),p^\ast(w_0,w_1,...)\big\rangle
		\end{align*}
	\end{aside}
	The function $p^\ast,q^\ast$ induce $\hat{\alpha} = p^\ast \oplus q^\ast: \ell^2 \lto \ell^2 \oplus \ell^2$ defined by
	\begin{equation}
		\hat{\alpha}(z_0,z_1,...) = \big((z_0,z_2,...),(z_1,z_3,...)\big)
	\end{equation}
	We make a few observations:
	\begin{lemma}\label{lem:operator_properties}
		The functions $p,q,p^\ast,q^\ast$ satisfy the following:
		\begin{itemize}
			\item $p^\ast p = \operatorname{id}_{\ell^2} = q^\ast q$,
			\item $p p^\ast + q q^\ast = \operatorname{id}_{\ell^2}$,
			\item $p^\ast q = 0 = q^\ast p$.
		\end{itemize}
	\end{lemma}
\end{example}
\begin{cor}\label{cor:operator_zero}
	Let $w$ be a word over the alphabet $\lbrace p, q, p^\ast, q^\ast, \operatorname{id}, \operatorname{id}^\ast\rbrace$, and let $\overline{w}$ denote the induced function given by reading $w$ as a composite of functions. Then $\overline{w} = 0$ if and only $w$ admits either $p^\ast q$ or $q^\ast p$ as a subword.
\end{cor}
\begin{proof}
	By the third dotpoint of Lemma \ref{lem:operator_properties}, it is clear that if $p^\ast q$ or $q^\ast p$ appears as a subword of $w$ then $\hat{w} = 0$.
	
	Conversely, say $\overline{w} = 0$. First, we can assume that $w$ is a word over the alphabet $\lbrace p, q, p^\ast, q^\ast \rbrace$. If $w$ admits no occurrence of $r^\ast$ where $r = p,q$ then $\overline{w} \neq 0$, this can be seen by considering $\overline{w}(\underline{e}^1)$. Hence, there is some occurrence of $r^\ast$ for $r = p,q$ in $w$. Moreover, by the first dotpoint of Lemma \ref{lem:operator_properties} it suffices to consider the case where $w$ does not admit $p^\ast p$ nor $q^\ast q$ as a subword.
	
	We proceed by induction on the integer $n$ given by the number of occurrences of letters $r^\ast$ for $r=p,q$ in $w$.
	
	Say there is only one such occurrence, $r^\ast$ where $r = p$ or $r = q$. If $w$ is a word of length $1$ then clearly $\overline{w} \neq 0$, so we can pick $r'$ to be a letter in $w$ so that either $r' r^\ast$ or $r^\ast r'$ appears as a subword of $w$. Say $r'r^\ast$ appears as a subword. Denote by $\overline{r'r^\ast}$ the function given by reading $r'r^\ast$ as a composite of functions. If $r = p$ then $\overline{r' r^\ast}(\underline{e}^0) \neq 0$, if $r = q$ then $\overline{r' r^\ast}(\underline{e}^1) \neq 0$. Hence $\overline{w}$ is a composite of two functions where each is not equal to the zero function, hence $\overline{w}$ is not the zero function.
	
	Thus, $r^\ast r'$ appears as a subword of $w$. We know that $r' = p$ or $r' = q$ and that $w$ does not contain $p^\ast p$ as a subword, and $w$ does not contain $q^\ast q$ as a subword. Hence, $r' =p$ if $r = q^\ast$, and $r' = q$ if $r = p^\ast$.
	
	Now for the inductive case. First we notice that if every letter $t$ in $w$ is of the form $t'^\ast$ where $t' = p$ or $t' = q$ then $\hat{w}\neq 0$, so this is not the case. We can thus choose an occurrence of a letter $r^\ast$ in $w$, where $r = p,q$, so that there exists some letter $r'$ so that either $r' r^\ast$ or $r^\ast r'$ appears as a subword in $w$. Say we are in the case of $r'r^\ast$. Then, as established in the base case, the function $\overline{r'r^\ast}$ induced by this word is not the $0$ function, so we can remove $r' r^\ast$ from $w$ and appeal to the inductive hypothesis. If we are in the case of $r^\ast r'$, then either the function $\overline{r^\ast r'}$ induced by $r^\ast r'$ is $0$, or not. In the former case, we can appeal to the base case of this argument (applied to the word $r^\ast r'$) and deduce that $r^\ast r'$ is either $q^\ast p$ or $p^\ast q$. If the function $\overline{r^\ast r'}$ is not the zero function, then we can remove $r^\ast r'$ from $w$ and appeal to the inductive hypothesis.
\end{proof}
\begin{cor}\label{cor:sum_operator_zero}
	Let $w_1,...,w_n$ be a set of words over the alphabet $\lbrace p, q, p^\ast, q^\ast, \operatorname{id}, \operatorname{id}^\ast\rbrace$ and consider their induced function $\hat{w}_1,...,\hat{w}_n$ as in Corollary \ref{cor:operator_zero}.
	
	Then $\sum_{i = 1}^n\hat{w}_i = 0$ if and only if for all $i = 1,...,n$ the either the subword $p^\ast q$ or the subword $q^\ast p$ appears in $w_i$.
\end{cor}
\begin{proof}
	We proceed by induction on $n$. The base case $n = 1$ is Corollary \ref{cor:operator_zero}.
	
	Now assume that $n>1$ and that the result is true for all $n - 1$. We will make use of the following observation, which follows easily from the definition of the functions $p,q$: if $u$ is a word over the alphabet $\lbrace p, q, p^\ast, q^\ast \rbrace$ which admits no occurrence of either $p^\ast q$ nor $q^\ast p$ as a subword. Then for all $i,j \geq 0$ the $j^{\text{th}}$ element of the sequence $\hat{u}(\underline{e}^i)$ is either equal to $0$ or $1$.
	
	Now say $\overline{w_1} + \hdots + \overline{w_n} = 0$. This implies $\overline{w_1} + \hdots + \overline{w}_{n-1} = -\hat{w}_n$. Hence, if $\overline{w}_n \neq 0$ then there exists integers $i,j$ such that the $j^{\text{th}}$ element of the sequence $\overline{w}_n(\underline{e}^i)$ is equal to $1$. This implies the $j^{\text{th}}$ entry of the sequence $\big(\overline{w}_1 + \hdots + \overline{w}_{n-1}\big)(\underline{e}^i)$ is $-1$. This contradicts the general statement proved in the previous paragraph.
	
	Thus, $\overline{w_n} = 0$ and so it follows from the inductive hypothesis that $\overline{w}_1 = \hdots = \overline{w_{n-1}} = 0$, completing the proof.
\end{proof}
\subsection{Persistent paths}
An axiom link with conclusions $\neg A, A$ is the translation of an axiom rule
\begin{center}
	\begin{prooftree}
		\AxiomC{}
		\RightLabel{$\ax$}
		\UnaryInfC{$A \vdash A$}
		\end{prooftree}
	\end{center}
and so we think of $\neg A$ as the \emph{hypothesis} and $A$ as the \emph{conclusion}. The role of the edges in a proof net are then to keep track of the occurrences of formulae which are logically related, that is, they represent ``the same" hypothesis or conclusion. For instance, the formulas coloured red in what follows are logically related.
% https://q.uiver.app/?q=WzAsMTEsWzEsMCwiXFxheCJdLFswLDEsIlxcbmVnIEEiXSxbMCwyLCJcXG9wZXJhdG9ybmFtZXtjfSJdLFsyLDEsIkEiXSxbMywyLCJcXG90aW1lcyJdLFszLDMsIkEgXFxvdGltZXMgXFxuZWcgQSJdLFs0LDEsIlxcbmVnIEEiXSxbNSwwLCJcXGF4Il0sWzYsMSwiQSJdLFs2LDIsIlxcb3BlcmF0b3JuYW1le2N9Il0sWzMsNCwiXFxvcGVyYXRvcm5hbWV7Y30iXSxbMCwxLCIiLDAseyJjdXJ2ZSI6Miwic3R5bGUiOnsiaGVhZCI6eyJuYW1lIjoibm9uZSJ9fX1dLFsxLDJdLFswLDMsIiIsMix7ImN1cnZlIjotMiwic3R5bGUiOnsiaGVhZCI6eyJuYW1lIjoibm9uZSJ9fX1dLFszLDQsIiIsMix7ImN1cnZlIjoyfV0sWzcsNiwiIiwyLHsiY3VydmUiOjIsInN0eWxlIjp7ImhlYWQiOnsibmFtZSI6Im5vbmUifX19XSxbNyw4LCIiLDAseyJjdXJ2ZSI6LTIsInN0eWxlIjp7ImhlYWQiOnsibmFtZSI6Im5vbmUifX19XSxbOCw5XSxbNiw0LCIiLDEseyJjdXJ2ZSI6LTJ9XSxbNCw1LCIiLDEseyJzdHlsZSI6eyJoZWFkIjp7Im5hbWUiOiJub25lIn19fV0sWzUsMTBdXQ==
\[\begin{tikzcd}[column sep = small, row sep = small]
	& \ax &&&& \ax \\
	{\textcolor{red}{\neg A}} && \textcolor{red}{A} && {\neg A} && A \\
	{\operatorname{c}} &&& \otimes &&& {\operatorname{c}} \\
	&&& {\textcolor{red}{A} \otimes \neg A} \\
	&&& {\operatorname{c}}
	\arrow[curve={height=12pt}, no head, from=1-2, to=2-1]
	\arrow[from=2-1, to=3-1]
	\arrow[curve={height=-12pt}, no head, from=1-2, to=2-3]
	\arrow[curve={height=12pt}, from=2-3, to=3-4]
	\arrow[curve={height=12pt}, no head, from=1-6, to=2-5]
	\arrow[curve={height=-12pt}, no head, from=1-6, to=2-7]
	\arrow[from=2-7, to=3-7]
	\arrow[curve={height=-12pt}, from=2-5, to=3-4]
	\arrow[no head, from=3-4, to=4-4]
	\arrow[from=4-4, to=5-4]
\end{tikzcd}\]
More precisely, axiom/cut links induce a bijection between the unoriented atoms of each conclusion, and tensor/par links induce injective functions from the unoriented atoms of the premises to the conclusions. In the tensor/par case, the pair of injective functions is moreover a surjective family.

Thus, a choice of unoriented atom in a proof structure has a set of edges associted to it according to the previously mentioned functions. We will show that if the chosen unoriented atom is part of a conclusion to a proof structure, then the induced set of edges induces a path which both begins and edges at a conclusion edge. If the chosen unoriented atom is \emph{not} part of a conclusion, then the induced path may not include \emph{any} conclusions, as is the case for any choice of unoriented atom in the following proof structure.
% https://q.uiver.app/?q=WzAsNCxbMSwwLCJcXGF4Il0sWzAsMSwiXFxuZWcgQSJdLFsxLDIsIlxcY3V0Il0sWzIsMSwiQSJdLFswLDEsIiIsMCx7ImN1cnZlIjoyLCJzdHlsZSI6eyJoZWFkIjp7Im5hbWUiOiJub25lIn19fV0sWzEsMiwiIiwwLHsiY3VydmUiOjJ9XSxbMCwzLCIiLDIseyJjdXJ2ZSI6LTIsInN0eWxlIjp7ImhlYWQiOnsibmFtZSI6Im5vbmUifX19XSxbMywyLCIiLDIseyJjdXJ2ZSI6LTJ9XV0=
\[\begin{tikzcd}[column sep = small, row sep = small]
	& \ax \\
	{\neg A} && A \\
	& \cut
	\arrow[curve={height=12pt}, no head, from=1-2, to=2-1]
	\arrow[curve={height=12pt}, from=2-1, to=3-2]
	\arrow[curve={height=-12pt}, no head, from=1-2, to=2-3]
	\arrow[curve={height=-12pt}, from=2-3, to=3-2]
\end{tikzcd}\]
In fact, this can only happen in a proof \emph{structure}, whereas \emph{any} choice of unoriented atom inside a proof net induces a conclusion-conclusion path. This is Corollary \ref{cor:looping_persistent}.
\begin{defn}\label{def:persistent_maps}
	Let $l$ be an axiom or cut link in a proof structure $\pi$. Let $A, \neg A$ denote the conclusions of $l$ in the case where $l$ is an axiom link, or the premises in the case where $l$ is a cut link. In both cases, the order of the sequences of oriented axioms of $A$ and $\neg A$ induces a bijection from the set of unoriented axioms of $A$ to that of $\neg A$. In the following diagram, $X_1 \coprod \ldots \coprod X_n$ is the set of unoriented atoms of $A$ and $Y_1 \coprod \ldots \coprod Y_n$ that of $\neg A$.
	
	% https://q.uiver.app/?q=WzAsMTQsWzEsMCwiXFxheCJdLFswLDEsIkEiXSxbMiwxLCJcXG5lZyBBIl0sWzAsMiwiXFx2ZG90cyJdLFsyLDIsIlxcdmRvdHMiXSxbMCwzLCJYXzEgXFxjb3Byb2QgXFxoZG90cyBcXGNvcHJvZCBYX24iXSxbMiwzLCJZXzEgXFxjb3Byb2QgXFxoZG90cyBcXGNvcHJvZCBZX24iXSxbNCwzLCJcXGN1dCJdLFszLDIsIkEiXSxbNSwyLCJcXG5lZyBBIl0sWzUsMSwiXFx2ZG90cyJdLFszLDEsIlxcdmRvdHMiXSxbMywwLCJYXzEgXFxjb3Byb2QgXFxoZG90cyBcXGNvcHJvZCBYX24iXSxbNSwwLCJZXzEgXFxjb3Byb2QgXFxoZG90cyBcXGNvcHJvZCBZX24iXSxbMCwxLCIiLDAseyJjdXJ2ZSI6Miwic3R5bGUiOnsiaGVhZCI6eyJuYW1lIjoibm9uZSJ9fX1dLFsxLDNdLFswLDIsIiIsMix7ImN1cnZlIjotMiwic3R5bGUiOnsiaGVhZCI6eyJuYW1lIjoibm9uZSJ9fX1dLFsyLDRdLFs1LDYsIkkiLDAseyJjb2xvdXIiOlsyNDAsNjAsNjBdLCJzdHlsZSI6eyJ0YWlsIjp7Im5hbWUiOiJhcnJvd2hlYWQifX19LFsyNDAsNjAsNjAsMV1dLFs3LDksIiIsMCx7ImN1cnZlIjoyLCJzdHlsZSI6eyJoZWFkIjp7Im5hbWUiOiJub25lIn19fV0sWzksMTBdLFs3LDgsIiIsMix7ImN1cnZlIjotMiwic3R5bGUiOnsiaGVhZCI6eyJuYW1lIjoibm9uZSJ9fX1dLFs4LDExXSxbMTIsMTMsIkkiLDAseyJjb2xvdXIiOlsyNDAsNjAsNjBdLCJzdHlsZSI6eyJ0YWlsIjp7Im5hbWUiOiJhcnJvd2hlYWQifX19LFsyNDAsNjAsNjAsMV1dXQ==
	\[\begin{tikzcd}[ampersand replacement=\&, column sep = small, row sep = small]
			\& \ax \&\& {X_1 \coprod \hdots \coprod X_n} \&\& {Y_1 \coprod \hdots \coprod Y_n} \\
			A \&\& {\neg A} \& \vdots \&\& \vdots \\
			\vdots \&\& \vdots \& A \&\& {\neg A} \\
			{X_1 \coprod \hdots \coprod X_n} \&\& {Y_1 \coprod \hdots \coprod Y_n} \&\& \cut
			\arrow[curve={height=12pt}, no head, from=1-2, to=2-1]
			\arrow[from=2-1, to=3-1]
			\arrow[curve={height=-12pt}, no head, from=1-2, to=2-3]
			\arrow[from=2-3, to=3-3]
			\arrow[color={rgb,255:red,92;green,92;blue,214}, tail reversed, from=4-1, to=4-3]
			\arrow[curve={height=12pt}, no head, from=4-5, to=3-6]
			\arrow[from=3-6, to=2-6]
			\arrow[curve={height=-12pt}, no head, from=4-5, to=3-4]
			\arrow[from=3-4, to=2-4]
			\arrow[ color={rgb,255:red,92;green,92;blue,214}, tail reversed, from=1-4, to=1-6]
	\end{tikzcd}\]
	Say $l$ is a tensor or par link, with premises $A,B$ and conclusion $C$, say. The order of the sequences of oriented axioms of $A,B,C$ induce two injective functions, both with codomain the set of unoriented axioms of $C$. The domain of one of these functions is the unoriented axioms of $A$ and the other is the set of unoriented axioms of $B$. In the following diagram, $\boxtimes$ is either $\otimes$ or $\parr$.
	% https://q.uiver.app/?q=WzAsMTAsWzIsMiwiXFxib3h0aW1lcyJdLFsxLDEsIkEiXSxbMywxLCJCIl0sWzIsMywiQSBcXGJveHRpbWVzIEIiXSxbMiw0LCJcXHZkb3RzIl0sWzEsMCwiXFx2ZG90cyJdLFszLDAsIlxcdmRvdHMiXSxbMCwxLCJYXzEgXFxjb3Byb2QgXFxoZG90cyBcXGNvcHJvZCBYX24iXSxbNCwxLCJZXzEgXFxjb3Byb2QgXFxoZG90cyBcXGNvcHJvZCBZX20iXSxbMiw1LCJYXzEgXFxjb3Byb2QgXFxoZG90cyBcXGNvcHJvZCBYX24gXFxjb3Byb2QgWV8xIFxcY29wcm9kIFxcaGRvdHMgXFxjb3Byb2QgWV9tIl0sWzIsMCwiIiwwLHsiY3VydmUiOi0yfV0sWzEsMCwiIiwyLHsiY3VydmUiOjJ9XSxbMCwzLCIiLDIseyJzdHlsZSI6eyJoZWFkIjp7Im5hbWUiOiJub25lIn19fV0sWzMsNF0sWzYsMiwiIiwwLHsic3R5bGUiOnsiaGVhZCI6eyJuYW1lIjoibm9uZSJ9fX1dLFs1LDEsIiIsMix7InN0eWxlIjp7ImhlYWQiOnsibmFtZSI6Im5vbmUifX19XSxbOCw5LCJJXjIiLDIseyJjb2xvdXIiOlsyNDAsNjAsNjBdfSxbMjQwLDYwLDYwLDFdXSxbNyw5LCJJXjEiLDAseyJjb2xvdXIiOlsyNDAsNjAsNjBdfSxbMjQwLDYwLDYwLDFdXV0=
	\[\begin{tikzcd}[ampersand replacement=\&, column sep = small, row sep = small]
			\& \vdots \&\& \vdots \\
			{X_1 \coprod \hdots \coprod X_n} \& A \&\& B \& {Y_1 \coprod \hdots \coprod Y_m} \\
			\&\& \boxtimes \\
			\&\& {A \boxtimes B} \\
			\&\& \vdots \\
			\&\& {X_1 \coprod \hdots \coprod X_n \coprod Y_1 \coprod \hdots \coprod Y_m}
			\arrow[curve={height=-12pt}, from=2-4, to=3-3]
			\arrow[curve={height=12pt}, from=2-2, to=3-3]
			\arrow[no head, from=3-3, to=4-3]
			\arrow[from=4-3, to=5-3]
			\arrow[no head, from=1-4, to=2-4]
			\arrow[no head, from=1-2, to=2-2]
			\arrow[color={rgb,255:red,92;green,92;blue,214}, from=2-5, to=6-3]
			\arrow[ color={rgb,255:red,92;green,92;blue,214}, from=2-1, to=6-3]
	\end{tikzcd}\]
	To every non-conclusion edge $e$, labelled $A$ say, of a proof structure $\pi$, there exists for each unoriented atom $X$ of $A$ two edges adjacent to $e$ admitting a copy of $X$ amonst the set of unoriented axioms of the labelling formulae according to the above functions.
\end{defn}
\begin{defn}\label{def:prepersistent_path}
	Consider a non-conclusion link $l$ along with an unoriented atom $X$ of a formula $A$ labelling an edge $e$ of $l$. There is at most one other link $l'$ admitting $e$ as an edge, in which case there is a unique edge $e' \neq e$ admitting a copy $X'$ of $X$ amongst the unoriented atoms of the labelling formula which is related to $X$ via the functions of Definition \ref{def:persistent_maps}. A path (ie, a sequence of edges $\nu = (e_1,\ldots, e_n)$ in $\pi$) for which there exists an unoriented atom $X$ of $e_1$ so that each pair $(e_i,e_{i+1})$, $i = 1,\ldots, n-1$ are related via some non-conclusion link $l$ is a \textbf{prepersistent-path}.
	\end{defn}

%\begin{defn}\label{def:persistent_path}
%	Let $\pi$ be a proof structure. Choose an unoriented atom $X$ occurring in some formula $A$ labelling some edge of $\pi$. A \textbf{prepersistent-path} of $X$ is a path (ie, a sequence of edges) $\nu$ in $\pi$ where each edge is connected to its adjacent ones via the following rule:
%	
%	The unoriented atom $X$ uniquely determines adjacent edges with a copy of $X$ amongst the set of unoriented atoms of the labelling formulae under one of $J, I^1, I^2$.
%
%
%
%	\begin{enumerate}
%		\item\label{def:pers_walk_e_1} The formula $A$ labels some edge $e$, the first edge $e_1$ of $\nu$ is $e$.
%		\item\label{def:pers_walk_e_i} If $i  > 1$ then $X$ uniquely determines an edge $e_{i} \neq e_{i-1}$ adjacent to a common vertex with $e_{i-1}$, and so that the label of $e_i$ admits a unique copy of $X$ amongst its set of unoriented axioms corresponding under one of $J,I^1,I^2$ to the copy of $X$ associated to $e_{i-1}$.
%	\end{enumerate}
%\end{defn}
It is not obvious from Definition \ref{def:prepersistent_path} but the set of prepersistent-paths of $X$, where $X$ is an unoriented atom of a formula labelling a conclusion, is finite. Thus, there exists a maximal length such prepersistent-path which will be of particular interest. First we prove that indeed this set is finite which will follow as a Corollary to the following Lemma.

Such a prepersistent-path may traverse the same edge more than once, but not the same unoriented atom more than once (see Lemma \ref{lem:non_repeating} below). The claim then follows by finiteness of the graph and all its formulas.

\begin{lemma}\label{lem:non_repeating}
	Let $\pi$ be a proof structure and $\nu = (e_1,\ldots, e_n)$ be a prepersistent-path of an unoriented atom $X$ of a conclusion $A$. For $i = 1,...,n$ say $e_i$ is labelled by $A_i$. Let $X_i$ denote the copy of $X$ in the set of unoriented atoms of $A_i$. If $i \neq j$ are such that $e_i = e_j$ then $X_i \neq X_j$.
\end{lemma}
\begin{proof}
	Let $e_i$ denote the $i^\text{th}$ element of the sequence $\nu$. Let $p(i)$ denote the following property.
	\begin{equation}
		p(i) := \exists j > i\text{ such that } e_i = e_j \text{ and }X_i = X_j
	\end{equation}
	First we claim that there exists $i$ such that $p(e_i)$ does \emph{not} hold. Say $p(e_1)$ holds. Then there exists $j$ such that $i \neq j$ and $e_i = e_j$. This means that the path $\nu$ eventually returns to the edge $e_1$ which is incident to a conclusion vertex. Hence so is $e_j$, so by Definition \ref{def:prepersistent_path}, the path $\nu$ is finite and $e_j$ is the final element of the path. It then follows that $p(e_j)$ does not hold as there is no $k > j$ such that $e_k$ exists.
	
	On the other hand, if it were the case that for all $i$ the statement $p(e_i)$ does not hold, then there is nothing to prove, so assume this is not the case.
	
	Hence we can let $i$ be the maximal integer such that $p(e_i)$ holds and $p(e_{i+1})$ does not hold. Let $j$ be such that $e_i = e_j$ and $X_i = X_j$. Let $e$ denote the edge $e_i = e_j$. Then $e$ is either a conclusion to a tensor or par link, or it is not. First consider the case when it is. Definition \ref{def:unoriented_axiom_maps} makes it clear that $\operatorname{im}I_l^1 \cap \operatorname{im}I_l^2 = \varnothing$, and so the statement $X_i = X_j$ is contradictory. Hence, the edge $e$ is not a conclusion to a tensor link nor to a par link.
	
	Say $e$ is premise to a cut link. Since $e$ is not the conclusion to a tensor link nor a par link, $e$ must be the conclusion to an axiom link. Denote the other conclusion to this axiom link by $e'$. We either have that $e' = e_{i-1}$ or $e'=e_{i+1}$. Say $e' = e_{i+1}$, then $X_1 = X_2$ implies that $X_{i+1}$ is either equal to $X_{j+1}$ or $X_{j-1}$, depending on the direction of the path at the point $e_j$, in either case though we contradict maximality of the integer $i$.
	
	Now say $e' = e_{i-1}$, we can similarly contradict maximality of $i$ by considering the other premise to the cut link which must be the edge $e_{i+1}$.
	
	The final case is when $e$ is conclusion to an axiom link. Then $e_{i+1}$ is either the other conclusion $e'$ to the axiom link, or a premise to some other link $l'$, each case contradicts maximality of $i$.
\end{proof}
\begin{cor}\label{cor:one_max_persistent}
	There is exactly one maximal prepersistent-path of $X$.
\end{cor}
\begin{proof}
	Given a prepersistent-path $\nu$ let $l(\nu)$ denote its length. Say $\nu,\nu'$ are prepersisten-paths of $X$ and $l(\nu) < l(\nu')$, we first claim that $\nu$ is a subpath of $\nu'$. This follows from an easy induction on $l(\nu')$.
	
	Hence it remains to show that prepersistent-paths of $X$ cannot be arbitrarily long. Although it is the case that prepersistent-paths can admit cycles, they cannot cycle indefinitely, this is because the set of unoriented atoms of each formula in $\pi$ is a finite set, and Lemma \ref{lem:non_repeating} shows that a distinct copy of $X$ is associated to each edge in a prepersistent-path.
\end{proof}
\begin{defn}
	A \textbf{persistent path} is a maximal length prepersistent-path of an unoriented atom $X$ of a conclusion $A$.
	\end{defn}
\begin{example}\label{ex:favourite_example}
	Let $\pi$ denote the following proof structure.
	% https://q.uiver.app/?q=WzAsMTYsWzEsMCwiXFxheCJdLFswLDEsIlxcbmVnIFgiXSxbMiwxLCJYIl0sWzMsMiwiXFxvdGltZXMiXSxbNCwxLCJcXG5lZyBYIl0sWzUsMCwiXFxheCJdLFs2LDEsIlgiXSxbMywzLCJYIFxcb3RpbWVzIFxcbmVnIFgiXSxbNywxLCJcXG5lZyBYIl0sWzgsMCwiXFxheCJdLFs5LDEsIlgiXSxbOCwyLCJcXHBhcnIiXSxbOCwzLCJcXG5lZyBYIFxccGFyciBYIl0sWzUsNCwiXFxjdXQiXSxbNiwyLCJcXG9wZXJhdG9ybmFtZXtjfSJdLFswLDIsIlxcb3BlcmF0b3JuYW1le2N9Il0sWzcsMTMsIiIsMCx7ImN1cnZlIjoyfV0sWzEyLDEzLCIiLDIseyJjdXJ2ZSI6LTJ9XSxbMTEsMTIsIiIsMix7InN0eWxlIjp7ImhlYWQiOnsibmFtZSI6Im5vbmUifX19XSxbMTAsMTEsIiIsMix7ImN1cnZlIjotMn1dLFs4LDExLCIiLDIseyJjdXJ2ZSI6Mn1dLFs5LDEwLCIiLDIseyJjdXJ2ZSI6LTIsInN0eWxlIjp7ImhlYWQiOnsibmFtZSI6Im5vbmUifX19XSxbOSw4LCIiLDEseyJjdXJ2ZSI6Miwic3R5bGUiOnsiaGVhZCI6eyJuYW1lIjoibm9uZSJ9fX1dLFs1LDQsIiIsMix7ImN1cnZlIjoyLCJzdHlsZSI6eyJoZWFkIjp7Im5hbWUiOiJub25lIn19fV0sWzUsNiwiIiwwLHsiY3VydmUiOi0yLCJzdHlsZSI6eyJoZWFkIjp7Im5hbWUiOiJub25lIn19fV0sWzYsMTRdLFs0LDMsIiIsMix7ImN1cnZlIjotMn1dLFsyLDMsIiIsMCx7ImN1cnZlIjoyfV0sWzAsMiwiIiwwLHsiY3VydmUiOi0yLCJzdHlsZSI6eyJoZWFkIjp7Im5hbWUiOiJub25lIn19fV0sWzAsMSwiIiwyLHsiY3VydmUiOjIsInN0eWxlIjp7ImhlYWQiOnsibmFtZSI6Im5vbmUifX19XSxbMSwxNV0sWzMsNywiIiwyLHsic3R5bGUiOnsiaGVhZCI6eyJuYW1lIjoibm9uZSJ9fX1dXQ==
	\begin{equation}\label{eq:standard_example}
		\begin{tikzcd}[column sep = small, row sep = small]
			& \ax &&&& \ax &&& \ax \\
			{\neg X} && X && {\neg X} && X & {\neg X} && X \\
			{\operatorname{c}} &&& \otimes &&& {\operatorname{c}} && \parr \\
			&&& {X \otimes \neg X} &&&&& {\neg X \parr X} \\
			&&&&& \cut
			\arrow[curve={height=12pt}, from=4-4, to=5-6]
			\arrow[curve={height=-12pt}, from=4-9, to=5-6]
			\arrow[no head, from=3-9, to=4-9]
			\arrow[curve={height=-12pt}, from=2-10, to=3-9]
			\arrow[curve={height=12pt}, from=2-8, to=3-9]
			\arrow[curve={height=-12pt}, no head, from=1-9, to=2-10]
			\arrow[curve={height=12pt}, no head, from=1-9, to=2-8]
			\arrow[curve={height=12pt}, no head, from=1-6, to=2-5]
			\arrow[curve={height=-12pt}, no head, from=1-6, to=2-7]
			\arrow[from=2-7, to=3-7]
			\arrow[curve={height=-12pt}, from=2-5, to=3-4]
			\arrow[curve={height=12pt}, from=2-3, to=3-4]
			\arrow[curve={height=-12pt}, no head, from=1-2, to=2-3]
			\arrow[curve={height=12pt}, no head, from=1-2, to=2-1]
			\arrow[from=2-1, to=3-1]
			\arrow[no head, from=3-4, to=4-4]
		\end{tikzcd}
	\end{equation}
	Indeed, $\pi$ is a proof \emph{net}, but we will not use this structure here. For clarity, we artificially place labels on the formulas so that we can refer to particular edges, but for all $i = 1,...,8$ the notation $Z_i$, where $Z = X, \neg X, X \otimes \neg X, \neg X \parr X$, denotes the formula $Z$.
	% https://q.uiver.app/?q=WzAsMTYsWzEsMCwiXFxheCJdLFswLDEsIlxcbmVnIFhfMSJdLFsyLDEsIlhfMiJdLFszLDIsIlxcb3RpbWVzIl0sWzQsMSwiXFxuZWcgWF8zIl0sWzUsMCwiXFxheCJdLFs2LDEsIlhfNCJdLFszLDMsIihYIFxcb3RpbWVzIFxcbmVnIFgpXzciXSxbNywxLCJcXG5lZyBYXzUiXSxbOCwwLCJcXGF4Il0sWzksMSwiWF82Il0sWzgsMiwiXFxwYXJyIl0sWzgsMywiKFxcbmVnIFggXFxwYXJyIFgpXzgiXSxbNSw0LCJcXGN1dCJdLFs2LDIsIlxcb3BlcmF0b3JuYW1le2N9Il0sWzAsMiwiXFxvcGVyYXRvcm5hbWV7Y30iXSxbNywxMywiIiwwLHsiY3VydmUiOjJ9XSxbMTIsMTMsIiIsMix7ImN1cnZlIjotMn1dLFsxMSwxMiwiIiwyLHsic3R5bGUiOnsiaGVhZCI6eyJuYW1lIjoibm9uZSJ9fX1dLFsxMCwxMSwiIiwyLHsiY3VydmUiOi0yfV0sWzgsMTEsIiIsMix7ImN1cnZlIjoyfV0sWzksMTAsIiIsMix7ImN1cnZlIjotMiwic3R5bGUiOnsiaGVhZCI6eyJuYW1lIjoibm9uZSJ9fX1dLFs5LDgsIiIsMSx7ImN1cnZlIjoyLCJzdHlsZSI6eyJoZWFkIjp7Im5hbWUiOiJub25lIn19fV0sWzUsNCwiIiwyLHsiY3VydmUiOjIsInN0eWxlIjp7ImhlYWQiOnsibmFtZSI6Im5vbmUifX19XSxbNSw2LCIiLDAseyJjdXJ2ZSI6LTIsInN0eWxlIjp7ImhlYWQiOnsibmFtZSI6Im5vbmUifX19XSxbNiwxNF0sWzQsMywiIiwyLHsiY3VydmUiOi0yfV0sWzIsMywiIiwwLHsiY3VydmUiOjJ9XSxbMCwyLCIiLDAseyJjdXJ2ZSI6LTIsInN0eWxlIjp7ImhlYWQiOnsibmFtZSI6Im5vbmUifX19XSxbMCwxLCIiLDIseyJjdXJ2ZSI6Miwic3R5bGUiOnsiaGVhZCI6eyJuYW1lIjoibm9uZSJ9fX1dLFsxLDE1XSxbMyw3LCIiLDIseyJzdHlsZSI6eyJoZWFkIjp7Im5hbWUiOiJub25lIn19fV1d
	\begin{equation}
		\begin{tikzcd}[column sep = small, row sep = small]
			& \ax &&&& \ax &&& \ax \\
			{\neg X_1} && {X_2} && {\neg X_3} && {X_4} & {\neg X_5} && {X_6} \\
			{\operatorname{c}} &&& \otimes &&& {\operatorname{c}} && \parr \\
			&&& {(X \otimes \neg X)_7} &&&&& {(\neg X \parr X)_8} \\
			&&&&& \cut
			\arrow[curve={height=12pt}, from=4-4, to=5-6]
			\arrow[curve={height=-12pt}, from=4-9, to=5-6]
			\arrow[no head, from=3-9, to=4-9]
			\arrow[curve={height=-12pt}, from=2-10, to=3-9]
			\arrow[curve={height=12pt}, from=2-8, to=3-9]
			\arrow[curve={height=-12pt}, no head, from=1-9, to=2-10]
			\arrow[curve={height=12pt}, no head, from=1-9, to=2-8]
			\arrow[curve={height=12pt}, no head, from=1-6, to=2-5]
			\arrow[curve={height=-12pt}, no head, from=1-6, to=2-7]
			\arrow[from=2-7, to=3-7]
			\arrow[curve={height=-12pt}, from=2-5, to=3-4]
			\arrow[curve={height=12pt}, from=2-3, to=3-4]
			\arrow[curve={height=-12pt}, no head, from=1-2, to=2-3]
			\arrow[curve={height=12pt}, no head, from=1-2, to=2-1]
			\arrow[from=2-1, to=3-1]
			\arrow[no head, from=3-4, to=4-4]
		\end{tikzcd}
	\end{equation}
	Consider the conclusion $\neg X_1$. This is conclusion to an axiom link, and so the next element of the persistent path of $\neg X_1$ is the edge labelled $X_2$. The unique directed path from $X_2$ to either a conclusion or a cut link is the path $X_2, (X \otimes \neg X)_7$. The formula $(X \otimes \neg X)_7$ is premise to a cut link with other premise $(\neg X \parr X)_8$. Now, the oriented atoms of $(\neg X \parr X)_8$ is the sequence $(\neg X, X)$ and $\neg X$ is the first element, hence the following edge in the persistent path is $\neg X_5$. We continue in this way to eventually obtain the following path, which is the unique maximal length persistent path of $\neg X_1$.
	\begin{equation}
		\neg X_1, X_2 (X \otimes \neg X)_7, (\neg X \parr X)_8, \neg X_5, X_6, (\neg X \parr X)_8, (X \otimes \neg X)_7, \neg X_3, X_4
	\end{equation}
\end{example}

\begin{lemma}
	Let $\pi$ be a proof net. Then every unoriented atom of every formula which appears in $\pi$ is associated to some edge of some persistent path.
	\end{lemma}
\begin{proof}
	Let $X$ be an unoriented atom of some formula $A$ labelling an edge $e: v \lto v'$ in $\pi$. There exists edges $e',e''$ incident respectively to $v,v'$ as determined by $X$ via the functions $J,I^1,I^2$. Thus, the edge $e$ can be extended to a path which either is a loop, or a conclusion to conclusion path.
	
	If the resulting path is conclusion to conclusion, then the resulting path is persistent, and we are done.
	
	On the other hand, if the resulting path is a loop, then a short trip can be extracted from this, contradicting the fact that $\pi$ is a proof net.
	\end{proof}





\begin{cor}\label{cor:looping_persistent}
	If a proof structure $\pi$ admits a looping prepersistent-path then $\pi$ is not a proof net.
	\end{cor}

\subsection{Proofs as operators}\label{sec:proofs_operators}
\begin{defn}\label{def:path_operator}
	Let $\pi$ be a proof structure, $\rho = (e_1,...,e_n)$ an undirected path in $\pi$, and assume moreover that each edge in $\pi$ is labelled by an element of the set $\lbrace p, q, \operatorname{id} \rbrace$. Hence, each edge $e_i$ in $\rho$ is associated to a letter $w_i$ in the alphabet $\lbrace p, q, \operatorname{id}, p^\ast, q^\ast, \operatorname{id}^\ast \rbrace$ where if $e_i$ is labelled $r_i$ and $e_i$ is traversed in the correct direction, $w_i = r_i$, and if $e_i$ is traversed in the reverse direction then $w_i = r_i^\ast$. To each of these letters $w_i$ let $\overline{w_i}$ denote the operator of the same name as $w_i$ given in Example \ref{ex:p_q}. We define the following operator induced by $\rho$.
	\begin{equation}
		o_{\rho} := \overline{w_n} \circ \hdots \circ \overline{w_1}
		\end{equation}
	\end{defn}

\begin{defn}\label{def:matrix_interpretation}
	Let $\pi$ be a proof structure, for each cut link $l$ with premises $\neg A, A$ replace the vertex labelled $\cut$ in $l$ by two vertices labelled $\operatorname{c}$ and make the edges $\neg A, A$ respectively be incident to one of these vertices labelled $\operatorname{c}$. The result is a proof structure $\pi'$, say this proof structure has conclusions $A_1,...,A_n$.  Let $m \leq n$ be an integer and assume that $A_1,...,A_n$ are ordered so that $A_1,...,A_m$ are conclusions of $\pi'$ but \emph{not} conclusions of $\pi$ (in other words, $A_1,...,A_m$ are the cut link premises of $\pi$). Assume also that $A_1,...,A_m$ are labelled so that for all odd $k < m $ there exists a cut link in $\pi$ with premises $A_k,A_{k+1}$.  We construct an $n \times n$ matrix $\llbracket \pi \rrbracket$ via the following procedure.
	\begin{enumerate}
		\item Label the left premise of every tensor link and every par link by $p$, label every right premise of every tensor link and every par link by $q$. Label the remaining edges by $\operatorname{id}$.
		\item For each $i, j \in \lbrace 1,...,n\rbrace$ there is a collection of maximal length persistent paths $\nu_1,...,\nu_l$ of an unoriented atom of $A_i$ and whose final edge is labelled $A_j$.
		\item\label{def:matrix_induced_operator} To each path $\nu_t$ let $o_{\nu_t}$ denote the corresponding operator as defined in Definition \ref{def:path_operator}.
		\item We then define:
		\begin{equation}
			(\llbracket \pi \rrbracket)_{ji} := o_1 + \hdots + o_l
		\end{equation}
	\end{enumerate}
\end{defn}
\begin{example}\label{ex:favourite_example_interpretation}
	Consider $\pi$ of Example \ref{ex:favourite_example}. We remove the cut link to obtain a proof-structure $\pi'$. Label the left premise of each tensor and each par link by $p$ and the right premise of each tensor and each par link by $q$ (indeed these are the same $p$ and $q$ as in Section \ref{sec:interal_sum_tens}). Label the remaining edges by the identity map $\operatorname{id}$ (this is the identity on the space $\ell^2$). For convenience, we have added artificial labels to the formulas.
	% https://q.uiver.app/?q=WzAsMTcsWzEsMCwiXFxheCJdLFswLDEsIlxcbmVnIFgiXSxbMiwxLCJYIl0sWzMsMiwiXFxvdGltZXMiXSxbNCwxLCJcXG5lZyBYIl0sWzUsMCwiXFxheCJdLFs2LDEsIlgiXSxbMywzLCJYIFxcb3RpbWVzIFxcbmVnIFgiXSxbNywxLCJcXG5lZyBYIl0sWzgsMCwiXFxheCJdLFs5LDEsIlgiXSxbOCwyLCJcXHBhcnIiXSxbOCwzLCJcXG5lZyBYIFxccGFyciBYIl0sWzYsMiwiXFxvcGVyYXRvcm5hbWV7Y30iXSxbMCwyLCJcXG9wZXJhdG9ybmFtZXtjfSJdLFszLDQsIlxcb3BlcmF0b3JuYW1le2N9Il0sWzgsNCwiXFxvcGVyYXRvcm5hbWV7Y30iXSxbMTEsMTIsIiIsMix7InN0eWxlIjp7ImhlYWQiOnsibmFtZSI6Im5vbmUifX19XSxbMTAsMTEsInEiLDAseyJjdXJ2ZSI6LTJ9XSxbOCwxMSwicCIsMix7ImN1cnZlIjoyfV0sWzksMTAsIiIsMix7ImN1cnZlIjotMiwic3R5bGUiOnsiaGVhZCI6eyJuYW1lIjoibm9uZSJ9fX1dLFs5LDgsIiIsMSx7ImN1cnZlIjoyLCJzdHlsZSI6eyJoZWFkIjp7Im5hbWUiOiJub25lIn19fV0sWzUsNCwiIiwyLHsiY3VydmUiOjIsInN0eWxlIjp7ImhlYWQiOnsibmFtZSI6Im5vbmUifX19XSxbNSw2LCIiLDAseyJjdXJ2ZSI6LTIsInN0eWxlIjp7ImhlYWQiOnsibmFtZSI6Im5vbmUifX19XSxbNiwxMywiXFxvcGVyYXRvcm5hbWV7aWR9Il0sWzQsMywicSIsMCx7ImN1cnZlIjotMn1dLFsyLDMsInAiLDIseyJjdXJ2ZSI6Mn1dLFswLDIsIiIsMCx7ImN1cnZlIjotMiwic3R5bGUiOnsiaGVhZCI6eyJuYW1lIjoibm9uZSJ9fX1dLFswLDEsIiIsMix7ImN1cnZlIjoyLCJzdHlsZSI6eyJoZWFkIjp7Im5hbWUiOiJub25lIn19fV0sWzEsMTQsIlxcb3BlcmF0b3JuYW1le2lkfSIsMl0sWzMsNywiIiwyLHsic3R5bGUiOnsiaGVhZCI6eyJuYW1lIjoibm9uZSJ9fX1dLFs3LDE1LCJcXG9wZXJhdG9ybmFtZXtpZH0iLDJdLFsxMiwxNiwiXFxvcGVyYXRvcm5hbWV7aWR9IiwyXV0=
	\begin{equation}
		\begin{tikzcd}[column sep = small, row sep = small]
			& \ax &&&& \ax &&& \ax \\
			{\neg X_1} && X_2 && {\neg X_3} && X_4 & {\neg X_5} && X_6 \\
			{\operatorname{c}} &&& \otimes &&& {\operatorname{c}} && \parr \\
			&&& {(X \otimes \neg X)_7} &&&&& {(\neg X \parr X)_8} \\
			&&& {\operatorname{c}} &&&&& {\operatorname{c}}
			\arrow[no head, from=3-9, to=4-9]
			\arrow["q", curve={height=-12pt}, from=2-10, to=3-9]
			\arrow["p"', curve={height=12pt}, from=2-8, to=3-9]
			\arrow[curve={height=-12pt}, no head, from=1-9, to=2-10]
			\arrow[curve={height=12pt}, no head, from=1-9, to=2-8]
			\arrow[curve={height=12pt}, no head, from=1-6, to=2-5]
			\arrow[curve={height=-12pt}, no head, from=1-6, to=2-7]
			\arrow["{\operatorname{id}}", from=2-7, to=3-7]
			\arrow["q", curve={height=-12pt}, from=2-5, to=3-4]
			\arrow["p"', curve={height=12pt}, from=2-3, to=3-4]
			\arrow[curve={height=-12pt}, no head, from=1-2, to=2-3]
			\arrow[curve={height=12pt}, no head, from=1-2, to=2-1]
			\arrow["{\operatorname{id}}"', from=2-1, to=3-1]
			\arrow[no head, from=3-4, to=4-4]
			\arrow["{\operatorname{id}}"', from=4-4, to=5-4]
			\arrow["{\operatorname{id}}"', from=4-9, to=5-9]
		\end{tikzcd}
	\end{equation}
Now we calculate the persistent paths in $\pi'$ along with their associated linear operators. These are as follows.
\begin{align}
	\nu_1 &= (\neg X_1, X_2, (X \otimes \neg X)_7) & o_{\nu_1} &= \operatorname{id} p\operatorname{id}^\ast = p\\
	\nu_2 &= ((X \otimes \neg X)_7, X_2, \neg X_1) & o_{\nu_2} &= \operatorname{id} p^\ast \operatorname{id}^\ast = p^\ast\\
	\nu_3 &= (X_4, \neg X_3, (X \otimes \neg X)_7) & o_{\nu_3} &=\operatorname{id}q\operatorname{id}^\ast = q\\
	\nu_4 &= ((X \otimes \neg X)_7, \neg X_3, X_4) & o_{\nu_4} &= \operatorname{id}q^\ast\operatorname{id}^\ast = q^\ast\\
	\nu_5 &= ((\neg X \parr X)_8, \neg X_5, X_6, (\neg X\parr X)_8) & o_{\nu_5} &= \operatorname{id}q p^\ast \operatorname{id}^\ast = q p^\ast\\
	\nu_6 &= ((\neg X \parr X)_8, X_6, \neg X_5, (\neg X \parr X)_8) & o_{\nu_6} &= \operatorname{id} p q^\ast \operatorname{id}^\ast = p q^\ast
	\end{align}
Hence $\llbracket \pi \rrbracket$ is the following matrix $4 \times 4$ matrix, where we assume respectively that index $1,2,3,4$ corresponds to conclusion $(X \otimes \neg X)_7, (\neg X \parr X)_8, \neg X_1, X_4$.
\begin{equation}
	\llbracket \pi \rrbracket =
	\begin{pmatrix}
		0 & 0 & p & q\\
		0 & qp^\ast + pq^\ast & 0 & 0\\
		p^\ast & 0 & 0 & 0\\
		q^\ast & 0 & 0 & 0
		\end{pmatrix}
	\end{equation}
	\begin{remark}
		There are more paths which begin and end at conclusions in $\pi'$ than the persistent paths $\nu_1,...,\nu_6$. For example, there is the following path.
		\begin{equation}
			\rho := (\neg X_1, X_2, \neg X_3, X_4)
			\end{equation}
		The path $\rho$ has corresponding operator $o_\rho = q^\ast p$. We notice that this is the zero operator. We will see in Proposition \ref{prop:not_persistent} that this is not a coincidence.
	\end{remark}
	\end{example}
\begin{proposition}\label{prop:not_persistent}
	Consider a path $\rho$ in a proof net $\pi$ which begins and ends respectively at conclusions $A$ and $B$ of $\pi$, and assume that $\rho$ never traverses the same edge twice in a row. Consider the corresponding operator $o_{\rho}$. We have that $o_{\rho} \neq 0$ if and only if $\rho$ is persistent (Definition \ref{def:persistent_path}).
\end{proposition}
\begin{proof}
	We can view $o_{\rho}$ as a word $w$ written over the alphabet $\lbrace p, q, p^\ast, q^\ast, \operatorname{id}, \operatorname{id}^\ast\rbrace$. Say $o_{\rho} = 0$. By Corollary \ref{cor:operator_zero} we have that $w$ admits either $q^\ast p$ or $p^\ast q$ as a subword. This means that $\rho$ admits a pair of consecutive edges $(e,e')$ which are premise to either a common tensor link or a common par link. This violates Definition \ref{def:persistent_path} and so $\rho$ is not persistent.
	
	Now assume that $o_{\rho} \neq 0$. Again, by Corollary \ref{cor:operator_zero} we have that there exists no subword of the form $q^\ast p$ nor of the form $p^\ast q$ in $w$. Write $w = w_1\hdots w_n$ so that $o_{\rho} = \overline{w_n} \circ \hdots \circ \overline{w_1}$ (see Definition \ref{def:path_operator}). First we will show that for each $i=2,...,n$ there exists some unoriented axiom of $A_i$ associated to some unoriented axiom of $A_{i-1}$ under one of $J,I^1,I^2$. Once this is established, we can then infer that for each $i=1,...,n$ there exists an unoriented axiom $Y_i$ of $A_i$ so that $Y_i$ corresponds under one of $J,I^1,I^2$ to $Y_{i+1}$. (Note, this choice of $Y_1,...,Y_n$ is not uniquely determined by $\rho$, but such a choice always exists). Then, we will show that $\rho$ is a persistent path of $Y_1$.
	
	Since $e_1$ is a conclusion of $\pi$, it is either a conclusion to an axiom, tensor, or par link. Say $e_1$ is conclusion to an axiom link $l$. The path $\rho$ does not traverse the same edge twice in a row, and so $e_2$ is necessarily the other conclusion of $l$. Hence, every unoriented atom of $A_2$ is associated to some unoriented atom of $A_1$ under $J$. Now say $l$ is a tensor or par link. Again, since $\rho$ does not traverse the same edge twice in a row, the edge $e_2$ is necessarily a premise of $l$. If $e_2$ is the left premise, then every unoriented axiom of $A_2$ corresponds under $I^1$ to an unoriented axiom of $A_1$, and if $e_2$ is the right premise, then every unoriented axiom of $A_2$ corresponds under $l^2$ to an unoriented axiom of $A_2$.
	
	Now say $i > 1$ and consider the subword $w_{i-1}w_i$ of $w$. If $e_{i-1}$ is not a premise to some tensor nor premise to some par link, then $e_i$ is completely determined by the fact that $\rho$ does not traverse the same edge twice in a row, that is, $e_i$ is either premise to a cut link or conclusion to an axiom link, in either case all the unoriented axioms of $A_i$ correspond under $J$ to some unoriented axiom of $A_{i-1}$.
	
	Now consider the case where $e_{i-1}$ is premise to a link $l$ where $l$ is either a tensor or a par link. We know from Corollary \ref{cor:operator_zero} that the word $w_{i-1}w_i$ is not $q^\ast p$ nor $p^\ast q$. Hence $e_i$ is not the other premise of $l$. This ensures that some unoriented axiom of $A_i$ corresponds under $J, I^1, I^2$ to an unoriented axiom of $A_{i-1}$.
	
	Hence, for each $i=1,...,n$ we can pick an unoriented axiom $Y_i$ of $A_i$ so that for each $i = 2,...,n$ the unoriented axiom $Y_i$ corresponds under one of $J,I^1,I^2$ to $Y_{i-1}$. We claim that $\rho$ is a persistent path of $Y_1$.
	
	This follows by construction. The only condition to check is \eqref{def:pers_path_e_i} of Definition \ref{def:persistent_path}, but this is fulfilled by the defining property of the sequence $Y_1,...,Y_n$.
\end{proof}
\begin{defn}\label{def:sigma}
	Let $\pi$ be a proof structure and $\zeta$ the corresponding proof structure given by removing all edges corresponding to cut-links in $\pi$. Then if $A_1,...,A_n$ are the conclusions of $\zeta$ and $m < n$ is such that $A_1,...,A_m$ are conclusions of $\zeta$ but \emph{not} of $\pi$, then we let $\sigma_m$ denote the $(2m + n) \times (2m + n)$ matrix whose top left $2m \times 2m$ minor is given by the following.
	\begin{equation}
		\begin{pmatrix}
			0 & 1\\
			1 & 0\\
			& & \ddots\\
			& & & 0 & 1\\
			& & & 1 & 0
		\end{pmatrix}
	\end{equation}
	The rest of the entries are 0.
\end{defn}

\begin{remark}
	Let $\pi$ be a proof structure admitting a conclusion $A$ and let $\pi'$ be a proof structure admitting a conclusion $\neg A$. Let $\zeta$ denote the proof structure given by placing $\pi$ and $\pi'$ side by side and constructing a cut link with one premise $A$ of $\pi$ and the other premise $\neg A$ of $\pi'$.
	
	% https://q.uiver.app/?q=WzAsNSxbMCwwLCJcXHBpIl0sWzIsMCwiXFxwaSciXSxbMCwxLCJBIl0sWzIsMSwiXFxuZWcgQSJdLFsxLDIsIlxcY3V0Il0sWzIsNCwiIiwwLHsiY3VydmUiOjJ9XSxbMyw0LCIiLDIseyJjdXJ2ZSI6LTJ9XSxbMSwzLCIiLDIseyJzdHlsZSI6eyJoZWFkIjp7Im5hbWUiOiJub25lIn19fV0sWzAsMiwiIiwwLHsic3R5bGUiOnsiaGVhZCI6eyJuYW1lIjoibm9uZSJ9fX1dXQ==
	\begin{equation}\label{eq:cut_result}
		\begin{tikzcd}[column sep = small, row sep = small]
			\pi && {\pi'} \\
			A && {\neg A} \\
			& \cut
			\arrow[curve={height=12pt}, from=2-1, to=3-2]
			\arrow[curve={height=-12pt}, from=2-3, to=3-2]
			\arrow[dashed, no head, from=1-3, to=2-3]
			\arrow[dashed, no head, from=1-1, to=2-1]
		\end{tikzcd}
	\end{equation}
	Let $X$ be some unoriented atom of $A$ and denote by $\nu$ the maximal persistent path in $\pi$ of $X$. Corresponding to $X$ is a copy of $X$ amongst the set of unoriented atoms of $\neg A$. Let $\mu$ denote the maximal persistent path in $\pi'$ of this copy of $X$. Finally, let $e,e'$ respevtively denote the left and right premise of the cut link in $\zeta$ displayed in \eqref{eq:cut_result}. Associated to $\nu$ is formula $B$ which is a conclusion of $\pi$ and labels the final edge of $\nu$. Amongst the set of unoriented atoms of $B$ is a copy of $X$. We now consider the maximal persistent path $\theta$ of this copy of $X$ associated to $B$ in the proof net $\zeta$.
	
	In fact, $\theta$ is easy to describe, it is the following concatenation, where $\cdot$ means concatenation of paths.
	\begin{equation}\label{eq:cut_relation}
		\theta = \nu \cdot e \cdot e' \cdot \mu
	\end{equation}
\end{remark}

\begin{lemma}\label{lem:grip}
	Let $\zeta$ be a proof structure and let $A_i,A_j$ be (possibly equal) conclusions of $\zeta$. Let $\theta_1,...,\theta_e$ denote the persistent paths in $\zeta$ corresponding to atomic axioms of $A_i$ and whose final edge is labelled $A_j$, where each of $\theta_1,...,\theta_e$ traverses a cut link in $\zeta$ exactly one time. Let $o_1,...,o_e$ denote the induced operators as described in Definition \ref{def:path_operator}. Then the $ji^{\text{th}}$ entry of $\llbracket \zeta \rrbracket \sigma_m \llbracket \zeta \rrbracket$ is given by the sum $o_1 + \hdots + o_e$.
\end{lemma}
\begin{proof}
	Let $A_1,...,A_n$ be the conclusions of the proof structure $\zeta'$ given by removing all the cut links from $\zeta$, moreover, assume that $A_1,...,A_n$ are ordered so that $A_1,...,A_m$ are premises to cut links in $\zeta$ and for all odd $k < m$ assume $A_k,A_{k+1}$ are premise to a common cut link $l_k$. For each $i,j \in \lbrace 1,..., n \rbrace$ and each $k$, let $\nu_1^k,...,\nu_l^k$ be the collection of maximal length persistent paths in $\zeta'$ corresponding to the unoriented atoms of $A_i$ such that the final edge in each $\nu_1^k,...,\nu_l^k$ is labelled $A_k$, and let $\mu_1^{k+1},...,\mu_{l'}^{k+1}$ be the collection of maximal length persistent paths in $\zeta'$ corresponding to the unoriented atoms of $A_{k+1}$ such that the final edge in each $\mu_1,...,\mu_l$ is labelled $A_j$. Then by \eqref{eq:cut_relation} we have that the set \eqref{eq:k_set} below is the formal sum of maximal length persistent paths in $\zeta$ (note: not $\zeta'$) corresponding to unoriented atoms of $A_i$ such that the final edge is labelled $A_j$, and where the cut link $l_k$ is traversed exactly one time.
	\begin{equation}\label{eq:k_set}
		\Big\lbrace \nu_p^k \cdot e \cdot e' \cdot \mu_q^{k+1} \mid 1 \leq p \leq l,\text{ }1 \leq q \leq l'\Big\rbrace
	\end{equation}
	Ranging over all $k$, we see that \eqref{eq:ji_nightmare} below is the set of all maximal length persistent paths in $\zeta$ corresponding to unoriented atoms of $A_i$ such that the final edge is labelled $A_j$, and where some cut link of $\zeta$ is traversed exactly one time is given as follows.
	\begin{equation}\label{eq:ji_nightmare}
		\Big\lbrace \nu_p^k \cdot e \cdot e' \cdot \mu_q^{k+1} \mid 1 \leq p \leq l,\text{ }1 \leq q \leq l',\text{ }0 < k,\text{ }k\text{ odd} \Big\rbrace
	\end{equation}
	We now show that the sum of the operators in \eqref{eq:ji_nightmare} is equal to the $ji^{\text{th}}$ entry of $\llbracket \zeta \rrbracket \sigma \llbracket \zeta \rrbracket$. This is a calculation.
	\begin{align}
		(\llbracket \zeta \rrbracket \sigma_m \llbracket \zeta \rrbracket)_{ji} &= \Big(\sum_{\substack{k < m \\ k\text{ odd}}}(o_{\mu_1}^{k+1} + \hdots + o_{\mu_{l'}}^{k+1})(o_{\nu_1}^k + \hdots + o_{\nu_l}^k)\Big)\\
		&= \sum_{\substack{k < m \\ k\text{ odd}}}\sum_{p = 1}^l\sum_{q = 1}^{l'}o_{\mu_q}^{k+1} \circ o_{\nu_p}^k
	\end{align}
\end{proof}
Lemma \ref{lem:grip} generalises easily.
\begin{cor}\label{cor:generalised_grip}
	Let $\zeta$ be a proof structure and let $A_i,A_j$ be (possibly equal) conclusions of $\zeta$. Let $\theta_1,...,\theta_e$ denote the maximal length persistent paths in $\zeta$ corresponding to the atomic axioms of $A_i$ and whose final edge is labelled $A_j$, where each of $\theta_1,...,\theta_e$ traverses a cut link in $\zeta$ exactly $n$ times. Let $o_1,...,o_e$ denote the induced operators as described in \eqref{def:matrix_induced_operator} of Definition \ref{def:matrix_interpretation}. Then the $ji^{\text{th}}$ entry of $\llbracket \zeta \rrbracket (\sigma_m \llbracket \zeta \rrbracket)^n$ is given by the sum $o_1 + \hdots + o_e$.
\end{cor}
\begin{cor}\label{cor:nilpotent}
	If $\pi$ is a proof net and $\sigma_m$ is as defined in Definition \ref{def:sigma} then there exists an integer $n > 0$ such that $\llbracket \pi \rrbracket (\sigma_m \llbracket \pi \rrbracket)^n = 0$.
\end{cor}
\begin{proof}
	Follows from Corollary \ref{cor:generalised_grip} along with the fact that maximal length persistent paths are finite, as established in the proof of Corollary \ref{cor:one_max_persistent}.
\end{proof}
\begin{defn}\label{def:execution_formula}
	We define
	\begin{equation}
		\operatorname{Ex}(\llbracket \pi \rrbracket) = \llbracket \pi \rrbracket + \llbracket \pi \rrbracket \sigma \llbracket \pi \rrbracket + \llbracket \pi \rrbracket \sigma \llbracket \pi \rrbracket \sigma \llbracket \pi \rrbracket + \hdots
	\end{equation}
	which by Corollary \ref{cor:nilpotent} is a well defined matrix.
\end{defn}
\begin{remark}\label{rmk:execution_grip}
	Let $\pi$ be a proof structure with conclusions $A_1,...,A_n$ It follows from Corollary \ref{cor:generalised_grip} that the $ji^{\text{th}}$ entry of $\operatorname{Ex}(\llbracket \pi \rrbracket)$ is the sum of operators corresponding under \eqref{def:matrix_induced_operator} of Definition \ref{def:matrix_interpretation} to maximal length persistent paths corresponding to an unoriented atom of $A_i$ and whose final edge is labelled $A_j$.
\end{remark}
\begin{example}
	We continue with $\pi$ from Examples \ref{ex:favourite_example} and \ref{ex:favourite_example_interpretation}. According to Definition \ref{def:sigma} we take $\sigma$ to be the matrix which permutes the first two columns.
	\begin{equation}
		\sigma =
		\begin{pmatrix}
			0 & 1 & 0 & 0\\
			1 & 0 & 0 & 0\\
			0 & 0 & 0 & 0\\
			0 & 0 & 0 & 0
		\end{pmatrix}
	\end{equation}
	and consider $\llbracket \pi \rrbracket \sigma \llbracket \pi \rrbracket$, which is a matrix whose $ij^{\text{th}}$ entry corresponds to the sum of operators corresponding to the paths in $\pi'$ which traverse the cut once, where the start of the path is the conclusion in $\pi'$ with label corresponding to column $j$, and whose end point is the conclusion with label corresponding to row $i$. In our current example this is given as follows, where $r = qp^\ast + pq^\ast$:
	\begin{equation}
		\llbracket \pi \rrbracket \sigma \llbracket \pi \rrbracket =
		\begin{pmatrix}
			0&0&0&0\\
			0&0&r p &r q\\
			0& p^\ast r &0&0\\
			0& q^\ast r &0&0
		\end{pmatrix}
	\end{equation}
	Multiplying by $\sigma \llbracket \pi \rrbracket$ yields:
	\begin{equation}\label{eq:cut_elimination_GoI}
		\llbracket \pi \rrbracket \sigma \llbracket \pi \rrbracket \sigma \llbracket \pi \rrbracket = 
		\begin{pmatrix}
			0&0&0&0\\
			0&0&0&0\\
			0&0& p^\ast r p & q^\ast r p\\
			0&0& p^\ast r q & q^\ast r q
		\end{pmatrix}
		=
		\begin{pmatrix}
			0&0&0&0\\
			0&0&0&0\\
			0&0& 0 & 1\\
			0&0& 1 & 0
		\end{pmatrix}
	\end{equation}
	What happens if we perform the same process to $\pi$ after we have performed cut-elimination? Under this process, $\pi$ corresponds to the proof consisting of a single axiom link:
	% https://q.uiver.app/?q=WzAsNixbMSwwLCJcXGF4Il0sWzAsMSwiXFxuZWcgWCJdLFsyLDEsIlgiXSxbMSwyXSxbMCwyLCJcXG9wZXJhdG9ybmFtZXtjfSJdLFsyLDIsIlxcb3BlcmF0b3JuYW1le2N9Il0sWzAsMSwiIiwwLHsiY3VydmUiOjIsInN0eWxlIjp7ImhlYWQiOnsibmFtZSI6Im5vbmUifX19XSxbMCwyLCIiLDIseyJjdXJ2ZSI6LTIsInN0eWxlIjp7ImhlYWQiOnsibmFtZSI6Im5vbmUifX19XSxbMSw0XSxbMiw1XV0=
	\begin{equation}
		\begin{tikzcd}[column sep = small, row sep = small]
			& \ax \\
			{\neg X} && X \\
			{\operatorname{c}} & {} & {\operatorname{c}}
			\arrow[curve={height=12pt}, no head, from=1-2, to=2-1]
			\arrow[curve={height=-12pt}, no head, from=1-2, to=2-3]
			\arrow[from=2-1, to=3-1]
			\arrow[from=2-3, to=3-3]
		\end{tikzcd}
	\end{equation}
	which corresponds to the matrix
	\begin{equation}
		\begin{pmatrix}
			0 & 1\\
			1 & 0
		\end{pmatrix}
	\end{equation}
	which appears as a minor in \eqref{eq:cut_elimination_GoI}. Proposition \ref{cor:GoI_one} states that this is not a coincidence.
	\end{example}

\begin{defn}\label{def:persistency}
	Let $\gamma: \pi \lto \pi'$ be a reduction, then there is a bijection between the set of persistent paths of $\pi$ and the set of persistent path on $\pi'$ (hence, \emph{persistent} paths). This is according the following schema:
	% https://q.uiver.app/?q=WzAsMTAsWzUsMCwiXFx2ZG90cyJdLFs1LDEsIlxcbmVnIEEiXSxbNSwyLCJcXHZkb3RzIl0sWzQsMCwiXFx2ZG90cyJdLFs0LDEsIlxcbmVnIEEiXSxbMywyLCJcXGN1dCJdLFsyLDEsIkEiXSxbMSwwLCJcXGF4Il0sWzAsMSwiXFxuZWcgQSJdLFswLDIsIlxcdmRvdHMiXSxbMSwyLCIiLDAseyJjb2xvdXIiOlszNTgsMTAwLDYwXX1dLFswLDEsIiIsMCx7ImNvbG91ciI6WzM1OCwxMDAsNjBdLCJzdHlsZSI6eyJoZWFkIjp7Im5hbWUiOiJub25lIn19fV0sWzgsOSwiIiwwLHsiY29sb3VyIjpbMzU4LDEwMCw2MF19XSxbNyw4LCIiLDAseyJjdXJ2ZSI6MiwiY29sb3VyIjpbMzU4LDEwMCw2MF0sInN0eWxlIjp7ImhlYWQiOnsibmFtZSI6Im5vbmUifX19XSxbNyw2LCIiLDIseyJjdXJ2ZSI6LTIsImNvbG91ciI6WzM1OCwxMDAsNjBdLCJzdHlsZSI6eyJoZWFkIjp7Im5hbWUiOiJub25lIn19fV0sWzYsNSwiIiwyLHsiY3VydmUiOjIsImNvbG91ciI6WzM1OCwxMDAsNjBdfV0sWzQsNSwiIiwwLHsiY3VydmUiOi0yLCJjb2xvdXIiOlszNTgsMTAwLDYwXX1dLFszLDQsIiIsMCx7ImNvbG91ciI6WzM1OCwxMDAsNjBdLCJzdHlsZSI6eyJoZWFkIjp7Im5hbWUiOiJub25lIn19fV1d
	\begin{equation}\label{eq:reduction_path_a_one}
		\begin{tikzcd}[column sep = small, row sep = small]
			& \ax &&& \vdots & \vdots \\
			{\neg A} && A && {\neg A} & {\neg A} \\
			\vdots &&& \cut && \vdots
			\arrow[draw={rgb,255:red,255;green,51;blue,58}, from=2-6, to=3-6]
			\arrow[draw={rgb,255:red,255;green,51;blue,58}, no head, from=1-6, to=2-6]
			\arrow[draw={rgb,255:red,255;green,51;blue,58}, from=2-1, to=3-1]
			\arrow[draw={rgb,255:red,255;green,51;blue,58}, curve={height=12pt}, no head, from=1-2, to=2-1]
			\arrow[draw={rgb,255:red,255;green,51;blue,58}, curve={height=-12pt}, no head, from=1-2, to=2-3]
			\arrow[draw={rgb,255:red,255;green,51;blue,58}, curve={height=12pt}, from=2-3, to=3-4]
			\arrow[draw={rgb,255:red,255;green,51;blue,58}, curve={height=-12pt}, from=2-5, to=3-4]
			\arrow[draw={rgb,255:red,255;green,51;blue,58}, no head, from=1-5, to=2-5]
		\end{tikzcd}
	\end{equation}
	% https://q.uiver.app/?q=WzAsMTAsWzUsMCwiXFx2ZG90cyJdLFs1LDEsIkEiXSxbNSwyLCJcXHZkb3RzIl0sWzQsMSwiQSJdLFszLDAsIlxcYXgiXSxbMiwxLCJcXG5lZyBBIl0sWzEsMiwiXFxjdXQiXSxbMCwxLCJBIl0sWzAsMCwiXFx2ZG90cyJdLFs0LDIsIlxcdmRvdHMiXSxbMyw5LCIiLDAseyJjb2xvdXIiOlszNTcsMTAwLDYwXX1dLFs0LDMsIiIsMCx7ImN1cnZlIjotMiwiY29sb3VyIjpbMzU3LDEwMCw2MF0sInN0eWxlIjp7ImhlYWQiOnsibmFtZSI6Im5vbmUifX19XSxbNCw1LCIiLDIseyJjdXJ2ZSI6MiwiY29sb3VyIjpbMzU3LDEwMCw2MF0sInN0eWxlIjp7ImhlYWQiOnsibmFtZSI6Im5vbmUifX19XSxbNSw2LCIiLDIseyJjdXJ2ZSI6LTIsImNvbG91ciI6WzM1NywxMDAsNjBdfV0sWzcsNiwiIiwwLHsiY3VydmUiOjIsImNvbG91ciI6WzM1NywxMDAsNjBdfV0sWzgsNywiIiwwLHsiY29sb3VyIjpbMzU3LDEwMCw2MF0sInN0eWxlIjp7ImhlYWQiOnsibmFtZSI6Im5vbmUifX19XSxbMSwyLCIiLDAseyJjb2xvdXIiOlszNTcsMTAwLDYwXX1dLFswLDEsIiIsMCx7ImNvbG91ciI6WzM1NywxMDAsNjBdLCJzdHlsZSI6eyJoZWFkIjp7Im5hbWUiOiJub25lIn19fV1d
	\begin{equation}\label{eq:reduction_path_a_two}
		\begin{tikzcd}[column sep = small, row sep = small]
			\vdots &&& \ax && \vdots \\
			A && {\neg A} && A & A \\
			& \cut &&& \vdots & \vdots
			\arrow[color={rgb,255:red,255;green,51;blue,61}, from=2-5, to=3-5]
			\arrow[color={rgb,255:red,255;green,51;blue,61}, curve={height=-12pt}, no head, from=1-4, to=2-5]
			\arrow[color={rgb,255:red,255;green,51;blue,61}, curve={height=12pt}, no head, from=1-4, to=2-3]
			\arrow[color={rgb,255:red,255;green,51;blue,61}, curve={height=-12pt}, from=2-3, to=3-2]
			\arrow[color={rgb,255:red,255;green,51;blue,61}, curve={height=12pt}, from=2-1, to=3-2]
			\arrow[color={rgb,255:red,255;green,51;blue,61}, no head, from=1-1, to=2-1]
			\arrow[color={rgb,255:red,255;green,51;blue,61}, from=2-6, to=3-6]
			\arrow[color={rgb,255:red,255;green,51;blue,61}, no head, from=1-6, to=2-6]
		\end{tikzcd}
	\end{equation}
	% https://q.uiver.app/?q=WzAsMjQsWzAsMCwiXFx2ZG90cyJdLFswLDEsIkEiXSxbMSwyLCJcXG90aW1lcyJdLFsxLDMsIkEgXFxvdGltZXMgQiJdLFsyLDEsIkIiXSxbMiwwLCJcXHZkb3RzIl0sWzQsMSwiXFxuZWcgQSJdLFs2LDEsIlxcbmVnIEIiXSxbNiwwLCJcXHZkb3RzIl0sWzQsMCwiXFx2ZG90cyJdLFs1LDIsIlxccGFyciJdLFs1LDMsIlxcbmVnIEEgXFxwYXJyIFxcbmVnIEIiXSxbMyw0LCJcXGN1dCJdLFswLDUsIlxcdmRvdHMiXSxbMiw1LCJcXHZkb3RzIl0sWzAsNiwiQSJdLFsyLDYsIkIiXSxbNCw1LCJcXHZkb3RzIl0sWzYsNSwiXFx2ZG90cyJdLFs0LDYsIlxcbmVnIEEiXSxbNiw2LCJcXG5lZyBCIl0sWzIsOCwiXFxjdXQiXSxbNSw3XSxbNSw4LCJcXGN1dCJdLFswLDEsIiIsMCx7ImNvbG91ciI6WzM1NCwxMDAsNjBdLCJzdHlsZSI6eyJoZWFkIjp7Im5hbWUiOiJub25lIn19fV0sWzUsNCwiIiwwLHsic3R5bGUiOnsiaGVhZCI6eyJuYW1lIjoibm9uZSJ9fX1dLFs0LDIsIiIsMCx7ImN1cnZlIjotMn1dLFsxLDIsIiIsMSx7ImN1cnZlIjoyLCJjb2xvdXIiOlszNTQsMTAwLDYwXX1dLFsyLDMsIiIsMSx7ImNvbG91ciI6WzM1NCwxMDAsNjBdLCJzdHlsZSI6eyJoZWFkIjp7Im5hbWUiOiJub25lIn19fV0sWzMsMTIsIiIsMSx7ImN1cnZlIjoyLCJjb2xvdXIiOlszNTQsMTAwLDYwXX1dLFsxMSwxMiwiIiwxLHsiY3VydmUiOi0yLCJjb2xvdXIiOlszNTQsMTAwLDYwXX1dLFsxMCwxMSwiIiwxLHsiY29sb3VyIjpbMzU0LDEwMCw2MF0sInN0eWxlIjp7ImhlYWQiOnsibmFtZSI6Im5vbmUifX19XSxbNywxMCwiIiwxLHsiY3VydmUiOi0yfV0sWzYsMTAsIiIsMSx7ImN1cnZlIjoyLCJjb2xvdXIiOlszNTQsMTAwLDYwXX1dLFs4LDcsIiIsMSx7InN0eWxlIjp7ImhlYWQiOnsibmFtZSI6Im5vbmUifX19XSxbOSw2LCIiLDEseyJjb2xvdXIiOlszNTQsMTAwLDYwXSwic3R5bGUiOnsiaGVhZCI6eyJuYW1lIjoibm9uZSJ9fX1dLFsxNiwyMywiIiwxLHsiY3VydmUiOjJ9XSxbMjAsMjMsIiIsMSx7ImN1cnZlIjotMn1dLFsxOSwyMSwiIiwxLHsiY3VydmUiOi0yLCJjb2xvdXIiOlszNTQsMTAwLDYwXX1dLFsxNSwyMSwiIiwxLHsiY3VydmUiOjIsImNvbG91ciI6WzM1NCwxMDAsNjBdfV0sWzEzLDE1LCIiLDEseyJjb2xvdXIiOlszNTQsMTAwLDYwXSwic3R5bGUiOnsiaGVhZCI6eyJuYW1lIjoibm9uZSJ9fX1dLFsxNCwxNiwiIiwxLHsic3R5bGUiOnsiaGVhZCI6eyJuYW1lIjoibm9uZSJ9fX1dLFsxNywxOSwiIiwxLHsiY29sb3VyIjpbMzU0LDEwMCw2MF0sInN0eWxlIjp7ImhlYWQiOnsibmFtZSI6Im5vbmUifX19XSxbMTgsMjAsIiIsMSx7InN0eWxlIjp7ImhlYWQiOnsibmFtZSI6Im5vbmUifX19XV0=
	\begin{equation}\label{eq:reduction_path_m_one}
		\begin{tikzcd}[column sep = small, row sep = small]
			\vdots && \vdots && \vdots && \vdots \\
			A && B && {\neg A} && {\neg B} \\
			& \otimes &&&& \parr \\
			& {A \otimes B} &&&& {\neg A \parr \neg B} \\
			&&& \cut \\
			\vdots && \vdots && \vdots && \vdots \\
			A && B && {\neg A} && {\neg B} \\
			&&&&& {} \\
			&& \cut &&& \cut
			\arrow[color={rgb,255:red,255;green,51;blue,71}, no head, from=1-1, to=2-1]
			\arrow[no head, from=1-3, to=2-3]
			\arrow[curve={height=-12pt}, from=2-3, to=3-2]
			\arrow[color={rgb,255:red,255;green,51;blue,71}, curve={height=12pt}, from=2-1, to=3-2]
			\arrow[color={rgb,255:red,255;green,51;blue,71}, no head, from=3-2, to=4-2]
			\arrow[color={rgb,255:red,255;green,51;blue,71}, curve={height=12pt}, from=4-2, to=5-4]
			\arrow[color={rgb,255:red,255;green,51;blue,71}, curve={height=-12pt}, from=4-6, to=5-4]
			\arrow[color={rgb,255:red,255;green,51;blue,71}, no head, from=3-6, to=4-6]
			\arrow[curve={height=-12pt}, from=2-7, to=3-6]
			\arrow[color={rgb,255:red,255;green,51;blue,71}, curve={height=12pt}, from=2-5, to=3-6]
			\arrow[no head, from=1-7, to=2-7]
			\arrow[color={rgb,255:red,255;green,51;blue,71}, no head, from=1-5, to=2-5]
			\arrow[curve={height=12pt}, from=7-3, to=9-6]
			\arrow[curve={height=-12pt}, from=7-7, to=9-6]
			\arrow[color={rgb,255:red,255;green,51;blue,71}, curve={height=-12pt}, from=7-5, to=9-3]
			\arrow[color={rgb,255:red,255;green,51;blue,71}, curve={height=12pt}, from=7-1, to=9-3]
			\arrow[color={rgb,255:red,255;green,51;blue,71}, no head, from=6-1, to=7-1]
			\arrow[no head, from=6-3, to=7-3]
			\arrow[color={rgb,255:red,255;green,51;blue,71}, no head, from=6-5, to=7-5]
			\arrow[no head, from=6-7, to=7-7]
	\end{tikzcd}\end{equation}
	% https://q.uiver.app/?q=WzAsMjQsWzAsMCwiXFx2ZG90cyJdLFswLDEsIkEiXSxbMSwyLCJcXG90aW1lcyJdLFsxLDMsIkEgXFxvdGltZXMgQiJdLFsyLDEsIkIiXSxbMiwwLCJcXHZkb3RzIl0sWzQsMSwiXFxuZWcgQSJdLFs2LDEsIlxcbmVnIEIiXSxbNiwwLCJcXHZkb3RzIl0sWzQsMCwiXFx2ZG90cyJdLFs1LDIsIlxccGFyciJdLFs1LDMsIlxcbmVnIEEgXFxwYXJyIFxcbmVnIEIiXSxbMyw0LCJcXGN1dCJdLFswLDUsIlxcdmRvdHMiXSxbMiw1LCJcXHZkb3RzIl0sWzAsNiwiQSJdLFsyLDYsIkIiXSxbNCw1LCJcXHZkb3RzIl0sWzYsNSwiXFx2ZG90cyJdLFs0LDYsIlxcbmVnIEEiXSxbNiw2LCJcXG5lZyBCIl0sWzIsOCwiXFxjdXQiXSxbNSw3XSxbNSw4LCJcXGN1dCJdLFswLDEsIiIsMCx7InN0eWxlIjp7ImhlYWQiOnsibmFtZSI6Im5vbmUifX19XSxbNSw0LCIiLDAseyJjb2xvdXIiOlszNTQsMTAwLDYwXSwic3R5bGUiOnsiaGVhZCI6eyJuYW1lIjoibm9uZSJ9fX1dLFs0LDIsIiIsMCx7ImN1cnZlIjotMiwiY29sb3VyIjpbMzU0LDEwMCw2MF19XSxbMSwyLCIiLDEseyJjdXJ2ZSI6Mn1dLFsyLDMsIiIsMSx7ImNvbG91ciI6WzM1NCwxMDAsNjBdLCJzdHlsZSI6eyJoZWFkIjp7Im5hbWUiOiJub25lIn19fV0sWzMsMTIsIiIsMSx7ImN1cnZlIjoyLCJjb2xvdXIiOlszNTQsMTAwLDYwXX1dLFsxMSwxMiwiIiwxLHsiY3VydmUiOi0yLCJjb2xvdXIiOlszNTQsMTAwLDYwXX1dLFsxMCwxMSwiIiwxLHsiY29sb3VyIjpbMzU0LDEwMCw2MF0sInN0eWxlIjp7ImhlYWQiOnsibmFtZSI6Im5vbmUifX19XSxbNywxMCwiIiwxLHsiY3VydmUiOi0yLCJjb2xvdXIiOlszNTQsMTAwLDYwXX1dLFs2LDEwLCIiLDEseyJjdXJ2ZSI6Mn1dLFs4LDcsIiIsMSx7ImNvbG91ciI6WzM1NCwxMDAsNjBdLCJzdHlsZSI6eyJoZWFkIjp7Im5hbWUiOiJub25lIn19fV0sWzksNiwiIiwxLHsic3R5bGUiOnsiaGVhZCI6eyJuYW1lIjoibm9uZSJ9fX1dLFsxNiwyMywiIiwxLHsiY3VydmUiOjIsImNvbG91ciI6WzM1NCwxMDAsNjBdfV0sWzIwLDIzLCIiLDEseyJjdXJ2ZSI6LTIsImNvbG91ciI6WzM1NCwxMDAsNjBdfV0sWzE5LDIxLCIiLDEseyJjdXJ2ZSI6LTJ9XSxbMTUsMjEsIiIsMSx7ImN1cnZlIjoyfV0sWzEzLDE1LCIiLDEseyJzdHlsZSI6eyJoZWFkIjp7Im5hbWUiOiJub25lIn19fV0sWzE0LDE2LCIiLDEseyJzdHlsZSI6eyJoZWFkIjp7Im5hbWUiOiJub25lIn19fV0sWzE3LDE5LCIiLDEseyJzdHlsZSI6eyJoZWFkIjp7Im5hbWUiOiJub25lIn19fV0sWzE4LDIwLCIiLDEseyJjb2xvdXIiOlszNTQsMTAwLDYwXSwic3R5bGUiOnsiaGVhZCI6eyJuYW1lIjoibm9uZSJ9fX1dXQ==
	\begin{equation}\label{eq:reduction_path_m_two}
		\begin{tikzcd}[column sep = small, row  sep = small]
			\vdots && \vdots && \vdots && \vdots \\
			A && B && {\neg A} && {\neg B} \\
			& \otimes &&&& \parr \\
			& {A \otimes B} &&&& {\neg A \parr \neg B} \\
			&&& \cut \\
			\vdots && \vdots && \vdots && \vdots \\
			A && B && {\neg A} && {\neg B} \\
			&&&&& {} \\
			&& \cut &&& \cut
			\arrow[no head, from=1-1, to=2-1]
			\arrow[color={rgb,255:red,255;green,51;blue,71}, no head, from=1-3, to=2-3]
			\arrow[color={rgb,255:red,255;green,51;blue,71}, curve={height=-12pt}, from=2-3, to=3-2]
			\arrow[curve={height=12pt}, from=2-1, to=3-2]
			\arrow[color={rgb,255:red,255;green,51;blue,71}, no head, from=3-2, to=4-2]
			\arrow[color={rgb,255:red,255;green,51;blue,71}, curve={height=12pt}, from=4-2, to=5-4]
			\arrow[color={rgb,255:red,255;green,51;blue,71}, curve={height=-12pt}, from=4-6, to=5-4]
			\arrow[color={rgb,255:red,255;green,51;blue,71}, no head, from=3-6, to=4-6]
			\arrow[color={rgb,255:red,255;green,51;blue,71}, curve={height=-12pt}, from=2-7, to=3-6]
			\arrow[curve={height=12pt}, from=2-5, to=3-6]
			\arrow[color={rgb,255:red,255;green,51;blue,71}, no head, from=1-7, to=2-7]
			\arrow[no head, from=1-5, to=2-5]
			\arrow[color={rgb,255:red,255;green,51;blue,71}, curve={height=12pt}, from=7-3, to=9-6]
			\arrow[color={rgb,255:red,255;green,51;blue,71}, curve={height=-12pt}, from=7-7, to=9-6]
			\arrow[curve={height=-12pt}, from=7-5, to=9-3]
			\arrow[curve={height=12pt}, from=7-1, to=9-3]
			\arrow[no head, from=6-1, to=7-1]
			\arrow[no head, from=6-3, to=7-3]
			\arrow[no head, from=6-5, to=7-5]
			\arrow[color={rgb,255:red,255;green,51;blue,71}, no head, from=6-7, to=7-7]
		\end{tikzcd}
	\end{equation}
\end{defn}

\begin{proposition}[Geometry of Interaction One]\label{cor:GoI_one}
	Let $\pi$ be a proof net and $\zeta$ the cut-free proof equivalent under cut elimination to $\pi$. Then the matrix $\llbracket \zeta\rrbracket$ exists as a minor in $Ex(\llbracket \pi \rrbracket)$ and any entry in $\operatorname{Ex}(\llbracket \pi \rrbracket)$ which is not in the minor corresponding to $\llbracket \zeta \rrbracket$ is equal to 0.
\end{proposition}
\begin{proof}
	It is clear by inspection of the rules in Definition \ref{def:persistency} that the transformations preserve persistency of paths. This establishes the first claim.
	
	Now say $x$ is an entry in $\operatorname{Ex}(\llbracket \pi \rrbracket)$ which is non-zero. Remark \ref{rmk:execution_grip} states that $x$ is the sum of operators induced by some collection of persistent paths. Reading \eqref{eq:reduction_path_a_one}, \eqref{eq:reduction_path_a_two}, \eqref{eq:reduction_path_m_one}, \eqref{eq:reduction_path_m_two} in the opposite direction to the first part of this proof, we obtain corresponding maximal length persistent paths in $\zeta$. By Proposition \ref{prop:not_persistent} we have that the sum of these paths correspond to some non-zero entry of $\llbracket \zeta \rrbracket$.
\end{proof}

\section{Exponentials}
\begin{defn}
	The new cut elimination rules are as follows.
	% https://q.uiver.app/?q=WzAsMTgsWzEsMCwiXFx2ZG90cyJdLFsxLDEsIkEiXSxbMSwzLCI/Il0sWzIsNSwiXFxvcGVyYXRvcm5hbWV7Y3V0fSJdLFszLDMsIiEiXSxbMywxLCJcXG5lZyBBIl0sWzMsMCwiXFx2ZG90cyJdLFsxLDQsIj9BIl0sWzMsNCwiIVxcbmVnIEEiXSxbMCwwLCJcXHZkb3RzIl0sWzQsMCwiXFx2ZG90cyJdLFswLDIsIlxcYnVsbGV0Il0sWzQsMiwiXFxidWxsZXQiXSxbNSwwLCJcXHZkb3RzIl0sWzcsMCwiXFx2ZG90cyJdLFs1LDEsIkEiXSxbNywxLCJcXG5lZyBBIl0sWzYsMiwiXFxjdXQiXSxbMCwxLCIiLDAseyJzdHlsZSI6eyJoZWFkIjp7Im5hbWUiOiJub25lIn19fV0sWzEsMl0sWzIsNywiIiwwLHsic3R5bGUiOnsiaGVhZCI6eyJuYW1lIjoibm9uZSJ9fX1dLFs3LDMsIiIsMCx7ImN1cnZlIjoyfV0sWzgsMywiIiwyLHsiY3VydmUiOi0yfV0sWzQsOCwiIiwyLHsic3R5bGUiOnsiaGVhZCI6eyJuYW1lIjoibm9uZSJ9fX1dLFs1LDRdLFs2LDUsIiIsMix7InN0eWxlIjp7ImhlYWQiOnsibmFtZSI6Im5vbmUifX19XSxbMTEsMTIsIiIsMix7InN0eWxlIjp7ImhlYWQiOnsibmFtZSI6Im5vbmUifX19XSxbOSwxMSwiIiwyLHsic3R5bGUiOnsiaGVhZCI6eyJuYW1lIjoibm9uZSJ9fX1dLFsxMCwxMiwiIiwwLHsic3R5bGUiOnsiaGVhZCI6eyJuYW1lIjoibm9uZSJ9fX1dLFsxNSwxNywiIiwwLHsiY3VydmUiOjJ9XSxbMTYsMTcsIiIsMix7ImN1cnZlIjotMn1dLFsxNCwxNiwiIiwyLHsic3R5bGUiOnsiaGVhZCI6eyJuYW1lIjoibm9uZSJ9fX1dLFsxMywxNSwiIiwwLHsic3R5bGUiOnsiaGVhZCI6eyJuYW1lIjoibm9uZSJ9fX1dXQ==
	\[\begin{tikzcd}[column sep = small, row sep = small]
		\vdots & \vdots && \vdots & \vdots & \vdots && \vdots \\
		& A && {\neg A} && A && {\neg A} \\
		\bullet &&&& \bullet && \cut \\
		& {?} && {!} \\
		& {?A} && {!\neg A} \\
		&& {\operatorname{cut}}
		\arrow[no head, from=1-2, to=2-2]
		\arrow[from=2-2, to=4-2]
		\arrow[no head, from=4-2, to=5-2]
		\arrow[curve={height=12pt}, from=5-2, to=6-3]
		\arrow[curve={height=-12pt}, from=5-4, to=6-3]
		\arrow[no head, from=4-4, to=5-4]
		\arrow[from=2-4, to=4-4]
		\arrow[no head, from=1-4, to=2-4]
		\arrow[no head, from=3-1, to=3-5]
		\arrow[no head, from=1-1, to=3-1]
		\arrow[no head, from=1-5, to=3-5]
		\arrow[curve={height=12pt}, from=2-6, to=3-7]
		\arrow[curve={height=-12pt}, from=2-8, to=3-7]
		\arrow[no head, from=1-8, to=2-8]
		\arrow[no head, from=1-6, to=2-6]
	\end{tikzcd}\]
% https://q.uiver.app/?q=WzAsMjYsWzEsNCwiP0EiXSxbNCw0LCIhXFxuZWcgQSJdLFs0LDMsIiEiXSxbNCwxLCJcXG5lZyBBIl0sWzIsNSwiXFxjdXQiXSxbMSwzLCI/Il0sWzMsMiwiXFxidWxsZXQiXSxbNSwyLCJcXGJ1bGxldCJdLFs1LDAsIlxcdmRvdHMiXSxbMywwLCJcXHZkb3RzIl0sWzIsMiwiXFxidWxsZXQiXSxbMiwwLCJcXHZkb3RzIl0sWzAsMCwiXFx2ZG90cyJdLFswLDIsIlxcYnVsbGV0Il0sWzEsMSwiQSJdLFs3LDEsIkEiXSxbOSwxLCJcXG5lZyBBIl0sWzQsMCwiXFx2ZG90cyJdLFsxLDAsIlxcdmRvdHMiXSxbNywwLCJcXHZkb3RzIl0sWzksMCwiXFx2ZG90cyJdLFs4LDIsIlxcb3BlcmF0b3JuYW1le2N1dH0iXSxbNiwzLCJcXGJ1bGxldCJdLFsxMCwzLCJcXGJ1bGxldCJdLFs2LDAsIlxcdmRvdHMiXSxbMTAsMCwiXFx2ZG90cyJdLFswLDQsIiIsMCx7ImN1cnZlIjoyfV0sWzEsNCwiIiwyLHsiY3VydmUiOi0yfV0sWzIsMSwiIiwyLHsic3R5bGUiOnsiaGVhZCI6eyJuYW1lIjoibm9uZSJ9fX1dLFszLDJdLFs1LDAsIiIsMCx7InN0eWxlIjp7ImhlYWQiOnsibmFtZSI6Im5vbmUifX19XSxbNiw3LCIiLDAseyJzdHlsZSI6eyJoZWFkIjp7Im5hbWUiOiJub25lIn19fV0sWzcsOCwiIiwwLHsic3R5bGUiOnsiaGVhZCI6eyJuYW1lIjoibm9uZSJ9fX1dLFs2LDksIiIsMix7InN0eWxlIjp7ImhlYWQiOnsibmFtZSI6Im5vbmUifX19XSxbMTMsMTAsIiIsMix7InN0eWxlIjp7ImhlYWQiOnsibmFtZSI6Im5vbmUifX19XSxbMTIsMTMsIiIsMix7InN0eWxlIjp7ImhlYWQiOnsibmFtZSI6Im5vbmUifX19XSxbMTEsMTAsIiIsMCx7InN0eWxlIjp7ImhlYWQiOnsibmFtZSI6Im5vbmUifX19XSxbMTQsNV0sWzE4LDE0LCIiLDAseyJzdHlsZSI6eyJoZWFkIjp7Im5hbWUiOiJub25lIn19fV0sWzE3LDMsIiIsMix7InN0eWxlIjp7ImhlYWQiOnsibmFtZSI6Im5vbmUifX19XSxbMTksMTUsIiIsMix7InN0eWxlIjp7ImhlYWQiOnsibmFtZSI6Im5vbmUifX19XSxbMjAsMTYsIiIsMix7InN0eWxlIjp7ImhlYWQiOnsibmFtZSI6Im5vbmUifX19XSxbMTUsMjEsIiIsMix7ImN1cnZlIjoyfV0sWzE2LDIxLCIiLDEseyJjdXJ2ZSI6LTJ9XSxbMjQsMjIsIiIsMSx7InN0eWxlIjp7ImhlYWQiOnsibmFtZSI6Im5vbmUifX19XSxbMjIsMjMsIiIsMSx7InN0eWxlIjp7ImhlYWQiOnsibmFtZSI6Im5vbmUifX19XSxbMjUsMjMsIiIsMSx7InN0eWxlIjp7ImhlYWQiOnsibmFtZSI6Im5vbmUifX19XV0=
\[\begin{tikzcd}[column sep = small, row sep = small]
	\vdots & \vdots & \vdots & \vdots & \vdots & \vdots & \vdots & \vdots && \vdots & \vdots \\
	& A &&& {\neg A} &&& A && {\neg A} \\
	\bullet && \bullet & \bullet && \bullet &&& {\operatorname{cut}} \\
	& {?} &&& {!} && \bullet &&&& \bullet \\
	& {?A} &&& {!\neg A} \\
	&& \cut
	\arrow[curve={height=12pt}, from=5-2, to=6-3]
	\arrow[curve={height=-12pt}, from=5-5, to=6-3]
	\arrow[no head, from=4-5, to=5-5]
	\arrow[from=2-5, to=4-5]
	\arrow[no head, from=4-2, to=5-2]
	\arrow[no head, from=3-4, to=3-6]
	\arrow[no head, from=3-6, to=1-6]
	\arrow[no head, from=3-4, to=1-4]
	\arrow[no head, from=3-1, to=3-3]
	\arrow[no head, from=1-1, to=3-1]
	\arrow[no head, from=1-3, to=3-3]
	\arrow[from=2-2, to=4-2]
	\arrow[no head, from=1-2, to=2-2]
	\arrow[no head, from=1-5, to=2-5]
	\arrow[no head, from=1-8, to=2-8]
	\arrow[no head, from=1-10, to=2-10]
	\arrow[curve={height=12pt}, from=2-8, to=3-9]
	\arrow[curve={height=-12pt}, from=2-10, to=3-9]
	\arrow[no head, from=1-7, to=4-7]
	\arrow[no head, from=4-7, to=4-11]
	\arrow[no head, from=1-11, to=4-11]
\end{tikzcd}\]
% https://q.uiver.app/?q=WzAsMTksWzEsMywiP0EiXSxbMywzLCIhXFxuZWcgQSJdLFsyLDQsIlxcY3V0Il0sWzEsMiwiXFxvcGVyYXRvcm5hbWV7Y3RyfSJdLFsyLDEsIj9BIl0sWzIsMCwiXFx2ZG90cyJdLFswLDAsIlxcdmRvdHMiXSxbMCwxLCI/QSJdLFszLDIsIlxcdmRvdHMiXSxbNCwxLCJcXHZkb3RzIl0sWzQsMiwiP0EiXSxbNSwzLCJcXGN1dCJdLFs2LDIsIiEgXFxuZWcgQSJdLFs2LDEsIlxcdmRvdHMiXSxbNywxLCJcXHZkb3RzIl0sWzcsMiwiP0EiXSxbOCwzLCJcXGN1dCJdLFs5LDIsIiFcXG5lZyBBIl0sWzksMSwiXFx2ZG90cyJdLFswLDIsIiIsMCx7ImN1cnZlIjoyfV0sWzEsMiwiIiwyLHsiY3VydmUiOi0yfV0sWzcsMywiIiwyLHsiY3VydmUiOjJ9XSxbNCwzLCIiLDAseyJjdXJ2ZSI6LTJ9XSxbMywwLCIiLDAseyJzdHlsZSI6eyJoZWFkIjp7Im5hbWUiOiJub25lIn19fV0sWzUsNCwiIiwwLHsic3R5bGUiOnsiaGVhZCI6eyJuYW1lIjoibm9uZSJ9fX1dLFs2LDcsIiIsMix7InN0eWxlIjp7ImhlYWQiOnsibmFtZSI6Im5vbmUifX19XSxbOCwxLCIiLDIseyJzdHlsZSI6eyJoZWFkIjp7Im5hbWUiOiJub25lIn19fV0sWzksMTAsIiIsMix7InN0eWxlIjp7ImhlYWQiOnsibmFtZSI6Im5vbmUifX19XSxbMTAsMTEsIiIsMix7ImN1cnZlIjoyfV0sWzEyLDExLCIiLDAseyJjdXJ2ZSI6LTJ9XSxbMTMsMTIsIiIsMCx7InN0eWxlIjp7ImhlYWQiOnsibmFtZSI6Im5vbmUifX19XSxbMTUsMTYsIiIsMCx7ImN1cnZlIjoyfV0sWzE3LDE2LCIiLDIseyJjdXJ2ZSI6LTJ9XSxbMTgsMTcsIiIsMix7InN0eWxlIjp7ImhlYWQiOnsibmFtZSI6Im5vbmUifX19XSxbMTQsMTUsIiIsMCx7InN0eWxlIjp7ImhlYWQiOnsibmFtZSI6Im5vbmUifX19XV0=
\[\begin{tikzcd}[column sep = small, row sep = small]
	\vdots && \vdots \\
	{?A} && {?A} && \vdots && \vdots & \vdots && \vdots \\
	& {\operatorname{ctr}} && \vdots & {?A} && {! \neg A} & {?A} && {!\neg A} \\
	& {?A} && {!\neg A} && \cut &&& \cut \\
	&& \cut
	\arrow[curve={height=12pt}, from=4-2, to=5-3]
	\arrow[curve={height=-12pt}, from=4-4, to=5-3]
	\arrow[curve={height=12pt}, from=2-1, to=3-2]
	\arrow[curve={height=-12pt}, from=2-3, to=3-2]
	\arrow[no head, from=3-2, to=4-2]
	\arrow[no head, from=1-3, to=2-3]
	\arrow[no head, from=1-1, to=2-1]
	\arrow[no head, from=3-4, to=4-4]
	\arrow[no head, from=2-5, to=3-5]
	\arrow[curve={height=12pt}, from=3-5, to=4-6]
	\arrow[curve={height=-12pt}, from=3-7, to=4-6]
	\arrow[no head, from=2-7, to=3-7]
	\arrow[curve={height=12pt}, from=3-8, to=4-9]
	\arrow[curve={height=-12pt}, from=3-10, to=4-9]
	\arrow[no head, from=2-10, to=3-10]
	\arrow[no head, from=2-8, to=3-8]
\end{tikzcd}\]

	\end{defn}
\begin{example}
	Consider the following sequence of reductions.
	% https://q.uiver.app/?q=WzAsMjMsWzIsMSwiXFxheCJdLFsxLDIsIlxcbmVnIFgiXSxbMywyLCJYIl0sWzEsNSwiP1xcbmVnIFgiXSxbMSw0LCI/Il0sWzMsNCwiISJdLFszLDUsIiFYIl0sWzUsNiwiXFxjdXQiXSxbNyw1LCI/XFxuZWcgWCJdLFs3LDQsIj8iXSxbNywyLCJcXG5lZyBYIl0sWzgsMSwiXFxheCJdLFs5LDIsIlgiXSxbOSw0LCIhIl0sWzksNSwiIVgiXSxbNCwzLCJcXGJ1bGxldCJdLFswLDMsIlxcYnVsbGV0Il0sWzYsMywiXFxidWxsZXQiXSxbMTAsMywiXFxidWxsZXQiXSxbMTAsMCwiXFxidWxsZXQiXSxbNiwwLCJcXGJ1bGxldCJdLFswLDAsIlxcYnVsbGV0Il0sWzQsMCwiXFxidWxsZXQiXSxbMCwxLCIiLDAseyJjdXJ2ZSI6Miwic3R5bGUiOnsiaGVhZCI6eyJuYW1lIjoibm9uZSJ9fX1dLFsxLDRdLFs0LDMsIiIsMCx7InN0eWxlIjp7ImhlYWQiOnsibmFtZSI6Im5vbmUifX19XSxbNSw2LCIiLDAseyJzdHlsZSI6eyJoZWFkIjp7Im5hbWUiOiJub25lIn19fV0sWzIsNV0sWzAsMiwiIiwxLHsiY3VydmUiOi0yLCJzdHlsZSI6eyJoZWFkIjp7Im5hbWUiOiJub25lIn19fV0sWzEwLDldLFs5LDgsIiIsMSx7InN0eWxlIjp7ImhlYWQiOnsibmFtZSI6Im5vbmUifX19XSxbMTMsMTQsIiIsMSx7InN0eWxlIjp7ImhlYWQiOnsibmFtZSI6Im5vbmUifX19XSxbMTIsMTNdLFsxMSwxMiwiIiwxLHsiY3VydmUiOi0yLCJzdHlsZSI6eyJoZWFkIjp7Im5hbWUiOiJub25lIn19fV0sWzExLDEwLCIiLDEseyJjdXJ2ZSI6Miwic3R5bGUiOnsiaGVhZCI6eyJuYW1lIjoibm9uZSJ9fX1dLFs4LDcsIiIsMSx7ImN1cnZlIjotMn1dLFs2LDcsIiIsMSx7ImN1cnZlIjoyfV0sWzE2LDE1LCIiLDEseyJzdHlsZSI6eyJoZWFkIjp7Im5hbWUiOiJub25lIn19fV0sWzE3LDE4LCIiLDEseyJzdHlsZSI6eyJoZWFkIjp7Im5hbWUiOiJub25lIn19fV0sWzIwLDE3LCIiLDEseyJzdHlsZSI6eyJoZWFkIjp7Im5hbWUiOiJub25lIn19fV0sWzIwLDE5LCIiLDEseyJzdHlsZSI6eyJoZWFkIjp7Im5hbWUiOiJub25lIn19fV0sWzE5LDE4LCIiLDEseyJzdHlsZSI6eyJoZWFkIjp7Im5hbWUiOiJub25lIn19fV0sWzIxLDIyLCIiLDEseyJzdHlsZSI6eyJoZWFkIjp7Im5hbWUiOiJub25lIn19fV0sWzIyLDE1LCIiLDEseyJzdHlsZSI6eyJoZWFkIjp7Im5hbWUiOiJub25lIn19fV0sWzIxLDE2LCIiLDEseyJzdHlsZSI6eyJoZWFkIjp7Im5hbWUiOiJub25lIn19fV1d
	\[\begin{tikzcd}[column sep = small, row sep = small]
		\bullet &&&& \bullet && \bullet &&&& \bullet \\
		&& \ax &&&&&& \ax \\
		& {\neg X} && X &&&& {\neg X} && X \\
		\bullet &&&& \bullet && \bullet &&&& \bullet \\
		& {?} && {!} &&&& {?} && {!} \\
		& {?\neg X} && {!X} &&&& {?\neg X} && {!X} \\
		&&&&& \cut
		\arrow[curve={height=12pt}, no head, from=2-3, to=3-2]
		\arrow[from=3-2, to=5-2]
		\arrow[no head, from=5-2, to=6-2]
		\arrow[no head, from=5-4, to=6-4]
		\arrow[from=3-4, to=5-4]
		\arrow[curve={height=-12pt}, no head, from=2-3, to=3-4]
		\arrow[from=3-8, to=5-8]
		\arrow[no head, from=5-8, to=6-8]
		\arrow[no head, from=5-10, to=6-10]
		\arrow[from=3-10, to=5-10]
		\arrow[curve={height=-12pt}, no head, from=2-9, to=3-10]
		\arrow[curve={height=12pt}, no head, from=2-9, to=3-8]
		\arrow[curve={height=-12pt}, from=6-8, to=7-6]
		\arrow[curve={height=12pt}, from=6-4, to=7-6]
		\arrow[no head, from=4-1, to=4-5]
		\arrow[no head, from=4-7, to=4-11]
		\arrow[no head, from=1-7, to=4-7]
		\arrow[no head, from=1-7, to=1-11]
		\arrow[no head, from=1-11, to=4-11]
		\arrow[no head, from=1-1, to=1-5]
		\arrow[no head, from=1-5, to=4-5]
		\arrow[no head, from=1-1, to=4-1]
	\end{tikzcd}\]
% https://q.uiver.app/?q=WzAsMTUsWzIsMSwiXFxheCJdLFsxLDIsIlxcbmVnIFgiXSxbMywyLCJYIl0sWzEsNiwiP1xcbmVnIFgiXSxbMSw1LCI/Il0sWzQsMywiXFxjdXQiXSxbNSwyLCJcXG5lZyBYIl0sWzYsMSwiXFxheCJdLFs3LDIsIlgiXSxbNyw1LCIhIl0sWzcsNiwiIVgiXSxbMCw0LCJcXGJ1bGxldCJdLFs4LDQsIlxcYnVsbGV0Il0sWzgsMCwiXFxidWxsZXQiXSxbMCwwLCJcXGJ1bGxldCJdLFswLDEsIiIsMCx7ImN1cnZlIjoyLCJzdHlsZSI6eyJoZWFkIjp7Im5hbWUiOiJub25lIn19fV0sWzEsNF0sWzQsMywiIiwwLHsic3R5bGUiOnsiaGVhZCI6eyJuYW1lIjoibm9uZSJ9fX1dLFswLDIsIiIsMSx7ImN1cnZlIjotMiwic3R5bGUiOnsiaGVhZCI6eyJuYW1lIjoibm9uZSJ9fX1dLFs5LDEwLCIiLDEseyJzdHlsZSI6eyJoZWFkIjp7Im5hbWUiOiJub25lIn19fV0sWzgsOV0sWzcsOCwiIiwxLHsiY3VydmUiOi0yLCJzdHlsZSI6eyJoZWFkIjp7Im5hbWUiOiJub25lIn19fV0sWzcsNiwiIiwxLHsiY3VydmUiOjIsInN0eWxlIjp7ImhlYWQiOnsibmFtZSI6Im5vbmUifX19XSxbNiw1LCIiLDEseyJjdXJ2ZSI6LTJ9XSxbMiw1LCIiLDEseyJjdXJ2ZSI6Mn1dLFsxMSwxMiwiIiwxLHsic3R5bGUiOnsiaGVhZCI6eyJuYW1lIjoibm9uZSJ9fX1dLFsxNCwxMSwiIiwxLHsic3R5bGUiOnsiaGVhZCI6eyJuYW1lIjoibm9uZSJ9fX1dLFsxMywxMiwiIiwxLHsic3R5bGUiOnsiaGVhZCI6eyJuYW1lIjoibm9uZSJ9fX1dLFsxNCwxMywiIiwxLHsic3R5bGUiOnsiaGVhZCI6eyJuYW1lIjoibm9uZSJ9fX1dXQ==
\[\begin{tikzcd}[column sep = small, row sep = small]
	\bullet &&&&&&&& \bullet \\
	&& \ax &&&& \ax \\
	& {\neg X} && X && {\neg X} && X \\
	&&&& \cut \\
	\bullet &&&&&&&& \bullet \\
	& {?} &&&&&& {!} \\
	& {?\neg X} &&&&&& {!X}
	\arrow[curve={height=12pt}, no head, from=2-3, to=3-2]
	\arrow[from=3-2, to=6-2]
	\arrow[no head, from=6-2, to=7-2]
	\arrow[curve={height=-12pt}, no head, from=2-3, to=3-4]
	\arrow[no head, from=6-8, to=7-8]
	\arrow[from=3-8, to=6-8]
	\arrow[curve={height=-12pt}, no head, from=2-7, to=3-8]
	\arrow[curve={height=12pt}, no head, from=2-7, to=3-6]
	\arrow[curve={height=-12pt}, from=3-6, to=4-5]
	\arrow[curve={height=12pt}, from=3-4, to=4-5]
	\arrow[no head, from=5-1, to=5-9]
	\arrow[no head, from=1-1, to=5-1]
	\arrow[no head, from=1-9, to=5-9]
	\arrow[no head, from=1-1, to=1-9]
\end{tikzcd}\]
% https://q.uiver.app/?q=WzAsMTEsWzEsMiwiXFxuZWcgWCJdLFsxLDUsIj9cXG5lZyBYIl0sWzEsNCwiPyJdLFsyLDEsIlxcYXgiXSxbMywyLCJYIl0sWzMsNCwiISJdLFszLDUsIiFYIl0sWzAsMywiXFxidWxsZXQiXSxbNCwzLCJcXGJ1bGxldCJdLFswLDAsIlxcYnVsbGV0Il0sWzQsMCwiXFxidWxsZXQiXSxbMCwyXSxbMiwxLCIiLDAseyJzdHlsZSI6eyJoZWFkIjp7Im5hbWUiOiJub25lIn19fV0sWzUsNiwiIiwxLHsic3R5bGUiOnsiaGVhZCI6eyJuYW1lIjoibm9uZSJ9fX1dLFs0LDVdLFszLDQsIiIsMSx7ImN1cnZlIjotMiwic3R5bGUiOnsiaGVhZCI6eyJuYW1lIjoibm9uZSJ9fX1dLFs3LDgsIiIsMSx7InN0eWxlIjp7ImhlYWQiOnsibmFtZSI6Im5vbmUifX19XSxbMywwLCIiLDEseyJjdXJ2ZSI6Miwic3R5bGUiOnsiaGVhZCI6eyJuYW1lIjoibm9uZSJ9fX1dLFsxMCw4LCIiLDEseyJzdHlsZSI6eyJoZWFkIjp7Im5hbWUiOiJub25lIn19fV0sWzksMTAsIiIsMSx7InN0eWxlIjp7ImhlYWQiOnsibmFtZSI6Im5vbmUifX19XSxbOSw3LCIiLDEseyJzdHlsZSI6eyJoZWFkIjp7Im5hbWUiOiJub25lIn19fV1d
\[\begin{tikzcd}[column sep = small, row sep = small]
	\bullet &&&& \bullet \\
	&& \ax \\
	& {\neg X} && X \\
	\bullet &&&& \bullet \\
	& {?} && {!} \\
	& {?\neg X} && {!X}
	\arrow[from=3-2, to=5-2]
	\arrow[no head, from=5-2, to=6-2]
	\arrow[no head, from=5-4, to=6-4]
	\arrow[from=3-4, to=5-4]
	\arrow[curve={height=-12pt}, no head, from=2-3, to=3-4]
	\arrow[no head, from=4-1, to=4-5]
	\arrow[curve={height=12pt}, no head, from=2-3, to=3-2]
	\arrow[no head, from=1-5, to=4-5]
	\arrow[no head, from=1-1, to=1-5]
	\arrow[no head, from=1-1, to=4-1]
\end{tikzcd}\]
Let $\pi$ denote the first of these proof nets, and let $\pi'$ denote the last of these. Following \cite{girard}, and beginning with the easier case, we associate the following matrix to $\pi'$, with indices $1,2$ respectively corresponding to $! X, ? \neg X$.
\begin{equation}
	\llbracket \pi' \rrbracket = 
	\begin{pmatrix}
		0 & (1 \otimes r)t^\ast\\
		t(1 \otimes r^\ast) & 0
		\end{pmatrix}
	\end{equation}
The matrix associated to $\pi$ is the following, with indices $1,2,3,4$ respectively corresponding to $!X, ? \neg X, !X, ?\neg X$, where the first $!X$ and the first $? \neg X$ are the ones premise to the cut link.
\begin{equation}
	\llbracket \pi \rrbracket = 
	\begin{pmatrix}
		0 & 0 & 0 & (1 \otimes r)t^\ast\\
		0 & 0 & t(1 \otimes r^\ast) & 0\\
		0 & (1 \otimes r)t^\ast & 0 & 0\\
		t(1 \otimes r^\ast) & 0 & 0 & 0
		\end{pmatrix}
	\end{equation}
We calculate $\operatorname{Ex}(\llbracket \pi \rrbracket)$.
\begin{equation}
	\llbracket \pi \rrbracket \sigma \llbracket \pi \rrbracket =
	\begin{pmatrix}
		(1 \otimes r^\ast)t^\ast t (1 \otimes r) & 0 & 0 & 0\\
		0 & t(1 \otimes r^\ast )(1 \otimes r)t^\ast & 0 & 0\\
		0 & 0 & 0 & \big((1 \otimes r)t^\ast\big)^2\\
		0 & 0 & \big(t(1 \otimes r^\ast)\big)^2 & 0
		\end{pmatrix}
	\end{equation}
Hence, for the Execution Formula of \cite{girard} to hold, we require:
\begin{equation}\label{eq:requirement}
	(1 \otimes r)t^\ast \stackrel{\text{Need}}{=} \big((1 \otimes r)t^\ast\big)^2
	\end{equation}
In Equation \eqref{eq:requirement} we have suppressed isomorphisms between $\bb{N}$ and $\bb{N}^2$. We write out the actual equation in full, to begin with we calculate what is on the right of \eqref{eq:requirement}. Let $\beta: \bb{N} \lto \bb{N} \times \bb{N}$ be an isomorphism, in the following, $1$ always means $1_{\bb{N}}$ unless stated otherwise.
\begin{align*}
	\big((1 \otimes r)t^\ast\big)^2 &= \beta^{-1}(1 \otimes \beta^{-1}r)\beta \beta^{-1}(\beta^{-1} \otimes 1)t^{-1}(1 \otimes \beta)\beta \beta^{-1}(\beta^{-1} \otimes 1)t^{-1}(1 \otimes \beta)\beta\\
	&= \beta^{-1}(1 \otimes \beta^{-1}r)(\beta^{-1} \otimes 1)t^{-1}(1 \otimes \beta)(\beta^{-1} \otimes 1)t^{-1}(1 \otimes \beta)\beta\\
	&= \beta^{-1}(\beta^{-1} \otimes \beta^{-1}r)t^{-1}(\beta^{-1} \otimes r)t^{-1}(1 \otimes \beta)\beta
	\end{align*}
Now we calculate what is on the left of \eqref{eq:requirement}.
\begin{align*}
	(1 \otimes r)t^\ast &= \beta^{-1}(1 \otimes \beta^{-1}r)(\beta^{-1} \otimes 1)t^{-1}(1 \otimes \beta)\beta\\
	&= \beta^{-1}(\beta^{-1} \otimes \beta^{-1}r)t^{-1}(1 \otimes \beta)\beta
	\end{align*}
Thus, \eqref{eq:requirement} holds if and only if the following does.
\begin{equation}\label{eq:actual_requirement}
	(1_{\bb{N} \otimes \bb{N}} \otimes \beta^{-1}r)t^{-1}(\beta^{-1} \otimes r) = 1_{\bb{N} \otimes \bb{N}} \otimes \beta^{-1}r: (\bb{N} \times \bb{N}) \times \bb{N}
	\end{equation}
We calculate the left hand side of \eqref{eq:actual_requirement} on an arbitrary value of $(\bb{N} \times \bb{N}) \times \bb{N}$. We use the following choices for $\beta$ and $r$:
\begin{align}
	\beta^{-1}: \bb{N} \times \bb{N} &\lto \bb{N} & r: \bb{N} &\lto \bb{N} \times \bb{N}\\
	(n_1,n_2) &\longmapsto 2^{n_1}(2n_2 + 1)-1 & n &\longmapsto (0,n)
	\end{align}
We calculate.
\begin{align*}
	\big((n_1,n_2),n_3\big) &\stackrel{\beta^{-1} \otimes r}{\longmapsto} \big(2^{n_1}(2n_2 + 1)-1, (0,n_3)\big)\\
	&\stackrel{t^{-1}}{\longmapsto} \big((2^{n_1}(2n_2 + 1)-1,0),n_3\big)\\
	&\stackrel{1\otimes \beta^{-1}r}{\longmapsto} \big((2^{n_1}(2n_2 + 1)-1,0),2(n_3 + 1)-1\big)
	\end{align*}
Now what is on the right of \eqref{eq:actual_requirement}.
\begin{equation*}
	\big((n_1,n_2),n_3\big)\stackrel{1 \otimes \beta^{-1}r}{\longmapsto} \big((n_1,n_2),2(n_3 + 1)-1\big)
	\end{equation*}
Thus, \eqref{eq:actual_requirement} holds if and only if $\beta(n_1,n_2) = n_1$, in fact, this does not hold for any bijection $\beta: \bb{N} \times \bb{N} \lto \bb{N}$.

Thus, we arrive at the following question.
\begin{question}
	Does there exists an appropriate choice of $\beta, r$ so that \eqref{eq:actual_requirement} holds?
	\end{question}

	\end{example}
	
	
	
	
	
	
	
	
	
	
	
	
	
	
	
	
	
	
	
	
	
	
	
	
	
	
	
	
	
	
	
	
	
	
	
	
	
	
	
	
	
	
	
	\bibliographystyle{amsalpha}
	\providecommand{\bysame}{\leavevmode\hbox to3em{\hrulefill}\thinspace}
	\providecommand{\href}[2]{#2}
	\begin{thebibliography}{99}
		\bibitem{ILSC} Intuitionistic, Linear Sequent Calculus, W. Troiani.
		
		\bibitem{linearlogic} \emph{Linear Logic}, J.Y. Girard
		
		\bibitem{multiplicatives} \emph{Multiplicatives}, J.Y. Girard.
		
		\bibitem{girard} \emph{Geometry of Interaction I}, J.Y. Girard.
		
		\bibitem{ILSC} \emph{Intuitionistic, linear sequent calculus} W. Troiani.
		
		\bibitem{proof nets} \emph{proof nets}, W Troiani
		
		\bibitem{laurent} \emph{An introduction to proof nets}
		
		\bibitem{AlgPnt} \emph{Elimination and cut-elimination in multiplicative linear logic}, W. Troiani, D. Murfet.
		
	\end{thebibliography}
\end{document}
