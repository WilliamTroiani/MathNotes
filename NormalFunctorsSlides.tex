\documentclass{beamer}

\usecolortheme{orchid}

\usefonttheme[onlymath]{serif}
%Information to be included in the title page:
\title{Normal funct\sout{ors}[ions], [the irrelevance of] power series, and [a new model of] $\lambda$-calculus.}
\author{Morgan Rogers, Thomas Seiller, William Troiani}
\institute{University of Sorbonne Paris Nord, University of Melbourne}
\date{2022}

\setbeamercolor{background canvas}{bg=gray!20}

\usepackage{amsthm}
\usepackage{amsmath}
\usepackage{amsfonts}
\usepackage{algorithm}
\usepackage{mathrsfs}
\usepackage{array}
\usepackage{amssymb}
\usepackage{units}
\usepackage{graphicx}
\usepackage{tikz-cd}
\usepackage{nicefrac}
\usepackage{hyperref}
\usepackage{bbm}
\usepackage{color}
\usepackage{tensor}
\usepackage{tipa}
\usepackage{bussproofs}
\usepackage{ stmaryrd }
\usepackage{ textcomp }
\usepackage{leftidx}
\usepackage{afterpage}
\usepackage{varwidth}
\usepackage{tasks}
\usepackage{ cmll }
\usepackage{makecell}
\usepackage{MnSymbol}
\usepackage{quiver}
\usepackage{adjustbox}
\usepackage{multirow}
\usepackage{booktabs}
\usepackage{xparse}
\usepackage{calc}
\usepackage[normalem]{ulem}

\newcommand\blankpage{
	\null
	\thispagestyle{empty}
	\addtocounter{page}{-1}
	\newpage
}

\graphicspath{ {images/} }

\theoremstyle{plain}
\newtheorem{thm}{Theorem}[subsection] % reset theorem numbering for each chapter
\newtheorem{proposition}[thm]{Proposition}
%\newtheorem{lemma}[thm]{Lemma}
%\newtheorem{fact}[thm]{Fact}
\newtheorem{cor}[thm]{Corollary}

\theoremstyle{definition}
\newtheorem{defn}[thm]{Definition} % definition numbers are dependent on theorem numbers
\newtheorem{exmp}[thm]{Example} % same for example numbers
\newtheorem{notation}[thm]{Notation}
\newtheorem{remark}[thm]{Remark}
\newtheorem{condition}[thm]{Condition}
\newtheorem{question}[thm]{Question}
\newtheorem{construction}[thm]{Construction}
\newtheorem{exercise}[thm]{Exercise}
%\newtheorem{example}[thm]{Example}
\newtheorem{aside}[thm]{Aside}

\def\doubleunderline#1{\underline{\underline{#1}}}
\newcommand{\bb}[1]{\mathbb{#1}}
\newcommand{\scr}[1]{\mathscr{#1}}
\newcommand{\call}[1]{\mathcal{#1}}
\newcommand{\psheaf}{\text{\underline{Set}}^{\scr{C}^{\text{op}}}}
\newcommand{\und}[1]{\underline{\hspace{#1 cm}}}
\newcommand{\adj}[1]{\text{\textopencorner}{#1}\text{\textcorner}}
\newcommand{\comment}[1]{}
\newcommand{\lto}{\longrightarrow}
\newcommand{\rone}{(\operatorname{R}\bold{1})}
\newcommand{\lone}{(\operatorname{L}\bold{1})}
\newcommand{\rimp}{(\operatorname{R} \multimap)}
\newcommand{\limp}{(\operatorname{L} \multimap)}
\newcommand{\rtensor}{(\operatorname{R}\otimes)}
\newcommand{\ltensor}{(\operatorname{L}\otimes)}
\newcommand{\rtrue}{(\operatorname{R}\top)}
\newcommand{\rwith}{(\operatorname{R}\\&)}
\newcommand{\lwithleft}{(\operatorname{L}\\&)_{\operatorname{left}}}
\newcommand{\lwithright}{(\operatorname{L}\\&)_{\operatorname{right}}}
\newcommand{\rplusleft}{(\operatorname{R}\oplus)_{\operatorname{left}}}
\newcommand{\rplusright}{(\operatorname{R}\oplus)_{\operatorname{right}}}
\newcommand{\lplus}{(\operatorname{L}\oplus)}
\newcommand{\prom}{(\operatorname{prom})}
\newcommand{\ctr}{(\operatorname{ctr})}
\newcommand{\der}{(\operatorname{der})}
\newcommand{\weak}{(\operatorname{weak})}
\newcommand{\exi}{(\operatorname{exists})}
\newcommand{\fa}{(\operatorname{for\text{ }all})}
\newcommand{\ex}{(\operatorname{ex})}
\newcommand{\cut}{(\operatorname{cut})}
\newcommand{\ax}{(\operatorname{ax})}
\newcommand{\negation}{\sim}
\newcommand{\true}{\top}
\newcommand{\false}{\bot}
\DeclareRobustCommand{\diamondtimes}{%
	\mathbin{\text{\rotatebox[origin=c]{45}{$\boxplus$}}}%
}
\newcommand{\tagarray}{\mbox{}\refstepcounter{equation}$(\theequation)$}
\newcommand{\startproof}[1]{
	\AxiomC{#1}
	\noLine
	\UnaryInfC{$\vdots$}
}
\newenvironment{scprooftree}[1]%
{\gdef\scalefactor{#1}\begin{center}\proofSkipAmount \leavevmode}%
	{\scalebox{\scalefactor}{\DisplayProof}\proofSkipAmount \end{center} }
\newcommand\Wider[2][3em]{%
	\makebox[\linewidth][c]{%
		\begin{minipage}{\dimexpr\textwidth+#1\relax}
			\raggedright#2
		\end{minipage}%
	}%
}
% https://tex.stackexchange.com/questions/63355/wrapping-cmidrule-in-a-macro
\ExplSyntaxOn
\makeatletter
\newcommand{\CMidRule}{\noalign\bgroup\@CMidRule{}}
\NewDocumentCommand{\@CMidRule}{
	m % Material to reinsert before cmidrule.
	O{0.0ex} % #1 = left adjust
	O{0.0ex} % #1 = right adjust
	m  %       #3 = columns to span
}{
	\peek_meaning_remove_ignore_spaces:NTF \CMidRule
	{ \@CMidRule { #1 \cmidrule[\cmidrulewidth](l{#2}r{#3}){#4} } }
	{ \egroup #1 \cmidrule[\cmidrulewidth](l{#2}r{#3}){#4} }
}
\makeatother
\ExplSyntaxOff

\DeclareMathOperator{\set}{Set}
\DeclareMathOperator{\finset}{FinSet}
\DeclareMathOperator{\vect}{Vect}
\DeclareMathOperator{\cat}{Cat}
\DeclareMathOperator{\CAT}{CAT}
\DeclareMathOperator{\psh}{PSh}
\DeclareMathOperator{\ind}{Ind}
\DeclareMathOperator{\Hom}{Hom}
\DeclareMathOperator{\Fun}{Fun}
\DeclareMathOperator{\ob}{ob}
\DeclareMathOperator{\colim}{colim}

\newcommand{\PhantC}{\phantom{\colon}}%
\newcommand{\PhantSQ}{\phantom{\sqrt{\hspace{0.3ex}}}}

\newcommand\showdiv[1]{\overline{\smash{)}#1}}


\begin{document}
	
	\frame{\titlepage}
	
	\begin{frame}
	\frametitle{Where did Linear Logic come from?}
	Girard was considering a categorical model of the untyped $\lambda$-calculus where each term $t$ in context $\{ x_1, \ldots, x_n \}$ is interpretted as a normal functor:

	\begin{equation*}
	\llbracket x_1, \ldots, x_n \mid t \rrbracket: \big(\underline{\operatorname{Set}}^{A}\big)^n \lto \underline{\operatorname{Set}}^{A}
	\end{equation*}
	where $A$ is any countably infinite set.

\begin{defn}
A functor $F: \underline{\operatorname{Set}}^A \lto \underline{\operatorname{Set}}$ is \textbf{normal} if it
\begin{itemize}
	\item preserves wide pullbacks,
	\item preserves filtered colimits.
\end{itemize}
\end{defn}
	\end{frame}
	\begin{frame}
	\frametitle{Girard's Normal Functor Theorem}
	\begin{thm}\label{thm:normal_form_theorem}
    Let $\scr{F}: \set^A \lto \set$ be a functor. Then the following are equivalent.
    \begin{itemize}
        \item The functor $\scr{F}$ is normal.
        \item The functor $\scr{F}$ is isomorphic to an analytic functor.
        \item The functor $\scr{F}$ satisfies the finite normal form property.
    \end{itemize}
\end{thm}
\begin{defn}\label{def:analytic}
		A functor $\scr{F}: \set^A \lto \set$ is \textbf{analytic} if there exists a family of sets $\lbrace C_{G}\rbrace_{G \in \set^A}$ such that for all objects $F \in \set^A$ and all morphisms $\mu: F \lto G$ we have
		\begin{equation*}
			\scr{F}(F) = \coprod_{G \in \operatorname{Int}(A) }( C_G\times \set^A(G,F))
			\end{equation*}
		\end{defn}
	\end{frame}
\begin{frame}
\frametitle{Is all of this machinery necessary?}
A critical definition of Girard's is the following.
\begin{defn}
Let $A$ be a set. Define
\begin{equation*}
\operatorname{Int}(A) \subseteq \set^A
\end{equation*}
to be the set of integral functors $G$. That is
\begin{equation*}
|\bigcup_{a \in A}G(a)| < \infty
\end{equation*}
and each $G(a) \in \bb{N}$. Ie, $G(a)$ is one of the following sets
\begin{equation*}
0 = \varnothing,\quad 1 = \{ 0 \} = \{ \varnothing \},\quad n = \{ 0, \ldots, n-1 \}, \ldots
\end{equation*}
\end{defn}
So... why not replace $\operatorname{I}(A)$ with the set of finite multisets of $A$?
\end{frame}

\begin{frame}
\frametitle{Our new model}
Girard's setup seems categorically unnatural. So we ``decategorified'' his model and came up with the following.
\begin{center}
\begin{tabular}{ |c|c| }
\hline
\textbf{Girard} & \textbf{Us} \\
$\set^A$ & $\call{Q}(A) := \{f: A \lto \bb{N} \cup \{ \infty \} \}$\\
$\operatorname{Int}(A)$ & $\call{I}(A) := \{ f: A \lto \bb{N} \mid f\text{ has finite support}\}$\\
\text{Normal} & \text{Preserves supremums}\\
\hline
\end{tabular}
\end{center}
We have the following important observation: let $f: \call{Q}(A) \lto \call{Q}(A)$ be an order preserving function which preserves supremums, and let $\underline{a} \in \call{Q}(A)$ be arbitrary.
\begin{align*}
f(\underline{a}) &= f(\operatorname{sup}_{\underline{a'} \in \call{I}(A)}\underline{a}'\cdot \delta_{\underline{a}' \leq \underline{a}})\\
&= \operatorname{sup}_{\underline{a}' \in \call{I}(A)}f(\underline{a}'\cdot \delta_{\underline{a}' \leq \underline{a}})\\
&= \operatorname{sup}_{\underline{a}' \leq \underline{a}}f(\underline{a}')
\end{align*}
Thus $f$ is determined by its restriction to $\call{I}(A)$.
\end{frame}
\begin{frame}
In fact we have a pair of functions
\begin{equation*}
\begin{tikzcd}[ampersand replacement = \&]
\operatorname{Norm}\Big(\call{Q}(A)^n \times \call{Q}(A), \call{Q}(A)\Big)\arrow[d,shift left, "(-)^+"]\\
\operatorname{Norm}\Big(\call{Q}(A)^n, \call{Q}(\call{I}(A) \times A)\Big)\arrow[u, shift left, "{(-)^{-}}"]
\end{tikzcd}
\end{equation*}
defined as follows, where $\alpha \in \call{Q}(A)^n, (\underline{a}, a) \in \call{I}(A) \times A$.
\begin{align*}
f^+(\alpha)(\underline{a}, a) &= f(\alpha, \underline{a})(a)\\
g^-(\underline{\alpha}, \underline{a})(a) &= \operatorname{sup}_{\underline{a}' \in \call{I}(A)}g(\alpha)(\underline{a}', a)\cdot \delta_{\underline{a}' \leq \underline{a}}
\end{align*}
We think of this as currying.

\begin{lemma}
We have that $(f^+)^- = f$, but in general $(g^-)^+ \neq g$.
\end{lemma}
\end{frame}

\begin{frame}
\frametitle{Terms}
\begin{defn}
		Let $\underline{x} = \{ x_1, \ldots, x_n \}$ be a set of variables and let $t$ be a $\lambda$-term for which $\underline{x}$ is a valid context.
		\begin{itemize}
			\item \textbf{The term $t$ is a variable $x_i \in \underline{x}$}. We define
			\begin{align*}
				\llbracket \underline{x} \mid x_i \rrbracket: \call{Q}(A)^n &\lto \call{Q}(A)\\
				(\underline{a}_1, \ldots, \underline{a}_n) &\longmapsto \underline{a}_i
			\end{align*}
			to be the projection map.
		\end{itemize}
	\end{defn}
\end{frame}

\begin{frame}
\frametitle{Application and abstraction}
Since $A$ is countably infinite, so is $\call{I}(A) \times A$. We fix a bijection $q: \call{I}(A) \times A \lto A$ which induces a bijection $\overline{q}: \call{Q}(A) \lto \call{Q}(\call{I}(A)\times A)$.
\begin{defn}
\textbf{The term $t$ is an application $t = t_1 t_2$}.
			\begin{equation*}
				(\overline{q} \llbracket \underline{x} \mid t_1 \rrbracket)^- \circ \llbracket \underline{x} \mid t_2 \rrbracket: \call{Q}(A)^n \lto \call{Q}(A)
			\end{equation*}
			\textbf{The term $t$ is an abstraction $t = \lambda y. t'$}.
			\begin{equation*}
				\llbracket \underline{x}, y \mid t' \rrbracket: \call{Q}(A)^{n+1} \lto \call{Q}(A)
			\end{equation*}
			We assume that this function is normal. We define
			\begin{equation*}
				\llbracket \underline{x} \mid t \rrbracket := \overline{q}^{-1} \llbracket \underline{x}, y \mid t' \rrbracket^+: \call{Q}(A)^n \lto \call{Q}(A)
			\end{equation*}
		\end{defn}
\end{frame}

\begin{frame}
\frametitle{Substitution Lemma, and denotation model Theorem}
\begin{lemma}
		Let $t,s$ be $\lambda$-terms and $\underline{x} = \{ x_1, \ldots, x_{n} \}$ and $y$ be such that $\underline{x} \cup \{ y \}$ is a valid context for $t$ and $\underline{x}$ is a valid context for $s$. Then for any $\alpha\in \call{Q}(A)^{n}$ we have
		\begin{equation*}\label{eq:sub_lem_cond}
			\llbracket \underline{x} \mid t[y := s]\rrbracket(\alpha) = \llbracket \underline{x}, y \mid t \rrbracket(\alpha, \llbracket \underline{x} \mid s \rrbracket (\alpha))
			\end{equation*}
		\end{lemma}

\begin{thm}\label{thm:denotational_model}
		This is a denotational model of the $\lambda$-calculus. That is, if $t$ is a $\lambda$-term and $\underline{x}$ a valid context for $t$ and for $s$, then we have the following equality.
		\begin{equation*}
			\llbracket \underline{x} \mid (\lambda y. t)s\rrbracket = \llbracket \underline{x} \mid t[y:=s]\rrbracket
			\end{equation*}
		\end{thm}
\end{frame}
	\begin{frame}
	\frametitle{Extending to Linear Logic}
\begin{itemize}
	\item $\call{I}(A)$ looks a lot like $!A$
\item Recall that a normal function $f: \call{Q}(A) \lto \call{Q}(A)$ is equivalent to an order preserving function $\call{I}(A) \lto \call{Q}(A)$.
\item Is there a property which $f$ may satisfy which means it is determined by its restriction to $A$? Yes! Assume $f$ is linear:
\end{itemize}
	  Let $\underline{a} \in \call{Q}(A)$
	\begin{align*}
	f(\underline{a}) &= f(\sum_{a \in A}\underline{a}(a) \cdot \delta_{a})\\
	&=\sum_{a \in A}\underline{a}(a)\cdot f(\delta_a)
	\end{align*}
	So this model has a concept of linearity: \emph{linearity}.
	\end{frame}
	\begin{frame}
	\frametitle{A genuine bijection}
	In fact we have a pair of bijections.
	\begin{equation*}
\begin{tikzcd}
\operatorname{Add}\big(\prod_{i = 1}^n \call{Q}(A_i) \times \call{Q}(A), \call{Q}(B)\big)\arrow[d, shift left, "{(-)^\times}"]\\
\operatorname{Add}\big(\prod_{i = 1}^n\call{Q}(A_i), \call{Q}(A \times B)\big)\arrow[u, shift left, "{(-)^\div}"]
\end{tikzcd}
\end{equation*}
Defined as follows, for $\alpha \in \prod_{i = 1}^n\call{Q}(A_i), \underline{a} \in \call{Q}(A), (a,b) \in A \times B$.
\begin{align*}
f^\times(\alpha)(a,b) &= f(\alpha, \delta_a)(b)\\
g^\div(\alpha, \underline{a}) &= \sum_{a \in A}\underline{a}(a)\cdot g(\alpha)(a,b)
\end{align*}
We use this to define a model of multiplicative, exponential linear logic.
	\end{frame}

\begin{frame}
\frametitle{A taste}
Say the last rule of $\pi$ is given by $\rimp$.
			\begin{center}
				\startproof{$\pi'$}
				\noLine
				\UnaryInfC{$\Gamma, A, \Delta \vdash B$}
				\RightLabel{$\rimp$}
				\UnaryInfC{$\Gamma, \Delta \vdash A \multimap B$}
				\DisplayProof
			\end{center}
			We define
			\begin{equation*}
				\llbracket \pi \rrbracket := \llbracket \pi' \rrbracket^\times
			\end{equation*}
\begin{center}
Say $\Gamma = A_1, \ldots, A_n, \Delta = B_1, \ldots, B_m$\vspace{1em}
\AxiomC{$\llbracket \pi' \rrbracket: \prod_{i = 1}^n\call{Q}(A_i) \times \call{Q}(A) \times \prod_{i = 1}^m\call{Q}(B_i) \lto \call{Q}(B)$}
\RightLabel{$\times$}
\UnaryInfC{$\llbracket \pi \rrbracket = \llbracket \pi' \rrbracket^\times: \prod_{i = 1}^n\call{Q}(A_i) \times \prod_{i = 1}^m\call{Q}(B_i) \lto \call{Q}(A \times B)$}
\DisplayProof
\end{center}
\end{frame}

\begin{frame}
\frametitle{A taste}
\begin{center}
	\startproof{$\pi'$}
	\noLine
	\UnaryInfC{$\Gamma \vdash A$}
	\startproof{$\pi''$}
	\noLine
	\UnaryInfC{$B, \Delta \vdash C$}
	\RightLabel{$(\operatorname{L}\multimap)$}
	\BinaryInfC{$A \multimap B, \Gamma, \Delta \vdash C$}
	\DisplayProof
\end{center}
\begin{align*}
	\llbracket \pi' \rrbracket: \prod_{i = 1}^n \call{Q}(A_i) &\lto \call{Q}(A)\\
	\llbracket \pi'' \rrbracket: \prod_{i = 1}^m \call{Q}(B) \times \call{Q}(B_i) &\lto \call{Q}(C)
\end{align*}
We define
\begin{align*}
\llbracket \pi \rrbracket : &\call{Q}(A \times B) \times \prod_{i = 1}^n \call{Q}(A_i) \times \prod_{i = 1}^m \call{Q}(B_i) \lto \call{Q}(C)\\
&(f, \underline{\alpha}, \beta) \longmapsto \llbracket \pi''\rrbracket\big(\beta, \sum_{a \in A}\llbracket \pi' \rrbracket(\alpha)(a) \cdot f(a, (-)\big)
\end{align*}
\end{frame}

\begin{frame}
\frametitle{$\cut$-reduction invariance}
In the special case where $f = \llbracket \zeta \rrbracket^\times(\gamma)$ for some $\gamma \in \prod_{i = 1}^k \call{Q}(C_i)$, we obtain
\begin{align*}
(\alpha, \beta, \gamma) &\longmapsto \llbracket \pi''\rrbracket\Big(\beta, \sum_{a \in A} \llbracket \pi' \rrbracket(\alpha)(a) \cdot \llbracket \zeta\rrbracket(\gamma)(a,-)\Big)\\
&= \llbracket \pi''\rrbracket\Big(\beta, (\llbracket \zeta\rrbracket^\times)^\div(\gamma, \llbracket \pi'\rrbracket)(\alpha)\Big)\\
&= \llbracket \pi''\rrbracket\Big(\beta, \llbracket \zeta\rrbracket(\gamma, \llbracket \pi'\rrbracket)(\alpha)\Big)
\end{align*}
This calculation proves equality of the interpretations of the two proofs:
\end{frame}

\begin{frame}
\begin{align*}
(\alpha, \beta, \gamma) &\longmapsto \llbracket \pi''\rrbracket\Big(\beta, \sum_{a \in A} \llbracket \pi' \rrbracket(\alpha)(a) \cdot \llbracket \zeta\rrbracket(\gamma)(a,-)\Big)\\
(\alpha, \beta, \gamma) &\longmapsto \llbracket \pi''\rrbracket\Big(\beta, \llbracket \zeta\rrbracket(\gamma, \llbracket \pi'\rrbracket(\alpha))\Big)
\end{align*}
\begin{center}
\startproof{$\zeta$}
\noLine
\UnaryInfC{$\Theta, A \vdash B$}
\RightLabel{$(\operatorname{R}\multimap)$}
\UnaryInfC{$\Theta \vdash A \multimap B$}
\startproof{$\pi'$}
\noLine
\UnaryInfC{$\Gamma \vdash A$}
\startproof{$\pi''$}
\noLine
\UnaryInfC{$\Delta, B \vdash C$}
\RightLabel{$(\operatorname{L}\multimap)$}
\BinaryInfC{$A \multimap B, \Gamma, \Delta \vdash C$}
\RightLabel{$\cut$}
\BinaryInfC{$\Theta, \Gamma, \Delta \vdash C$}
\DisplayProof\\
\vspace{0.5em}
$\lto$\\\vspace{0.5em}
\startproof{$\pi'$}
\noLine
\UnaryInfC{$\Gamma \vdash A$}
\startproof{$\zeta$}
\noLine
\UnaryInfC{$\Theta, A \vdash B$}
\RightLabel{$\cut$}
\BinaryInfC{$\Gamma, \Theta \vdash B$}
\startproof{$\pi''$}
\noLine
\UnaryInfC{$\Delta, B \vdash C$}
\RightLabel{$\cut$}
\BinaryInfC{$\Gamma, \Theta, \Delta \vdash C$}
\DisplayProof
\end{center}
\end{frame}

\begin{frame}
\frametitle{Can we go further?}
\begin{itemize}
	\item Now that we have decategorified Girard's model, can we re-categorify it? At the moment, we are \emph{not} anticipating our model to be an instance of a $\ast$-autonomous category with a comonad satisfying the relevant conditions, it seems as though we have something else.
	\item Once a more general framework is established, can we recover the famous relational model? Taking $\call{Q} = \call{P}$ and relaxing the requirement that our functions be order preserving is a start...
	\item Once we have recovered the relational model, can we transfer the differential structure across to obtain a model of the differential $\lambda$-calculus?
\end{itemize}
\end{frame}

\begin{frame}[allowframebreaks]
	\begin{thebibliography}{9}
		\bibitem{Boole} G.~Boole, \textsl{An Investigation into the Laws of Thought} (1854).
		\bibitem{Grobner} D. Cox, J. Little, D. O'Shea, \emph{Ideals, Varieties, and Algorithms} Fourth Edition, Springer (2015).
		
		\bibitem{girard_llogic}
		J.-Y.~Girard, \textsl{Linear Logic}, Theoretical Computer Science 50 (1), 1--102 (1987).
		
		\bibitem{Girard} J.-Y.~Girard, \emph{Multiplicatives}, Logic and Computer Science: New Trends and Applications. Rosenberg \& Sellier. pp. 11–34 (1987).
		
		\bibitem{towards_goi}
		J.-Y.~Girard, \textsl{Towards a geometry of interaction}, In J.~W.~Gray and A.~Scedrov, editors, Categories in Computer Science and Logic, volume 92 of Contemporary Mathematics, 69--108, AMS (1989).
		
		\bibitem{bs}
		J.-Y.~Girard, \textsl{The Blind Spot: lectures on logic}, European Mathematical Society, (2011).
		
		\bibitem{proofstypes}
		J.-Y.~Girard, Y.~Lafont, and P.~Taylor, \textsl{Proofs and Types}, Cambridge Tracts in Theoretical Computer Science 7 ,Cambridge University Press (1989).
		
		\bibitem{howard} W.~A.~Howard, \textsl{The formulae-as-types notion of construction}, in Seldin and Hindley \textsl{To H.B.Curry: essays on Combinatory logic, Lambda calculus and Formalism}, Academic press (1980).
		
		\bibitem{Laurent} O. Laurent, \emph{An Introduction to Proof Nets}, \url{http://perso.ens-lyon.fr/olivier.laurent/pn.pdf} (2013).
		
		\bibitem{murfet_ll}
		D.~Murfet, \textsl{Logic and Linear Algebra: An Introduction}, preprint \url{https://arxiv.org/abs/1407.2650v3} (2017).
		
		\bibitem{gmz} D.~Murfet and W.~Troiani, \textsl{{G}entzen-{M}ints-{Z}ucker duality}, preprint \url{https://arxiv.org/abs/2008.10131} (2020).
		
		\bibitem{Troiani} W. Troiani, \emph{Linear logic}, lecture notes \url{https://williamtroiani.github.io/MathNotes/LinearLogic.pdf} (2020).
		\end{thebibliography}
	\end{frame}
	
\end{document}